% This is an empty shell file placed for you by the 'examiner' script.
% You can now fill in the TeX for your question here.

% Now, down to brasstacks. ** Writing good solutions is an Art **. 
% Eventually, you will find your own style. But here are some thoughts 
% to get you started: 
%
%   1. Write the solution as if you are writing it for your favorite
%      14-17 year old to help him/her understand. Could be your nephew, 
%      your niece, a cousin perhaps or probably even you when you 
%      were that age. Just write for them.
%
%   2. Use margin-notes to "talk" to students about the critical insights
%      in the question. The tone can be - in fact, should be - informal
%
%   3. Don't shy away from creating margin-figures you think will help
%      students understand. Yes, it is a little more work per question. 
%      But the question & solution will be written only once. Make that
%      attempt at writing a solution count.
%
%   4. At the same time, do not be too verbose. A long solution can
%      - at first sight - make the student think, "God, that is a lot to know".
%      Our aim is not to scare students. Rather, our aim should be to 
%      create many "Aha!" moments everyday in classrooms around the world
% 
%   5. Ensure that there are *no spelling mistakes anywhere*. We are an 
%      education company. Bad spellings suggest that we ourselves 
%      don't have any education. Also, use American spellings by default
% 
%   6. If a question has multiple parts, then first delete lines 40-41
%   7. If a question does not have parts, then first delete lines 43-69

\question Find the expression for the sum of the first $n$ terms of the series 
$4^2 + 7^2 + 10^2 + 13^2 \ldots$

\insertQR{}

\ifprintanswers
\fi 

\begin{solution}
	Each term in the series - $a_n$ - is of the form $(3n+1)^2$. Therefore, the required sum is
	\begin{align}
		S_n &= \sum_{k=1}^{n}(3n+1)^2 = \sum_{k=1}^{n}9n^2+6n+1 \\
		&= 9\cdot\sum_{k=1}^{n}n^2 + 6\cdot\sum_{k=1}^{n}n + \sum_{k=1}^{n}1 \\
		&= 9\cdot\eSumOfSquares{n} + 6\cdot\eSumOfN{n} + n \\
		&= \dfrac{n}{2}\cdot\left[ 3\cdot(n+1)\cdot(2n+1) + 6\cdot(n+1) + 2\right] \\
		&= \dfrac{n}{2}\cdot\left[ 6n^2 + 15n + 11\right]
	\end{align}
\end{solution}
