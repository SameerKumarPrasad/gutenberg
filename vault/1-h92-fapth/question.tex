% This is an empty shell file placed for you by the 'examiner' script.
% You can now fill in the TeX for your question here.

% Now, down to brasstacks. ** Writing good solutions is an Art **. 
% Eventually, you will find your own style. But here are some thoughts 
% to get you started: 
%
%   1. Write to be understood - but be crisp. Your own solution should not take 
%      more space than you will give to the student. Hence, if you take more than 
%      a half-page to write a solution, then give the student a full-page and so on...
%
%   2. Use margin-notes to "talk" to students about the critical insights
%      in the question. The tone can be - in fact, should be - informal
%
%   3. Don't shy away from creating margin-figures you think will help
%      students understand. Yes, it is a little more work per question. 
%      But the question & solution will be written only once. Make that
%      attempt at writing a solution count.
%      
%      3b. Use bc_to_fig.tex. Its an easier way to generate plots & graphs 
% 
%   4. Ensure that there are *no spelling mistakes anywhere*. We are an 
%      education company. Bad spellings suggest that we ourselves 
%      don't have any education. Also, use American spellings by default
% 
%   5. If a question has multiple parts, then first delete lines 40-41
%   6. If a question does not have parts, then first delete lines 43-69
%   
%   7. Create versions of the question when possible. Use commands defined in 
%      tufte-tweaks.sty to do so. Its easier than you think

% \noprintanswers
% \setcounter{rolldice}{3}

\ifnumequal{\value{rolldice}}{0}{
  % variables 
  \renewcommand{\vbone}{5}
  \renewcommand{\vbtwo}{10}
  \renewcommand{\vbthree}{5}
}{
  \ifnumequal{\value{rolldice}}{1}{
    % variables 
    \renewcommand{\vbone}{7}
    \renewcommand{\vbtwo}{15}
    \renewcommand{\vbthree}{6}
  }{
    \ifnumequal{\value{rolldice}}{2}{
      % variables 
      \renewcommand{\vbone}{10}
      \renewcommand{\vbtwo}{21}
      \renewcommand{\vbthree}{7}
    }{
      % variables 
      \renewcommand{\vbone}{6}
      \renewcommand{\vbtwo}{45}
      \renewcommand{\vbthree}{9}
    }
  }
}

\gcalcexpr[0]\tmp{2 * \vbtwo}

\question[2] A city's metro network has $\vbone$ lines. On one of these lines, there are $\vbtwo$ 
possible \textit{fares} that can be charged - depending on which two stations one is going between.
 How many stations are there on this line? 

\insertQR{}

\watchout

\ifprintanswers
\fi 

\begin{solution}[\mcq]
   It is reasonable to assume that the fare between $A$ and $B$ is the same as that 
   between $B$ and $A$. And hence, if there be $N$ stations on the line, then
   \begin{align}
   		N_{\texttt{fares}} &= \encr{N}{2} = \vbtwo \\
   		\Rightarrow \fncr{N}{2} &= \vbtwo \text{ or } N\cdot(N-1) = \tmp \\
   		\Rightarrow N &= \vbthree
   \end{align}
\end{solution}
