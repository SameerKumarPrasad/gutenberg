% This is an empty shell file placed for you by the 'examiner' script.
% You can now fill in the TeX for your question here.

% Now, down to brasstacks. ** Writing good solutions is an Art **. 
% Eventually, you will find your own style. But here are some thoughts 
% to get you started: 
%
%   1. Write the solution as if you are writing it for your favorite
%      14-17 year old to help him/her understand. Could be your nephew, 
%      your niece, a cousin perhaps or probably even you when you 
%      were that age. Just write for them.
%
%   2. Use margin-notes to "talk" to students about the critical insights
%      in the question. The tone can be - in fact, should be - informal
%
%   3. Don't shy away from creating margin-figures you think will help
%      students understand. Yes, it is a little more work per question. 
%      But the question & solution will be written only once. Make that
%      attempt at writing a solution count.
%
%   4. At the same time, do not be too verbose. A long solution can
%      - at first sight - make the student think, "God, that is a lot to know".
%      Our aim is not to scare students. Rather, our aim should be to 
%      create many "Aha!" moments everyday in classrooms around the world
% 
%   5. Ensure that there are *no spelling mistakes anywhere*. We are an 
%      education company. Bad spellings suggest that we ourselves 
%      don't have any education. Also, use American spellings by default
% 
%   6. If a question has multiple parts, then first delete lines 40-41
%   7. If a question does not have parts, then first delete lines 43-69


%\printrubric
%\setcounter{rolldice}{2}
%\noprintanswers

\ifnumequal{\value{rolldice}}{0}{
  % variables 
  \renewcommand{\vbone}{5} %m
  \renewcommand{\vbtwo}{3} %n
  \renewcommand{\vbthree}{4} % first-numerator
  \renewcommand{\vbfour}{7} % 2nd numert
  \renewcommand{\vbfive}{10} % 3rd numertr
}{
	\ifnumequal{\value{rolldice}}{1}{
		\renewcommand{\vbone}{7}
    \renewcommand{\vbtwo}{2}
  	\renewcommand{\vbthree}{3}
  	\renewcommand{\vbfour}{5}
  	\renewcommand{\vbfive}{7}
	}{
	  \ifnumequal{\value{rolldice}}{2}{
      \renewcommand{\vbone}{9}
      \renewcommand{\vbtwo}{4}
      \renewcommand{\vbthree}{5}
      \renewcommand{\vbfour}{9}
      \renewcommand{\vbfive}{13}
	  }{
      \renewcommand{\vbone}{9}
      \renewcommand{\vbtwo}{5}
      \renewcommand{\vbthree}{6}
      \renewcommand{\vbfour}{11}
      \renewcommand{\vbfive}{16}
	  }
	}
}

\gcalcexpr[0]\tp{\vbone - 1}
\gcalcexpr[0]\tq{\vbone + \vbtwo - 1}
\gcalcexpr[0]\tr{\vbone * \tq}
\gcalcexpr[0]\ts{\tp * \tp }


\question[4] Find the sum of the infinite series $1 + \frac{\vbthree}{\vbone} + \frac{\vbfour}{\vbone^2} + 
\frac{\vbfive}{\vbone^3}\ldots$

\watchout
\insertQR[-25pt]{QRC}

\ifprintanswers
\fi 

\begin{solution}[\halfpage]
	If $S$ be the required sum, then
	\begin{align}
		S &= 1 + \frac{\vbthree}{\vbone} + \frac{7}{\vbone^2} + \frac{10}{\vbone^3}\ldots = 
      \sum_{k=0}^{\infty}\dfrac{\vbtwo k+1}{\vbone^k} \\
		\Rightarrow\dfrac{S}{\vbone} &= \sum_{k=0}^{\infty}\dfrac{\vbtwo k+1}{\vbone^{k+1}} \\
		\Rightarrow \underbrace{S - \dfrac{S}{\vbone}}_{= \frac{\tp S}{\vbone}} &= \underbrace{1 + \sum_{m=1}^{\infty}\dfrac{\vbtwo m+1}{\vbone^m}}
		_{\text{first term}}- \sum_{k=0}^{\infty}\dfrac{\vbtwo k+1}{\vbone^{k+1}} \\
		&= 1 + \sum_{k=0}^{\infty}\dfrac{\vbtwo (k+1)+1}{\vbone^{k+1}}-\sum_{k=0}^{\infty}\dfrac{\vbtwo k+1}{\vbone^{k+1}} \\
		&= 1 + \sum_{k=0}^{\infty}\dfrac{\vbtwo }{\vbone^{k+1}} \\
		&= 1 + \vbtwo\cdot\left( \dfrac{1}{\vbone} + \dfrac{1}{\vbone^2} + \dfrac{1}{\vbone^3}\ldots \right) \\
		&= 1 + \vbtwo\cdot\dfrac{\frac{1}{\vbone}}{1-\frac{1}{\vbone}} = 1 + \dfrac{\vbtwo}{\tp } = \dfrac{\tq}{\tp } \\
		\Rightarrow S &= \dfrac{\vbone}{\tp }\cdot\dfrac{\tq}{\tp } = \WRITEFRAC{\tr}{\ts}
	\end{align}
\end{solution}

\ifprintrubric
  \begin{table}
  	\begin{tabular}{ p{5cm}p{5cm} }
  		\toprule % in brief (4-6 words), what should a grader be looking for for insights & formulations
  		  \sc{\textcolor{blue}{Insight}} & \sc{\textcolor{blue}{Formulation}} \\ 
  		\midrule % ***** Insights & formulations ******
        Divide original expression of sum to get a new series & \\
        Subtract the new series from the original & \\
        Resulting series is reduceable & \\
  		\toprule % final numerical answers for the various versions
        \sc{\textcolor{blue}{If question has $\ldots$}} & \sc{\textcolor{blue}{Final answer}} \\
  		\midrule % ***** Numerical answers (below) **********
        $S = 1 + \frac{4}{5} + \frac{7}{5^2}\ldots$ & $35/16$ \\
        $S = 1 + \frac{3}{7} + \frac{5}{7^2}\ldots$ & $14/9$ \\
        $S = 1 + \frac{5}{9} + \frac{9}{9^2}\ldots$ & $27/16$ \\
        $S = 1 + \frac{6}{9} + \frac{11}{9^2}\ldots$ & $117/64$ \\
  		\bottomrule
  	\end{tabular}
  \end{table}
\fi
