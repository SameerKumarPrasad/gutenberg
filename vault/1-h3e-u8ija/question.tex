% This is an empty shell file placed for you by the 'examiner' script.
% You can now fill in the TeX for your question here.

% Now, down to brasstacks. ** Writing good solutions is an Art **. 
% Eventually, you will find your own style. But here are some thoughts 
% to get you started: 
%
%   1. Write the solution as if you are writing it for your favorite
%      14-17 year old to help him/her understand. Could be your nephew, 
%      your niece, a cousin perhaps or probably even you when you 
%      were that age. Just write for them.
%
%   2. Use margin-notes to "talk" to students about the critical insights
%      in the question. The tone can be - in fact, should be - informal
%
%   3. Don't shy away from creating margin-figures you think will help
%      students understand. Yes, it is a little more work per question. 
%      But the question & solution will be written only once. Make that
%      attempt at writing a solution count.
%
%   4. At the same time, do not be too verbose. A long solution can
%      - at first sight - make the student think, "God, that is a lot to know".
%      Our aim is not to scare students. Rather, our aim should be to 
%      create many "Aha!" moments everyday in classrooms around the world
% 
%   5. Ensure that there are *no spelling mistakes anywhere*. We are an 
%      education company. Bad spellings suggest that we ourselves 
%      don't have any education. Also, use American spellings by default
% 
%   6. If a question has multiple parts, then first delete lines 40-41
%   7. If a question does not have parts, then first delete lines 43-69

\question[2] A tower and a building stand opposite each other. If the angle of elevation
from the bottom of the tower to the top of the building is $\ang{30}$ and the angle
of elevation from the bottom of the building to the top of the tower is $\ang{60}$, then
what is the height of the building given that the tower is 50 meters tall

\insertQR{QRC}

\ifprintanswers
  % stuff to be shown only in the answer key - like explanatory margin figures
  \begin{marginfigure}
    \figinit{pt}
      \figpt 100: $A$(0,0)
      \figpt 101: $B$(70,0)
      \figpt 102: $N$(70,50)
      \figpt 103: $M$(0,25)
      \figpt 104: $F$(65,45)
      \figpt 105: $H_{tower} = 50m$(67,25)
      \figpt 106: $\ang{30}$(55,5)
      \figpt 107: $\ang{60}$(10,5)
    \figdrawbegin{}
      \figdrawline [100,101,102,100]
      \figdrawline [101,103,100]
      \figdrawarccircP 100 ; 12 [101,102] 
      \figdrawarccircP 101 ; 12 [103,100] 
    \figdrawend
    \figvisu{\figBoxA}{}{%
      \figwrites 100:(2)
      \figwrites 101:(2)
      \figwritee 102:(2)
      \figwritew 103:(2)
      \figwritee 105:(4)
      \figwritew 106:(2)
      \figwritee 107:(2)
    }
    \centerline{\box\figBoxA}
  \end{marginfigure}
\fi 

\begin{solution}[\halfpage]
	The situation is \asif. And one can see that,
	
	\begin{align}
		AB = \dfrac{BN}{\tan\ang{60}} &= \dfrac{AM}{\tan\ang{30}} \\
		\Rightarrow \dfrac{AM}{BN} &= \dfrac{\tan\ang{30}}{\tan\ang{60}} \\
		\Rightarrow AM &= BN\cdot\dfrac{\frac{1}{\sqrt{3}}}{\sqrt{3}} = \dfrac{BN}{3} \\
		               &= \dfrac{50}{3} = 16\frac{2}{3}\text{ meters}
	\end{align}
\end{solution}

