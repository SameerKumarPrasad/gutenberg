% This is an empty shell file placed for you by the 'examiner' script.
% You can now fill in the TeX for your question here.

% Now, down to brasstacks. ** Writing good solutions is an Art **. 
% Eventually, you will find your own style. But here are some thoughts 
% to get you started: 
%
%   1. Write to be understood - but be crisp. Your own solution should not take 
%      more space than you will give to the student. Hence, if you take more than 
%      a half-page to write a solution, then give the student a full-page and so on...
%
%   2. Use margin-notes to "talk" to students about the critical insights
%      in the question. The tone can be - in fact, should be - informal
%
%   3. Don't shy away from creating margin-figures you think will help
%      students understand. Yes, it is a little more work per question. 
%      But the question & solution will be written only once. Make that
%      attempt at writing a solution count.
%      
%      3b. Use bc_to_fig.tex. Its an easier way to generate plots & graphs 
% 
%   4. Ensure that there are *no spelling mistakes anywhere*. We are an 
%      education company. Bad spellings suggest that we ourselves 
%      don't have any education. Also, use American spellings by default
% 
%   5. If a question has multiple parts, then first delete lines 40-41
%   6. If a question does not have parts, then first delete lines 43-69
%   
%   7. Create versions of the question when possible. Use commands defined in 
%      tufte-tweaks.sty to do so. Its easier than you think

%\noprintanswers
%\setcounter{rolldice}{0}

\ifnumequal{\value{rolldice}}{0}{
  % variables 
  \renewcommand{\vbone}{1} % a 
  \renewcommand{\vbtwo}{2} % b
  \renewcommand{\vbthree}{5} % m
  \renewcommand{\vbfour}{3} % n
}{
  \ifnumequal{\value{rolldice}}{1}{
    % variables 
    \renewcommand{\vbone}{2}
    \renewcommand{\vbtwo}{3}
    \renewcommand{\vbthree}{7}
    \renewcommand{\vbfour}{11}
  }{
    \ifnumequal{\value{rolldice}}{2}{
      % variables 
      \renewcommand{\vbone}{9}
      \renewcommand{\vbtwo}{1}
      \renewcommand{\vbthree}{11}
      \renewcommand{\vbfour}{16}
    }{
      % variables 
      \renewcommand{\vbone}{5}
      \renewcommand{\vbtwo}{8}
      \renewcommand{\vbthree}{9}
      \renewcommand{\vbfour}{17}
    }
  }
}

\gcalcexpr[0]{\vbfive}{\vbfour * \vbone + \vbthree * \vbtwo}
\gcalcexpr[0]{\vbsix}{\vbtwo + \vbfour}

\question[3] A ray of light coming from a point $A = (\vbone, \vbtwo)$ is reflected at a point $B$ \textit{on the x-axis} 
before passing through a point $C = (\vbthree, \vbfour)$ - \asif. Find the coordinates of point $B$. The angle at which light
hits $B$ - and reflected back - is $\theta$

\insertQR{QRC}

\watchout

  \begin{marginfigure}
  % 1. Definition of characteristic points
\figinit{pt}
\def\Xmin{0}
\def\Ymin{1.57894}
\def\Xmax{60.00000}
\def\Ymax{61.57894}
\def\Xori{0}
\def\Yori{-1.57894}
\figpt0:(\Xori,\Yori)

\figpt 100: $A$(11,37) % A
\figpt 101: $C$(48,35) % C
\figpt 102: $B$(30,0) % B
\figpt 103: $\theta$(24,4)
\figpt 104: $\theta$(36,4)
\figpt 105: $M$(53,-1)

% 2. Creation of the graphical file
\figdrawbegin{}
\def\Xmaxx{\Xmax} % To customize the position
\def\Ymaxx{\Ymax} % of the arrow-heads of the axes.
\figset arrowhead(length=4, fillmode=yes) % styling the arrowheads
\figdrawaxes 0(\Xmin, \Xmaxx, \Ymin, \Ymaxx)
\figdrawlineC(
0 59.99999,
1.53846 56.84210,
3.07692 53.68421,
4.61538 50.52631,
6.15384 47.36842,
7.69230 44.21052,
9.23076 41.05263,
10.76923 37.89473,
12.30769 34.73684,
13.84615 31.57894,
15.38461 28.42105,
16.92307 25.26315,
18.46153 22.10526,
19.99999 18.94736,
21.53846 15.78947,
23.07692 12.63157,
24.61538 9.47368,
26.15384 6.31578,
27.69230 3.15789,
29.23076 0,
30.76923 0,
32.30769 3.15789,
33.84615 6.31578,
35.38461 9.47368,
36.92307 12.63157,
38.46153 15.78947,
39.99999 18.94736,
41.53846 22.10526,
43.07692 25.26315,
44.61538 28.42105,
46.15384 31.57894,
47.69230 34.73684,
49.23076 37.89473,
50.76923 41.05263,
52.30769 44.21052,
53.84615 47.36842,
55.38461 50.52631,
56.92307 53.68421,
58.46153 56.84210,
59.99999 59.99999
)
\figdrawend
% 3. Writing text on the figure
\figvisu{\figBoxA}{}{%
\figptsaxes 1:0(\Xmin, \Xmaxx, \Ymin, \Ymaxx)
% Points 1 and 2 are the end points of the arrows
\figwritee 1:(5pt)     \figwriten 2:(5pt)
\figptsaxes 1:0(\Xmin, \Xmax, \Ymin, \Ymax)
\figwritew 103:(2)
\figwritee 104:(2)
\figset write(mark=$\bullet$)
\figwritee 100:(2)
\figwritee 101:(2)
\figwrites 102:(4)
\figwrites 105:(4)
}
\centerline{\box\figBoxA}


  \end{marginfigure}


\begin{solution}[\halfpage]
	The slop of the incoming beam of light - $AB$ - is $\tan\angle{MBA} = \tan (\pi-\theta) = -\tan\theta$ 
	and of the reflected beam - $BC$ - is $\tan\angle{MBC} = \tan\theta$
	
	Also, note that as $B$ is on the x-axis, its coordinates are of the form $(x,0)$
	
	\begin{align}
		\Rightarrow \underbrace{\dfrac{0-\vbtwo}{x-\vbone}}_{\tan\angle{MBA}} &= 
		-\underbrace{\dfrac{0-\vbfour}{x-\vbthree}}_{\tan\angle{MBC}} \\
		\Rightarrow \vbtwo\cdot(x-\vbthree) &= -\vbfour\cdot x + \vbfour\cdot\vbone \\
		\Rightarrow x &= \dfrac{\vbfive}{\vbsix}
	\end{align}
	Hence, the light hits the flat surface at $B = (\dfrac{\vbfive}{\vbsix}, 0)$
	
\end{solution}
