% This is an empty shell file placed for you by the 'examiner' script.
% You can now fill in the TeX for your question here.

% Now, down to brasstacks. ** Writing good solutions is an Art **. 
% Eventually, you will find your own style. But here are some thoughts 
% to get you started: 
%
%   1. Write the solution as if you are writing it for your favorite
%      14-17 year old to help him/her understand. Could be your nephew, 
%      your niece, a cousin perhaps or probably even you when you 
%      were that age. Just write for them.
%
%   2. Use margin-notes to "talk" to students about the critical insights
%      in the question. The tone can be - in fact, should be - informal
%
%   3. Don't shy away from creating margin-figures you think will help
%      students understand. Yes, it is a little more work per question. 
%      But the question & solution will be written only once. Make that
%      attempt at writing a solution count.
%
%   4. At the same time, do not be too verbose. A long solution can
%      - at first sight - make the student think, "God, that is a lot to know".
%      Our aim is not to scare students. Rather, our aim should be to 
%      create many "Aha!" moments everyday in classrooms around the world
% 
%   5. Ensure that there are *no spelling mistakes anywhere*. We are an 
%      education company. Bad spellings suggest that we ourselves 
%      don't have any education. Also, use American spellings by default
% 
%   6. If a question has multiple parts, then first delete lines 40-41
%   7. If a question does not have parts, then first delete lines 43-69

\question[3] Suppose there are 2 bowls of cookies. Bowl A has 10 chocolate chip and 30 
	plain cookies while Bowl B has 20 of each. Fred picks a bowl at random 
	and from the picked bowl, a cookie at random. The cookie turns out to be 
	a plain one. What is the probability that Fred picked it out of Bowl A?

\insertQR{QRC}

\ifprintanswers
	\begin{table}
		\begin{tabular}{ccc}
		   \toprule
		   & A & B \\
		   \midrule
		   P(picks bowl $X$) & 0.5 & 0.5 \\
		   P(plain cookie $\vert X$) & $\frac{30}{40} = 0.75$ & $\frac{20}{40} = 0.5$ \\
		   P(chocolate cookie $\vert X$) & 0.25 & 0.5 \\
		   \bottomrule
		\end{tabular}
	\end{table}
\fi 

\begin{solution}[\halfpage]
	If $P(A)$ and $P(B)$ be the probabilities that a cookie is picked from bowls A and B respectively,
	then the probability we need is, 
	
	\begin{align}
		P(A \vert plain) &= \dfrac{P(plain \vert A) \cdot P(A)}{P(plain)} \\
		  &= \dfrac{P(plain \vert A) \cdot P(A)}{P(plain \vert A)\cdot P(A) + P(plain \vert B)\cdot P(B)} \\
		  &= \dfrac{0.75 \times 0.5}{0.75\times 0.5 + 0.5\times 0.5} \\
		  &= 0.6
	\end{align}

\end{solution}
