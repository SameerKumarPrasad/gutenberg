% This is an empty shell file placed for you by the 'examiner' script.
% You can now fill in the TeX for your question here.

% Now, down to brasstacks. ** Writing good solutions is an Art **. 
% Eventually, you will find your own style. But here are some thoughts 
% to get you started: 
%
%   1. Write to be understood - but be crisp. Your own solution should not take 
%      more space than you will give to the student. Hence, if you take more than 
%      a half-page to write a solution, then give the student a full-page and so on...
%
%   2. Use margin-notes to "talk" to students about the critical insights
%      in the question. The tone can be - in fact, should be - informal
%
%   3. Don't shy away from creating margin-figures you think will help
%      students understand. Yes, it is a little more work per question. 
%      But the question & solution will be written only once. Make that
%      attempt at writing a solution count.
%      
%      3b. Use bc_to_fig.tex. Its an easier way to generate plots & graphs 
% 
%   4. Ensure that there are *no spelling mistakes anywhere*. We are an 
%      education company. Bad spellings suggest that we ourselves 
%      don't have any education. Also, use American spellings by default
% 
%   5. If a question has multiple parts, then first delete lines 40-41
%   6. If a question does not have parts, then first delete lines 43-69
%   
%   7. Create versions of the question when possible. Use commands defined in 
%      tufte-tweaks.sty to do so. Its easier than you think

% \noprintanswers
% \setcounter{rolldice}{0}
% \printrubric

\ifnumequal{\value{rolldice}}{0}{
  % variables 
  \renewcommand{\vbone}{}
  \renewcommand{\vbtwo}{}
  \renewcommand{\vbthree}{}
  \renewcommand{\vbfour}{}
  \renewcommand{\vbfive}{}
  \renewcommand{\vbsix}{}
  \renewcommand{\vbseven}{}
  \renewcommand{\vbeight}{}
  \renewcommand{\vbnine}{}
  \renewcommand{\vbten}{}
}{
  \ifnumequal{\value{rolldice}}{1}{
    % variables 
    \renewcommand{\vbone}{}
    \renewcommand{\vbtwo}{}
    \renewcommand{\vbthree}{}
    \renewcommand{\vbfour}{}
    \renewcommand{\vbfive}{}
    \renewcommand{\vbsix}{}
    \renewcommand{\vbseven}{}
    \renewcommand{\vbeight}{}
    \renewcommand{\vbnine}{}
    \renewcommand{\vbten}{}
  }{
    \ifnumequal{\value{rolldice}}{2}{
      % variables 
      \renewcommand{\vbone}{}
      \renewcommand{\vbtwo}{}
      \renewcommand{\vbthree}{}
      \renewcommand{\vbfour}{}
      \renewcommand{\vbfive}{}
      \renewcommand{\vbsix}{}
      \renewcommand{\vbseven}{}
      \renewcommand{\vbeight}{}
      \renewcommand{\vbnine}{}
      \renewcommand{\vbten}{}
    }{
      % variables 
      \renewcommand{\vbone}{}
      \renewcommand{\vbtwo}{}
      \renewcommand{\vbthree}{}
      \renewcommand{\vbfour}{}
      \renewcommand{\vbfive}{}
      \renewcommand{\vbsix}{}
      \renewcommand{\vbseven}{}
      \renewcommand{\vbeight}{}
      \renewcommand{\vbnine}{}
      \renewcommand{\vbten}{}
    }
  }
}

\question Consider the curve given by $xy^2-x^3y=6$.

\insertQR{}


\ifprintanswers
  % stuff to be shown only in the answer key - like explanatory margin figures
  \begin{marginfigure}
    \figinit{pt}
      \figpt 100:(0,0)
      \figpt 101:(0,0)
    \figdrawbegin{}
      \figdrawline [100,101]
    \figdrawend
    \figvisu{\figBoxA}{}{%
    }
    \centerline{\box\figBoxA}
  \end{marginfigure}
\fi 

\begin{parts}
  \part[2] Show that $\dydx = \dfrac{3x^2y-y^2}{2xy-x^3}$.

\begin{solution}[\mcq]
    \begin{align}
                  &xy^2-x^3y=6 \\
      \Rightarrow &x(2y)\dydx+y^2\cdot 1-\left[x^3\dydx +y(3x^2)\right]=0 \\
      \Rightarrow &\dydx = \dfrac{3x^2y-y^2}{2xy-x^3}    
    \end{align}
  \end{solution}

  \part[3] Find all the points on the curve whose \textit{x-coordinate} is $1$ and 
  write an equation for the tangent line at each of these points.

\begin{solution}[\mcq]
    Substitute $x=1$ in the equation of the curve,
    \begin{align}
      &y^2-y=6 \Rightarrow y=3 \text{ or } y=-2
    \end{align}
    To find the equation for the tangent line we need the $derivative$ (slope)
    of the function at these two points.
    \begin{align}
      \dydx {\displaystyle\vert}_{(1,3)} &=0 \Rightarrow y=3\\ 
      \dydx {\displaystyle\vert}_{(1,-2)}&=2 \Rightarrow y+2=2(x-1) \\
                                         &\quad\quad\Rightarrow y=2x-4)
    \end{align}
  \end{solution}

  \part[3] Find the \textit{x-coordinate} of each point on the curve where the tangent
  line is vertical.
\begin{solution}[\mcq]
    The tangent is vertical when $\dfrac{\ud x}{\ud y}$ is $0$. Therefore,
    \begin{align}
                  &\dfrac{2xy-x^3}{3x^2y-y^2}=0 \\
      \Rightarrow &2xy-x^3=0 \textit{ (and $3x^2y-y^2\neq 0$)} \\
      \Rightarrow &2xy-x^3=0 \Rightarrow x(2y-x^2)=0 \\
      \Rightarrow &x=0 \text{ or } y=\dfrac{1}{2}x^2
    \end{align}
    $x=0$ rejected, therefore using $y =\dfrac{1}{2}x^2$
    \begin{align}
                  &\dfrac{1}{4}x^5-\dfrac{1}{2}x^5=6 \\
      \Rightarrow &x^5 =-24 \Rightarrow x=\sqrt[5]{-24}
    \end{align}
  \end{solution}
\end{parts}

\ifprintrubric
  \begin{table}
  	\begin{tabular}{ p{5cm}p{5cm} }
  		\toprule % in brief (4-6 words), what should a grader be looking for for insights & formulations
  		  \sc{\textcolor{blue}{Insight}} & \sc{\textcolor{blue}{Formulation}} \\ 
  		\midrule % ***** Insights & formulations ******
  		\toprule % final numerical answers for the various versions
        \sc{\textcolor{blue}{If question has $\ldots$}} & \sc{\textcolor{blue}{Final answer}} \\
  		\midrule % ***** Numerical answers (below) **********
  		\bottomrule
  	\end{tabular}
  \end{table}
\fi
