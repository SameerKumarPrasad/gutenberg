% This is an empty shell file placed for you by the 'examiner' script.
% You can now fill in the TeX for your question here.

% Now, down to brasstacks. ** Writing good solutions is an Art **. 
% Eventually, you will find your own style. But here are some thoughts 
% to get you started: 
%
%   1. Write the solution as if you are writing it for your favorite
%      14-17 year old to help him/her understand. Could be your nephew, 
%      your niece, a cousin perhaps or probably even you when you 
%      were that age. Just write for them.
%
%   2. Use margin-notes to "talk" to students about the critical insights
%      in the question. The tone can be - in fact, should be - informal
%
%   3. Don't shy away from creating margin-figures you think will help
%      students understand. Yes, it is a little more work per question. 
%      But the question & solution will be written only once. Make that
%      attempt at writing a solution count.
%
%   4. At the same time, do not be too verbose. A long solution can
%      - at first sight - make the student think, "God, that is a lot to know".
%      Our aim is not to scare students. Rather, our aim should be to 
%      create many "Aha!" moments everyday in classrooms around the world
% 
%   5. Ensure that there are *no spelling mistakes anywhere*. We are an 
%      education company. Bad spellings suggest that we ourselves 
%      don't have any education. And, use American spellings

\question[3] ABC Bank is offering to double your money! They say that if you invest with them, at 6\% interest compounded quarterly, your money will double in no time. But their advertisement fails to mention how long it would actually take! Can you find out?

\ifprintanswers
  % stuff to be shown only in the answer key - like explanatory margin figures
\fi 

\begin{solution}[\halfpage]
	Let the time it would take to double your money be n years. It follows that,
	\begin{align}
			2P &= P\left(1+\dfrac{6}{100}\right)^\text{4n} \\
		 	 2 &= \left(\dfrac{106}{100}\right)^\text{4n} \\
		\log 2 &= 4n\left(\log 106 -2\right) \\
			 n &= \dfrac{\log 2}{4\left(\log 106 -2\right)} \\
			 n &= 3.7625
	\end{align}
	ABC Bank is therefore offering to double your money in 3.76 years!
\end{solution}
