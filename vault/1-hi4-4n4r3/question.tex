
%\noprintanswers
%\setcounter{rolldice}{3}
%\printrubric

\ifnumequal{\value{rolldice}}{0}{
  % variables 
  \renewcommand{\vbone}{4}
  \renewcommand{\vbtwo}{5}
  \renewcommand{\vbthree}{3}
}{
  \ifnumequal{\value{rolldice}}{1}{
    % variables 
    \renewcommand{\vbone}{6}
    \renewcommand{\vbtwo}{7}
    \renewcommand{\vbthree}{4}
  }{
    \ifnumequal{\value{rolldice}}{2}{
      % variables 
      \renewcommand{\vbone}{4}
      \renewcommand{\vbtwo}{9}
      \renewcommand{\vbthree}{5}
    }{
      % variables 
      \renewcommand{\vbone}{5}
      \renewcommand{\vbtwo}{8}
      \renewcommand{\vbthree}{3}
    }
  }
}

\ADD\vbone{1}\a
\SUBTRACT\vbtwo\vbthree\b
\MULTIPLY\vbone\vbtwo\c
\DIVIDE\c\b\ans

\question Find $N$ if $\dfrac{\enpr{N}\vbone}{\enpr{N-1}\vbone} = \dfrac{\vbtwo}{\vbthree}$

\insertQR[-20pt]{}

\watchout

\ifprintanswers
\fi 

\begin{solution}
	\begin{align}
		\dfrac{\enpr{N}\vbone}{\enpr{N-1}\vbone} &= \fnpr{N}\vbone\times\overbrace{\dfrac{(N-1-\vbone)!}{(N-1)!}}^{(N-\a)!} \\
		&= \dfrac{N}{N-\vbone} \\
		\text{ And }\therefore \dfrac{N}{N-\vbone} &= \dfrac{\vbtwo}{\vbthree} \Rightarrow N = \ans
	\end{align}
\end{solution}

\ifprintrubric
  \begin{table}
  	\begin{tabular}{ p{5cm}p{5cm} }
  		\toprule % in brief (4-6 words), what should a grader be looking for for insights & formulations
  		  \sc{\textcolor{blue}{Look for the following}} \\ 
  		\midrule % ***** Insights & formulations ******
        Not much. Must apply formula for $\enpr{N}{M}$ correctly & \\
  		\toprule % final numerical answers for the various versions
        \sc{\textcolor{blue}{If question has $\ldots$}} & \sc{\textcolor{blue}{Final answer}} \\
  		\midrule % ***** Numerical answers (below) **********
        Ratio = $\frac{5}{3}$ & $N=10$ \\
        Ratio = $\frac{7}{4}$ & $N=14$ \\
        Ratio = $\frac{9}{5}$ & $N=9$ \\
        Ratio = $\frac{8}{3}$ & $N=8$ \\
  		\bottomrule
  	\end{tabular}
  \end{table}
\fi
