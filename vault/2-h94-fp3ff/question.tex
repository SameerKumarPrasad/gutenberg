% This is an empty shell file placed for you by the 'examiner' script.
% You can now fill in the TeX for your question here.

% Now, down to brasstacks. ** Writing good solutions is an Art **. 
% Eventually, you will find your own style. But here are some thoughts 
% to get you started: 
%
%   1. Write to be understood - but be crisp. Your own solution should not take 
%      more space than you will give to the student. Hence, if you take more than 
%      a half-page to write a solution, then give the student a full-page and so on...
%
%   2. Use margin-notes to "talk" to students about the critical insights
%      in the question. The tone can be - in fact, should be - informal
%
%   3. Don't shy away from creating margin-figures you think will help
%      students understand. Yes, it is a little more work per question. 
%      But the question & solution will be written only once. Make that
%      attempt at writing a solution count.
%      
%      3b. Use bc_to_fig.tex. Its an easier way to generate plots & graphs 
% 
%   4. Ensure that there are *no spelling mistakes anywhere*. We are an 
%      education company. Bad spellings suggest that we ourselves 
%      don't have any education. Also, use American spellings by default
% 
%   5. If a question has multiple parts, then first delete lines 40-41
%   6. If a question does not have parts, then first delete lines 43-69
%   
%   7. Create versions of the question when possible. Use commands defined in 
%      tufte-tweaks.sty to do so. Its easier than you think

% \noprintanswers
% \setcounter{rolldice}{0}

\ifnumequal{\value{rolldice}}{0}{
  % variables 
  \renewcommand{\vbone}{}
  \renewcommand{\vbtwo}{}
  \renewcommand{\vbthree}{}
  \renewcommand{\vbfour}{}
  \renewcommand{\vbfive}{}
  \renewcommand{\vbsix}{}
  \renewcommand{\vbseven}{}
  \renewcommand{\vbeight}{}
  \renewcommand{\vbnine}{}
  \renewcommand{\vbten}{}
}{
  \ifnumequal{\value{rolldice}}{1}{
    % variables 
    \renewcommand{\vbone}{}
    \renewcommand{\vbtwo}{}
    \renewcommand{\vbthree}{}
    \renewcommand{\vbfour}{}
    \renewcommand{\vbfive}{}
    \renewcommand{\vbsix}{}
    \renewcommand{\vbseven}{}
    \renewcommand{\vbeight}{}
    \renewcommand{\vbnine}{}
    \renewcommand{\vbten}{}
  }{
    \ifnumequal{\value{rolldice}}{2}{
      % variables 
      \renewcommand{\vbone}{}
      \renewcommand{\vbtwo}{}
      \renewcommand{\vbthree}{}
      \renewcommand{\vbfour}{}
      \renewcommand{\vbfive}{}
      \renewcommand{\vbsix}{}
      \renewcommand{\vbseven}{}
      \renewcommand{\vbeight}{}
      \renewcommand{\vbnine}{}
      \renewcommand{\vbten}{}
    }{
      % variables 
      \renewcommand{\vbone}{}
      \renewcommand{\vbtwo}{}
      \renewcommand{\vbthree}{}
      \renewcommand{\vbfour}{}
      \renewcommand{\vbfive}{}
      \renewcommand{\vbsix}{}
      \renewcommand{\vbseven}{}
      \renewcommand{\vbeight}{}
      \renewcommand{\vbnine}{}
      \renewcommand{\vbten}{}
    }
  }
}

\question The figure alongside shows the graph of $f'$, the derivative of a
function $f$. The domain of $f$ is the set of all Real numbers $x$ such that
$-3<x<5$.
  \insertQR{QRC}

  \begin{marginfigure}
    \figinit{pt}
      \figpt 10:(0,0)
      \figpt 15:(-35,10)
      \figpt 20:(-30,10)
      \figpt 30:(-20,0)
      \figpt 40:(-10,-20)
      \figpt 50:(0,-10)
      \figpt 60:(10,0)
      \figpt 61:(12,-1)
      \figpt 70:(30,-30)
      \figpt 80:(40,0)
      \figpt 90:(50,10)
      \figpt 95:(55,10)
      % extremeties
      \def\Xmax{60}
      \def\Ymax{40}
      \def\Xmin{-60}
      \def\Ymin{-40}
      % pts for numbering the axes (5, 10, 15...)
      \figpt 201:$\tiny\text{-3}$(-30,0)
      \figpt 202:$\tiny\text{-2}$(-20,0)
      \figpt 203:$\tiny\text{-1}$(-10,0)
      \figpt 204:$\tiny\text{O}$(0,0)
      \figpt 205:$\tiny\text{1}$(10,0)
      \figpt 206:$\tiny\text{2}$(20,0)
      \figpt 207:$\tiny\text{3}$(30,0)
      \figpt 208:$\tiny\text{4}$(40,0)
      \figpt 209:$\tiny\text{5}$(50,0)
      % label graph
      \figpt 100:(60,0)
      \figpt 101:(0,40)
      \figpt 102:(0,-60)
    \figdrawbegin{}
      \figdrawcurve [15,20,30,40,50,61,70,80,90,95]
      \figset arrowhead(length=4, fillmode=yes)
      \figdrawaxes 10(\Xmin, \Xmax, \Ymin, \Ymax)
      \figset(dash=5)
      \figdrawline [201,20]
      \figdrawline [203,40]
      \figdrawline [207,70]
      \figdrawline [209,90]
    \figdrawend
    \figvisu{\figBoxA}{}{%
      \figwritee 100:$\text{x}$(2pt)
      \figwriten 101:$\text{f'(x)}$(2pt)
      \figwrites 201,202,203,204,205,206,207,208,209 :(2pt)
      \figwritec[102] {\textit{y = f'(x)}}
    }
    \centerline{\box\figBoxA}
  \end{marginfigure}

\begin{parts}
  \part[2] For what values of $x$ does $f$ have a relative maximum? Why?

\begin{solution}[\mcq]
    For a function to have a relative maximum, it's first derivative must go
    from positive to negative. In this case $f'(x)$ goes from positive to 
    negative at $x=-2$. \\
    This means the function $f$ is increasing to the left of $x=-2$ and 
    decreasing to the right of $x=-2$.
  \end{solution}

  \part[2] For what values of $x$ does $f$ have a relative minimum? Why?

\begin{solution}[\mcq]
    Similar logic. A relative minimum occurs when the first derivative goes from
    negative to positive. In this case $f'(x)$ goes from negative to positive
    at $x=4$.
    Therefore function $f$ is decreasing to the left of $x=4$ and increasing
    to the right of $x=4$.
  \end{solution}

  \part[3] On what intervals is the graph of $f$ concave upwards. Use $f'$ to
  justify your answer.

\begin{solution}[\mcq]
    For the function $f$ to concave upwards, it's first derivative (or rate of 
    change) must be increasing. In the given graph $f'$ is increasing in the
    intervals $(-1,1)$ and $(3,5)$.
  \end{solution}

  \part[3] Suppose that $f(1)=0$. Draw a sketch that shows the general shape of the
  graph of the function $f$ on the open interval $0<x<2$. \\  
  
  \begin{solution}
  \end{solution}  
    \figinit{pt}
      \figpt 0:(0,0)
      \figpt 50:(20,20)%arc center      
      \figpt 60:(20,-20)%arc center
      % extremeties
      \def\Xmax{60}
      \def\Ymax{40}
      \def\Xmin{-20}
      \def\Ymin{-40}
      % pts for numbering the axes (5, 10, 15...) and graph
      \figpt 203:$\tiny\text{-1}$(\Xmin,0)
      \figpt 204:$\tiny\text{0}$(0,0)
      \figpt 205:$\tiny\text{1}$(20,0)
      \figpt 206:$\tiny\text{2}$(40,0)
      \figpt 207:$\tiny\text{3}$(\Xmax,0)
      \figpt 208:$\tiny\text{-2}$(0,\Ymin)
      \figpt 209:$\tiny\text{2}$(0,\Ymax)
      \figpt 210:$\tiny\text{2}$(\Xmin,\Ymin)
      \figpt 211:$\tiny\text{2}$(\Xmax,\Ymin)
      \figpt 212:$\tiny\text{2}$(\Xmax,\Ymax)
      \figpt 213:$\tiny\text{2}$(\Xmin,\Ymax)
      \figpt 214:(20,\Ymin) %label
    \figdrawbegin{}
\ifprintanswers
  % stuff to be shown only in the answer key - like explanatory margin figures
      \figdrawarccirc 50 ; 20 (180,270)
      \figdrawarccirc 60 ; 20 (0,90)
\fi  
      \figset arrowhead(length=4, fillmode=yes)
      \figdrawaxes 0(\Xmin, \Xmax, \Ymin, \Ymax)
      \figset general(dash=5)
      \figdrawmesh 4,4 [210,211,212,213]
    \figdrawend
    \figvisu{\figBoxA}{}{%
      \figwritee 207:$\text{x}$(2 pt)
      \figwriten 209:$\text{f(x)}$(2 pt)
      \figwritese 203,204,205,206,207 :(2 pt)
      \figwrites 214:{\textit{y = f(x)}}(5 pt)
    }
    \centerline{\box\figBoxA}

\end{parts}
