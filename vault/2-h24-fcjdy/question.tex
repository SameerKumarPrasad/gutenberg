% This is an empty shell file placed for you by the 'examiner' script.
% You can now fill in the TeX for your question here.

% Now, down to brasstacks. ** Writing good solutions is an Art **. 
% Eventually, you will find your own style. But here are some thoughts 
% to get you started: 
%
%   1. Write the solution as if you are writing it for your favorite
%      14-17 year old to help him/her understand. Could be your nephew, 
%      your niece, a cousin perhaps or probably even you when you 
%      were that age. Just write for them.
%
%   2. Use margin-notes to "talk" to students about the critical insights
%      in the question. The tone can be - in fact, should be - informal
%
%   3. Don't shy away from creating margin-figures you think will help
%      students understand. Yes, it is a little more work per question. 
%      But the question & solution will be written only once. Make that
%      attempt at writing a solution count.
%
%   4. At the same time, do not be too verbose. A long solution can
%      - at first sight - make the student think, "God, that is a lot to know".
%      Our aim is not to scare students. Rather, our aim should be to 
%      create many "Aha!" moments everyday in classrooms around the world
% 
%   5. Ensure that there are *no spelling mistakes anywhere*. We are an 
%      education company. Bad spellings suggest that we ourselves 
%      don't have any education. Also, use American spellings by default
% 
%   6. If a question has multiple parts, then first delete lines 40-41
%   7. If a question does not have parts, then first delete lines 43-69

\question[4] Find the sum of the first 10 terms that are \textit{common} to the series 
$A = 17,21,25,29 ...$ and $B = 16,21,26,31 ...$

\insertQR{QRC}

\ifprintanswers
  % stuff to be shown only in the answer key - like explanatory margin figures
\fi 

\begin{solution}[\halfpage]
  \begin{align}
  	A &= 17 + 4n_1, n_1 \in Z\\
  	B &= 16 + 5n_2, n_2 \in Z \\
  	\text{Common terms when } 17+4n_1 &= 16+5n_2 \\ 
  	\text{ or, when } n_2 = \dfrac{1}{5}(4n_1 + 1)
  \end{align}
  Now, if $n_2$ is an integer, then $(4n_1+1)$ must be a multiple of 5 $\Rightarrow (4n_1+1)$
  must end in either a 0 or a 5 $\Rightarrow 4n_1$ must end in a 9 or a 4
  
  No multiple of 4 can end in a 9. However, numbers of the form $4\times (5k+1), k \in Z$
  will end in a 4
  
  And so, terms in $A$ where $n_1 = 5k+1, k \in Z$ would also be found in $B$
  
  The \textit{common} terms are therefore
  \begin{align}
  	C_k &= 17 + 4\cdot(5k + 1) = 21 + 20k \\
  \end{align}
  This looks like another series where the first term is 21 and the common difference is 20
  
  And hence, the sum of the first 10 terms is
  \begin{align}
  	S_{10} &= \dfrac{10}{2}\cdot\left[ 2\cdot 21 + (10-1)\cdot 20 \right] \\
  	       &= 1110
  \end{align}
\end{solution} 
