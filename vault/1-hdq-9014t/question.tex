%\noprintanswers
%\setcounter{rolldice}{0}
%\printrubric

\ifnumequal{\value{rolldice}}{0}{
  % variables 
  \renewcommand{\vbone}{13}
  \renewcommand{\vbtwo}{11}
  \renewcommand{\vbthree}{4}
  \renewcommand{\vbfour}{3}
  \renewcommand{\vbfive}{eight}
}{
  \ifnumequal{\value{rolldice}}{1}{
    % variables 
    \renewcommand{\vbone}{12}
    \renewcommand{\vbtwo}{9}
    \renewcommand{\vbthree}{2}
    \renewcommand{\vbfour}{3}
    \renewcommand{\vbfive}{second}
  }{
    \ifnumequal{\value{rolldice}}{2}{
      % variables 
      \renewcommand{\vbone}{11}
      \renewcommand{\vbtwo}{13}
      \renewcommand{\vbthree}{4}
      \renewcommand{\vbfour}{3}
      \renewcommand{\vbfive}{seventh}
    }{
      % variables 
      \renewcommand{\vbone}{10}
      \renewcommand{\vbtwo}{12}
      \renewcommand{\vbthree}{5}
      \renewcommand{\vbfour}{3}
      \renewcommand{\vbfive}{third}
    }
  }
}

\EXPR[0]\p{(\vbtwo - (\vbthree + \vbfour))}
\SUBTRACT\vbone{2}\q

\question In a $\vbone-$storey building, $\vbtwo$ people enter an elevator. 
It is known that they will leave the elevator in groups of 
$\vbthree$, $\vbfour$ and $\p$ at different storeys. In how many ways can they do 
so if the elevator does not stop at the $\vbfive$ storey?

\insertQR{}

\watchout[-1cm]

\ifprintanswers
\fi 

\begin{solution}
  Right upfront, we know that two floors are off-limits from consideration - the $\vbfive$ floor
  and the floor at which everyone got on the elevator. 

  Which means, the 3-groups have a choice of $\q$ floors. And therefore they could get off 
  in $\encr\q{3}\times 3! = \enpr\q{3}$ ways
\end{solution}

\ifprintrubric
  \begin{table}
  	\begin{tabular}{ p{5cm}p{5cm} }
  		\toprule % in brief (4-6 words), what should a grader be looking for for insights & formulations
  		  \sc{\textcolor{blue}{Insight}} & \sc{\textcolor{blue}{Formulation}} \\ 
  		\midrule % ***** Insights & formulations ******
        All floors - except two - are valid for getting off & \\
        $\Rightarrow$ if there are $N$ total floors, then there are $N-2$ valid floors & \\
        & Each person knows what floor they will off on. So, no need to make groups by permuting persons \\ 
  		\toprule % final numerical answers for the various versions
        \sc{\textcolor{blue}{If question has $\ldots$}} & \sc{\textcolor{blue}{Final answer}} \\
  		\midrule % ***** Numerical answers (below) **********
        12-storey & $\enpr{10}{3} = 720$\\
        13-storey & $\enpr{11}{3} = 990$\\
        11-storey & $\enpr{9}{3} = 504$\\
        10-storey & $\enpr{8}{3} = 336$\\
  		\bottomrule
  	\end{tabular}
  \end{table}
\fi
