% This is an empty shell file placed for you by the 'examiner' script.
% You can now fill in the TeX for your question here.

% Now, down to brasstacks. ** Writing good solutions is an Art **. 
% Eventually, you will find your own style. But here are some thoughts 
% to get you started: 
%
%   1. Write the solution as if you are writing it for your favorite
%      14-17 year old to help him/her understand. Could be your nephew, 
%      your niece, a cousin perhaps or probably even you when you 
%      were that age. Just write for them.
%
%   2. Use margin-notes to "talk" to students about the critical insights
%      in the question. The tone can be - in fact, should be - informal
%
%   3. Don't shy away from creating margin-figures you think will help
%      students understand. Yes, it is a little more work per question. 
%      But the question & solution will be written only once. Make that
%      attempt at writing a solution count.
%
%   4. At the same time, do not be too verbose. A long solution can
%      - at first sight - make the student think, "God, that is a lot to know".
%      Our aim is not to scare students. Rather, our aim should be to 
%      create many "Aha!" moments everyday in classrooms around the world
% 
%   5. Ensure that there are *no spelling mistakes anywhere*. We are an 
%      education company. Bad spellings suggest that we ourselves 
%      don't have any education. And, use American spellings

\question[3]  Suppose I want to have \texteuro{60,000} in my account for my children's college education. 
I expect to earn interest at an \textit{annualized} rate of 12\% compounded $4$ times an year. 
If I deposit \texteuro{25,000} in my account today, how many years will it take me to meet my goal 
( \texteuro{60,000} in my account)?

\insertQR{QRC}

\ifprintanswers
  % stuff to be shown only in the answer key - like explanatory margin figures
  \marginnote[2.2cm] {\textit{4N} since the interest is compounded 4 times a year}
\fi 

\begin{solution}[\halfpage]
  If it will take $N$ years to get from \texteuro{25,000} (today) to \texteuro{60,000} (eventually), then 
	\begin{align}
    \text{\texteuro{60,000}} &= \text{\texteuro{25,000}}\times
    \underbrace{\left( 1 + \dfrac{\frac{12}{4}}{100} \right)^{4\text{ times a year}\times N\text{ year}}}
    _{\texttt{3\% compounding every quarter}}\\
    \Rightarrow \log \dfrac{60,000}{25,000} &= 4N\cdot\log 1.03 \\
    \Rightarrow N &= \dfrac{\log 2.4}{4\cdot\log 1.03} = 7.404\text{ years}
	\end{align}
\end{solution}
