% This is an empty shell file placed for you by the 'examiner' script.
% You can now fill in the TeX for your question here.

% Now, down to brasstacks. ** Writing good solutions is an Art **. 
% Eventually, you will find your own style. But here are some thoughts 
% to get you started: 
%
%   1. Write the solution as if you are writing it for your favorite
%      14-17 year old to help him/her understand. Could be your nephew, 
%      your niece, a cousin perhaps or probably even you when you 
%      were that age. Just write for them.
%
%   2. Use margin-notes to "talk" to students about the critical insights
%      in the question. The tone can be - in fact, should be - informal
%
%   3. Don't shy away from creating margin-figures you think will help
%      students understand. Yes, it is a little more work per question. 
%      But the question & solution will be written only once. Make that
%      attempt at writing a solution count.
%
%   4. At the same time, do not be too verbose. A long solution can
%      - at first sight - make the student think, "God, that is a lot to know".
%      Our aim is not to scare students. Rather, our aim should be to 
%      create many "Aha!" moments everyday in classrooms around the world
% 
%   5. Ensure that there are *no spelling mistakes anywhere*. We are an 
%      education company. Bad spellings suggest that we ourselves 
%      don't have any education. And, use American spellings

\question[4] The angle of elevation of a hot air baloon, climbing vertically, changes from $\ang{30}$ at 10.00 AM to $\ang{60}$ at 10:10 AM. The point of observation is 300 meters from the point of take off. What is the average upward speed of the baloon? \\
\textit{(use Speed = Distance/Time)}

\ifprintanswers
	\begin{marginfigure}
  		% 1. Definition of characteristic points
  		\figinit{pt}
		%Vertices of the triangle
		\figpt 1:(0, 0)
		\figpt 2:(100, 0)
		\figpt 3:(100, 58)
		\figpt 4:(100, 173)
		% 2. Creation of the graphical file
		\figdrawbegin{}
			\figdrawline[1,2,3,4,1]
			\figdrawline[3,1]
		\figdrawend
		% 3. Draw the angles
%		\psarrowcirc 1; 0.5(0, 30)
%		\psarrowcirc 1; 0.75(0, 60)
  		% 3. Writing text on the figure
		\figvisu{\figBoxA} {Figure}
		{
			\figwritesw 1:$O$(4)
			\figwritese 2:$A$(4)
			\figwritee  3:$B$(4)
			\figwritene 4:$C$(4)
		}
		\centerline{\box\figBoxA}
	\end{marginfigure}
\fi 

\begin{solution}
	Let us suppose the baloon was released at a Point A (see figure) and began climbing. It reached Point B at exactly 10AM and Point C, 10 minutes after that. In order to compute speed we need to find the height that the baloon gained between the two points in time (10.00AM and 10.10AM). \\
	Since we know the distance OA from the base of the baloon to the point of obervation, we can use $\tan$ ratios:
	\begin{align}
		\tan\ang{30} &= \dfrac{AB}{OA}	\\		
		\tan\ang{60} &= \dfrac{AB+BC}{OA}
	\end{align}
	
	From (1) we get,
	\begin{align}
		AB = \dfrac{300}{\sqrt{3}} m
	\end{align}	
	
	Using (2) and (3), we get,
	\begin{align}
		\sqrt{3}AB &= \dfrac{AB+BC}{\sqrt{3}}	\\		
		\Rightarrow 3AB &= AB+BC				\\
		\Rightarrow 2AB &= BC
	\end{align}

	Using (3) and (6) therefore, we get,
	\begin{align}
		\text{Speed of rise} &= \dfrac{\text{gain in height}}{\text{time elapsed}} \\
							 &= \dfrac{\frac{600}{\sqrt{3}}(m)}{600(s)}	\\
							 &= \dfrac{1}{\sqrt{3}}(m/s)
	\end{align}	
		
\end{solution}
