% This is an empty shell file placed for you by the 'examiner' script.
% You can now fill in the TeX for your question here.

% Now, down to brasstacks. ** Writing good solutions is an Art **. 
% Eventually, you will find your own style. But here are some thoughts 
% to get you started: 
%
%   1. Write the solution as if you are writing it for your favorite
%      14-17 year old to help him/her understand. Could be your nephew, 
%      your niece, a cousin perhaps or probably even you when you 
%      were that age. Just write for them.
%
%   2. Use margin-notes to "talk" to students about the critical insights
%      in the question. The tone can be - in fact, should be - informal
%
%   3. Don't shy away from creating margin-figures you think will help
%      students understand. Yes, it is a little more work per question. 
%      But the question & solution will be written only once. Make that
%      attempt at writing a solution count.
%
%   4. At the same time, do not be too verbose. A long solution can
%      - at first sight - make the student think, "God, that is a lot to know".
%      Our aim is not to scare students. Rather, our aim should be to 
%      create many "Aha!" moments everyday in classrooms around the world
% 
%   5. Ensure that there are *no spelling mistakes anywhere*. We are an 
%      education company. Bad spellings suggest that we ourselves 
%      don't have any education. Also, use American spellings by default
% 
%   6. If a question has multiple parts, then first delete lines 40-41
%   7. If a question does not have parts, then first delete lines 43-69

\question A 2-digit number id such that the product of its digits is 20. If 9 
is added to the number, then its digits interchange their places. What is the 
number? 

\insertQR{}

\ifprintanswers
  % stuff to be shown 
\fi 

\begin{solution}
	If $x$ and $y$ be the two digits in the units and tens place respectively, then
	\begin{align}
		\text{ Number } &= 10\cdot y + x 
	\end{align}
	With the digits interchanged, the \textit{new} number is 
	\begin{align}
		\text{New number } &= 10\cdot x + y \\
		                   &= (10\cdot y + x) + 9 \\
		   \Rightarrow x - y &= 1
	\end{align}
	Given that $xy = 20$, the above reduces to 
	\begin{align}
		x - \dfrac{20}{x} &= 1 \\
		\Rightarrow x^2 - x - 20 &= 0 \\
		\text{or, } x &= -4, 5 \Rightarrow y = \dfrac{20}{x} = -5, 4
	\end{align}
	The two possible numbers - yes, there would be two - are $10\cdot4 + 5 = 45$ 
	and $10\cdot (-5) + (-4) = -54$
\end{solution}
