% This is an empty shell file placed for you by the 'examiner' script.
% You can now fill in the TeX for your question here.

% Now, down to brasstacks. ** Writing good solutions is an Art **. 
% Eventually, you will find your own style. But here are some thoughts 
% to get you started: 
%
%   1. Write to be understood - but be crisp. Your own solution should not take 
%      more space than you will give to the student. Hence, if you take more than 
%      a half-page to write a solution, then give the student a full-page and so on...
%
%   2. Use margin-notes to "talk" to students about the critical insights
%      in the question. The tone can be - in fact, should be - informal
%
%   3. Don't shy away from creating margin-figures you think will help
%      students understand. Yes, it is a little more work per question. 
%      But the question & solution will be written only once. Make that
%      attempt at writing a solution count.
%      
%      3b. Use bc_to_fig.tex. Its an easier way to generate plots & graphs 
% 
%   4. Ensure that there are *no spelling mistakes anywhere*. We are an 
%      education company. Bad spellings suggest that we ourselves 
%      don't have any education. Also, use American spellings by default
% 
%   5. If a question has multiple parts, then first delete lines 40-41
%   6. If a question does not have parts, then first delete lines 43-69
%   
%   7. Create versions of the question when possible. Use commands defined in 
%      tufte-tweaks.sty to do so. Its easier than you think

% \noprintanswers
% \setcounter{rolldice}{0}

\ifnumequal{\value{rolldice}}{0}{
  % variables 
  \renewcommand{\vbone}{LATE}
  \renewcommand{\vbtwo}{A}
  \renewcommand{\vbthree}{E}
  \renewcommand{\vbfour}{T}
  \renewcommand{\vbfive}{L}
  \renewcommand{\vbsix}{LAET}
  \renewcommand{\vbseven}{20}
  \renewcommand{\vbeight}{}
  \renewcommand{\vbnine}{}
  \renewcommand{\vbten}{}
}{
  \ifnumequal{\value{rolldice}}{1}{
    % variables 
    \renewcommand{\vbone}{WALK}
    \renewcommand{\vbtwo}{A}
    \renewcommand{\vbthree}{K}
    \renewcommand{\vbfour}{L}
    \renewcommand{\vbfive}{W}
    \renewcommand{\vbsix}{WAKL}
    \renewcommand{\vbseven}{20}
    \renewcommand{\vbeight}{}
    \renewcommand{\vbnine}{}
    \renewcommand{\vbten}{}
  }{
    \ifnumequal{\value{rolldice}}{2}{
      % variables 
      \renewcommand{\vbone}{TIME}
      \renewcommand{\vbtwo}{E}
      \renewcommand{\vbthree}{I}
      \renewcommand{\vbfour}{M}
      \renewcommand{\vbfive}{T}
      \renewcommand{\vbsix}{TEIM, TEME, TIEM}
      \renewcommand{\vbseven}{22}
      \renewcommand{\vbeight}{}
      \renewcommand{\vbnine}{}
      \renewcommand{\vbten}{}
    }{
      % variables 
      \renewcommand{\vbone}{YEAR}
      \renewcommand{\vbtwo}{A}
      \renewcommand{\vbthree}{E}
      \renewcommand{\vbfour}{R}
      \renewcommand{\vbfive}{Y}
      \renewcommand{\vbsix}{YAER, YARE}
      \renewcommand{\vbseven}{21}
      \renewcommand{\vbeight}{}
      \renewcommand{\vbnine}{}
      \renewcommand{\vbten}{}
    }
  }
}

\question[3] If the letters of the word $\vbone$ are permutated and all the
 words formed are arranged in alphabetical order, what would be the rank of 
 the word $\vbone$.


\insertQR{QRC}

\watchout

\ifprintanswers
  % stuff to be shown only in the answer key - like explanatory margin figures
  \begin{marginfigure}
    \figinit{pt}
      \figpt 100:(0,0)
      \figpt 101:(0,0)
    \figdrawbegin{}
      \figdrawline [100,101]
    \figdrawend
    \figvisu{\figBoxA}{}{%
    }
    \centerline{\box\figBoxA}
  \end{marginfigure}
\fi 

\begin{solution}[\mcq]
  In alphabetical order, all permutations of the word $\vbone$ that begin
  with $\vbtwo$, $\vbthree$ or $\vbfour$ would be ranked ahead of the word
  $\vbone$. Let us begin by counting these:
  \begin{align}
    \text{Number of words not beginning with $\vbfive$} &= 3\times3! \\
    													&= 18
  \end{align}
  Word(s) beginning with $\vbfive$ that precede the word $\vbone$ when taken
  in alphabetical order
  \begin{align}
    {\vbsix}
  \end{align}
  Adding the number of words in these two sets we get the rank of $\vbone$ 
  as $\vbseven$.
\end{solution}
