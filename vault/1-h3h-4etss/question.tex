% This is an empty shell file placed for you by the 'examiner' script.
% You can now fill in the TeX for your question here.

% Now, down to brasstacks. ** Writing good solutions is an Art **. 
% Eventually, you will find your own style. But here are some thoughts 
% to get you started: 
%
%   1. Write the solution as if you are writing it for your favorite
%      14-17 year old to help him/her understand. Could be your nephew, 
%      your niece, a cousin perhaps or probably even you when you 
%      were that age. Just write for them.
%
%   2. Use margin-notes to "talk" to students about the critical insights
%      in the question. The tone can be - in fact, should be - informal
%
%   3. Don't shy away from creating margin-figures you think will help
%      students understand. Yes, it is a little more work per question. 
%      But the question & solution will be written only once. Make that
%      attempt at writing a solution count.
%
%   4. At the same time, do not be too verbose. A long solution can
%      - at first sight - make the student think, "God, that is a lot to know".
%      Our aim is not to scare students. Rather, our aim should be to 
%      create many "Aha!" moments everyday in classrooms around the world
% 
%   5. Ensure that there are *no spelling mistakes anywhere*. We are an 
%      education company. Bad spellings suggest that we ourselves 
%      don't have any education. Also, use American spellings by default
% 
%   6. If a question has multiple parts, then first delete lines 40-41
%   7. If a question does not have parts, then first delete lines 43-69
% \noprintanswers
% \setcounter{rolldice}{3}

\ifnumequal{\value{rolldice}}{0}{
  \renewcommand{\vbone}{2}
  \renewcommand{\vbtwo}{\dfrac{2}{3}}
  \renewcommand{\vbthree}{\dfrac{8\pi}{3}}
  \renewcommand{\vbfour}{}
  \renewcommand{\vbfive}{}
}{
  \ifnumodd{\value{rolldice}}{
    \renewcommand{\vbone}{3}
    \renewcommand{\vbtwo}{\dfrac{3}{3}}
    \renewcommand{\vbthree}{\dfrac{27\pi}{2}}
    \renewcommand{\vbfour}{}
    \renewcommand{\vbfive}{}
  }{
    \renewcommand{\vbone}{4}
    \renewcommand{\vbtwo}{\dfrac{4}{3}}
    \renewcommand{\vbthree}{\dfrac{128\pi}{3}}
    \renewcommand{\vbfour}{}
    \renewcommand{\vbfive}{}
  }
}

\question[4] Compute the volume of the solid obtained by revolving - about the $y-axis$ -
the figure bounded by the parabola $y=\vbone x-x^2$ and the $x-axis$

\watchout
\insertQR{QRC}

\ifprintanswers
  % stuff to be shown only in the answer key - like explanatory margin figures
  \begin{marginfigure}
% 1. Definition of characteristic points
\figinit{pt}
\def\Xmin{-8.00000}
\def\Ymin{-25.22999}
\def\Xmax{72.00000}
\def\Ymax{44.77000}
\def\Xori{8.00000}
\def\Yori{25.22999}
\figpt0:(\Xori,\Yori)
\figpt 100:$\vbone,0$(71,24)
\figpt 101:$0,0$(9,24)
\figpt 102:$x_0$(47,26)
\figpt 103:(47,68)
% 2. Creation of the graphical file
\figdrawbegin{}
\def\Xmaxx{\Xmax} % To customize the position
\def\Ymaxx{\Ymax} % of the arrow-heads of the axes.
\figset arrowhead(length=4, fillmode=yes) % styling the arrowheads
\figdrawaxes 0(\Xmin, \Xmaxx, \Ymin, \Ymaxx)
\figdrawline [102,103]
\figdrawlineC(
0 0,
2.75862 9.33333,
5.51724 17.99999,
8.27586 25.99999,
11.03448 33.33333,
13.79310 39.99999,
16.55172 45.99999,
19.31034 51.33333,
22.06896 55.99999,
24.82758 59.99999,
27.58620 63.33333,
30.34482 65.99999,
33.10344 67.99999,
35.86206 69.33333,
38.62068 69.99999,
41.37931 69.99999,
44.13793 69.33333,
46.89655 68.00000,
49.65517 66.00000,
52.41379 63.33333,
55.17241 60.00000,
57.93103 56.00000,
60.68965 51.33333,
63.44827 46.00000,
66.20689 40.00000,
68.96551 33.33333,
71.72413 26.00000,
74.48275 18.00000,
77.24137 9.33333,
79.99999 0
)
\figdrawend
% 3. Writing text on the figure
\figvisu{\figBoxA}{}{%
\figptsaxes 1:0(\Xmin, \Xmaxx, \Ymin, \Ymaxx)
% Points 1 and 2 are the end points of the arrows
\figwritee 1:(5pt)     \figwriten 2:(5pt)
\figptsaxes 1:0(\Xmin, \Xmax, \Ymin, \Ymax)
\figwritesw 100:(2)
\figwritese 101:(2)
\figwrites 102:(2)
}
\centerline{\box\figBoxA}

  \end{marginfigure}
\fi 

\begin{solution}[\halfpage]
   Imagine a thin strip at $x = x_0$ that is rotated around the y-axis. The volume 
   added to the total by this thin strip would be 
   \begin{align}
     \ud V &= \underbrace{2\pi\cdot x_0}_{\text{circumference}}\times y_0\ud x
   \end{align}
   
   This suggests that the total volume of the generated solid would be
   \begin{align}
   	 V &= \int_0^{\vbone} (2\pi\cdot x)\times(\vbone x-x^2)\ud x \\
   	   &= 2\pi\int_0^{\vbone} (\vbone x^2-x^3)\ud x \\
   	   &= 2\pi\times\left[ \left( \vbtwo x^3\right)_0^{\vbone} - \left( \dfrac{x^4}{4}\right)_0^{\vbone}\right] \\
   	   &= \vbthree
   \end{align}
\end{solution}
