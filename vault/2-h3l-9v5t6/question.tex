% This is an empty shell file placed for you by the 'examiner' script.
% You can now fill in the TeX for your question here.

% Now, down to brasstacks. ** Writing good solutions is an Art **. 
% Eventually, you will find your own style. But here are some thoughts 
% to get you started: 
%
%   1. Write the solution as if you are writing it for your favorite
%      14-17 year old to help him/her understand. Could be your nephew, 
%      your niece, a cousin perhaps or probably even you when you 
%      were that age. Just write for them.
%
%   2. Use margin-notes to "talk" to students about the critical insights
%      in the question. The tone can be - in fact, should be - informal
%
%   3. Don't shy away from creating margin-figures you think will help
%      students understand. Yes, it is a little more work per question. 
%      But the question & solution will be written only once. Make that
%      attempt at writing a solution count.
%
%   4. At the same time, do not be too verbose. A long solution can
%      - at first sight - make the student think, "God, that is a lot to know".
%      Our aim is not to scare students. Rather, our aim should be to 
%      create many "Aha!" moments everyday in classrooms around the world
% 
%   5. Ensure that there are *no spelling mistakes anywhere*. We are an 
%      education company. Bad spellings suggest that we ourselves 
%      don't have any education. Also, use American spellings by default
% 
%   6. If a question has multiple parts, then first delete lines 40-41
%   7. If a question does not have parts, then first delete lines 43-69

\question How many distinct ways can the faces of a cube be numbered $1$ through $6$ such that each face gets a unique number and no two cubes look the same in any orientation.

\insertQR{}

\ifprintanswers
  % stuff to be shown only in the answer key - like explanatory margin figures
\fi 

\begin{solution}
  First solution: \\
  Let us begin by numbering the bottom face of the cube $1$. \\
  Number of options to number the top face,
  \begin{align}
     N_{top} = 5 \nonumber
  \end{align}
  The remaining $4$ numbers would occupy the four vertical faces of the cube which have no reference point. Therefore number of options is,
  \begin{align}
     N_{vertical} = 3! \nonumber
  \end{align}
  Therefore the total number of unique configurations is,
  \begin{align}
    N_{unique} &= N_{top} \times N_{vertical} \nonumber \\
               &= 5 \times 3! = 30 \nonumber 
  \end{align}
  Alternate solution:\\
  Maximum ways in which the cube can be numbered is 
  \begin{align}
    N_{maximum} = 6! \nonumber
  \end{align}
  Number of orientations possible for any given configuration of the cube,
  \begin{align}
    N_{orientations} = 24 \quad\text{(four with each number as top face)} \nonumber
  \end{align}
  Therefore factoring those out of the maximum we get,
  \begin{align}
    N_{unique} &= \dfrac{N_{maximum}}{N_{orientations}} \nonumber \\
               &= \dfrac{6!}{24} = 30 \nonumber
  \end{align}   
\end{solution}

