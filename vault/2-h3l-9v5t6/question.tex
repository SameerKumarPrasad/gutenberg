% This is an empty shell file placed for you by the 'examiner' script.
% You can now fill in the TeX for your question here.

% Now, down to brasstacks. ** Writing good solutions is an Art **. 
% Eventually, you will find your own style. But here are some thoughts 
% to get you started: 
%
%   1. Write the solution as if you are writing it for your favorite
%      14-17 year old to help him/her understand. Could be your nephew, 
%      your niece, a cousin perhaps or probably even you when you 
%      were that age. Just write for them.
%
%   2. Use margin-notes to "talk" to students about the critical insights
%      in the question. The tone can be - in fact, should be - informal
%
%   3. Don't shy away from creating margin-figures you think will help
%      students understand. Yes, it is a little more work per question. 
%      But the question & solution will be written only once. Make that
%      attempt at writing a solution count.
%
%   4. At the same time, do not be too verbose. A long solution can
%      - at first sight - make the student think, "God, that is a lot to know".
%      Our aim is not to scare students. Rather, our aim should be to 
%      create many "Aha!" moments everyday in classrooms around the world
% 
%   5. Ensure that there are *no spelling mistakes anywhere*. We are an 
%      education company. Bad spellings suggest that we ourselves 
%      don't have any education. Also, use American spellings by default
% 
%   6. If a question has multiple parts, then first delete lines 40-41
%   7. If a question does not have parts, then first delete lines 43-69

%\noprintanswers

\question[6] In how many ways can the six faces of a cube be numbered - using 1 through 6 
so that no two cubes that have been numbered differently look the same in any orientation?
\texttt{Hint:} Unroll the cube and lay its faces flat out on a table

\insertQR[15pt]{QRC}

\ifprintanswers
  % stuff to be shown only in the answer key - like explanatory margin figures
  \begin{marginfigure}
    \figinit{pt}
      \figpt 100:(20,0)
      \figpt 101:(40,0)
      \figpt 102:(40,20)
      \figpt 103:(40,40)
      \figpt 104:(60,40)
      \figpt 105:(60,60)
      \figpt 106:(40,60)
      \figpt 107:(40,80)
      \figpt 108:(20,80) 
      \figpt 109:(20,60)
      \figpt 110:(0,60)
      \figpt 111:(0,40)
      \figpt 112:(20,40)
      \figpt 113:(20,20)
      \figpt 200:$1$(30,70)
      \figpt 201:$2$(50,50)
      \figpt 202:$3$(30,30)
      \figpt 203:$4$(10,50)
    \figdrawbegin{}
      \figdrawline[100,101,102,103,104,105,106,107,108,109,110,111,112,113,100]
      \figset (dash=8)
      \figdrawline[103,106,109,112,103]
      \figset (fillmode=yes, color=0.7)
      \figdrawline[100,101,102,113,100]
    \figdrawend
    \figvisu{\figBoxA}{}{%
      \figwriten 200:(0)
      \figwritee 201:(0)
      \figwrites 202:(0)
      \figwritew 203:(0)
    }
    \centerline{\box\figBoxA}
  \end{marginfigure}
  
  \marginnote[10pt]{These $30$ cubes are called MacMohan cubes after the British mathematician Percy Alexander MacMohan}
\fi 

\begin{solution}[\halfpage]
  First solution: \\
  A cube without any numbers has no point of reference on it. By picking
  any number for say the face it is resting on, we give it a point of
  reference. Let us begin by numbering the bottom face of the cube $1$ 
  (we can pick any of the numbers and it won't affect the counting). \\
  By doing this, the \textit{top} face is uniquely identified as the face
  opposite the bottom, whereas the four side faces are symmetrical.
  Next let us count the number of ways we can number the top face,
  \begin{align}
     N_{top} = 5 \nonumber
  \end{align}
  The remaining $4$ numbers would occupy the four vertical faces of the cube.
  These four faces are in a kind of circular symmetry. For counting the number
  of ways in which objects can be arranged in a circle, we must begin by 
  fixing the position of any one object as the reference point.
  By doing so we reduce a circular permutation to a linear permutation of the
  remaining objects in the remaining places. Therefore the number of ways we
  can arrange the remaining 4 numbers on the vertical faces is,
  \begin{align}
     N_{vertical} = 3! \nonumber
  \end{align}
  Therefore the total number of unique configurations is,
  \begin{align}
    N_{unique} &= N_{top} \times N_{vertical} \nonumber \\
               &= 5 \times 3! = 30 \nonumber 
  \end{align}
  Alternate solution:\\
  Maximum ways in which the cube can be numbered is 
  \begin{align}
    N_{maximum} = 6! \nonumber
  \end{align}
  Number of orientations possible for any given configuration of the cube,
  \begin{align}
    N_{orientations} = 24 \quad\text{(four with each number as top face)} \nonumber
  \end{align}
  Therefore factoring those out of the maximum we get,
  \begin{align}
    N_{unique} &= \dfrac{N_{maximum}}{N_{orientations}} \nonumber \\
               &= \dfrac{6!}{24} = 30 \nonumber
  \end{align}   
\end{solution}

