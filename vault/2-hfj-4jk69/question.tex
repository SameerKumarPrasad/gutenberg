% This is an empty shell file placed for you by the 'examiner' script.
% You can now fill in the TeX for your question here.

% Now, down to brasstacks. ** Writing good solutions is an Art **. 
% Eventually, you will find your own style. But here are some thoughts 
% to get you started: 
%
%   1. Write to be understood - but be crisp. Your own solution should not take 
%      more space than you will give to the student. Hence, if you take more than 
%      a half-page to write a solution, then give the student a full-page and so on...
%
%   2. Use margin-notes to "talk" to students about the critical insights
%      in the question. The tone can be - in fact, should be - informal
%
%   3. Don't shy away from creating margin-figures you think will help
%      students understand. Yes, it is a little more work per question. 
%      But the question & solution will be written only once. Make that
%      attempt at writing a solution count.
%      
%      3b. Use bc_to_fig.tex. Its an easier way to generate plots & graphs 
% 
%   4. Ensure that there are *no spelling mistakes anywhere*. We are an 
%      education company. Bad spellings suggest that we ourselves 
%      don't have any education. Also, use American spellings by default
% 
%   5. If a question has multiple parts, then first delete lines 40-41
%   6. If a question does not have parts, then first delete lines 43-69
%   
%   7. Create versions of the question when possible. Use commands defined in 
%      tufte-tweaks.sty to do so. Its easier than you think

% \noprintanswers
% \setcounter{rolldice}{0}
% \printrubric

\ifnumequal{\value{rolldice}}{0}{
  % variables 
  \renewcommand{\vbone}{}
  \renewcommand{\vbtwo}{}
  \renewcommand{\vbthree}{}
  \renewcommand{\vbfour}{}
  \renewcommand{\vbfive}{}
  \renewcommand{\vbsix}{}
  \renewcommand{\vbseven}{}
  \renewcommand{\vbeight}{}
  \renewcommand{\vbnine}{}
  \renewcommand{\vbten}{}
}{
  \ifnumequal{\value{rolldice}}{1}{
    % variables 
    \renewcommand{\vbone}{}
    \renewcommand{\vbtwo}{}
    \renewcommand{\vbthree}{}
    \renewcommand{\vbfour}{}
    \renewcommand{\vbfive}{}
    \renewcommand{\vbsix}{}
    \renewcommand{\vbseven}{}
    \renewcommand{\vbeight}{}
    \renewcommand{\vbnine}{}
    \renewcommand{\vbten}{}
  }{
    \ifnumequal{\value{rolldice}}{2}{
      % variables 
      \renewcommand{\vbone}{}
      \renewcommand{\vbtwo}{}
      \renewcommand{\vbthree}{}
      \renewcommand{\vbfour}{}
      \renewcommand{\vbfive}{}
      \renewcommand{\vbsix}{}
      \renewcommand{\vbseven}{}
      \renewcommand{\vbeight}{}
      \renewcommand{\vbnine}{}
      \renewcommand{\vbten}{}
    }{
      % variables 
      \renewcommand{\vbone}{}
      \renewcommand{\vbtwo}{}
      \renewcommand{\vbthree}{}
      \renewcommand{\vbfour}{}
      \renewcommand{\vbfive}{}
      \renewcommand{\vbsix}{}
      \renewcommand{\vbseven}{}
      \renewcommand{\vbeight}{}
      \renewcommand{\vbnine}{}
      \renewcommand{\vbten}{}
    }
  }
}

\question The rate at which a baby bird gains weight is proportional to the difference 
between its adult weight and its current weight. At time $t=0$, when the bird is first
weighed, it's weight is $20$ grams. If $B(t)$ is the weight of the bird, in grams, at
time $t$ days after it is first weighed, then
\begin{align}
  \dfrac{\ud B}{\ud t} = \dfrac{1}{5}(100 - B) \nonumber
\end{align}
Let $y=B(t)$ be the solution to the differential equation above with initial condition
$B(0)=20$. 
\insertQR{}

%\watchout

\ifprintanswers
  % stuff to be shown only in the answer key - like explanatory margin figures
  \begin{marginfigure}
    \figinit{pt}
      \figpt 100:(0,0)
      \figpt 101:(0,0)
    \figdrawbegin{}
      \figdrawline [100,101]
    \figdrawend
    \figvisu{\figBoxA}{}{%
    }
    \centerline{\box\figBoxA}
  \end{marginfigure}
\fi 

\begin{parts}
  \part Is the bird gaining weight faster when it weighs $40$ grams or when it weights
  $70$ grams? Explain your reasoning.
  \begin{solution}
    We can use the rate of weight gain expression to calculate this.
    \begin{align}
      \dfrac{\ud B}{\ud t}|_{B=40} = \dfrac{1}{5}(60) = 12 
    \end{align}
    \begin{align}
      \dfrac{\ud B}{\ud t}|_{B=70} = \dfrac{1}{5}(30) = 6 
    \end{align}
    Form results $(1)$ and $(2)$, it is clear that the bird was gaining weight faster when 
    it weighed $40$ grams.
  \end{solution}

  \part Find $\dfrac{\ud^2 B}{\ud t^2}$ in terms of $B$. Use $\dfrac{\ud^2 B}{\ud t^2}$ to 
  explain why the graph of $B$ cannot resemble the adjoining graph.
  \begin{marginfigure}
    \figinit{pt}
      \figpt 100:(-5,0)
      \figpt 101:(120,0)
      \figpt 102:(0,-5)
      \figpt 103:(0,120)
      \figpt 104:(0,0)
      \figpt 10:(-40,20)
      \figpt 20:(0,20)
      \figpt 30:(30,30)
      \figpt 40:(60,80)
      \figpt 50:(100,90)
      \figpt 60:(160,100)
      \figpt 200:(0,20)
      \figpt 201:(0,90)
    \figdrawbegin{}
      \figdrawline [100,101]
      \figdrawline [102,103]
      \figdrawcurve [10,20,30,40,50,60]
    \figdrawend
    \figvisu{\figBoxA}{}{%
      \figwritesw 104: $O$(2 pt)
      \figsetmark {$-$}
      \figwritew 200: $20$(2 pt)
      \figwritew 201: $100$(2 pt)
    }
    \centerline{\box\figBoxA}
  \end{marginfigure}
  \begin{solution}
    Let us first determine $\dfrac{\ud^2 B}{\ud t^2}$ in terms of $B$,
    \begin{align}
      \dfrac{\ud^2 B}{\ud t^2} &= -\dfrac{1}{5}\dfrac{\ud B}{\ud t} \\ 
                               &= -\dfrac{1}{5}\times\dfrac{1}{5}(100-B) \\
                               &= -\dfrac{1}{25}(100-B) 
    \end{align}
    Since the second order differential function is negative therefore, the graph is
    concave down for $20 \leq B < 100$ whereas in the figure a portion of the graph is
    concave up. Therefore this cannot resemble the graph of $B$.
  \end{solution}

  \part Use the separation of variables to find $y=B(t)$, the particular solution to the
  differential equation with initial condition $B(0)=20$.
  \begin{solution}
    \begin{align}
      \dfrac{\ud B}{\ud t}       &= \dfrac{1}{5}(100-B) \\ 
      \dfrac{\ud B}{(100-B)}     &= \dfrac{1}{5}\ud t \\
      \int\dfrac{\ud B}{(100-B)} &= \int\dfrac{1}{5}\ud t \\       
      \ln |100-B|                &= \dfrac{1}{5}t+C
    \end{align}
    Since $20 \leq B < 100$, therefore $|100-B| = 100-B$. Use $B=20$
    to find the value of $C$,
    \begin{align}
      -\ln(100-20)=\dfrac{1}{5}(0)+C \Rightarrow -\ln(80) = C
    \end{align}
    Substitute into original equation,
    \begin{align}
      \ln (100-B) &= \dfrac{1}{5}-\ln(80) \\
      100-B       &= 80 e^{-t/5} \\
      B(t)        &= 100 - 80 e^{-t/5}\text{,}\quad t\geq 0
    \end{align}
  \end{solution}

\end{parts}

\ifprintrubric
  \begin{table}
  	\begin{tabular}{ p{5cm}p{5cm} }
  		\toprule % in brief (4-6 words), what should a grader be looking for for insights & formulations
  		  \sc{\textcolor{blue}{Insight}} & \sc{\textcolor{blue}{Formulation}} \\ 
  		\midrule % ***** Insights & formulations ******
  		\toprule % final numerical answers for the various versions
        \sc{\textcolor{blue}{If question has $\ldots$}} & \sc{\textcolor{blue}{Final answer}} \\
  		\midrule % ***** Numerical answers (below) **********
  		\bottomrule
  	\end{tabular}
  \end{table}
\fi
