
\ifnumequal{\value{rolldice}}{0}{
  % variables 
  \renewcommand{\vbone}{11}
  \renewcommand{\vbtwo}{2}
  \renewcommand{\vbthree}{8}
  \renewcommand{\vbfour}{10}
  \renewcommand{\vbfive}{4}
  \renewcommand{\vbsix}{11}
  \renewcommand{\vbseven}{-1}
  \renewcommand{\vbeight}{1}
  \renewcommand{\vbnine}{2}
  \renewcommand{\vbten}{3}
}{
  \ifnumequal{\value{rolldice}}{1}{
    % variables 
    \renewcommand{\vbone}{2}
    \renewcommand{\vbtwo}{2}
    \renewcommand{\vbthree}{8}
    \renewcommand{\vbfour}{4}
    \renewcommand{\vbfive}{3}
    \renewcommand{\vbsix}{2}
    \renewcommand{\vbseven}{-1}
    \renewcommand{\vbeight}{-2}
    \renewcommand{\vbnine}{1}
    \renewcommand{\vbten}{-6}
  }{
    \ifnumequal{\value{rolldice}}{2}{
      % variables 
      \renewcommand{\vbone}{1}
      \renewcommand{\vbtwo}{3}
      \renewcommand{\vbthree}{8}
      \renewcommand{\vbfour}{2}
      \renewcommand{\vbfive}{3}
      \renewcommand{\vbsix}{5}
      \renewcommand{\vbseven}{2}
      \renewcommand{\vbeight}{5}
      \renewcommand{\vbnine}{-9}
      \renewcommand{\vbten}{13}
  }{
      % variables 
      \renewcommand{\vbone}{4}
      \renewcommand{\vbtwo}{4}
      \renewcommand{\vbthree}{1}
      \renewcommand{\vbfour}{1}
      \renewcommand{\vbfive}{2}
      \renewcommand{\vbsix}{2}
      \renewcommand{\vbseven}{-2}
      \renewcommand{\vbeight}{2}
      \renewcommand{\vbnine}{-2}
      \renewcommand{\vbten}{3}
    }
  }
}

\question[3] Find the length and the foot of the perpendicular drawn from the point $(2,-1,5)$ 
to the line $\dfrac{x - \vbone}{\vbfour} = \dfrac{y + \vbtwo}{-\vbfive} = \dfrac{z+\vbthree}{-\vbsix}$

\insertQR{}

\watchout

\ifprintanswers
  % stuff to be shown only in the answer key - like explanatory margin figures
  \begin{marginfigure}
    \figinit{pt}
      \figpt 100:(0,0)
      \figpt 101:(0,0)
    \figdrawbegin{}
      \figdrawline [100,101]
    \figdrawend
    \figvisu{\figBoxA}{}{%
    }
    \centerline{\box\figBoxA}
  \end{marginfigure}
\fi 

\begin{solution}[\fullpage]
  Let the foot of the perpendicular drawn from point $P(2,-1,5)$ to the above line be $Q$.
  Let us express the equation of the line as,
  \begin{align}
    \dfrac{x - \vbone}{\vbfour} = \dfrac{y + \vbtwo}{-\vbfive} = \dfrac{z+\vbthree}{-\vbsix} = \lambda
  \end{align}
  Since the point $Q$ lies on the line, it's co-ordinates can be of the form,
  \begin{align}
    (\vbfour \lambda + \vbone \text{,} -\vbfive \lambda - \vbtwo \text{,} -\vbsix \lambda - \vbthree)
  \end{align}
  Since the point $Q$ is the foot of the perpendicular therefore, the \textbf{dot} product of the two
  would be zero.
  \begin{align}
    &(\vbfour \lambda + \vbone - 2)(\vbfour) + (\vbfive \lambda + \vbtwo + 1)(\vbfive) + (\vbsix \lambda - 5)(\vbsix) \\
    &\Rightarrow \lambda = \vbseven
  \end{align}
  Therefore we get the coordinates as $(\vbeight, \vbnine, \vbten)$.  
  \ifnumequal{\value{rolldice}}{0}{
    Using the formula for distance, we get, length of the perpendicular as $\sqrt{14}$.
  }{
    \ifnumequal{\value{rolldice}}{1}{
    Using the formula for distance, we get, length of the perpendicular as $\sqrt{141}$.
  }{
    \ifnumequal{\value{rolldice}}{2}{
      Using the formula for distance, we get, length of the perpendicular as $\sqrt{14}$.
  }{
     Using the formula for distance, we get, length of the perpendicular as $\sqrt{141}$.
    }
  }
}
  
  
\end{solution}

\ifprintrubric
  \begin{table}
  	\begin{tabular}{ p{5cm}p{5cm} }
  		\toprule % in brief (4-6 words), what should a grader be looking for for insights & formulations
  		  \sc{\textcolor{blue}{Insight}} & \sc{\textcolor{blue}{Formulation}} \\ 
  		\midrule % ***** Insights & formulations ******
  		\toprule % final numerical answers for the various versions
        \sc{\textcolor{blue}{If question has $\ldots$}} & \sc{\textcolor{blue}{Final answer}} \\
  		\midrule % ***** Numerical answers (below) **********
  		\bottomrule
  	\end{tabular}
  \end{table}
\fi
