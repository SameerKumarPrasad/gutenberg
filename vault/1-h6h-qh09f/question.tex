% This is an empty shell file placed for you by the 'examiner' script.
% You can now fill in the TeX for your question here.

% Now, down to brasstacks. ** Writing good solutions is an Art **. 
% Eventually, you will find your own style. But here are some thoughts 
% to get you started: 
%
%   1. Write the solution as if you are writing it for your favorite
%      14-17 year old to help him/her understand. Could be your nephew, 
%      your niece, a cousin perhaps or probably even you when you 
%      were that age. Just write for them.
%
%   2. Use margin-notes to "talk" to students about the critical insights
%      in the question. The tone can be - in fact, should be - informal
%
%   3. Don't shy away from creating margin-figures you think will help
%      students understand. Yes, it is a little more work per question. 
%      But the question & solution will be written only once. Make that
%      attempt at writing a solution count.
%
%   4. At the same time, do not be too verbose. A long solution can
%      - at first sight - make the student think, "God, that is a lot to know".
%      Our aim is not to scare students. Rather, our aim should be to 
%      create many "Aha!" moments everyday in classrooms around the world
% 
%   5. Ensure that there are *no spelling mistakes anywhere*. We are an 
%      education company. Bad spellings suggest that we ourselves 
%      don't have any education. Also, use American spellings by default
% 
%   6. If a question has multiple parts, then first delete lines 40-41
%   7. If a question does not have parts, then first delete lines 43-69

\question[4] Evaluate the expression for the sum of the first $n$ terms of the 
series $1 + 5 + 11 + 19 + 29 \ldots$

\insertQR{QRC}

\ifprintanswers
	\marginnote{We chose to call the first term $a_1$ in this question. Which means, $k$ ran from $1$ to $n$}
	\marginnote[5pt]{But in a lot of scientific literature, the convention is to call the first term $a_0$. Had we chosen to do that, then $k$ would have run from $0$ to $(n-1)$ - not $n$ (because we still want only $n$ terms) \textit{and} $a_k = (k+1)^2+k$} 
	\marginnote[5pt]{The result would be the same in both cases - obviously - because how you index terms
	does not change the unerlying math}
\fi 

\begin{solution}[\halfpage]
	If you figured out that the $k^{th}$ term of the series - $a_k = k^2 + (k-1),\, k \geq 1$, then
	you have won more than half the battle. Because then, the sum of the first $n$ terms - $S_n$ -
	is given by
	
	\begin{align}
		S_n &= \sum_{k=1}^{n}a_k = \sum_{k=1}^{n}\lbrace k^2 + (k-1)\rbrace \\
		&= \sum_{k=1}^{n}k^2 + \sum_{k=1}^{n}k - \sum_{k=1}^{n}1 \\
		&= \eSumOfSquares{n} + \eSumOfN{n} - n \\
		&= \dfrac{n}{6}\cdot(2n^2+3n+1 + 3n+3 - 6) = \dfrac{n}{6}\cdot(2n^2+6n-2) \\
		&= \dfrac{n}{3}\cdot(n^2+3n-1)
	\end{align}
\end{solution}
