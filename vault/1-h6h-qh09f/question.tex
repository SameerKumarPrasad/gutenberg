% This is an empty shell file placed for you by the 'examiner' script.
% You can now fill in the TeX for your question here.

% Now, down to brasstacks. ** Writing good solutions is an Art **. 
% Eventually, you will find your own style. But here are some thoughts 
% to get you started: 
%
%   1. Write the solution as if you are writing it for your favorite
%      14-17 year old to help him/her understand. Could be your nephew, 
%      your niece, a cousin perhaps or probably even you when you 
%      were that age. Just write for them.
%
%   2. Use margin-notes to "talk" to students about the critical insights
%      in the question. The tone can be - in fact, should be - informal
%
%   3. Don't shy away from creating margin-figures you think will help
%      students understand. Yes, it is a little more work per question. 
%      But the question & solution will be written only once. Make that
%      attempt at writing a solution count.
%
%   4. At the same time, do not be too verbose. A long solution can
%      - at first sight - make the student think, "God, that is a lot to know".
%      Our aim is not to scare students. Rather, our aim should be to 
%      create many "Aha!" moments everyday in classrooms around the world
% 
%   5. Ensure that there are *no spelling mistakes anywhere*. We are an 
%      education company. Bad spellings suggest that we ourselves 
%      don't have any education. Also, use American spellings by default
% 
%   6. If a question has multiple parts, then first delete lines 40-41
%   7. If a question does not have parts, then first delete lines 43-69

%\noprintanswers
%\setcounter{rolldice}{3}

\ifnumequal{\value{rolldice}}{0}{
  % variables 
  \renewcommand{\vbone}{2} %p
  \renewcommand{\vbtwo}{3} %q
}{
	\ifnumequal{\value{rolldice}}{1}{
		\renewcommand{\vbone}{3}
    \renewcommand{\vbtwo}{7}
	}{
	  \ifnumequal{\value{rolldice}}{2}{
      \renewcommand{\vbone}{2}
      \renewcommand{\vbtwo}{5}
	  }{
      \renewcommand{\vbone}{3}
      \renewcommand{\vbtwo}{5}
	  }
	}
}

% variables for question text
\gcalcexpr[0]\af{1-\vbone + \vbtwo}
\gcalcexpr[0]\as{4-(2*\vbone)+\vbtwo}
\gcalcexpr[0]\at{9-(3*\vbone)+\vbtwo}
\gcalcexpr[0]\afo{16-(4*\vbone)+\vbtwo}

% variables for solution
\gcalcexpr[0]\tp{3 * \vbone}
\gcalcexpr[0]\tq{6 * \vbtwo}
\gcalcexpr[0]\tr{3 * (\vbone - 1)}
\gcalcexpr[0]\ts{\tq - \tp + 1}

\question[4] Evaluate the expression for the sum of the first $n$ terms of the 
series $\af + \as + \at + \afo \ldots$. \texttt{Hint:} Each term in the series is 
of the form $a_k = k^2 - qk + r, \, k \geq 1$ and $q,r \in\aleph$

\watchout
\insertQR{QRC}

\ifprintanswers
\fi 

\begin{solution}[\halfpage]
  You were given a hint! And so we can tell you straight up that $a_k = k^2 - \vbone k + \vbtwo,\, k \geq 1$
	
	\begin{align}
		S_n &= \sum_{k=1}^{n}a_k = \sum_{k=1}^{n}\lbrace k^2 - \vbone k + \vbtwo \rbrace \\
		&= \sum_{k=1}^{n}k^2 - \vbone\cdot\sum_{k=1}^{n}k + \vbtwo\cdot\sum_{k=1}^{n}1 \\
		&= \eSumOfSquares{n} - \vbone\cdot\eSumOfN{n} + \vbtwo\cdot n \\
    &= \dfrac{n}{6}\lbrace (n+1)\cdot(2n+1) - \tp\cdot(n+1) + \tq \rbrace \\
    &= \dfrac{n}{6}\left[ \lbrace (n+1)\cdot(2n + 1 - \tp)\rbrace + \tq \right] \\
    &= \dfrac{n}{6}\left( 2n^2 - \tr n + \ts \right)
	\end{align}
\end{solution}
