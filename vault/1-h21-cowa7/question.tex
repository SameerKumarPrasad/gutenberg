% This is an empty shell file placed for you by the 'examiner' script.
% You can now fill in the TeX for your question here.

% Now, down to brasstacks. ** Writing good solutions is an Art **. 
% Eventually, you will find your own style. But here are some thoughts 
% to get you started: 
%
%   1. Write the solution as if you are writing it for your favorite
%      14-17 year old to help him/her understand. Could be your nephew, 
%      your niece, a cousin perhaps or probably even you when you 
%      were that age. Just write for them.
%
%   2. Use margin-notes to "talk" to students about the critical insights
%      in the question. The tone can be - in fact, should be - informal
%
%   3. Don't shy away from creating margin-figures you think will help
%      students understand. Yes, it is a little more work per question. 
%      But the question & solution will be written only once. Make that
%      attempt at writing a solution count.
%
%   4. At the same time, do not be too verbose. A long solution can
%      - at first sight - make the student think, "God, that is a lot to know".
%      Our aim is not to scare students. Rather, our aim should be to 
%      create many "Aha!" moments everyday in classrooms around the world
% 
%   5. Ensure that there are *no spelling mistakes anywhere*. We are an 
%      education company. Bad spellings suggest that we ourselves 
%      don't have any education. Also, use American spellings by default
% 
%   6. If a question has multiple parts, then first delete lines 40-41
%   7. If a question does not have parts, then first delete lines 43-69


\question 
\begin{fullwidth} An insurance company charges younger drivers a higher premium because younger drivers 
as a group tend to have more accidents. The company has 3 age groups - Group A includes those
under 25 years old, 22\% of all its policyholders. Group B includes those that are 25-39 years
old,  43\% of its policyholders. Group C includes those over 40 years old and older. Company 
records show that in any given \textit{1 year} period, 11\% of its Group A policyholders have 
an accident. Figures for Groups B and C are 3\% and 2\% respectively
\end{fullwidth}


\ifprintanswers
  % stuff to be shown only in the answer key - like explanatory margin figures
\fi 

\begin{parts}
  \part[2] What percentage of the company's policyholders are expected to have an accident during 
  the next 12 months?

  \insertQR{QRC}
  \ifprintanswers
  	\begin{table}
  	  \begin{tabular}{cccc}
  	     \toprule
  		 & A & B & C \\
  		 \midrule
  		 P($x \in X$) & 0.22 & 0.43 & 0.35 \\
  		 P(accident $\vert x \in X$) & 0.11 & 0.03 & 0.02 \\
  		 \bottomrule
  		\end{tabular}
  	\end{table}
  \fi
\begin{solution}[\mcq]
  	If $P(E)$ be the probability of an accident occuring in the next 12 months and $P(A)$, $P(B)$
  	, $P(C)$ the probabilities that a policyholder belongs to Group A, B or C respectively, then
  	
  	\begin{align}
  		P(E) &= P(E \vert A)\cdot P(A) + P(E \vert B)\cdot P(B) + P(E \vert C)\cdot P(C) \\
  		     &= 0.22 \times 0.11 + 0.43 \times 0.03 + 0.35 \times 0.02 \\
  		     &= 0.0441 = 4.41\%
  	\end{align}
  \end{solution}

  \part[2] Suppose Mr. X has just had a car accident. If he is one of the company's policyholders, 
  then what is the probability that he is under 25?

  \insertQR{QRC}
\begin{solution}[\halfpage]
      The probability we are seeking is $P(A\vert E)$, which Baye's theorem tells us would be
      \begin{align}
         P(A\vert E) &= \dfrac{P(E\vert A)\cdot P(A)}{P(E)} \\
                     &= \dfrac{0.11 \times 0.22}{0.0441} \\
                     &= 0.55 = 55\%
      \end{align}
  \end{solution}

\end{parts}
