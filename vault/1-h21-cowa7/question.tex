
% \noprintanswers
% \setcounter{rolldice}{0}
% \printrubric

\ifnumequal{\value{rolldice}}{0}{
  % variables 
  \renewcommand{\vbone}{22}
  \renewcommand{\vbtwo}{43}
  \renewcommand{\vbthree}{11}
  \renewcommand{\vbfour}{3}
  \renewcommand{\vbfive}{2}
  \renewcommand{\vbsix}{441}
  \renewcommand{\vbseven}{10000}
  \renewcommand{\vbeight}{4.41}
  \renewcommand{\vbnine}{242}
  \renewcommand{\vbten}{441}
}{
  \ifnumequal{\value{rolldice}}{1}{
    % variables 
    \renewcommand{\vbone}{15}
    \renewcommand{\vbtwo}{39}
    \renewcommand{\vbthree}{12}
    \renewcommand{\vbfour}{4}
    \renewcommand{\vbfive}{3}
    \renewcommand{\vbsix}{237}
    \renewcommand{\vbseven}{5000}
    \renewcommand{\vbeight}{4.74}
    \renewcommand{\vbnine}{30}
    \renewcommand{\vbten}{79}
  }{
    \ifnumequal{\value{rolldice}}{2}{
      % variables 
      \renewcommand{\vbone}{18}
      \renewcommand{\vbtwo}{45}
      \renewcommand{\vbthree}{16}
      \renewcommand{\vbfour}{5}
      \renewcommand{\vbfive}{2}
      \renewcommand{\vbsix}{1019}
      \renewcommand{\vbseven}{10000}
      \renewcommand{\vbeight}{10.19}
      \renewcommand{\vbnine}{720}
      \renewcommand{\vbten}{1019}
    }{
      % variables 
      \renewcommand{\vbone}{24}
      \renewcommand{\vbtwo}{38}
      \renewcommand{\vbthree}{16}
      \renewcommand{\vbfour}{4}
      \renewcommand{\vbfive}{3}
      \renewcommand{\vbsix}{13}
      \renewcommand{\vbseven}{200}
      \renewcommand{\vbeight}{6.5}
      \renewcommand{\vbnine}{192}
      \renewcommand{\vbten}{325}
    }
  }
}

\FRACADD\vbone{100}\vbtwo{100}\a\b
\FRACMINUS{1}{1}\a\b\c\d

\question 
An insurance company charges younger drivers a higher premium because younger drivers 
as a group tend to have more accidents. The company has 3 age groups - Group A includes those
under 25 years of age, \vbone\% of all its policyholders. Group B includes those that are 25-39 years
old,  \vbtwo\% of its policyholders. Group C includes those over 40. Company 
records show that in any given \textit{1 year} period, \vbthree\% of its Group A policyholders have 
an accident. Figures for Groups B and C are \vbfour\% and \vbfive\% respectively

\watchout[-80pt]

\ifprintanswers
  % stuff to be shown only in the answer key - like explanatory margin figures
\fi 

\begin{parts}
  \part[2] What percentage of the company's policyholders are expected to have an accident during 
  the next 12 months?

  \insertQR[-20pt]{QRC}
  \ifprintanswers
  	\begin{table}
  	  \begin{tabular}{cccc}
  	     \toprule
  		 & A & B & C \\
  		 \midrule
  		 P($x \in X$) & $\WRITEFRAC\vbone{100}$ & $\WRITEFRAC\vbtwo{100}$ & $\WRITEFRAC\c\d$ \\
  		 P(accident $\vert x \in X$) & $\WRITEFRAC\vbthree{100}$ & 
                                     $\WRITEFRAC\vbfour{100}$ & 
                                     $\WRITEFRAC\vbfive{100}$ \\
  		 \bottomrule
  		\end{tabular}
  	\end{table}
  \fi
\begin{solution}[\mcq]
  	If $P(E)$ be the probability of an accident occuring in the next 12 months and $P(A)$, $P(B)$
  	, $P(C)$ the probabilities that a policyholder belongs to Group A, B or C respectively, then
  	
  	\begin{align}
      P(C) &= 1 - \left( \dfrac\vbone{100} + \dfrac\vbtwo{100} \right) = \dfrac\c\d
  	\end{align}

    And therefore, 
  	\begin{align}
  		P(E) &= P(E \vert A)\cdot P(A) + P(E \vert B)\cdot P(B) + P(E \vert C)\cdot P(C) \\
           &= \WRITEFRAC\vbthree{100}\times\WRITEFRAC\vbone{100} + 
              \WRITEFRAC\vbfour{100}\times\WRITEFRAC\vbtwo{100} + 
              \WRITEFRAC\vbfive{100}\times\dfrac\c\d \\
           &= \dfrac\vbsix\vbseven = \vbeight\%
  	\end{align}
  \end{solution}

  \part[2] Suppose Mr. X has just had a car accident. If he is one of the company's policyholders, 
  then what is the probability that he is under 25?

  \insertQR{QRC}
\begin{solution}[\halfpage]
      The probability we are seeking is $P(A\vert E)$, which Baye's theorem tells us would be
      \begin{align}
         P(A\vert E) &= \dfrac{P(E\vert A)\cdot P(A)}{P(E)} \\
                     &= \left(\dfrac{\WRITEFRAC\vbthree{100} \times \WRITEFRAC\vbone{100} }{\frac\vbsix\vbseven}\right) \\
                     &= \dfrac\vbnine\vbten 
      \end{align}
  \end{solution}

\end{parts}

\ifprintrubric
  \begin{table}
  	\begin{tabular}{ p{5cm}p{5cm} }
  		\toprule % in brief (4-6 words), what should a grader be looking for for insights & formulations
  		  \sc{\textcolor{blue}{Insight}} & \sc{\textcolor{blue}{Formulation}} \\ 
  		\midrule % ***** Insights & formulations ******
          $P(C) = 1 - P(A) - P(B)$ & Correctly applied Baye's formula in both parts \\
  		\toprule % final numerical answers for the various versions
        \sc{\textcolor{blue}{If question has $\ldots$}} & \sc{\textcolor{blue}{Final answer}} \\
          $P(A) = 15\%$ & $P(\text{accident}) = 4.74\%$ \\ 
                        & $P(A\vert\text{accident}) = \frac{30}{79}$ \\
          $P(A) = 18\%$ & $P(\text{accident}) = 10.19\%$ \\ 
                        & $P(A\vert\text{accident}) = \frac{720}{1019}$ \\
          $P(A) = 22\%$ & $P(\text{accident}) = 4.41\%$ \\ 
                        & $P(A\vert\text{accident}) = \frac{232}{441}$ \\
          $P(A) = 24\%$ & $P(\text{accident}) = 6.5\%$ \\ 
                        & $P(A\vert\text{accident}) = \frac{192}{325}$ \\
  		\midrule % ***** Numerical answers (below) **********
  		\bottomrule
  	\end{tabular}
  \end{table}
\fi
