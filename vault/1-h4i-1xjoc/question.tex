% This is an empty shell file placed for you by the 'examiner' script.
% You can now fill in the TeX for your question here.

% Now, down to brasstacks. ** Writing good solutions is an Art **. 
% Eventually, you will find your own style. But here are some thoughts 
% to get you started: 
%
%   1. Write the solution as if you are writing it for your favorite
%      14-17 year old to help him/her understand. Could be your nephew, 
%      your niece, a cousin perhaps or probably even you when you 
%      were that age. Just write for them.
%
%   2. Use margin-notes to "talk" to students about the critical insights
%      in the question. The tone can be - in fact, should be - informal
%
%   3. Don't shy away from creating margin-figures you think will help
%      students understand. Yes, it is a little more work per question. 
%      But the question & solution will be written only once. Make that
%      attempt at writing a solution count.
%
%   4. At the same time, do not be too verbose. A long solution can
%      - at first sight - make the student think, "God, that is a lot to know".
%      Our aim is not to scare students. Rather, our aim should be to 
%      create many "Aha!" moments everyday in classrooms around the world
% 
%   5. Ensure that there are *no spelling mistakes anywhere*. We are an 
%      education company. Bad spellings suggest that we ourselves 
%      don't have any education. Also, use American spellings by default
% 
%   6. If a question has multiple parts, then first delete lines 40-41
%   7. If a question does not have parts, then first delete lines 43-69

\question[3] The angles of a triangle are in arithmetic progression and the ratio of 
the number of \textit{radians} in the \textit{smallest} angle to the number of 
\textit{degrees} of the \textit{mean} angle is $\frac{1}{120}$. Find the angles 
of the triangle - in radians 

\insertQR{QRC}

\ifprintanswers
\fi 

\begin{solution}[\halfpage]
  \begin{fullwidth}
   Let the three angles be $A=a$, $B=a+d$ and $C=a+2d$ - either all in degrees or all in radians. 
   Moreover, let $A_R$ be the \textit{number of radians} in $\angle A$ and $B_D$ the
   \textit{number of degrees} in (mean) $\angle B$
   
   And so, 
   \begin{align}
       \dfrac{a}{a+d} &= \overbrace{\dfrac{A_R}{\frac{\pi}{180}\cdot B_D}}^{\texttt{everything in radians}} \\
       \text{where } \dfrac{A_R}{B_D} &= \dfrac{1}{120} \\
       \Rightarrow \dfrac{a}{a+d} &= \dfrac{180}{\pi}\cdot\dfrac{1}{120} = \dfrac{3}{2\pi} \\
       \Rightarrow d &= \left( \dfrac{2\pi}{3}-1\right)\cdot a 
   \end{align}
   The three angles, therefore, are 
   \begin{align}
      A &= a \\
      B &= a + d = a\cdot\left(1 + \dfrac{2\pi}{3} - 1 \right) = \dfrac{2\pi}{3}a\\
      C &= a + 2d = a\cdot\left( 1 + \dfrac{4\pi}{3} - 2\right) = \left( \dfrac{4\pi}{3} - 1\right) \\
      \text{Moreover, if } A + B + C &= \pi \text{ then } 
      a\cdot\left[1 + \dfrac{2\pi}{3} + \dfrac{4\pi}{3} - 1 \right] = \pi \\
      \Rightarrow A = a &= \dfrac{1}{2} \\
      B &= \dfrac{2\pi}{3}a = \dfrac{\pi}{3} \\
      C &= \left(\dfrac{2\pi}{3} - \dfrac{1}{2} \right)
   \end{align}
 \end{fullwidth}
\end{solution}
