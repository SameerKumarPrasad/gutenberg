% This is an empty shell file placed for you by the 'examiner' script.
% You can now fill in the TeX for your question here.

% Now, down to brasstacks. ** Writing good solutions is an Art **. 
% Eventually, you will find your own style. But here are some thoughts 
% to get you started: 
%
%   1. Write the solution as if you are writing it for your favorite
%      14-17 year old to help him/her understand. Could be your nephew, 
%      your niece, a cousin perhaps or probably even you when you 
%      were that age. Just write for them.
%
%   2. Use margin-notes to "talk" to students about the critical insights
%      in the question. The tone can be - in fact, should be - informal
%
%   3. Don't shy away from creating margin-figures you think will help
%      students understand. Yes, it is a little more work per question. 
%      But the question & solution will be written only once. Make that
%      attempt at writing a solution count.
%
%   4. At the same time, do not be too verbose. A long solution can
%      - at first sight - make the student think, "God, that is a lot to know".
%      Our aim is not to scare students. Rather, our aim should be to 
%      create many "Aha!" moments everyday in classrooms around the world
% 
%   5. Ensure that there are *no spelling mistakes anywhere*. We are an 
%      education company. Bad spellings suggest that we ourselves 
%      don't have any education. And, use American spellings

\question[5] If the base of a conical bucket has radius 8cm, its open end has radius 
of 20 cm and the height of the bucket is 16 cm, then what are the volume and outer 
surface area of the bucket? You can use $\pi = \frac{22}{7}$

\ifprintanswers
  % stuff to be shown only in the answer key - like explanatory margin figures
\fi 

\begin{solution}[\fullpage]
	From the top to the base, the radius falls from $20cm$ to $8cm$ over a height 
	of $16cm$
	
	This suggests that, 
	\begin{align}
		\dfrac{20-8}{16} &= \dfrac{20-0}{16+h} \\
	\end{align}
	, where $h$ is the extra depth that would need to be added to the bucket
	to turn it into a full-cone
	
	Solving the above, we get 
	\begin{align}
		16 + h &= \dfrac{80}{3}
	\end{align}
	
	The volume of the bucket is simply the volume of the full-cone 
	minus the volume of the bit that would need to be added to make 
	the  bucket a full-cone
	
	\begin{align}
		\Rightarrow V &= \dfrac{1}{3}\pi\left[R^2(16+h) - r^2(h)\right] \\
		              &= \dfrac{1}{3}\pi\left[400\times\dfrac{80}{3} - 64\times\dfrac{32}{3}\right]
	\end{align}
	
\end{solution}
