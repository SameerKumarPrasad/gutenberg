% This is an empty shell file placed for you by the 'examiner' script.
% You can now fill in the TeX for your question here.

% Now, down to brasstacks. ** Writing good solutions is an Art **. 
% Eventually, you will find your own style. But here are some thoughts 
% to get you started: 
%
%   1. Write the solution as if you are writing it for your favorite
%      14-17 year old to help him/her understand. Could be your nephew, 
%      your niece, a cousin perhaps or probably even you when you 
%      were that age. Just write for them.
%
%   2. Use margin-notes to "talk" to students about the critical insights
%      in the question. The tone can be - in fact, should be - informal
%
%   3. Don't shy away from creating margin-figures you think will help
%      students understand. Yes, it is a little more work per question. 
%      But the question & solution will be written only once. Make that
%      attempt at writing a solution count.
%
%   4. At the same time, do not be too verbose. A long solution can
%      - at first sight - make the student think, "God, that is a lot to know".
%      Our aim is not to scare students. Rather, our aim should be to 
%      create many "Aha!" moments everyday in classrooms around the world
% 
%   5. Ensure that there are *no spelling mistakes anywhere*. We are an 
%      education company. Bad spellings suggest that we ourselves 
%      don't have any education. Also, use American spellings by default
% 
%   6. If a question has multiple parts, then first delete lines 40-41
%   7. If a question does not have parts, then first delete lines 43-69

\ifnumequal{\value{diceroll}}{0}{
  \renewcommand{\vbone}{3}
  \renewcommand{\vbtwo}{4}
  \renewcommand{\vbthree}{2}
  \renewcommand{\vbfour}{1}
  \renewcommand{\vbfive}{3}
  \renewcommand{\vbsix}{14}
}{
  \ifnumequal{\value{diceroll}}{1}{
    \renewcommand{\vbone}{2}
    \renewcommand{\vbtwo}{3}
    \renewcommand{\vbthree}{7}
    \renewcommand{\vbfour}{2}
    \renewcommand{\vbfive}{4}
    \renewcommand{\vbsix}{33\frac{1}{3}}
  }{
    \ifnumequal{\value{diceroll}}{2}{
      \renewcommand{\vbone}{3}
      \renewcommand{\vbtwo}{5}
      \renewcommand{\vbthree}{4}
      \renewcommand{\vbfour}{3}
      \renewcommand{\vbfive}{5}
      \renewcommand{\vbsix}{33\frac{1}{3}}
    }{
      \renewcommand{\vbone}{3}
      \renewcommand{\vbtwo}{2}
      \renewcommand{\vbthree}{2}
      \renewcommand{\vbfour}{1}
      \renewcommand{\vbfive}{4}
      \renewcommand{\vbsix}{54}
    }
  }
} 

\question[2] Find the area of the figure bounded by the parabola $y=\vbone x^2 - \vbtwo x + \vbthree$
the $x-axis$ and the straight lines $x=\vbfour$ and $x=\vbfive$

\insertQR{QRC}

\ifprintanswers
  \begin{marginfigure}
% 1. Definition of characteristic points
\figinit{pt}
\def\Xmin{-6.77966}
\def\Ymin{-4.32654}
\def\Xmax{73.22033}
\def\Ymax{108.68282}
\def\Xori{6.77966}
\def\Yori{-8.68282}
\figpt0:(\Xori,\Yori)
\figpt 200: (45,\Yori)
\figpt 201: (45, 7)
\figpt 202: (71,\Yori)
\figpt 203: (71,62)
\figpt 204: (60,-1)
% 2. Creation of the graphical file
\figdrawbegin{}
\def\Xmaxx{\Xmax} % To customize the position
\def\Ymaxx{\Ymax} % of the arrow-heads of the axes.
\figset arrowhead(length=4, fillmode=yes) % styling the arrowheads
\figdrawaxes 0(\Xmin, \Xmaxx, \Ymin, \Ymaxx)
\figdrawline [200,201]
\figdrawline [202,203]
\figdrawlineC(
0 54.05208,
2.75862 45.60796,
5.51724 37.88016,
8.27586 30.86869,
11.03448 24.57354,
13.79310 18.99471,
16.55172 14.13221,
19.31034 9.98603,
22.06896 6.55618,
24.82758 3.84265,
27.58620 1.84544,
30.34482 .56456,
33.10344 0,
35.86206 .15176,
38.62068 1.01985,
41.37931 2.60426,
44.13793 4.90499,
46.89655 7.92205,
49.65517 11.65543,
52.41379 16.10514,
55.17241 21.27117,
57.93103 27.15352,
60.68965 33.75220,
63.44827 41.06720,
66.20689 49.09852,
68.96551 57.84617,
71.72413 67.31014,
74.48275 77.49043,
77.24137 88.38705,
79.99999 99.99999
)
\figdrawend
% 3. Writing text on the figure
\figvisu{\figBoxA}{}{%
\figptsaxes 1:0(\Xmin, \Xmaxx, \Ymin, \Ymaxx)
% Points 1 and 2 are the end points of the arrows
\figwritee 1:(5pt)     \figwriten 2:(5pt)
\figptsaxes 1:0(\Xmin, \Xmax, \Ymin, \Ymax)
\figwriten 204:$R$(2)
}
\centerline{\box\figBoxA}

  \end{marginfigure}
\fi 

\begin{solution}[\halfpage]
  The required area $A$ - of region $R$ - \asif - is
  \begin{align}
    A &= \int_{\vbfour}^{\vbfive} (\vbone x^2-\vbtwo x+\vbthree)\ud x \\
      &= \left[ \vbone\cdot\dfrac{x^3}{3}-\dfrac{\vbtwo}{2}x^2+\vbthree x\right]_{\vbfour}^{\vbfive} \\
      &= \vbsix
  \end{align}
\end{solution}
