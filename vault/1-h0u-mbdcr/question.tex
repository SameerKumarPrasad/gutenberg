% This is an empty shell file placed for you by the 'examiner' script.
% You can now fill in the TeX for your question here.

% Now, down to brasstacks. ** Writing good solutions is an Art **. 
% Eventually, you will find your own style. But here are some thoughts 
% to get you started: 
%
%   1. Write the solution as if you are writing it for your favorite
%      14-17 year old to help him/her understand. Could be your nephew, 
%      your niece, a cousin perhaps or probably even you when you 
%      were that age. Just write for them.
%
%   2. Use margin-notes to "talk" to students about the critical insights
%      in the question. The tone can be - in fact, should be - informal
%
%   3. Don't shy away from creating margin-figures you think will help
%      students understand. Yes, it is a little more work per question. 
%      But the question & solution will be written only once. Make that
%      attempt at writing a solution count.
%
%   4. At the same time, do not be too verbose. A long solution can
%      - at first sight - make the student think, "God, that is a lot to know".
%      Our aim is not to scare students. Rather, our aim should be to 
%      create many "Aha!" moments everyday in classrooms around the world
% 
%   5. Ensure that there are *no spelling mistakes anywhere*. We are an 
%      education company. Bad spellings suggest that we ourselves 
%      don't have any education. Also, use American spellings by default
% 
%   6. If a question has multiple parts, then first delete lines 40-41
%   7. If a question does not have parts, then first delete lines 43-69

\question Compute the area of the figure bounded by the parabolas 
$y = x^2-6x+10$ and $y=6x-x^2$

\insertQR{}

\ifprintanswers
  \begin{marginfigure}
% 1. Definition of characteristic points
\figinit{pt}
\def\Xmin{-5.33333}
\def\Ymin{-2.28571}
\def\Xmax{74.66666}
\def\Ymax{77.71428}
\def\Xori{5.33333}
\def\Yori{2.28571}
\figpt0:(\Xori,\Yori)
% 2. Creation of the graphical file
\figdrawbegin{}
\def\Xmaxx{\Xmax} % To customize the position
\def\Ymaxx{\Ymax} % of the arrow-heads of the axes.
\figset arrowhead(length=4, fillmode=yes) % styling the arrowheads
\figdrawaxes 0(\Xmin, \Xmaxx, \Ymin, \Ymaxx)
\figpt 100: (70,-30)
\figpt 101: (70, 77)
\figpt 102: (32, 27)
\figdrawlineC(
0 -12.57142,
2.75862 -4.60132,
5.51724 2.75726,
8.27586 9.50433,
11.03448 15.63988,
13.79310 21.16392,
16.55172 26.07643,
19.31034 30.37744,
22.06896 34.06692,
24.82758 37.14489,
27.58620 39.61134,
30.34482 41.46628,
33.10344 42.70969,
35.86206 43.34160,
38.62068 43.36198,
41.37931 42.77085,
44.13793 41.56820,
46.89655 39.75403,
49.65517 37.32835,
52.41379 34.29114,
55.17241 30.64243,
57.93103 26.38219,
60.68965 21.51044,
63.44827 16.02717,
66.20689 9.93239,
68.96551 3.22609,
71.72413 -4.09172,
74.48275 -12.02106,
77.24137 -20.56191,
79.99999 -29.71428
)
\figdrawlineC(
0 62.85714,
2.75862 54.88703,
5.51724 47.52845,
8.27586 40.78138,
11.03448 34.64582,
13.79310 29.12179,
16.55172 24.20927,
19.31034 19.90827,
22.06896 16.21878,
24.82758 13.14081,
27.58620 10.67436,
30.34482 8.81943,
33.10344 7.57601,
35.86206 6.94411,
38.62068 6.92373,
41.37931 7.51486,
44.13793 8.71751,
46.89655 10.53167,
49.65517 12.95736,
52.41379 15.99456,
55.17241 19.64328,
57.93103 23.90351,
60.68965 28.77526,
63.44827 34.25853,
66.20689 40.35332,
68.96551 47.05962,
71.72413 54.37744,
74.48275 62.30677,
77.24137 70.84763,
79.99999 79.99999
)
\figdrawend
% 3. Writing text on the figure
\figvisu{\figBoxA}{}{%
\figptsaxes 1:0(\Xmin, \Xmaxx, \Ymin, \Ymaxx)
% Points 1 and 2 are the end points of the arrows
\figwritee 1:(5pt)     \figwriten 2:(5pt)
\figptsaxes 1:0(\Xmin, \Xmax, \Ymin, \Ymax)
\figwrites 100: $y=6x-x^2$(2)
\figwriten 101: $y=x^2-6x+10$(2)
\figwritee 102: $R$(1)
}
\centerline{\box\figBoxA}

  \end{marginfigure}
\fi 

\begin{solution}
	The two parabolas intersect at points where
	\begin{align}
	   x^2-6x+10 &= 6x-x^2 \\
	   \Rightarrow 2x^2 - 12x + 10 &= 0 \\
	   \Rightarrow x &= 1,5
	\end{align}
	
	The required area $A$ of region $R$ is therefore
	\begin{align}
	  A &= \int_1^5 [(6x-x^2) - (x^2 - 6x + 10)] \ud x \\
	    &= \int_1^5 (12x - 2x^2 -10) \ud x \\
	    &= \left[ 6x^2 -\dfrac{2}{3}x^3 - 10x\right]_1^5 \\
	    &= \left( 150 - \frac{250}{3} -50\right) - \left( 6 - \frac{2}{3} - 10\right) \\
	    &= 104 - \frac{248}{3} = 21\frac{1}{3}
	\end{align}
\end{solution}
