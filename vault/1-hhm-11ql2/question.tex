
% \noprintanswers
% \setcounter{rolldice}{3}
 % \printrubric

\ifnumequal{\value{rolldice}}{0}{
  % variables 
  \renewcommand{\vbone}{5}
  \renewcommand{\vbtwo}{2}
  \renewcommand{\vbthree}{6}
}{
  \ifnumequal{\value{rolldice}}{1}{
    % variables 
    \renewcommand{\vbone}{7}
    \renewcommand{\vbtwo}{4}
    \renewcommand{\vbthree}{3}
  }{
    \ifnumequal{\value{rolldice}}{2}{
      % variables 
      \renewcommand{\vbone}{3}
      \renewcommand{\vbtwo}{2}
      \renewcommand{\vbthree}{2}
    }{
      % variables 
      \renewcommand{\vbone}{9}
      \renewcommand{\vbtwo}{4}
      \renewcommand{\vbthree}{5}
    }
  }
}

\renewcommand{\vbfour}{\vbone + \vbtwo\cdot\sqrt{\vbthree}}
\renewcommand{\vbfive}{\vbone - \vbtwo\cdot\sqrt{\vbthree}}
\POWER\vbone{2}\a
\EXPR[0]{\b}{(\vbtwo * \vbtwo * \vbthree)}
\ADD\vbone\vbone\c
\SQUARE\c\d
\SUBTRACT\d{2}\ans

\question[3] If $a = \vbone + \vbtwo\cdot\sqrt{\vbthree}$ and $b = \frac{1}{a}$, then what is the value of $a^2 + b^2$?

\insertQR[-25pt]{qrc}

\watchout

\begin{solution}[\halfpage]
	\begin{align}
		b = \dfrac{1}{a} &= \dfrac{1}{\vbfour} = \dfrac{1}{\vbfour}\times\dfrac{\vbfive}{\vbfive} \\
		&= \dfrac{\vbfive}{\a-\b} = \vbfive
	\end{align}
	Moreover, \begin{align}
		a^2 + b^2 &= (a+b)^2 - 2ab \\
		\text{where } (a+b) &= (\vbfour) + (\vbfive) = \c \\
		\text{and } a\cdot b &= (\vbfour)\cdot (\vbfive) = 1 \\
		\Rightarrow a^2 + b^2 &= \c^{2} - 2 = \ans
	\end{align}
\end{solution}

\ifprintrubric
  \begin{table}
  	\begin{tabular}{ p{5cm}p{5cm} }
  		\toprule % in brief (4-6 words), what should a grader be looking for for insights & formulations
  		  \sc{\textcolor{blue}{Look for the following}} &  \\ 
  		\midrule % ***** Insights & formulations ******
  			Got a simplified expression for $b$ using rationalization &  Recognized that $a^2+b^2 = (a+b)^2 - 2ab$ \\
  		\toprule % final numerical answers for the various versions
        \sc{\textcolor{blue}{If question has $\ldots$}} & \sc{\textcolor{blue}{Final answer}} \\
  		\midrule % ***** Numerical answers (below) **********
  			$a = 5 + 2\sqrt{6}$ & $98$ \\
  			$a = 7 + 4\sqrt{3}$ & $194$ \\
  			$a = 3 + 2\sqrt{2}$ & $34$ \\
  			$a = 9 + 4\sqrt{5}$ & $322$ \\
  		\bottomrule
  	\end{tabular}
  \end{table}
\fi
