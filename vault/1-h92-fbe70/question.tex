% This is an empty shell file placed for you by the 'examiner' script.
% You can now fill in the TeX for your question here.

% Now, down to brasstacks. ** Writing good solutions is an Art **. 
% Eventually, you will find your own style. But here are some thoughts 
% to get you started: 
%
%   1. Write to be understood - but be crisp. Your own solution should not take 
%      more space than you will give to the student. Hence, if you take more than 
%      a half-page to write a solution, then give the student a full-page and so on...
%
%   2. Use margin-notes to "talk" to students about the critical insights
%      in the question. The tone can be - in fact, should be - informal
%
%   3. Don't shy away from creating margin-figures you think will help
%      students understand. Yes, it is a little more work per question. 
%      But the question & solution will be written only once. Make that
%      attempt at writing a solution count.
%      
%      3b. Use bc_to_fig.tex. Its an easier way to generate plots & graphs 
% 
%   4. Ensure that there are *no spelling mistakes anywhere*. We are an 
%      education company. Bad spellings suggest that we ourselves 
%      don't have any education. Also, use American spellings by default
% 
%   5. If a question has multiple parts, then first delete lines 40-41
%   6. If a question does not have parts, then first delete lines 43-69
%   
%   7. Create versions of the question when possible. Use commands defined in 
%      tufte-tweaks.sty to do so. Its easier than you think

% \noprintanswers
% \setcounter{rolldice}{3}

\ifnumequal{\value{rolldice}}{0}{
  % variables 
  \renewcommand{\vbone}{1,5,6,7}
  \renewcommand{\vbtwo}{5671}
  \renewcommand{\vbthree}{1} % units
  \renewcommand{\vbfour}{7} % tens
  \renewcommand{\vbfive}{6} % hundreds
  \renewcommand{\vbsix}{5} % thousands
  \renewcommand{\vbseven}{6,7} % guaranteed > (thousands) 
  \renewcommand{\vbeight}{7} % hundreds
  \renewcommand{\vbnine}{7} % tens
  \renewcommand{\vbten}{5,6,7} % units
  \gcalcexpr[0]\nth{2}
  \gcalcexpr[0]\nh{1}
  \gcalcexpr[0]\nten{1}
  \gcalcexpr[0]\nu{3}
}{
  \ifnumequal{\value{rolldice}}{1}{
    % variables 
    \renewcommand{\vbone}{2,4,7,9}
    \renewcommand{\vbtwo}{4792}
    \renewcommand{\vbthree}{2}
    \renewcommand{\vbfour}{9}
    \renewcommand{\vbfive}{7}
    \renewcommand{\vbsix}{4}
    \renewcommand{\vbseven}{7,9}
    \renewcommand{\vbeight}{9}
    \renewcommand{\vbnine}{9}
    \renewcommand{\vbten}{4,7,9}
    \gcalcexpr[0]\nth{2}
    \gcalcexpr[0]\nh{1}
    \gcalcexpr[0]\nten{1}
    \gcalcexpr[0]\nu{3}
  }{
    \ifnumequal{\value{rolldice}}{2}{
      % variables 
      \renewcommand{\vbone}{3,4,6,7}
      \renewcommand{\vbtwo}{6473}
      \renewcommand{\vbthree}{3}
      \renewcommand{\vbfour}{7}
      \renewcommand{\vbfive}{4}
      \renewcommand{\vbsix}{6}
      \renewcommand{\vbseven}{7}
      \renewcommand{\vbeight}{6,7}
      \renewcommand{\vbnine}{7}
      \renewcommand{\vbten}{4,6,7}
      \gcalcexpr[0]\nth{1}
      \gcalcexpr[0]\nh{2}
      \gcalcexpr[0]\nten{1}
      \gcalcexpr[0]\nu{3}
    }{
      % variables 
      \renewcommand{\vbone}{2,5,6,9}
      \renewcommand{\vbtwo}{6592}
      \renewcommand{\vbthree}{2}
      \renewcommand{\vbfour}{9}
      \renewcommand{\vbfive}{5}
      \renewcommand{\vbsix}{6}
      \renewcommand{\vbseven}{9}
      \renewcommand{\vbeight}{6,9}
      \renewcommand{\vbnine}{9}
      \renewcommand{\vbten}{5,6,9}
      \gcalcexpr[0]\nth{1}
  	  \gcalcexpr[0]\nh{2}
  	  \gcalcexpr[0]\nten{1}
  	  \gcalcexpr[0]\nu{3}
    }
  }
}

\gcalcexpr[0]\tp{\nth * 4 * 4 * 4}
\gcalcexpr[0]\tq{\nh * 4 * 4}
\gcalcexpr[0]\tr{\nten * \nu}
\gcalcexpr[0]\ngreater{\tp + \tq + \tr}
\gcalcexpr[0]\nlessequal{256 - \ngreater}
\gcalcexpr[0]\nreqd{136 + \nlessequal}

\question[5] How many numbers $\leq \vbtwo$ can be formed using $\vbone$ if the digits are allowed to repeat?

\insertQR{QRC}

\watchout

\ifprintanswers
\fi 

\begin{solution}[\halfpage]
	Any one-, two- or three-digit number made using digits from $A=\lbrace \vbone \rbrace$ 
	is guaranteed to be $\leq \vbtwo$. They are also easy to count
	\begin{align}
		N_{\texttt{< 4-digits}} &= N_1 + N_2 + N_3 \\ 
		 &= 4 + \encr{4}{2}\cdot 2^2 + \encr{4}{3}\cdot 3^3 = 136
	\end{align}
	
	It is when counting the 4-digit numbers that one has to be careful. But here is what we know about what 
	would make a 4-digit number $ > \vbtwo$ (strictly greater)
	
	
	\begin{tabular}{ccccc}
		\toprule
		Thousand & Hundreds & Tens & Units & \# possibilities \\
		\midrule
		$\lbrace\vbseven\rbrace$ & 4 & 4 & 4 & \tp \\ % thousands
		$\lbrace\vbsix\rbrace$ & $\lbrace\vbeight\rbrace$ & 4 & 4 & \tq \\ % hundreds
		$\lbrace\vbsix\rbrace$ & $\lbrace\vbfive\rbrace$ & $\lbrace\vbnine\rbrace$ & $\lbrace\vbten\rbrace$ & \tr \\ % tens 
	    \bottomrule
	\end{tabular}
	
	This means that there are $\ngreater (= \tp + \tq + \tr)$ 4-digit numbers $ > \vbtwo$ that can be formed from 
	$A = \lbrace\vbone\rbrace$. And hence, there are $\nlessequal( = 4^4 - \ngreater)$ numbers that are $ \leq \vbtwo$
	
	So, in total, there are $\nreqd (= \nlessequal + 136)$ numbers that meet our initial criterion
	
\end{solution}
