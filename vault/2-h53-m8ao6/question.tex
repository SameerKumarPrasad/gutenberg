% This is an empty shell file placed for you by the 'examiner' script.
% You can now fill in the TeX for your question here.

% Now, down to brasstacks. ** Writing good solutions is an Art **. 
% Eventually, you will find your own style. But here are some thoughts 
% to get you started: 
%
%   1. Write to be understood - but be crisp. Your own solution should not take 
%      more space than you will give to the student. Hence, if you take more than 
%      a half-page to write a solution, then give the student a full-page and so on...
%
%   2. Use margin-notes to "talk" to students about the critical insights
%      in the question. The tone can be - in fact, should be - informal
%
%   3. Don't shy away from creating margin-figures you think will help
%      students understand. Yes, it is a little more work per question. 
%      But the question & solution will be written only once. Make that
%      attempt at writing a solution count.
%      
%      3b. Use bc_to_fig.tex. Its an easier way to generate plots & graphs 
% 
%   4. Ensure that there are *no spelling mistakes anywhere*. We are an 
%      education company. Bad spellings suggest that we ourselves 
%      don't have any education. Also, use American spellings by default
% 
%   5. If a question has multiple parts, then first delete lines 40-41
%   6. If a question does not have parts, then first delete lines 43-69
%   
%   7. Create versions of the question when possible. Use commands defined in 
%      tufte-tweaks.sty to do so. Its easier than you think

% \noprintanswers
% \setcounter{rolldice}{0}

\ifnumequal{\value{rolldice}}{0}{
  % variables 
  \renewcommand{\vbone}{-1}
  \renewcommand{\vbtwo}{0}
  \renewcommand{\vbthree}{7}
  \renewcommand{\vbfour}{2}
  \renewcommand{\vbfive}{5}
  \renewcommand{\vbsix}{-2}
}{
  \ifnumequal{\value{rolldice}}{1}{
    % variables 
    \renewcommand{\vbone}{1}
    \renewcommand{\vbtwo}{0}
    \renewcommand{\vbthree}{-7}
    \renewcommand{\vbfour}{2}
    \renewcommand{\vbfive}{-5}
    \renewcommand{\vbsix}{-2}
  }{
    \ifnumequal{\value{rolldice}}{2}{
      % variables 
      \renewcommand{\vbone}{-1}
      \renewcommand{\vbtwo}{0}
      \renewcommand{\vbthree}{7}
      \renewcommand{\vbfour}{-2}
      \renewcommand{\vbfive}{5}
      \renewcommand{\vbsix}{2}
    }{
      % variables 
      \renewcommand{\vbone}{1}
      \renewcommand{\vbtwo}{0}
      \renewcommand{\vbthree}{-7}
      \renewcommand{\vbfour}{-2}
      \renewcommand{\vbfive}{-5}
      \renewcommand{\vbsix}{2}
    }
  }
}

\question[4] Show that the perpendicular bisectors of the sides of a triangle 
with vertices $(\vbone , \vbtwo )$, $(\vbthree , \vbfour)$ and 
$(\vbfive , \vbsix )$ are concurrent .Also find the coordinates of the point 
of concurrence.

\insertQR{QRC}

\watchout

\ifprintanswers
  % stuff to be shown only in the answer key - like explanatory margin figures
  \begin{marginfigure}
    \figinit{pt}
      \figpt 10: (-10, 0)
      \figpt 20: (70, 20)
      \figpt 30: (50, -20)
    \figdrawbegin{}
      \figdrawline [10, 20]
      \figdrawline [20, 30]
      \figdrawline [30, 10]
    \figdrawend
    \figvisu{\figBoxA}{}{%
      \figwritesw 10:$-1, 0$(5pt)
      \figwritene 20:$7, 2$(5pt)
      \figwritese 30:$5, -2$(5pt)
    }
    \centerline{\box\figBoxA}
  \end{marginfigure}
\fi 

\begin{solution}[\fullpage]
  Let us begin by finding the equations for the perpendicular bisectors 
  of the three lines. This can be done with the Slope-Point form.
  For the perpendicular of the line joining $(\vbone, \vbtwo)$ and 
  $(\vbthree, \vbfour)$, the calculations look like this  
  \begin{align}
	\text{Slope} &= -\dfrac{\vbthree - (\vbone)}
	    				   {\vbfour - \vbtwo} = -4  	   \nonumber \\
    \text{Point} &= \dfrac{\vbone+\vbthree}{2}\text{,}
    				\dfrac{\vbtwo+\vbfour}{2} = 3\text{,}1 \nonumber \\ 
    \text{Line}  &\Rightarrow y = -4x + 13
  \end{align}
  Similar, equations for perpendicular bisectors of other two lines are
  \begin{align}
  	y &= -\frac{1}{2}x + 3 \\
  	y &= 3x - 7
  \end{align}
  Solving (2) and (3) we get $x=\dfrac{20}{7}$, $y=\dfrac{11}{7}$.
  Concurrence means all three perpendicular bisectors pass through a point.
  Substituting $x$ and $y$ in (1) verifies concurrence.
  \begin{align}
  	\text{LHS} &= y + 4x \nonumber \\
  			   &= \dfrac{11}{7} + 4\dfrac{20}{7} \nonumber \\
  			   &= 13 = \text{RHS} \nonumber
  \end{align}
  Perpendicular bisectors are concurrent through 
  $(\dfrac{20}{7}, \dfrac{11}{7})$.
\end{solution}
