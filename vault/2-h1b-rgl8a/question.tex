% This is an empty shell file placed for you by the 'examiner' script.
% You can now fill in the TeX for your question here.

% Now, down to brasstacks. ** Writing good solutions is an Art **. 
% Eventually, you will find your own style. But here are some thoughts 
% to get you started: 
%
%   1. Write the solution as if you are writing it for your favorite
%      14-17 year old to help him/her understand. Could be your nephew, 
%      your niece, a cousin perhaps or probably even you when you 
%      were that age. Just write for them.
%
%   2. Use margin-notes to "talk" to students about the critical insights
%      in the question. The tone can be - in fact, should be - informal
%
%   3. Don't shy away from creating margin-figures you think will help
%      students understand. Yes, it is a little more work per question. 
%      But the question & solution will be written only once. Make that
%      attempt at writing a solution count.
%
%   4. At the same time, do not be too verbose. A long solution can
%      - at first sight - make the student think, "God, that is a lot to know".
%      Our aim is not to scare students. Rather, our aim should be to 
%      create many "Aha!" moments everyday in classrooms around the world
% 
%   5. Ensure that there are *no spelling mistakes anywhere*. We are an 
%      education company. Bad spellings suggest that we ourselves 
%      don't have any education. Also, use American spellings by default
% 
%   6. If a question has multiple parts, then first delete lines 40-41
%   7. If a question does not have parts, then first delete lines 43-69

\question The derivative of a function $f$ is defined by 
\begin{align}
  f'(x) = \left\{
            \begin{array}{l l l}
              g(x)           & \text{for} & -4 \leq x \leq 0 \\
              5 e^{-(x/3)}-3 & \text{for} & 0 \leq x \leq 4  \\
            \end{array} 
          \right. \nonumber 
\end{align}
The graph of the continuous function $f'$, shown in figure alongside, has an x-intercepts at $x=-2$ and $x=3\ln (\dfrac{5}{3})$. The graph of $g$ on $-4 \leq x \leq 0$ is a semicircle, and $f(0)=5$.

\begin{marginfigure}
\figinit{pt}
%Origin
\figpt 0:$O$ (0,0)
%Curve
\figpt 13: (-40, 40) 
%axes
\figpt 1: (0,40)
\figpt 2: (30,0)
\figpt 3: (80,-38)
\figpt 4: (20, 0)
\figpt 5: (40, 0)
\figpt 6: (60, 0)
\figpt 7: (-20, 0)
\figpt 8: (-40, 0)
\figpt 9: (-60, 0)
\figpt 10: (0, 20)
\figpt 11: (0, 40)
\figpt 12: (0, -20)
\figpt 13: (-40, 40)
\figpt 14: (65, -25)
\figpt 15: (-3, 10.6)

\figpt 100: (90,0)
\figpt 101: (0,60)
%draw
\figdrawbegin{}
\figset arrowhead(length=4, fillmode=yes) % styling the arrowheads
\figdrawaxes 0(-80,80, -40,60)
\figdrawarccirc 13;40 (180, 360)
\figdrawcurve [15,1,2,14,3]
\figdrawend
%write
\figvisu{\figBoxA}{Graph of f'}{
\figwritesw 0:$O$(5pt)
\figwritee 100:$x$(5pt)
\figwriten 101:$y$(5pt)
\figsetmark{$|$}
\figwrites 4:$1$(5)
\figwrites 5:$2$(5)
\figwrites 6:$3$(5)
\figwrites 7:$-1$(5)
\figwrites 8:$-2$(5)
\figwrites 9:$-3$(5)
\figsetmark{$-$}
\figwritew 10:$1$(5)
\figwritew 11:$ $(5)
\figwritew 12:$ $(5)
}
\centerline{\box\figBoxA}
\end{marginfigure}


\ifprintanswers
  % stuff to be shown only in the answer key - like explanatory margin figures
\fi 

\begin{parts}

  \part[3] For $-4 < x < 4$, find all values of $x$ at which the graph of $f$ has a point of inflection. Justify your answer.
  \insertQR{QRC}
\begin{solution}[\halfpage]
    A point of inflection is when the slope of a function changes behavior. It goes either from increasing to decreasing or decreasing to increasing. In this case by looking at the graph of slope $f'(x)$, we can tell that it is decreasing till $x=-2$ and then starts increasing and continues to increase till $x=0$ at which point it starts decreasing again. Therefore the two points of inflection are $x=-2$ and $x=0$.
  \end{solution}

  \part[3] Find $f(-4)$ and $f(4)$.
  \insertQR{QRC}
\begin{solution}[\halfpage]
    Area under the curve of the derivative of a function represents the change in value of the original function over a given interval. In other words for any function $f(x)$,
    \begin{align}
      \int_a^b f'(x)\ud x = f(b) - f(a)
    \end{align}
    Consider the portion of the curve from $-4$ to $0$, 
    \begin{align}
       \int_{-4}^0 g(x)\ud x &= f(0) - f(-4) \\
       8-\dfrac{\pi 4}{2}    &= 5 - f(-4) \\
       f(-4)                 &= 2\pi - 3
    \end{align}
    Consider the portion of the curve from $0$ to $-4$, 
    \begin{align}
       \int_{0}^{4} (5 e^{-(x/3)}-3)\ud x   &= f(4) - f(0) \\
       \left[5(-3)e^{-(x/3)}-3x \right]_0^4 &= f(4) - f(0) \\
       f(4)                                 &= 8 - 15 e^{(-4/3)}
    \end{align}
  \end{solution}
\nextpg
  \part[3] For $-4 \leq x \leq 4$, find the value of $x$ at which $f$ has an absolute maximum. Justify your answer.
  \insertQR{QRC}
\begin{solution}[\halfpage]
    $f'(x)$ is decreasing but positive from $x=-4$ to $x=-2$. Thereafter it starts increasing and keeps increasing till $x=0$. After this the slope remains positive till $x=\ln(5/3)$ after which it becomes negative. This means that $f(x)$ is steadily increasing from $x=-4$ till $x=\ln (3/2)$ and only then does it start decreasing. Therefore the absolute maxima of the function in the given range is at $x=\ln(3/2)$.
  \end{solution}

\end{parts}
