% This is an empty shell file placed for you by the 'examiner' script.
% You can now fill in the TeX for your question here.

% Now, down to brasstacks. ** Writing good solutions is an Art **. 
% Eventually, you will find your own style. But here are some thoughts 
% to get you started: 
%
%   1. Write to be understood - but be crisp. Your own solution should not take 
%      more space than you will give to the student. Hence, if you take more than 
%      a half-page to write a solution, then give the student a full-page and so on...
%
%   2. Use margin-notes to "talk" to students about the critical insights
%      in the question. The tone can be - in fact, should be - informal
%
%   3. Don't shy away from creating margin-figures you think will help
%      students understand. Yes, it is a little more work per question. 
%      But the question & solution will be written only once. Make that
%      attempt at writing a solution count.
%      
%      3b. Use bc_to_fig.tex. Its an easier way to generate plots & graphs 
% 
%   4. Ensure that there are *no spelling mistakes anywhere*. We are an 
%      education company. Bad spellings suggest that we ourselves 
%      don't have any education. Also, use American spellings by default
% 
%   5. If a question has multiple parts, then first delete lines 40-41
%   6. If a question does not have parts, then first delete lines 43-69
%   
%   7. Create versions of the question when possible. Use commands defined in 
%      tufte-tweaks.sty to do so. Its easier than you think

 %\noprintanswers
 %\setcounter{rolldice}{0}
 %\printrubric

\ifnumequal{\value{rolldice}}{0}{
  % variables 
  \renewcommand{\vbone}{2}   %y = 2
  \renewcommand{\vbtwo}{1}    %coeff x
  \renewcommand{\vbthree}{-4}%coeff y
  \renewcommand{\vbfour}{-4}  %coeff const
  \renewcommand{\vbfive}{4}  %4sqrt(2) dist
}{
  \ifnumequal{\value{rolldice}}{1}{
    % variables 
    \renewcommand{\vbone}{1} 
    \renewcommand{\vbtwo}{2} 
    \renewcommand{\vbthree}{-3}
    \renewcommand{\vbfour}{3} 
    \renewcommand{\vbfive}{3} 
  }{
    \ifnumequal{\value{rolldice}}{2}{
      % variables 
      \renewcommand{\vbone}{4}
      \renewcommand{\vbtwo}{3}
      \renewcommand{\vbthree}{-2}
      \renewcommand{\vbfour}{-2}
      \renewcommand{\vbfive}{4}
    }{
      % variables 
      \renewcommand{\vbone}{3}
      \renewcommand{\vbtwo}{2}
      \renewcommand{\vbthree}{-1}
      \renewcommand{\vbfour}{-1}
      \renewcommand{\vbfive}{3}
     }
  }
}

\SOLVELINEARSYSTEM(0, 1; \vbtwo, \vbthree)(\vbone,\vbfour)(\p,\q)
\gcalcexpr[0]{\vbseven}{(\vbfive - 1) * 10}
\gcalcexpr[0]{\vbpreal}{\p * 10}
\gcalcexpr[0]{\vbqreal}{\q * 10}
\gcalcexpr[0]{\pplusq}{\p + \q}
\gcalcexpr[0]{\cthree}{(\vbfive^2 * 2) - 2}
\gcalcexpr[0]{\afive}{\p - 1} % coeff a eqn 5
\gcalcexpr[0]{\bfive}{\q - 1} 
\gcalcexpr[0]{\asix}{\p + 1}  % coeff a eqn 6
\gcalcexpr[0]{\bsix}{\q + 1}
\gcalcexpr[0]{\csix}{\cthree - \pplusq}
\gcalcexpr[0]{\a}{-1 + \vbfive}
\gcalcexpr[0]{\b}{-1 + \vbfive}
\gcalcexpr[0]{\c}{2 * \a}
\gcalcexpr[0]{\midpt}{(\vbseven - 10)/2}

\question[4] Find the equation of a straight line that passes
 through the \textit{intersection} of lines $y=\vbone$ and 
 $\vbtwo x \vbthree y = \vbfour$ and is at a distance of $\vbfive\sqrt{2}$
 units from $(-1, -1)$.

\insertQR{QRC}

\watchout

\ifprintanswers
  % stuff to be shown only in the answer key - like explanatory margin figures
  \begin{marginfigure}[+40 pt]
    \figinit{pt}
      \figpt 100:(-10,-10)
      \figpt 101:(\vbpreal,\vbqreal)
      \figpt 102:(\vbseven,\vbseven)
      \figpt 103:(\midpt, \midpt)
    \figdrawbegin{}
      \figdrawline [100,101]
      \figdrawline [101,102]
      \figdrawaltitude 5[100,102,101]
    \figdrawend
    \figvisu{\figBoxA}{}{%
      \figwritesw 100: ${(-1,-1)}$(2 pt)
      \figwritese 101: ${(\p,\q)}$(2 pt)
      \figwritene 102: ${(a,b)}$(2 pt)
      \figwritenw 103: $\vbfive\sqrt{2}$(2 pt)
    }
    \centerline{\box\figBoxA}
  \end{marginfigure}
  \marginnote[275 pt]{Though the equation is of quadratic form and will have
  two roots, we can choose any one of the roots since the question asks for
  a (one) straight line only}
\fi 

\begin{solution}[\fullpage]
	Let us begin by finding the intersection point of the given lines,
	\begin{align}
		y &= \vbone \nonumber \\
		\vbtwo x \vbthree y &= \vbfour \nonumber \\
		\Rightarrow x       &= \p \text{ and } y = \q
	\end{align}	
	Let the coordinates of the point where the perpendicular from $(-1,-1)$
	meets the line in question, be $(a,b)$ \textit{(see figure)}.\\
	$(a,b)$ is $\vbfive\sqrt{2}$ units away from $(-1,-1)$. Using the standard
	formula for distance between two points,
	\begin{align}
	  \sqrt{(b+1)^2+(a+1)^2}&=\vbfive\sqrt{2} \\
	  \Rightarrow a^2 + b^2 &= -2a -2b + \cthree
	\end{align}
	Also, the line joining $(a,b)$ with $(-1,-1)$ is perpendicular to the 
	line joining $(a,b)$ with $(\p,\q)$, therefore the product of their
	slopes is $-1$,
	\begin{align}
	  \left(\dfrac{b-\q}{a-\p}\right)\times
	    \left(\dfrac{b+1}{a+1}\right)&=-1 \\
	  \Rightarrow a^2 + b^2          &= \afive a +\bfive b +\pplusq
	\end{align}
	Using equations (3) and (5) we get,
	\begin{align}
	  \asix a + \bsix b=\csix
	\end{align}
	Substitute result (6) in equation (2) or (4). Solve for $a$ and $b$ to
	get,
	\begin{align}
	  a=\a \text{, }b=\b
	\end{align}
	Therefore, equation of the line joining $(\p,\q)$ and $(\a,\b)$ is,
	\begin{align}
	  y-\q &=\dfrac{\a-\q}{\b-\p}(x-\p) \\
	  x + y&=\c
	\end{align}	
\end{solution}
\ifprintrubric
  \begin{table}
        \begin{tabular}{ p{5cm}p{5cm} }
                \toprule % in brief (4-6 words), what should a grader be looking for for insights & formulations
                  \sc{\textcolor{blue}{Insight}} & \sc{\textcolor{blue}{Formulation}} \\
                \midrule % ***** Insights & formulations ******
        Intersection point of two lines & \\
        Distance between $(-1,-1)$ and $(a,b)$ equals given value & \\
        Product of slopes of the two line segments is $-1$  & \\
                \toprule % final numerical answers for the various versions
        \sc{\textcolor{blue}{If question has $\ldots$}} & \sc{\textcolor{blue}{Final answer}} \\
                \midrule % ***** Numerical answers (below) **********
        $y=2$, $4\sqrt{2}$, $x-4y+4=0$  & $(3,3)$, $x+y=6$ \\
        $y=1$, $3\sqrt{2}$, $2x-3y-3=0$ & $(2,2)$, $x+y=4$ \\
        $y=4$, $4\sqrt{2}$, $3x-2y+2=0$ & $(3,3)$, $x+y=6$ \\
        $y=2$, $3\sqrt{2}$, $2x-y+1=0$  & $(2,2)$, $x+y=4$ \\
                \bottomrule
        \end{tabular}
  \end{table}
\fi
