% This is an empty shell file placed for you by the 'examiner' script.
% You can now fill in the TeX for your question here.

% Now, down to brasstacks. ** Writing good solutions is an Art **. 
% Eventually, you will find your own style. But here are some thoughts 
% to get you started: 
%
%   1. Write to be understood - but be crisp. Your own solution should not take 
%      more space than you will give to the student. Hence, if you take more than 
%      a half-page to write a solution, then give the student a full-page and so on...
%
%   2. Use margin-notes to "talk" to students about the critical insights
%      in the question. The tone can be - in fact, should be - informal
%
%   3. Don't shy away from creating margin-figures you think will help
%      students understand. Yes, it is a little more work per question. 
%      But the question & solution will be written only once. Make that
%      attempt at writing a solution count.
%      
%      3b. Use bc_to_fig.tex. Its an easier way to generate plots & graphs 
% 
%   4. Ensure that there are *no spelling mistakes anywhere*. We are an 
%      education company. Bad spellings suggest that we ourselves 
%      don't have any education. Also, use American spellings by default
% 
%   5. If a question has multiple parts, then first delete lines 40-41
%   6. If a question does not have parts, then first delete lines 43-69
%   
%   7. Create versions of the question when possible. Use commands defined in 
%      tufte-tweaks.sty to do so. Its easier than you think

 %\noprintanswers
 %\setcounter{rolldice}{0}

\ifnumequal{\value{rolldice}}{0}{
  % variables 
  \renewcommand{\vbone}{2}
  \renewcommand{\vbtwo}{3}
  \renewcommand{\vbfour}{3}
}{
  \ifnumequal{\value{rolldice}}{1}{
    % variables 
    \renewcommand{\vbone}{3} % passing - x
    \renewcommand{\vbtwo}{4} % passing - y
    \renewcommand{\vbfour}{3} % distance
  }{
    \ifnumequal{\value{rolldice}}{2}{
      % variables 
      \renewcommand{\vbone}{4}
      \renewcommand{\vbtwo}{7}
      \renewcommand{\vbfour}{4} % k
    }{
      % variables 
      \renewcommand{\vbone}{1}
      \renewcommand{\vbtwo}{2}
      \renewcommand{\vbfour}{2}
     }
  }
}

\gcalcexpr[0]{\vbthree}{(\vbone + \vbtwo) / (2-\vbtwo + \vbone)}
\gcalcexpr[0]{\vbseven}{(\vbthree - \vbtwo) / (\vbtwo - \vbone)}
\gcalcexpr[0]{\vbeight}{\vbseven + 1}
\gcalcexpr[0]{\vbfive}{\vbseven + \vbfour} % from-x
\gcalcexpr[0]{\vbsix}{\vbeight - \vbfour} % from-y
\gcalcexpr[0]{\vbnine}{\vbthree - \vbeight}
\gcalcexpr[0]{\vbten}{\vbeight + \vbseven}

\question Find the equation of the straight lines that pass through the \textit{intersection} of lines 
$x-y+1 = 0$ and $\vbone x - \vbtwo y + \vbthree = 0$ and is at a distance of 
$\vbfour\sqrt{2}$ units from $(\vbfive, \vbsix)$

\insertQR{}

\watchout

\ifprintanswers

\fi 

\begin{solution}
	The two given lines would intersect when
	\begin{align}
		y &= x + 1 = \dfrac{\vbone}{\vbtwo}x + \dfrac{\vbthree}{\vbtwo} \\
		\Rightarrow x &= \vbseven \text{ and } y = \vbeight
	\end{align}
	Any line that passes through $(\vbseven, \vbeight)$ will satify the equation
	\begin{align}
		\dfrac{y-\vbeight}{x-\vbseven} &= m \text{ (as yet unknown) } \\
		\Rightarrow mx - y + (\vbeight - \vbseven \cdot m) &= 0
	\end{align}
	And the distance of $(\vbfive, \vbsix)$ from it would be given by 
	\begin{align}
		\dfrac{\vert\, \vbfive m - \vbsix + (\vbeight - \vbseven\cdot m)\,\vert}{\sqrt{1 + m^2}} &= d = \vbfour\sqrt{2}\\
		\Rightarrow \vbfour^2\cdot(m+1)^2 &= \vbfour^2\cdot 2\cdot(1 + m^2) \\
		\Rightarrow m &= -1 
	\end{align}
	Hence, the line is $-x-y-\vbten = 0$
\end{solution}
