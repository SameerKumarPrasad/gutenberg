% This is an empty shell file placed for you by the 'examiner' script.
% You can now fill in the TeX for your question here.

% Now, down to brasstacks. ** Writing good solutions is an Art **. 
% Eventually, you will find your own style. But here are some thoughts 
% to get you started: 
%
%   1. Write the solution as if you are writing it for your favorite
%      14-17 year old to help him/her understand. Could be your nephew, 
%      your niece, a cousin perhaps or probably even you when you 
%      were that age. Just write for them.
%
%   2. Use margin-notes to "talk" to students about the critical insights
%      in the question. The tone can be - in fact, should be - informal
%
%   3. Don't shy away from creating margin-figures you think will help
%      students understand. Yes, it is a little more work per question. 
%      But the question & solution will be written only once. Make that
%      attempt at writing a solution count.
%
%   4. At the same time, do not be too verbose. A long solution can
%      - at first sight - make the student think, "God, that is a lot to know".
%      Our aim is not to scare students. Rather, our aim should be to 
%      create many "Aha!" moments everyday in classrooms around the world
% 
%   5. Ensure that there are *no spelling mistakes anywhere*. We are an 
%      education company. Bad spellings suggest that we ourselves 
%      don't have any education. Also, use American spellings by default
% 
%   6. If a question has multiple parts, then first delete lines 40-41
%   7. If a question does not have parts, then first delete lines 43-69

\question[4] The figure alongside shows the plot of the curve $y=(x^2+2x)\cdot e^{-x}$.
Find the area contained between it and the $x-axis$.

\insertQR{QRC}

\ifprintanswers
\fi 

\begin{marginfigure}
  % 1. Definition of characteristic points
\figinit{pt}
\def\Xmin{-66.66666}
\def\Ymin{-14.57502}
\def\Xmax{13.33333}
\def\Ymax{65.42497}
\def\Xori{66.66666}
\def\Yori{14.57502}
\figpt0:(\Xori,\Yori)
\figpt 100: (32,13)
% 2. Creation of the graphical file
\figdrawbegin{}
\def\Xmaxx{\Xmax} % To customize the position
\def\Ymaxx{\Ymax} % of the arrow-heads of the axes.
\figset arrowhead(length=4, fillmode=yes) % styling the arrowheads
\figdrawaxes 0(\Xmin, \Xmaxx, \Ymin, \Ymaxx)
\figdrawlineC(
0 79.99999,
2.75862 59.42819,
5.51724 43.17884,
8.27586 30.51152,
11.03448 20.79838,
13.79310 13.50832,
16.55172 8.19326,
19.31034 4.47623,
22.06896 2.04110,
24.82758 .62368,
27.58620 .00406,
30.34482 0,
33.10344 .46119,
35.86206 1.26443,
38.62068 2.30933,
41.37931 3.51477,
44.13793 4.81577,
46.89655 6.16088,
49.65517 7.50989,
52.41379 8.83194,
55.17241 10.10387,
57.93103 11.30887,
60.68965 12.43526,
63.44827 13.47555,
66.20689 14.42559,
68.96551 15.28387,
71.72413 16.05097,
74.48275 16.72903,
77.24137 17.32142,
79.99999 17.83234
)
\figdrawend
% 3. Writing text on the figure
\figvisu{\figBoxA}{}{%
\figptsaxes 1:0(\Xmin, \Xmaxx, \Ymin, \Ymaxx)
% Points 1 and 2 are the end points of the arrows
\figwritee 1:(5pt)     \figwriten 2:(5pt)
\figptsaxes 1:0(\Xmin, \Xmax, \Ymin, \Ymax)
\figwrites 100:$R$(2)
}
\centerline{\box\figBoxA}

\end{marginfigure}

\begin{solution}[\fullpage]
 \begin{align}
    (x^2+2x)\cdot e^{-x} &= x(x+2)\cdot e^{-x}
 \end{align}
 $\Rightarrow$ the curve cuts the $x-axis$ at $x=0$ and $x=-2$
 
 The required area $A$ of region $R$ is therefore
 \begin{align}
    A &= \int_{-2}^0(x^2+2x)e^{-x}\ud x \\
      &= \int_{-2}^0 (x^2\cdot e^{-x})\ud x + \int_{-2}^0 (2x\cdot e^{-x})\ud x
  \end{align}
  
  Setting $-x = z$, we get 
  \begin{align}
    A &= -\int_2^0(z^2\cdot e^z)\ud z + \int_2^0 (2z\cdot e^z)\ud z \\
      &= \int_0^2(z^2\cdot e^z)\ud z + \int_2^0 (2z\cdot e^z)\ud z \\
      &= \underbrace{\left[ e^{z}z^2 - \int e^{z}(2z)\ud z \right]_0^2}_{\texttt{integration by parts}}
      + \int_2^0 (2z\cdot e^z)\ud z \\
      &= \left( e^{z}z^2 \right)_0^2 + 4\cdot\int_2^0 ze^{z}\ud z \\
      &= 4e^4 + 4\cdot\left( e^{z}(z-1)\right)_2^0 \\
      &= 4e^2 + 4\cdot(-1-e^2) \\
      &= -4
 \end{align}
 As area cannot be negative, we simply take the absolute value of the calculated integral.
 And hence, the required area $A = 4$ units
\end{solution}
