% This is an empty shell file placed for you by the 'examiner' script.
% You can now fill in the TeX for your question here.

% Now, down to brasstacks. ** Writing good solutions is an Art **. 
% Eventually, you will find your own style. But here are some thoughts 
% to get you started: 
%
%   1. Write to be understood - but be crisp. Your own solution should not take 
%      more space than you will give to the student. Hence, if you take more than 
%      a half-page to write a solution, then give the student a full-page and so on...
%
%   2. Use margin-notes to "talk" to students about the critical insights
%      in the question. The tone can be - in fact, should be - informal
%
%   3. Don't shy away from creating margin-figures you think will help
%      students understand. Yes, it is a little more work per question. 
%      But the question & solution will be written only once. Make that
%      attempt at writing a solution count.
%      
%      3b. Use bc_to_fig.tex. Its an easier way to generate plots & graphs 
% 
%   4. Ensure that there are *no spelling mistakes anywhere*. We are an 
%      education company. Bad spellings suggest that we ourselves 
%      don't have any education. Also, use American spellings by default
% 
%   5. If a question has multiple parts, then first delete lines 40-41
%   6. If a question does not have parts, then first delete lines 43-69
%   
%   7. Create versions of the question when possible. Use commands defined in 
%      tufte-tweaks.sty to do so. Its easier than you think

%\noprintanswers
%\setcounter{rolldice}{1}
%\printrubric

\ifnumequal{\value{rolldice}}{0}{
  % variables 
  \renewcommand{\vbone}{\dfrac{2}{3}}
  \renewcommand{\vbtwo}{\dfrac{3}{2}}
  \renewcommand{\vbthree}{6}
  \renewcommand{\vbfour}{-2}
  \renewcommand{\vbfive}{\dfrac{135}{4}}
}{
  \ifnumequal{\value{rolldice}}{1}{
    % variables 
    \renewcommand{\vbone}{\dfrac{4}{7}}
    \renewcommand{\vbtwo}{\dfrac{7}{4}}
    \renewcommand{\vbthree}{7}
    \renewcommand{\vbfour}{-3}
    \renewcommand{\vbfive}{-\dfrac{7203}{64}}
  }{
    \ifnumequal{\value{rolldice}}{2}{
      % variables 
      \renewcommand{\vbone}{\dfrac{3}{5}}
      \renewcommand{\vbtwo}{\dfrac{5}{3}}
      \renewcommand{\vbthree}{5}
      \renewcommand{\vbfour}{-1}
      \renewcommand{\vbfive}{-\dfrac{50}{3}}
    }{
      % variables 
      \renewcommand{\vbone}{\dfrac{5}{7}}
      \renewcommand{\vbtwo}{\dfrac{7}{5}}
      \renewcommand{\vbthree}{6}
      \renewcommand{\vbfour}{-2}
      \renewcommand{\vbfive}{\dfrac{147}{5}}
    }
  }
}

\gcalcexpr[0]\tp{(\vbthree + \vbfour) / 2}

\question[3] What is the \textit{coefficient} of $x^{\vbfour}$ in $\left( \vbone x - \vbtwo\cdot\dfrac{1}{x} \right)^{\vbthree}$ ?

\insertQR[-30pt]{QRC}

\watchout

\ifprintanswers
\fi 

\begin{solution}[\halfpage]
  If the terms in the expansion are $a_0, a_1, a_2 \ldots$, then the $m^{\text{th}}$ term 
  in the expansion is of the form 
	\begin{align}
	   a_m &= \encr{\vbthree}{m}\cdot\left( \vbone x \right)^{m}\cdot\left( -\vbtwo\cdot\dfrac{1}{x}\right)^{\vbthree - m} \\
	   &= \encr{\vbthree}{m}\cdot\left(\vbone\right)^{m}\cdot\left( -\vbtwo \right)^{\vbthree - m}\times\dfrac{x^m}{x^{\vbthree - m}}
	\end{align}
	We will get a $x^{\vbfour}$ term when 
	\begin{align}
		\dfrac{x^m}{x^{\vbthree - m}} &= x^{\vbfour} \Rightarrow 2m - \vbthree = \vbfour \Rightarrow m = \tp
	\end{align}
	And the \textit{coefficient} then would be 
	\begin{align}
		C_{\tp} &= \encr{\vbthree}{\tp}\cdot\left(\vbone\right)^{\tp}\cdot\left( -\vbtwo \right)^{\vbthree - \tp} \\
		&= \vbfive
	\end{align}
\end{solution}

\ifprintrubric
  \begin{table}
  	\begin{tabular}{ p{5cm}p{5cm} }
  		\toprule % in brief (4-6 words), what should a grader be looking for for insights & formulations
  		  \sc{\textcolor{blue}{Insight}} & \sc{\textcolor{blue}{Formulation}} \\ 
  		\midrule % ***** Insights & formulations ******
        $T_m = \encr{n}{m}(ax)^m\cdot\left(\dfrac{b}{x}\right)^{n-m}$ & \\
        $\dfrac{x^m}{x^{n-m}} = x^{\texttt{R}}$ & \\
  		\toprule % final numerical answers for the various versions
        \sc{\textcolor{blue}{If question has $\ldots$}} & \sc{\textcolor{blue}{Final answer}} \\
  		\midrule % ***** Numerical answers (below) **********
        $\left(\dfrac{2}{3} - \dfrac{3}{2x} \right)^{6}$ & $\dfrac{135}{4}$ \\
        $\left(\dfrac{4}{7} - \dfrac{7}{4x} \right)^{7}$ & $-\dfrac{7203}{64}$ \\
        $\left(\dfrac{3}{5} - \dfrac{5}{3x} \right)^{5}$ & $-\dfrac{50}{3}$ \\
        $\left(\dfrac{5}{7} - \dfrac{7}{5x} \right)^{6}$ & $\dfrac{147}{5}$ \\
  		\bottomrule
  	\end{tabular}
  \end{table}
\fi
