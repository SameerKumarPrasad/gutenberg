% This is an empty shell file placed for you by the 'examiner' script.
% You can now fill in the TeX for your question here.

% Now, down to brasstacks. ** Writing good solutions is an Art **. 
% Eventually, you will find your own style. But here are some thoughts 
% to get you started: 
%
%   1. Write to be understood - but be crisp. Your own solution should not take 
%      more space than you will give to the student. Hence, if you take more than 
%      a half-page to write a solution, then give the student a full-page and so on...
%
%   2. Use margin-notes to "talk" to students about the critical insights
%      in the question. The tone can be - in fact, should be - informal
%
%   3. Don't shy away from creating margin-figures you think will help
%      students understand. Yes, it is a little more work per question. 
%      But the question & solution will be written only once. Make that
%      attempt at writing a solution count.
%      
%      3b. Use bc_to_fig.tex. Its an easier way to generate plots & graphs 
% 
%   4. Ensure that there are *no spelling mistakes anywhere*. We are an 
%      education company. Bad spellings suggest that we ourselves 
%      don't have any education. Also, use American spellings by default
% 
%   5. If a question has multiple parts, then first delete lines 40-41
%   6. If a question does not have parts, then first delete lines 43-69
%   
%   7. Create versions of the question when possible. Use commands defined in 
%      tufte-tweaks.sty to do so. Its easier than you think

% \noprintanswers
% \setcounter{rolldice}{0}
% \printrubric

\ifnumequal{\value{rolldice}}{0}{
  % variables 
  \renewcommand{\vbone}{}
  \renewcommand{\vbtwo}{}
  \renewcommand{\vbthree}{}
  \renewcommand{\vbfour}{}
  \renewcommand{\vbfive}{}
  \renewcommand{\vbsix}{}
  \renewcommand{\vbseven}{}
  \renewcommand{\vbeight}{}
  \renewcommand{\vbnine}{}
  \renewcommand{\vbten}{}
}{
  \ifnumequal{\value{rolldice}}{1}{
    % variables 
    \renewcommand{\vbone}{}
    \renewcommand{\vbtwo}{}
    \renewcommand{\vbthree}{}
    \renewcommand{\vbfour}{}
    \renewcommand{\vbfive}{}
    \renewcommand{\vbsix}{}
    \renewcommand{\vbseven}{}
    \renewcommand{\vbeight}{}
    \renewcommand{\vbnine}{}
    \renewcommand{\vbten}{}
  }{
    \ifnumequal{\value{rolldice}}{2}{
      % variables 
      \renewcommand{\vbone}{}
      \renewcommand{\vbtwo}{}
      \renewcommand{\vbthree}{}
      \renewcommand{\vbfour}{}
      \renewcommand{\vbfive}{}
      \renewcommand{\vbsix}{}
      \renewcommand{\vbseven}{}
      \renewcommand{\vbeight}{}
      \renewcommand{\vbnine}{}
      \renewcommand{\vbten}{}
    }{
      % variables 
      \renewcommand{\vbone}{}
      \renewcommand{\vbtwo}{}
      \renewcommand{\vbthree}{}
      \renewcommand{\vbfour}{}
      \renewcommand{\vbfive}{}
      \renewcommand{\vbsix}{}
      \renewcommand{\vbseven}{}
      \renewcommand{\vbeight}{}
      \renewcommand{\vbnine}{}
      \renewcommand{\vbten}{}
    }
  }
}

\question Let $f$ be a continuous function defined on $[-4,3]$ whose graph,
consisting of three line segments and a semi-circle centered at the origin,
is given in the figure shown. Let $g$ be the function given by $g(x) = 
\int_1^x f(t)\ud t$.

\insertQR{}

\watchout

%\ifprintanswers
  \begin{marginfigure}
    \figinit{pt}
      \figpt 10:(0,0)
      \figpt 20:(-80,20)
      \figpt 30:(-40,60)
      \figpt 40:(-20,0)
      \figpt 50:(0,-20)
      \figpt 60:(20,0)
      \figpt 70:(60,-20)
      % extremeties
      \def\Xmax{100}
      \def\Ymax{80}
      \def\Xmin{-100}
      \def\Ymin{-40}
      % pts for numbering the axes (5, 10, 15...)
      \figpt 200:$\tiny\text{-4}$(-80,0)
      \figpt 201:$\tiny\text{-3}$(-60,0)
      \figpt 202:$\tiny\text{-2}$(-40,0)
      \figpt 203:$\tiny\text{-1}$(-20,0)
      \figpt 204:$\tiny\text{O}$(0,0)
      \figpt 205:$\tiny\text{1}$(20,0)
      \figpt 206:$\tiny\text{2}$(40,0)
      \figpt 207:$\tiny\text{3}$(60,0)
      \figpt 208:$\tiny\text{4}$(80,0)
      \figpt 209:$\tiny\text{5}$(100,0)
      \figpt 210:$\tiny\text{-2}$(0,-40)
      \figpt 211:$\tiny\text{-1}$(0,-20)
      \figpt 212:$\tiny\text{1}$(0,20)
      \figpt 213:$\tiny\text{2}$(0,40)
      \figpt 214:$\tiny\text{3}$(0,60)
      % label graph
      \figpt 100:(\Xmax,0)
      \figpt 101:(0,\Ymax)
      \figpt 102:(0,\Ymin)
    \figdrawbegin{}
      \figset arrowhead(length=4, fillmode=yes)
      \figdrawaxes 10(\Xmin, \Xmax, \Ymin, \Ymax)
      \figdrawline [20,30]
      \figdrawarccirc 10 ; 20 (180, 360)
      \figdrawline [30,40]
      \figdrawline [60,70]
    \figdrawend
    \figvisu{\figBoxA}{}{%
      \figwritee 100: $x$(2 pt)
      \figsetmark{$\tiny\text{|}$}
      \figwrites 200,201,202,203,205,206,207,208,209 :(2 pt)
      \figwritesw 204: (2 pt)
      \figsetmark{$\tiny\text{-}$}
      \figwrites 210,211,212,213,214 :(2 pt)
      \figsetmark{}
      \figwritew 20:$\text{(-4,1)}$(2 pt)
      \figwritee 30:$\text{(-2,3)}$(2 pt)
      \figwriten 60:$\text{(1,0)}$(2 pt)
      \figwrites 70:$\text{(3,-1)}$(2 pt)
      \figwrites 102: $\text{Graph of f}$(10 pt)
    }
    \centerline{\box\figBoxA}
  \end{marginfigure}
%\fi 

\begin{solution}
\end{solution}

\begin{parts}
  \part Find the values of $g(2)$ and $g(-2)$.
  \begin{solution}
    $g(x)$ is the area under the curve $f$ in the
    said interval. Therefore,
    \begin{align}
      g(2) &= \int_1^2 f(t)\ud t = -\dfrac{1}{2}(1)(\dfrac{1}{2}) = -\dfrac{1}{4}
    \end{align}
    \begin{align}
      g(-2) &= \int_1^{-2} f(t)\ud t = -\int_{-2}^1f(t)\ud t \nonumber \\
            &= -\left(\dfrac{3}{2} - \dfrac{\pi}{2}\right) = \dfrac{\pi}{2} - \dfrac{3}{2}
    \end{align}
  \end{solution}

  \part For $g'(-3)$ and $g"(-3)$, find the value if it exists, 
  or state that it does not exist.
  \begin{solution}
    $g'(x)$ = $f(x)$ and $g"(x) = f'(x)$. Therefore, from a simple
    examination of the graph shown in the figure,
    \begin{align}
      g'(-3) = f(-3) = 2 \nonumber
    \end{align}
    \begin{align}
      g"(-3) = f'(-3) = 1 \nonumber
    \end{align}
  \end{solution}

  \part Find the $x-coordinate$ of each point at which the graph of $g$ has
  a horizontal tangent line. For each of these points, determine whether $g$ 
  has a relative maximum, a relative minimum, or neither a maximum nor a 
  minimum at that point. Justify your answers.
  
  \begin{solution}
    The graph of $g$ would have a tangent where $g'(x) = f(x) = 0$. This 
    occurs at $x=-1$ and $x=1$. \\
    $g'(x)$ changes sign from positive to negative at $x = -1$. Therefore
    $g$ has a relative maximum at $x=-1$. \\  
    $g'(x)$ does not change sign at $x=1$. Therefore, $g$ has neither a 
    relative maximum nor a relative minimum at $x=1$.   
  \end{solution}

  \part For $-4 \leq 3$, find all values of $x$ for which the graph of $g$ 
  has a point of inflection. Explain your reasoning.
  \begin{solution}
    A point of inflection occurs when the second derivative of the function 
    changes sign. For the graph of $g$, $g"(x) = f'(x)$ which changes sign
    at $x=-2$, $x=0$ and $x=1$.  
  \end{solution}

\end{parts}

\ifprintrubric
  \begin{table}
  	\begin{tabular}{ p{5cm}p{5cm} }
  		\toprule % in brief (4-6 words), what should a grader be looking for for insights & formulations
  		  \sc{\textcolor{blue}{Insight}} & \sc{\textcolor{blue}{Formulation}} \\ 
  		\midrule % ***** Insights & formulations ******
  		\toprule % final numerical answers for the various versions
        \sc{\textcolor{blue}{If question has $\ldots$}} & \sc{\textcolor{blue}{Final answer}} \\
  		\midrule % ***** Numerical answers (below) **********
  		\bottomrule
  	\end{tabular}
  \end{table}
\fi
