% This is an empty shell file placed for you by the 'examiner' script.
% You can now fill in the TeX for your question here.

% Now, down to brasstacks. ** Writing good solutions is an Art **. 
% Eventually, you will find your own style. But here are some thoughts 
% to get you started: 
%
%   1. Write the solution as if you are writing it for your favorite
%      14-17 year old to help him/her understand. Could be your nephew, 
%      your niece, a cousin perhaps or probably even you when you 
%      were that age. Just write for them.
%
%   2. Use margin-notes to "talk" to students about the critical insights
%      in the question. The tone can be - in fact, should be - informal
%
%   3. Don't shy away from creating margin-figures you think will help
%      students understand. Yes, it is a little more work per question. 
%      But the question & solution will be written only once. Make that
%      attempt at writing a solution count.
%
%   4. At the same time, do not be too verbose. A long solution can
%      - at first sight - make the student think, "God, that is a lot to know".
%      Our aim is not to scare students. Rather, our aim should be to 
%      create many "Aha!" moments everyday in classrooms around the world
% 
%   5. Ensure that there are *no spelling mistakes anywhere*. We are an 
%      education company. Bad spellings suggest that we ourselves 
%      don't have any education. Also, use American spellings by default
% 
%   6. If a question has multiple parts, then first delete lines 40-41
%   7. If a question does not have parts, then first delete lines 43-69

\question The angles of a triangle are in arithmetic progression and the ratio of 
the number of \textit{degrees} of the \textit{smallest} angle to the number of 
\textit{radians} of the \textit{largest} angle is $\frac{60}{\pi}$. Find the angles 
of the triangle - in degrees 

\insertQR{}

\ifprintanswers
  \marginnote[4cm]{You can also write (1) as $\dfrac{a}{a+2d} = \dfrac{A_D}{\frac{180}{\pi}\cdot C_R}$. 
  The key is to remember that when taking ratios, both the values must have the same units }
\fi 

\begin{solution}
   Let the three angles be $A=a$, $B=a+d$ and $C=a+2d$ - either all in degrees or all in radians. 
   Moreover, let $A_D$ be the \textit{number of degrees} in $\angle A$ and $C_R$ the
   \textit{number of radians} in $\angle C$
   
   And so, 
   \begin{align}
       \dfrac{a}{a+2d} &= \overbrace{\dfrac{\frac{\pi}{180}\times A_D}{C_R}}^{\texttt{everything in radians}} \\
       \text{where } \dfrac{A_D}{C_R} &= \dfrac{60}{\pi} \\
       \Rightarrow \dfrac{a}{a+2d} &= \dfrac{\pi}{180}\cdot\dfrac{60}{\pi} = \dfrac{1}{3} \\
       \Rightarrow a &= d 
   \end{align}
   And therefore, $A=a$, $B = 2a$ and $C = 3a$. Given that $ A+B+C = \ang{180}$, we get 
   $A = \ang{30}$, $B=\ang{60}$ and $C=\ang{90}$
\end{solution}
