% This is an empty shell file placed for you by the 'examiner' script.
% You can now fill in the TeX for your question here.

% Now, down to brasstacks. ** Writing good solutions is an Art **. 
% Eventually, you will find your own style. But here are some thoughts 
% to get you started: 
%
%   1. Write the solution as if you are writing it for your favorite
%      14-17 year old to help him/her understand. Could be your nephew, 
%      your niece, a cousin perhaps or probably even you when you 
%      were that age. Just write for them.
%
%   2. Use margin-notes to "talk" to students about the critical insights
%      in the question. The tone can be - in fact, should be - informal
%
%   3. Don't shy away from creating margin-figures you think will help
%      students understand. Yes, it is a little more work per question. 
%      But the question & solution will be written only once. Make that
%      attempt at writing a solution count.
%
%   4. At the same time, do not be too verbose. A long solution can
%      - at first sight - make the student think, "God, that is a lot to know".
%      Our aim is not to scare students. Rather, our aim should be to 
%      create many "Aha!" moments everyday in classrooms around the world
% 
%   5. Ensure that there are *no spelling mistakes anywhere*. We are an 
%      education company. Bad spellings suggest that we ourselves 
%      don't have any education. And, use American spellings

\question[3] The angle of elevation of the top of a tower from a point $A$ 
due south of it is $\alpha$ and $\beta$ from a point $B$ due east of it. If $AB=d$, then prove that the height of the tower - $H$ - is equal to $\dfrac{d}{\sqrt{\cot^2\alpha + \cot^2\beta}}$

\ifprintanswers
  % stuff to be shown only in the answer key - like explanatory margin figures
	\begin{marginfigure}
		\figinit{cm}
		\figpt 1:$O$(0,0)
		\figpt 2:$Y$(0,-3)
		\figpt 3:$Z$(4,0)
		\figsetmark{+}
		\figdrawbegin{}
			\figdrawline[1,2,3,1]
		\figdrawend
		\figvisu{\figBoxA}{Figure}{}
		\centerline{\box\figBoxA}
	\end{marginfigure}
\fi 

\begin{solution}[\halfpage]
	From the adjoining figures, we know the following:
	\begin{align}
		\text{Top-view} &: OA^2 + OB^2 = d^2 \\
		\text{South-side} &: OX = H = OA \times\tan\alpha \Rightarrow OA = H\cdot\cot\alpha \\
		\text{East-side} &: OX = H = OB \times\tan\beta \Rightarrow OB = H\cdot\cot\beta
	\end{align}
	
	Substituting (2) and (3) in (1), we get
	\begin{align}
		(H\cdot\cot\alpha)^2 + (H\cdot\cot\beta)^2 &= d^2 \\
		\Rightarrow H^2 &= \dfrac{d^2}{\cot^2\alpha + \cot^2\beta} \\
		\Rightarrow H &= \dfrac{d}{\sqrt{\cot^2\alpha + \cot^2\beta}}
	\end{align}
\end{solution}
