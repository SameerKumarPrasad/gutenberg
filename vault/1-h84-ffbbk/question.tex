% This is an empty shell file placed for you by the 'examiner' script.
% You can now fill in the TeX for your question here.

% Now, down to brasstacks. ** Writing good solutions is an Art **. 
% Eventually, you will find your own style. But here are some thoughts 
% to get you started: 
%
%   1. Write to be understood - but be crisp. Your own solution should not take 
%      more space than you will give to the student. Hence, if you take more than 
%      a half-page to write a solution, then give the student a full-page and so on...
%
%   2. Use margin-notes to "talk" to students about the critical insights
%      in the question. The tone can be - in fact, should be - informal
%
%   3. Don't shy away from creating margin-figures you think will help
%      students understand. Yes, it is a little more work per question. 
%      But the question & solution will be written only once. Make that
%      attempt at writing a solution count.
%      
%      3b. Use bc_to_fig.tex. Its an easier way to generate plots & graphs 
% 
%   4. Ensure that there are *no spelling mistakes anywhere*. We are an 
%      education company. Bad spellings suggest that we ourselves 
%      don't have any education. Also, use American spellings by default
% 
%   5. If a question has multiple parts, then first delete lines 40-41
%   6. If a question does not have parts, then first delete lines 43-69
%   
%   7. Create versions of the question when possible. Use commands defined in 
%      tufte-tweaks.sty to do so. Its easier than you think

% \noprintanswers
% \setcounter{rolldice}{0}

\ifnumequal{\value{rolldice}}{0}{
  % variables 
  \renewcommand{\vbone}{5} % a1
  \renewcommand{\vbtwo}{6} % b1
  \renewcommand{\vbthree}{1} % c1
  \renewcommand{\vbfour}{3} %a2
  \renewcommand{\vbfive}{5} % c2
  \renewcommand{\vbsix}{3} % k = b1/b2
}{
  \ifnumequal{\value{rolldice}}{1}{
    % variables 
    \renewcommand{\vbone}{3}
    \renewcommand{\vbtwo}{4}
    \renewcommand{\vbthree}{7}
    \renewcommand{\vbfour}{4}
    \renewcommand{\vbfive}{5}
    \renewcommand{\vbsix}{2}
  }{
    \ifnumequal{\value{rolldice}}{2}{
      % variables 
      \renewcommand{\vbone}{4}
      \renewcommand{\vbtwo}{7}
      \renewcommand{\vbthree}{11}
      \renewcommand{\vbfour}{2}
      \renewcommand{\vbfive}{13}
      \renewcommand{\vbsix}{1}
    }{
      % variables 
      \renewcommand{\vbone}{5}
      \renewcommand{\vbtwo}{8}
      \renewcommand{\vbthree}{7}
      \renewcommand{\vbfour}{3}
      \renewcommand{\vbfive}{9}
      \renewcommand{\vbsix}{2}
    }
  }
}

\gcalcexpr[0]{\vbseven}{\vbtwo / \vbsix} % b2
\gcalcexpr[2]{\vbeight}{(\vbsix * \vbfive - \vbthree)/(\vbone - \vbsix * \vbfour)} % x of intersection
\gcalcexpr[2]{\vbnine}{-(\vbthree + \vbone * \vbeight)/\vbtwo}

\question Find the equation of the straight line that passes through the point of intersection of 
$\vbone x + \vbtwo y + \vbthree = 0$ and $\vbfour x + \vbseven y + \vbfive = 0$ \textit{and} is 
perpendicular to the line $3x-5y+11 = 0$

\insertQR{}

\watchout

\ifprintanswers
\fi 

\begin{solution}
	First, we must find that point of intersection. The two lines will intersect at an $x$ where
	\begin{align}
	    \vbone x + \vbtwo y + \vbthree &= 0 \text{ or } \vbtwo y = -(\vbone x + \vbthree) \\
	    \vbfour x + \vbseven y + \vbfive &= 0 \text { or } \vbseven y = -(\vbfour x + \vbfive) \\
	    \Rightarrow \dfrac{\vbone x + \vbthree}{\vbfour x + \vbfive} &= \vbsix \text{ or } x = \vbeight \\
	    \text{ which means that } y &= \dfrac{-(\vbthree + \vbone\cdot\vbeight)}{\vbtwo} = \vbnine
	\end{align}
	So, the point of intersection is at $(\vbeight, \vbnine)$
	
	Now, the slope of the line \textit{ perpendicular } to 
	$3x - 5y + 11 = 0 \Rightarrow y = \dfrac{3}{5}x + \dfrac{11}{5}$ is $-\dfrac{5}{3}$. And so, its equation
	would be 
	
	\gcalcexpr[2]{\vbten}{5*\vbeight + 3*\vbnine}
	
	\begin{align}
		\dfrac{y - \vbnine}{x - \vbeight} &= -\dfrac{5}{3} \\
		\Rightarrow 3y - 3\times\vbnine &= -5x + 5\times\vbeight \\
		\text{ or } 3y + 5x + \vbten &= 0
	\end{align}
	
\end{solution}
