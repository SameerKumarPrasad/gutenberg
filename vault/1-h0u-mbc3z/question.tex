% This is an empty shell file placed for you by the 'examiner' script.
% You can now fill in the TeX for your question here.

% Now, down to brasstacks. ** Writing good solutions is an Art **. 
% Eventually, you will find your own style. But here are some thoughts 
% to get you started: 
%
%   1. Write the solution as if you are writing it for your favorite
%      14-17 year old to help him/her understand. Could be your nephew, 
%      your niece, a cousin perhaps or probably even you when you 
%      were that age. Just write for them.
%
%   2. Use margin-notes to "talk" to students about the critical insights
%      in the question. The tone can be - in fact, should be - informal
%
%   3. Don't shy away from creating margin-figures you think will help
%      students understand. Yes, it is a little more work per question. 
%      But the question & solution will be written only once. Make that
%      attempt at writing a solution count.
%
%   4. At the same time, do not be too verbose. A long solution can
%      - at first sight - make the student think, "God, that is a lot to know".
%      Our aim is not to scare students. Rather, our aim should be to 
%      create many "Aha!" moments everyday in classrooms around the world
% 
%   5. Ensure that there are *no spelling mistakes anywhere*. We are an 
%      education company. Bad spellings suggest that we ourselves 
%      don't have any education. Also, use American spellings by default
% 
%   6. If a question has multiple parts, then first delete lines 40-41
%   7. If a question does not have parts, then first delete lines 43-69

\question After 4 seconds of motion, the speed of a moving particle is $1$ cm/s. 
Given that the speed of the particle is propotional to square of the time, what is
the distance travelled by the particle in the first $10$ seconds?

\insertQR{}

\ifprintanswers
  % stuff to be shown only in the answer key - like explanatory margin figures
\fi 

\begin{solution}
  \begin{align}
     \text{Speed} &= \dfrac{\ud x}{\ud t} \propto t^2 \\
     \Rightarrow \dfrac{\ud x}{\ud t} &= k\cdot t^2
  \end{align}
  And so, if at $t=4$, $\dfrac{\ud x}{\ud t} = 1$, then 
  \begin{align}
     \dfrac{\ud x}{\ud t} = 1 &= k\cdot 4^2 \\
     \Rightarrow k &= \frac{1}{16} \\
     \therefore \dfrac{\ud x}{\ud t} &= \dfrac{t^2}{16}
  \end{align}
  
  Now that we have an equation for speed, finding the distance covered \textit{upto}
  a time $t$ is easy
  \begin{align}
     \text{Distance covered} &= \int_{0}^{t} \ud x = \int_0^{t}\dfrac{t^2}{16}\ud t \\
     &= \left[ \dfrac{t^3}{48}\right]_0^{t} = \dfrac{t^3}{48}
  \end{align}
  
  And hence, the distance covered in the first 10 seconds is $ = \dfrac{10^3}{48} = 20\frac{5}{6}cm$
\end{solution}
