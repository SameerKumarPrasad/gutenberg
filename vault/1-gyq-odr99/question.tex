% This is an empty shell file placed for you by the 'examiner' script.
% You can now fill in the TeX for your question here.

% Now, down to brasstacks. ** Writing good solutions is an Art **. 
% Eventually, you will find your own style. But here are some thoughts 
% to get you started: 
%
%   1. Write the solution as if you are writing it for your favorite
%      14-17 year old to help him/her understand. Could be your nephew, 
%      your niece, a cousin perhaps or probably even you when you 
%      were that age. Just write for them.
%
%   2. Use margin-notes to "talk" to students about the critical insights
%      in the question. The tone can be - in fact, should be - informal
%
%   3. Don't shy away from creating margin-figures you think will help
%      students understand. Yes, it is a little more work per question. 
%      But the question & solution will be written only once. Make that
%      attempt at writing a solution count.
%
%   4. At the same time, do not be too verbose. A long solution can
%      - at first sight - make the student think, "God, that is a lot to know".
%      Our aim is not to scare students. Rather, our aim should be to 
%      create many "Aha!" moments everyday in classrooms around the world
% 
%   5. Ensure that there are *no spelling mistakes anywhere*. We are an 
%      education company. Bad spellings suggest that we ourselves 
%      don't have any education. Also, use American spellings by default
% 
%   6. If a question has multiple parts, then first delete lines 40-41
%   7. If a question does not have parts, then first delete lines 43-69

\question[3] Compute the area of the figure contained between the curve 
$y = \dfrac{1}{1+x^2}$ and the parabola $y=\dfrac{x^2}{2}$

\insertQR{QRC}

\ifprintanswers
  \begin{marginfigure}
  % 1. Definition of characteristic points
\figinit{pt}
\def\Xmin{-39.99999}
\def\Ymin{-31.99999}
\def\Xmax{39.99999}
\def\Ymax{47.99999}
\def\Xori{39.99999}
\def\Yori{31.99999}
\figpt0:(\Xori,\Yori)
\figpt 100: (42,56)
% 2. Creation of the graphical file
\figdrawbegin{}
\def\Xmaxx{\Xmax} % To customize the position
\def\Ymaxx{\Ymax} % of the arrow-heads of the axes.
\figset arrowhead(length=4, fillmode=yes) % styling the arrowheads
\figdrawaxes 0(\Xmin, \Xmaxx, \Ymin, \Ymaxx)
\figdrawlineC(
0 45.12820,
2.75862 46.46152,
5.51724 47.96736,
8.27586 49.66531,
11.03448 51.57325,
13.79310 53.70431,
16.55172 56.06214,
19.31034 58.63400,
22.06896 61.38191,
24.82758 64.23235,
27.58620 67.06735,
30.34482 69.72159,
33.10344 71.99182,
35.86206 73.66347,
38.62068 74.55282,
41.37931 74.55282,
44.13793 73.66347,
46.89655 71.99182,
49.65517 69.72159,
52.41379 67.06735,
55.17241 64.23235,
57.93103 61.38191,
60.68965 58.63400,
63.44827 56.06214,
66.20689 53.70431,
68.96551 51.57325,
71.72413 49.66531,
74.48275 47.96736,
77.24137 46.46152,
79.99999 45.12820
)
\figdrawlineC(
0 79.99999,
2.75862 73.60760,
5.51724 67.67181,
8.27586 62.19262,
11.03448 57.17003,
13.79310 52.60404,
16.55172 48.49464,
19.31034 44.84185,
22.06896 41.64565,
24.82758 38.90606,
27.58620 36.62306,
30.34482 34.79667,
33.10344 33.42687,
35.86206 32.51367,
38.62068 32.05707,
41.37931 32.05707,
44.13793 32.51367,
46.89655 33.42687,
49.65517 34.79667,
52.41379 36.62306,
55.17241 38.90606,
57.93103 41.64565,
60.68965 44.84185,
63.44827 48.49464,
66.20689 52.60404,
68.96551 57.17003,
71.72413 62.19262,
74.48275 67.67181,
77.24137 73.60760,
79.99999 79.99999
)
\figdrawend
% 3. Writing text on the figure
\figvisu{\figBoxA}{}{%
\figptsaxes 1:0(\Xmin, \Xmaxx, \Ymin, \Ymaxx)
% Points 1 and 2 are the end points of the arrows
\figwritee 1:(5pt)     \figwriten 2:(5pt)
\figptsaxes 1:0(\Xmin, \Xmax, \Ymin, \Ymax)
\figwritee 100:$R$(2)
}
\centerline{\box\figBoxA}

  \end{marginfigure}
\fi 

\begin{solution}[\fullpage]
  The two curves will intersect at points where 
  \begin{align}
     \dfrac{1}{1+x^2} &= \dfrac{x^2}{2} \\
     \Rightarrow x^4+x^2-2 &= 0 \\
     \text{Setting } z = x^2, \text{ we get }
     z^2+z-2 &= 0 \\
     \Rightarrow z = 1, -2
  \end{align}
  As $z = x^2$, it has to be > 0. And therefore, we go with $z = 1 \Rightarrow x = \pm 1$
  
  Given how symmetrical everything is, we can evaluate the area $A$ of the said
  region $R$ as,
  \begin{align}
     A &= 2\cdot\left[ \int_0^1 \dfrac{1}{1+x^2}\ud x - \int_0^1\dfrac{x^2}{2}\ud x\right] \\
     &= 2\cdot\left[ \left( \tan^{-1} x\right)_0^1 - \left( \dfrac{x^3}{6}\right)_0^1\right] \\
     &= \dfrac{\pi}{2} - \dfrac{1}{3} 
  \end{align}
\end{solution}
