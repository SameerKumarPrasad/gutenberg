% This is an empty shell file placed for you by the 'examiner' script.
% You can now fill in the TeX for your question here.

% Now, down to brasstacks. ** Writing good solutions is an Art **. 
% Eventually, you will find your own style. But here are some thoughts 
% to get you started: 
%
%   1. Write the solution as if you are writing it for your favorite
%      14-17 year old to help him/her understand. Could be your nephew, 
%      your niece, a cousin perhaps or probably even you when you 
%      were that age. Just write for them.
%
%   2. Use margin-notes to "talk" to students about the critical insights
%      in the question. The tone can be - in fact, should be - informal
%
%   3. Don't shy away from creating margin-figures you think will help
%      students understand. Yes, it is a little more work per question. 
%      But the question & solution will be written only once. Make that
%      attempt at writing a solution count.
%
%   4. At the same time, do not be too verbose. A long solution can
%      - at first sight - make the student think, "God, that is a lot to know".
%      Our aim is not to scare students. Rather, our aim should be to 
%      create many "Aha!" moments everyday in classrooms around the world
% 
%   5. Ensure that there are *no spelling mistakes anywhere*. We are an 
%      education company. Bad spellings suggest that we ourselves 
%      don't have any education. Also, use American spellings by default
% 
%   6. If a question has multiple parts, then first delete lines 40-41
%   7. If a question does not have parts, then first delete lines 43-69

\question[4] Find the sum of an infinitely decreasing geometric progression if thesum the first three terms is $3$ and the sum of the first three odd terms is $\dfrac{21}{4}$.

\insertQR{QRC}

\ifprintanswers
  % stuff to be shown only in the answer key - like explanatory margin figures
\fi 

\begin{solution}[\fullpage]
  Let the progression have initial term $a$ and common ration $r$. From the question statement we know that,
  \begin{align}
    a+ar+ar^2   &= 3 \\
    a+ar^2+ar^4 &= \dfrac{21}{4}
  \end{align}
  Rewrite equation (2) as follows,
  \begin{align}
    a(1+r^2+r^4)               &= \dfrac{21}{4} \\
	a((1+r+r^2)^2-2r(1+r+r^2)) &= \dfrac{21}{4}
  \end{align}
  Using equation (1) and (4)
  \begin{align}
    a((\dfrac{3}{a})^2-2r(\dfrac{3}{a}) &= \dfrac{21}{4} \\
    \dfrac{3}{a}-2r                     &= \dfrac{7}{4} \\
    r(1+r+r^2-2r)                       &= 7 \\
    r                                   &= -\dfrac{1}{2},\dfrac{3}{2}
  \end{align}
  Common ratio $\dfrac{3}{2}$ is rejected as it will not result in a converging series. Substituting common ratio $-\dfrac{1}{2}$ in equation (1), we get initial term as $\dfrac{8}{3}$.

\end{solution}
