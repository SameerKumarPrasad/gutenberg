% This is an empty shell file placed for you by the 'examiner' script.
% You can now fill in the TeX for your question here.

% Now, down to brasstacks. ** Writing good solutions is an Art **. 
% Eventually, you will find your own style. But here are some thoughts 
% to get you started: 
%
%   1. Write the solution as if you are writing it for your favorite
%      14-17 year old to help him/her understand. Could be your nephew, 
%      your niece, a cousin perhaps or probably even you when you 
%      were that age. Just write for them.
%
%   2. Use margin-notes to "talk" to students about the critical insights
%      in the question. The tone can be - in fact, should be - informal
%
%   3. Don't shy away from creating margin-figures you think will help
%      students understand. Yes, it is a little more work per question. 
%      But the question & solution will be written only once. Make that
%      attempt at writing a solution count.
%
%   4. At the same time, do not be too verbose. A long solution can
%      - at first sight - make the student think, "God, that is a lot to know".
%      Our aim is not to scare students. Rather, our aim should be to 
%      create many "Aha!" moments everyday in classrooms around the world
% 
%   5. Ensure that there are *no spelling mistakes anywhere*. We are an 
%      education company. Bad spellings suggest that we ourselves 
%      don't have any education. Also, use American spellings by default
% 
%   6. If a question has multiple parts, then first delete lines 40-41
%   7. If a question does not have parts, then first delete lines 43-69

\question For the series $1 + \dfrac{1}{1+2} + \dfrac{1}{1+2+3} + \dfrac{1}{1+2+3+4}\ldots$

\insertQR{}

\ifprintanswers
\fi 

\begin{parts}
  \part Find an expression for the sum of the first $n$ terms

  \insertQR{}
  \begin{solution}
  	\begin{align}
  		S_n &= 1 + \dfrac{1}{1+2} + \dfrac{1}{\underbrace{1+2+3}_{\text{Sum of first $k$ integers}}} + \dfrac{1}{1+2+3+4}\ldots \\
  		&= \sum_{k=1}^{n}\dfrac{1}{\eSumOfN{k}} = 2\cdot\sum_{k=1}^{n}\dfrac{1}{k\cdot(k+1)} \\
  		&= 2\cdot\sum_{k=1}^{n}\dfrac{k+1-k}{k\cdot(k+1)} \\
  		&= 2\cdot\left[ \sum_{k=1}^{n}\dfrac{1}{k} - \sum_{k=1}^{n}\dfrac{1}{k+1} \right] \\
  		&= 2\cdot\left[ 1 + \underbrace{\sum_{k=2}^{n}\dfrac{1}{k} - \sum_{k=1}^{n-1}\dfrac{1}{k+1}}
  		_{\text{= 0. Convinced?}}- \dfrac{1}{n+1}\right] \\
  		&= \dfrac{2n}{n+1}
  	\end{align}
  \end{solution}

  \part The sum if the same series went upto infinity

  \insertQR{}
  \begin{solution}
  	The above expression for $S_n$ can be re-written as
  	\begin{align}
  		S_n &= \dfrac{2}{1+\frac{1}{n}} \\
  		\Rightarrow\text{ that when } n &= \infty, \text{ then } S_n = 2
  	\end{align}
  \end{solution}

\end{parts}
