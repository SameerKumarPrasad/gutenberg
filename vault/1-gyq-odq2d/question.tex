% This is an empty shell file placed for you by the 'examiner' script.
% You can now fill in the TeX for your question here.

% Now, down to brasstacks. ** Writing good solutions is an Art **. 
% Eventually, you will find your own style. But here are some thoughts 
% to get you started: 
%
%   1. Write the solution as if you are writing it for your favorite
%      14-17 year old to help him/her understand. Could be your nephew, 
%      your niece, a cousin perhaps or probably even you when you 
%      were that age. Just write for them.
%
%   2. Use margin-notes to "talk" to students about the critical insights
%      in the question. The tone can be - in fact, should be - informal
%
%   3. Don't shy away from creating margin-figures you think will help
%      students understand. Yes, it is a little more work per question. 
%      But the question & solution will be written only once. Make that
%      attempt at writing a solution count.
%
%   4. At the same time, do not be too verbose. A long solution can
%      - at first sight - make the student think, "God, that is a lot to know".
%      Our aim is not to scare students. Rather, our aim should be to 
%      create many "Aha!" moments everyday in classrooms around the world
% 
%   5. Ensure that there are *no spelling mistakes anywhere*. We are an 
%      education company. Bad spellings suggest that we ourselves 
%      don't have any education. And, use American spellings

\question[6] A peacock - a hungry peacock - is sitting on top of a pillar 9 meters high.
A snake emerges from a hole 27 meters from the base of the pillar and starts moving 
towards the pole. Seeing the snake, the peacock pounces on it. If the speed at which 
the peacock comes down is the same at which the snake is crawling, then at what distance
from its hole is the snake caught? Assume that the peacock descends in a straight line                          

\ifprintanswers
  % stuff to be shown only in the answer key - like explanatory margin figures
  \begin{marginfigure}
    \figinit{cm}
    \figpt 1: (0,0)
    \figpt 2: (5,0)
    \figpt 3: (5,3)
    \figpt 4: (3,0)
    \figpt 5: (1.5,0)
    \figpt 6: (4.75,0)
    \figpt 7: (1,0.35)
    \figpt 8: (3.6,0.35)
  	\figdrawbegin{}
    \figdrawline [1,2,3,1]
    \figdrawline [3,4]
    \figdrawend
    \figvisu{\figBoxA}{Figure 1}{%
      \figwritew 1:A(0.1)
      \figwritee 3:B(0.1)
      \figwrites 4:C(0.1)
      \figwritee 2:D(0.1)
      \figwritese 5:$x$(0.2)
      \figwritesw 6:$27-x$(0.2)
      \figwritee 7:$\alpha$(0.2)
      \figwritee 8:$2\alpha$(0.2)
      \psarc{<->}{1}{0}{30}
      \psarc{<->}(3,0){0.65}{0}{60}
    }
    \centerline{\box\figBoxA}
  \end{marginfigure}
  \marginnote[1cm] {$\angle BCD = 2\alpha$ because $x = BC$}
  \marginnote[0.5cm]{ $x = BC $ because the snake and peacock start at the same time
  from their initial positions, move at the same speed and meet at the same time }
  \marginnote[1cm] { That said, the snake surely didn't want to 'meet' the peacock }
\fi 

\begin{solution}[\fullpage]
	In the figure alongside, let $x$ be the distance the snake travels before
	it is caught
	
	Now, right upfront we know at least the following:
	\begin{align}
		\tan\alpha &= \dfrac{9}{27} = \dfrac{1}{3} \\
		\angle BCD &= 2\cdot\alpha \\
	    \Rightarrow \tan\angle BCD &= \tan 2\alpha = \dfrac{2\tan\alpha}{1-\tan^2\alpha}
	                               = \dfrac{9}{27-x}
	\end{align}
	Solving (3), we get 
	\begin{align}
		\tan\angle BCD &= \dfrac{9}{27-x} = \dfrac{2\cdot\frac{1}{3}}{1-(\frac{1}{3})^2} \\
		\Rightarrow \dfrac{9}{27-x} &= \dfrac{3}{4} \\
		\Rightarrow x &= 15
	\end{align}
	
	So, the snake goes 15 meters before it is caught!
	
	
\end{solution}
