
% \noprintanswers
%\setcounter{rolldice}{1}
% \printrubric

\ifnumequal{\value{rolldice}}{0}{
  % variables 
  \renewcommand{\vbone}{9}
  \renewcommand{\vbtwo}{27}
}{
  \ifnumequal{\value{rolldice}}{1}{
    % variables 
    \renewcommand{\vbone}{10}
    \renewcommand{\vbtwo}{50}
  }{
    \ifnumequal{\value{rolldice}}{2}{
      % variables 
      \renewcommand{\vbone}{12}
      \renewcommand{\vbtwo}{36}
    }{
      % variables 
      \renewcommand{\vbone}{8}
      \renewcommand{\vbtwo}{32}
    }
  }
}

\FRACTIONSIMPLIFY\vbone\vbtwo\a\b %\tan\alpha
\FRACMULT{2}{1}\a\b\c\d % 2\tan\alpha
\FRACMULT\a\b\a\b\e\f %\tan^{2}\alpha
\FRACMINUS{1}{1}\e\f\x\y % 1- \tan^2\alpha
\FRACDIV\c\d\x\y\m\n

\FRACMULT\n\m\vbone{1}\j\k
\FRACMINUS\vbtwo{1}\j\k\ansx\ansy

\question[4] Snake eagles are predatory birds that hunt, well, snakes. 
Brave birds - and one such bird is perched on a tree $\vbone$ meters high when 
a snake emerges from a hole $\vbtwo$ meters from the base of the tree and starts moving 
towards it. Seeing the snake, the eagle pounces on it. If the speed at which 
the eagle comes down is the same at which the snake is crawling, then at what distance
from its hole is the snake caught? Assume that the snake descends in a straight line                          

\insertQR[30pt]{QRC}
\watchout[-70pt]

\ifprintanswers
  % stuff to be shown only in the answer key - like explanatory margin figures
  \begin{marginfigure}
    \figinit{cm}
    \figpt 1: (0,0)
    \figpt 2: (5,0)
    \figpt 3: (5,3)
    \figpt 4: (3,0)
    \figpt 5: (1.5,0)
    \figpt 6: (4.75,0)
    \figpt 7: (1,0.35)
    \figpt 8: (3.6,0.35)
  	\figdrawbegin{}
    \figdrawline [1,2,3,1]
    \figdrawline [3,4]
    \figdrawend
    \figvisu{\figBoxA}{Figure 1}{%
      \figwritew 1:A(0.1)
      \figwritee 3:B(0.1)
      \figwrites 4:C(0.1)
      \figwritee 2:D(0.1)
      \figwritese 5:$x$(0.2)
      \figwritesw 6:$\vbtwo-x$(0.2)
      \figwritee 7:$\alpha$(0.2)
      \figwritee 8:$2\alpha$(0.2)
      \psarc{<->}{1}{0}{30}
      \psarc{<->}(3,0){0.65}{0}{60}
    }
    \centerline{\box\figBoxA}
  \end{marginfigure}
  \marginnote[1cm] {$\angle BCD = 2\alpha$ because $x = BC$}
  \marginnote[0.5cm]{ $x = BC $ because the snake and peacock start at the same time
  from their initial positions, move at the same speed and meet at the same time }
\fi 


\begin{solution}[\fullpage]
	In the figure alongside, let $x$ be the distance the snake travels before
	it is caught
	
	Now, right upfront we know at least the following:
	\begin{align}
		\tan\alpha &= \dfrac\vbone\vbtwo = \dfrac\a\b \\
		\angle BCD &= 2\cdot\alpha \\
	    \Rightarrow \tan\angle BCD &= \tan 2\alpha = \dfrac{2\tan\alpha}{1-\tan^2\alpha}
	                               = \dfrac\vbone{\vbtwo - x}
	\end{align}
	Solving (3), we get 
	\begin{align}
		\tan\angle BCD &= \dfrac{\vbone}{\vbtwo-x} = \dfrac{2\cdot\frac\a\b}{1-(\frac\a\b)^2} \\
		\Rightarrow \dfrac{\vbone}{\vbtwo-x} &= \dfrac\m\n \\
		\Rightarrow x &= \WRITEFRAC\ansx\ansy
	\end{align}
	
	So, the snake goes $\WRITEFRAC\ansx\ansy$ meters before it is caught!
\end{solution}


\ifprintrubric
  \begin{table}
  	\begin{tabular}{ p{5cm}p{5cm} }
  		\toprule % in brief (4-6 words), what should a grader be looking for for insights & formulations
  		  \sc{\textcolor{blue}{Insight}} & \sc{\textcolor{blue}{Formulation}} \\ 
  		\midrule % ***** Insights & formulations ******
        $\angle{BCD} = 2\times\alpha$ & Applied $\tan 2\alpha = \dfrac{2\tan\alpha}{1-\tan^2\alpha}$ \\
  		\toprule % final numerical answers for the various versions
        \sc{\textcolor{blue}{If question has $\ldots$}} & \sc{\textcolor{blue}{Final answer}} \\
  		\midrule % ***** Numerical answers (below) **********
        Tree height = 8m & $x=17$ \\
        Tree height = 9m & $x=15$ \\
        Tree height = 10m & $x=26$ \\
        Tree height = 12m & $x=20$ \\
  		\bottomrule
  	\end{tabular}
  \end{table}
\fi
