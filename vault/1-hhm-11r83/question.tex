
% \noprintanswers
% \setcounter{rolldice}{3}
% \printrubric

\ifnumequal{\value{rolldice}}{0}{
  % variables 
  \renewcommand{\vbone}{7}
  \renewcommand{\vbtwo}{5}
}{
  \ifnumequal{\value{rolldice}}{1}{
    % variables 
    \renewcommand{\vbone}{6}
    \renewcommand{\vbtwo}{3}
  }{
    \ifnumequal{\value{rolldice}}{2}{
      % variables 
      \renewcommand{\vbone}{5}
      \renewcommand{\vbtwo}{7}
    }{
      % variables 
      \renewcommand{\vbone}{7}
      \renewcommand{\vbtwo}{3}
    }
  }
}

\renewcommand{\vbthree}{\vbone + \sqrt{\vbtwo}}
\renewcommand{\vbfour}{\vbone - \sqrt{\vbtwo}}
\SQUARE\vbone\a
\SUBTRACT\a\vbtwo\b
\MULTIPLY\vbone{4}\c
\FRACTIONSIMPLIFY\c\b\p\q

\question[3] If $\dfrac{\vbthree}{\vbfour} - \dfrac{\vbfour}{\vbthree} = a + b\cdot\sqrt{\vbtwo}$, then what 
are the values of $a$ and $b$?

\insertQR[-10pt]{}

\watchout

\begin{solution}[\halfpage]
	\begin{align}
		\dfrac{\vbthree}{\vbfour} &- \dfrac{\vbfour}{\vbthree} = \dfrac{(\vbthree)^2 - (\vbfour)^2}{\vbone^{2}-\vbtwo} \\
		&= \dfrac{(\vbone^2 + 2\cdot\vbone\cdot\sqrt{\vbtwo} + \vbtwo) - 
		(\vbone^2 - 2\cdot\vbone\cdot\sqrt{\vbtwo} + \vbtwo)}{\b} \\
		&= \dfrac{\p}{\q}\cdot\sqrt{\vbtwo}
	\end{align}
	On comparing (3) with the expression $a + b\cdot\sqrt{\vbtwo}$, we can see that $a=0$ and $b = \frac{\p}{\q}$
\end{solution}

\ifprintrubric
  \begin{table}
  	\begin{tabular}{ p{5cm}p{5cm} }
  		\toprule % final numerical answers for the various versions
        \sc{\textcolor{blue}{If question has $\ldots$}} & \sc{\textcolor{blue}{Final answer}} \\
  		\midrule % ***** Numerical answers (below) **********
  			$7+\sqrt{5}$ & $a=0$ and $b = \frac{7}{11}$ \\
  			$6+\sqrt{3}$ & $a=0$ and $b = \frac{8}{11}$ \\
  			$5+\sqrt{7}$ & $a=0$ and $b = \frac{10}{9}$ \\
  			$7+\sqrt{3}$ & $a=0$ and $b = \frac{14}{23}$ \\
  		\bottomrule
  	\end{tabular}
  \end{table}
\fi
