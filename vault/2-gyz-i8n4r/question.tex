% This is an empty shell file placed for you by the 'examiner' script.
% You can now fill in the TeX for your question here.

% Now, down to brasstacks. ** Writing good solutions is an Art **. 
% Eventually, you will find your own style. But here are some thoughts 
% to get you started: 
%
%   1. Write the solution as if you are writing it for your favorite
%      14-17 year old to help him/her understand. Could be your nephew, 
%      your niece, a cousin perhaps or probably even you when you 
%      were that age. Just write for them.
%
%   2. Use margin-notes to "talk" to students about the critical insights
%      in the question. The tone can be - in fact, should be - informal
%
%   3. Don't shy away from creating margin-figures you think will help
%      students understand. Yes, it is a little more work per question. 
%      But the question & solution will be written only once. Make that
%      attempt at writing a solution count.
%
%   4. At the same time, do not be too verbose. A long solution can
%      - at first sight - make the student think, "God, that is a lot to know".
%      Our aim is not to scare students. Rather, our aim should be to 
%      create many "Aha!" moments everyday in classrooms around the world
% 
%   5. Ensure that there are *no spelling mistakes anywhere*. We are an 
%      education company. Bad spellings suggest that we ourselves 
%      don't have any education. And, use American spellings

\question[4] Calculate $2\left(\dfrac{\cos 58^\circ}{\sin 32^\circ}\right) - \sqrt{3} \left(\dfrac{\cos 38^\circ \cos 52^\circ}{\tan 15^\circ \tan 60^\circ \tan 75^\circ}\right)$?
\ifprintanswers
  % stuff to be shown only in the answer key - like explanatory margin figures
\fi 
\begin{solution}
  \begin{align}
    \text{Let}\: &\theta   = 32^\circ \\ 
         	     &\phi     = 38^\circ \\ 
                 &\epsilon = 15^\circ 
  \end{align}
  Rewriting the expression using (1), (2) and (3), we get \linebreak
  \begin{align}
    \text{E} &= 2\left(\dfrac{\cos(\frac{\pi}{2}-\theta)}{\sin \theta}\right) - 
  	       		 \sqrt{3}
  	       		 \left(\dfrac{\cos \phi \cos(\frac{\pi}{2}-\phi)}
  	       			   {\tan \epsilon \tan \frac{\pi}{3} \tan(\frac{\pi}{2}-\epsilon)}
		   		\right) \\
    \text{E} &= 2\left(\dfrac{\sin \theta}{\sin \theta}\right) - 
  	       		 \sqrt{3}
  	       		 \left(\dfrac{\cos \phi \sec \phi}
  	       			   {\tan \epsilon \sqrt{3} \cot \epsilon}
		   		\right) \\
    \text{E} &= 2 - \sqrt{3}\left(\dfrac{1}{\sqrt{3}}\right) \nonumber \\
    \text{E} &= 1
  \end{align}    
\end{solution}
