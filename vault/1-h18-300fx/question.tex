% This is an empty shell file placed for you by the 'examiner' script.
% You can now fill in the TeX for your question here.

% Now, down to brasstacks. ** Writing good solutions is an Art **. 
% Eventually, you will find your own style. But here are some thoughts 
% to get you started: 
%
%   1. Write the solution as if you are writing it for your favorite
%      14-17 year old to help him/her understand. Could be your nephew, 
%      your niece, a cousin perhaps or probably even you when you 
%      were that age. Just write for them.
%
%   2. Use margin-notes to "talk" to students about the critical insights
%      in the question. The tone can be - in fact, should be - informal
%
%   3. Don't shy away from creating margin-figures you think will help
%      students understand. Yes, it is a little more work per question. 
%      But the question & solution will be written only once. Make that
%      attempt at writing a solution count.
%
%   4. At the same time, do not be too verbose. A long solution can
%      - at first sight - make the student think, "God, that is a lot to know".
%      Our aim is not to scare students. Rather, our aim should be to 
%      create many "Aha!" moments everyday in classrooms around the world
% 
%   5. Ensure that there are *no spelling mistakes anywhere*. We are an 
%      education company. Bad spellings suggest that we ourselves 
%      don't have any education. Also, use American spellings by default
% 
%   6. If a question has multiple parts, then first delete lines 40-41
%   7. If a question does not have parts, then first delete lines 43-69

\question Compute the area of parabolic area $R$ - as shown in the adjoining 
figure - with base $a = 10cm$ and altitude $h=6cm$. The base is perpendicular 
to the axis of the parabola

\insertQR{}

\ifprintanswers
  % stuff to be shown only in the answer key - like explanatory margin figures
\fi 
\begin{marginfigure}
% 1. Definition of characteristic points
\figinit{pt}
\def\Xmin{-35.00000}
\def\Ymin{-10.00000}
\def\Xmax{35.00000}
\def\Ymax{70.00000}
\def\Xori{35.00000}
\def\Yori{10.00000}
\figpt0:(\Xori,\Yori)
\figpt 100: (0,70)
\figpt 101: (70,70)
\figpt 102: (70,10)
\figpt 103: (52,70)
\figpt 104: (70,40)
% 2. Creation of the graphical file
\figdrawbegin{}
\def\Xmaxx{\Xmax} % To customize the position
\def\Ymaxx{\Ymax} % of the arrow-heads of the axes.
\figset arrowhead(length=4, fillmode=yes) % styling the arrowheads
\figdrawaxes 0(\Xmin, \Xmaxx, \Ymin, \Ymaxx)
\figdrawline[100,101,102]
\figdrawarrowhead [100,101]
\figdrawarrowhead [101,100]
\figdrawarrowhead [101,102]
\figdrawarrowhead [102,101]
\figdrawlineC(
0 70.00000,
2.41379 62.00951,
4.82758 54.58977,
7.24137 47.74078,
9.65517 41.46254,
12.06896 35.75505,
14.48275 30.61831,
16.89655 26.05231,
19.31034 22.05707,
21.72413 18.63258,
24.13793 15.77883,
26.55172 13.49583,
28.96551 11.78359,
31.37931 10.64209,
33.79310 10.07134,
36.20689 10.07134,
38.62068 10.64209,
41.03448 11.78359,
43.44827 13.49583,
45.86206 15.77883,
48.27586 18.63258,
50.68965 22.05707,
53.10344 26.05231,
55.51724 30.61831,
57.93103 35.75505,
60.34482 41.46254,
62.75862 47.74078,
65.17241 54.58977,
67.58620 62.00951,
69.99999 69.99999
)
\figdrawend
% 3. Writing text on the figure
\figvisu{\figBoxA}{}{%
\figptsaxes 1:0(\Xmin, \Xmaxx, \Ymin, \Ymaxx)
% Points 1 and 2 are the end points of the arrows
\figwritee 1:(5pt)     \figwriten 2:(5pt)
\figptsaxes 1:0(\Xmin, \Xmax, \Ymin, \Ymax)
\figwriten 103: $a$(4)
\figwritee 104: $h$(2)
}
\centerline{\box\figBoxA}

\end{marginfigure}

\begin{solution}
   Given that the parabola is symmetrical about the $y-axis$, we know that it has been 
   plotted between $\left( \frac{-a}{2}, \frac{a}{2}\right)$ or $(-5,5)$
   
   Also, if the equation of the parabola be $y = ax^2$, where $a$ is an as yet unknown constant, 
   then given that $y(5) = 6 = a\cdot(5^2)$, we get $a = \frac{6}{25}$
   
   This gives us the equation of the parabola - which is $y = \dfrac{6}{25}x^2$
   
   The required area $A$ of region $R$ is therefore
   \begin{align}
   	  A &= Area(rectangle) - Area(\textit{under the parabola}) \\
   	    &= 10\times6 - \int_{-5}^5 \dfrac{6}{25}x^2 \ud x \\
   	    &= 60 -  \dfrac{6}{25}\left[ \dfrac{x^3}{3}\right]_{-5}^5 \\
   	    &= 40\text{ }cm^2
   \end{align}
\end{solution}
