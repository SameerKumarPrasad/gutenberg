% This is an empty shell file placed for you by the 'examiner' script.
% You can now fill in the TeX for your question here.

% Now, down to brasstacks. ** Writing good solutions is an Art **. 
% Eventually, you will find your own style. But here are some thoughts 
% to get you started: 
%
%   1. Write the solution as if you are writing it for your favorite
%      14-17 year old to help him/her understand. Could be your nephew, 
%      your niece, a cousin perhaps or probably even you when you 
%      were that age. Just write for them.
%
%   2. Use margin-notes to "talk" to students about the critical insights
%      in the question. The tone can be - in fact, should be - informal
%
%   3. Don't shy away from creating margin-figures you think will help
%      students understand. Yes, it is a little more work per question. 
%      But the question & solution will be written only once. Make that
%      attempt at writing a solution count.
%
%   4. At the same time, do not be too verbose. A long solution can
%      - at first sight - make the student think, "God, that is a lot to know".
%      Our aim is not to scare students. Rather, our aim should be to 
%      create many "Aha!" moments everyday in classrooms around the world
% 
%   5. Ensure that there are *no spelling mistakes anywhere*. We are an 
%      education company. Bad spellings suggest that we ourselves 
%      don't have any education. And, use American spellings

\question[5] The National Bank of the Shire has a 3-year investment scheme 
for its residents whereby they get 4\% interest compounded annually in the 
first year and - if they do not make a withdrawl at the end of the first year - 
4.5\% interest compounded annually in the next 2 years. However, if they do make
a withdrawl, then they get the same 4\% for the next 2 years too

Frodo Baggins invests in this scheme. But at the end of the first year, 
he needed cash in hand and was forced to withdraw 20\% of his original principal amount.
How much did he lose in the way of interest at the end of the 3-year period because
of the withdrawl?

\ifprintanswers
  % stuff to be shown only in the answer key - like explanatory margin figures
\fi 

\begin{solution}[\fullpage]
	Had Frodo \textit{not} withdrawn the amount at the end of the first year, he would have
	 got - at the end of the 3 years,
	\begin{align}
		P_1 &= P_0\cdot\left(1+\dfrac{4}{100}\right)\cdot\left(1+\dfrac{4.5}{100}\right)^{2} \\
		    &= 1.136\cdot P_0
	\end{align}
	
	Which means an interest payment of
	\begin{align}
		I_1 &= 1.136 P_0 - P_0 = 0.136 P_0
	\end{align}
	
	But he did. And so, he would now get at the end of the third year
	\begin{align}
		P_2 &= \overbrace{(1.04-0.2)\cdot P_0}^{\texttt{new start amount}}\cdot
		\underbrace{\left( 1 + \dfrac{4}{100}\right)^2}_{\texttt{still getting 4\%}} \\
		&= 0.9085\cdot P_0
	\end{align}
	
	Which means, that over the 3-years, Frodo earned interest equal to
	\begin{align}
		I_2 &= \overbrace{(1.04 - 1)\cdot P_0}^{\texttt{first year}} + 
		       \overbrace{(0.9085 - 0.84)\cdot P_0}^{\texttt{Years 2 \& 3}} \\
		    &= 	0.1085\cdot P_0
	\end{align}
	
	Hence, Frodo lost = $(0.136 - 0.1085)\cdot P_0 = 0.0275\cdot P_0$ - or 2.75\% - as interest 
	payment because of the withdrawl
	
\end{solution}
