% This is an empty shell file placed for you by the 'examiner' script.
% You can now fill in the TeX for your question here.

% Now, down to brasstacks. ** Writing good solutions is an Art **. 
% Eventually, you will find your own style. But here are some thoughts 
% to get you started: 
%
%   1. Write the solution as if you are writing it for your favorite
%      14-17 year old to help him/her understand. Could be your nephew, 
%      your niece, a cousin perhaps or probably even you when you 
%      were that age. Just write for them.
%
%   2. Use margin-notes to "talk" to students about the critical insights
%      in the question. The tone can be - in fact, should be - informal
%
%   3. Don't shy away from creating margin-figures you think will help
%      students understand. Yes, it is a little more work per question. 
%      But the question & solution will be written only once. Make that
%      attempt at writing a solution count.
%
%   4. At the same time, do not be too verbose. A long solution can
%      - at first sight - make the student think, "God, that is a lot to know".
%      Our aim is not to scare students. Rather, our aim should be to 
%      create many "Aha!" moments everyday in classrooms around the world
% 
%   5. Ensure that there are *no spelling mistakes anywhere*. We are an 
%      education company. Bad spellings suggest that we ourselves 
%      don't have any education. Also, use American spellings by default
% 
%   6. If a question has multiple parts, then first delete lines 40-41
%   7. If a question does not have parts, then first delete lines 43-69

\question[3] Using second derivatives, find the $x$ at which the function 
$y = \dfrac{x}{\ln x}$ has its extreme value. Also, state whether the extreme 
value is a maxima or a minima? 

\insertQR{QRC}

\ifprintanswers
  % stuff to be shown only in the answer key - like explanatory margin figures
\fi 

\begin{solution}[\fullpage]
	\begin{align}
	   y &= \dfrac{x}{\ln x} \\
	   \Rightarrow \dfrac{\ud y}{\ud x} &= \dfrac{1}{\ln x}\dfrac{\ud}{\ud x}x 
	             + x\dfrac{\ud}{\ud x}\ln x \\
	     &= \dfrac{1}{\ln x} - \dfrac{1}{(\ln x)^2} \\
	     &= \dfrac{1}{(\ln x)^2}\cdot(\ln x - 1) \\
	     \Rightarrow \text{ an extrema when } \ln x &= 1 \text{ or } x = e
	\end{align}
	
	To know whether $x=e$ is a maxima or a minima, we need to find the second derivative
	
	\begin{align}
	   \dfrac{\ud y}{\ud x} &= \dfrac{1}{\ln x} - \dfrac{1}{(\ln x)^2} \\
	   \Rightarrow \dfrac{\ud^2 y}{\ud x^2} &= \dfrac{-1}{(\ln x)^2}\cdot\dfrac{1}{x}
	                                        + \dfrac{2}{(\ln x)^3}\cdot\dfrac{1}{x} \\
	               &= \dfrac{1}{x\cdot(\ln x)^2}\left[ \dfrac{2}{\ln x} - 1\right] \\
	               &= \dfrac{1}{e} \text{ when } x = e \\
	               \Rightarrow \text{ minima at } x = e \text{ with } y_{min} = e
	\end{align}
\end{solution}
