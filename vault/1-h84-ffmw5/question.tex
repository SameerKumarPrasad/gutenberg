% This is an empty shell file placed for you by the 'examiner' script.
% You can now fill in the TeX for your question here.

% Now, down to brasstacks. ** Writing good solutions is an Art **. 
% Eventually, you will find your own style. But here are some thoughts 
% to get you started: 
%
%   1. Write to be understood - but be crisp. Your own solution should not take 
%      more space than you will give to the student. Hence, if you take more than 
%      a half-page to write a solution, then give the student a full-page and so on...
%
%   2. Use margin-notes to "talk" to students about the critical insights
%      in the question. The tone can be - in fact, should be - informal
%
%   3. Don't shy away from creating margin-figures you think will help
%      students understand. Yes, it is a little more work per question. 
%      But the question & solution will be written only once. Make that
%      attempt at writing a solution count.
%      
%      3b. Use bc_to_fig.tex. Its an easier way to generate plots & graphs 
% 
%   4. Ensure that there are *no spelling mistakes anywhere*. We are an 
%      education company. Bad spellings suggest that we ourselves 
%      don't have any education. Also, use American spellings by default
% 
%   5. If a question has multiple parts, then first delete lines 40-41
%   6. If a question does not have parts, then first delete lines 43-69
%   
%   7. Create versions of the question when possible. Use commands defined in 
%      tufte-tweaks.sty to do so. Its easier than you think

%\printrubric
%\noprintanswers
%\setcounter{rolldice}{0}

\ifnumequal{\value{rolldice}}{0}{
  % variables 
  \renewcommand{\vbone}{2}
  \renewcommand{\vbtwo}{-3}
  \renewcommand{\vbthree}{4}
  \renewcommand{\vbfour}{2}
  \renewcommand{\vbfive}{4}
}{
  \ifnumequal{\value{rolldice}}{1}{
    % variables 
    \renewcommand{\vbone}{3}
    \renewcommand{\vbtwo}{4}
    \renewcommand{\vbthree}{5}
    \renewcommand{\vbfour}{7}
    \renewcommand{\vbfive}{3}
  }{
    \ifnumequal{\value{rolldice}}{2}{
      % variables 
      \renewcommand{\vbone}{-1}
      \renewcommand{\vbtwo}{-1}
      \renewcommand{\vbthree}{2}
      \renewcommand{\vbfour}{5}
      \renewcommand{\vbfive}{4}
    }{
      % variables 
      \renewcommand{\vbone}{0}
      \renewcommand{\vbtwo}{-1}
      \renewcommand{\vbthree}{2}
      \renewcommand{\vbfour}{4}
      \renewcommand{\vbfive}{5}
     }
  }
}

\SUBTRACT\vbfour\vbtwo\aa
\SUBTRACT\vbthree\vbone\ab
\FRACTIONSIMPLIFY\aa\ab\ma\mb
\FRACDIV{-1}{1}\ma\mb\pa\pb
\MULTIPLY{-\vbfive}\pa\c

\question[4] Find the equation of the line that has $x$-intercept of $\vbfive$ units and is perpendicular
to the line joining points $(\vbone, \vbtwo)$ and $(\vbthree, \vbfour)$

\insertQR[-5pt]{QRC}

\watchout

\ifprintanswers
\fi 

\begin{solution}[\halfpage]
	The slope of the line joining $(\vbone, \vbtwo)$ and $(\vbthree, \vbfour)$ is,
	\begin{align}
		m &= \dfrac{\vbfour-\vbtwo}{\vbthree-\vbone} = \WRITEFRAC\ma\mb
	\end{align}
	And so, the slope of the required line - which is perpendicular to the above line - is 
	\begin{align}
		m\cdot m_{\perp} &= -1 \Rightarrow m_{\perp} = \WRITEFRAC\pa\pb
	\end{align}
  Now, if the $x$-intercept is $\vbfive$ units, then it means the required line passes through $(\vbfive,0)$

  In which case,
  \begin{align}
    \dfrac{y - 0}{x - \vbfive} &= \WRITEFRAC\pa\pb \\
    \Rightarrow \pb y &= \pa x \gsign\c
  \end{align}
\end{solution}

\ifprintrubric
  \begin{table}
  	\begin{tabular}{ p{5cm}p{5cm} }
  		\toprule % in brief (4-6 words), what should a grader be looking for for insights & formulations
  		  \sc{\textcolor{blue}{Insight}} & \sc{\textcolor{blue}{Formulation}} \\ 
  		\midrule % ***** Insights & formulations ******
        Inferred the slope of the perpendicular & \\ 
  		\toprule % final numerical answers for the various versions
        \sc{\textcolor{blue}{If question has $\ldots$}} & \sc{\textcolor{blue}{Final answer}} \\
  		\midrule % ***** Numerical answers (below) **********
        $(2,-3)$ and $(4,2)$ & $2x + 5y = 8$ \\
        $(3,4)$ and $(5,7)$ & $2x + 3y = 6$ \\
        $(-1,-1)$ and $(2,5)$ & $ x + 2y = 4$\\
        $(0,-1)$ and $(2,4)$ & $2x + 5y = 10$\\
  		\bottomrule
  	\end{tabular}
  \end{table}
\fi
