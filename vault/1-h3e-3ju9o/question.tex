% This is an empty shell file placed for you by the 'examiner' script.
% You can now fill in the TeX for your question here.

% Now, down to brasstacks. ** Writing good solutions is an Art **. 
% Eventually, you will find your own style. But here are some thoughts 
% to get you started: 
%
%   1. Write the solution as if you are writing it for your favorite
%      14-17 year old to help him/her understand. Could be your nephew, 
%      your niece, a cousin perhaps or probably even you when you 
%      were that age. Just write for them.
%
%   2. Use margin-notes to "talk" to students about the critical insights
%      in the question. The tone can be - in fact, should be - informal
%
%   3. Don't shy away from creating margin-figures you think will help
%      students understand. Yes, it is a little more work per question. 
%      But the question & solution will be written only once. Make that
%      attempt at writing a solution count.
%
%   4. At the same time, do not be too verbose. A long solution can
%      - at first sight - make the student think, "God, that is a lot to know".
%      Our aim is not to scare students. Rather, our aim should be to 
%      create many "Aha!" moments everyday in classrooms around the world
% 
%   5. Ensure that there are *no spelling mistakes anywhere*. We are an 
%      education company. Bad spellings suggest that we ourselves 
%      don't have any education. Also, use American spellings by default
% 
%   6. If a question has multiple parts, then first delete lines 40-41
%   7. If a question does not have parts, then first delete lines 43-69

\question[3] There are three boxes: box X has 10 bulbs of which 4 are defective, 
box Y has 6 bulbs of which 1 is defective and box Z has 8 bulbs of which 3 are defective. 
A box is chosen at random and a bulb drawn from it. If the bulb is \textit{not} defective,
then what is the probability that it came from box Z?

\insertQR{QRC}

\ifprintanswers
  % stuff to be shown only in the answer key - like explanatory margin figures
  \begin{table}[ht]
    \begin{tabular}{cccc}
      \toprule
      D = bulb is defective & X & Y & Z \\
      \midrule
      $P(D\,\vert\, box)$ & $\frac{4}{10}$ & $\frac{1}{6}$ & $\frac{3}{8}$ \\
      $P(\overline{D}\,\vert\, box)$ & $\frac{6}{10}$ & $\frac{5}{6}$ & $\frac{5}{8}$ \\
      \bottomrule
    \end{tabular}
  \end{table}
\fi 

\begin{solution}[\fullpage]
   The probability we are seeking is $\bayesp{Z}{\textoverline{D}}$, where $\overline{D}$ 
   is the event that a bulb is \textit{not} defective
   
   \begin{align}
      \bayesp{Z}{\textoverline{D}} &= \dfrac{\bayesf{\textoverline{D}}{Z}}
      {\bayesf{\textoverline{D}}{X} + \bayesf{\textoverline{D}}{Y} + \bayesf{\textoverline{D}}{Z}} \\
      &= \dfrac{\frac{5}{8}\cdot\frac{1}{3}}{\frac{1}{3}\cdot\left(\frac{5}{8} +
      \frac{3}{5} + \frac{5}{6}\right)} \\
      &= \dfrac{150}{494} = \dfrac{75}{247}
   \end{align}
   
\end{solution}
