% This is an empty shell file placed for you by the 'examiner' script.
% You can now fill in the TeX for your question here.

% Now, down to brasstacks. ** Writing good solutions is an Art **. 
% Eventually, you will find your own style. But here are some thoughts 
% to get you started: 
%
%   1. Write to be understood - but be crisp. Your own solution should not take 
%      more space than you will give to the student. Hence, if you take more than 
%      a half-page to write a solution, then give the student a full-page and so on...
%
%   2. Use margin-notes to "talk" to students about the critical insights
%      in the question. The tone can be - in fact, should be - informal
%
%   3. Don't shy away from creating margin-figures you think will help
%      students understand. Yes, it is a little more work per question. 
%      But the question & solution will be written only once. Make that
%      attempt at writing a solution count.
%      
%      3b. Use bc_to_fig.tex. Its an easier way to generate plots & graphs 
% 
%   4. Ensure that there are *no spelling mistakes anywhere*. We are an 
%      education company. Bad spellings suggest that we ourselves 
%      don't have any education. Also, use American spellings by default
% 
%   5. If a question has multiple parts, then first delete lines 40-41
%   6. If a question does not have parts, then first delete lines 43-69
%   
%   7. Create versions of the question when possible. Use commands defined in 
%      tufte-tweaks.sty to do so. Its easier than you think

% \noprintanswers
% \setcounter{rolldice}{0}

\ifnumequal{\value{rolldice}}{0}{
  % variables 
  \renewcommand{\vbone}{5}
  \renewcommand{\vbtwo}{4}
  \renewcommand{\vbthree}{15}
  \renewcommand{\vbfour}{4}
  \renewcommand{\vbfive}{6}
  \renewcommand{\vbsix}{4}
  \renewcommand{\vbseven}{1}
  \renewcommand{\vbeight}{}
  \renewcommand{\vbnine}{}
  \renewcommand{\vbten}{}
}{
  \ifnumequal{\value{rolldice}}{1}{
    % variables 
    \renewcommand{\vbone}{5}
    \renewcommand{\vbtwo}{3}
    \renewcommand{\vbthree}{7}
    \renewcommand{\vbfour}{3}
    \renewcommand{\vbfive}{3}
    \renewcommand{\vbsix}{1}
    \renewcommand{\vbseven}{}
    \renewcommand{\vbeight}{}
    \renewcommand{\vbnine}{}
    \renewcommand{\vbten}{}
  }{
    \ifnumequal{\value{rolldice}}{2}{
      % variables 
      \renewcommand{\vbone}{4}
      \renewcommand{\vbtwo}{4}
      \renewcommand{\vbthree}{15}
      \renewcommand{\vbfour}{4}
      \renewcommand{\vbfive}{6}
      \renewcommand{\vbsix}{4}
      \renewcommand{\vbseven}{1}
      \renewcommand{\vbeight}{}
      \renewcommand{\vbnine}{}
      \renewcommand{\vbten}{}
    }{
      % variables 
      \renewcommand{\vbone}{4}
      \renewcommand{\vbtwo}{3}
      \renewcommand{\vbthree}{7}
      \renewcommand{\vbfour}{3}
      \renewcommand{\vbfive}{3}
      \renewcommand{\vbsix}{1}
      \renewcommand{\vbseven}{}
      \renewcommand{\vbeight}{}
      \renewcommand{\vbnine}{}
      \renewcommand{\vbten}{}
    }
  }
}

\gcalcexpr[0]{\total}{\vbone * \vbthree}

\question[3] There are $\vbone$ multiple-choice type questions in an assignment. 
Each question presents $\vbtwo$ possible options. How many unique answer combinations are possible for the assignment, provided each question has at
least one correct option.

\insertQR{QRC}

\watchout

\ifprintanswers
  % stuff to be shown only in the answer key - like explanatory margin figures
  \begin{marginfigure}
    \figinit{pt}
      \figpt 100:(0,0)
      \figpt 101:(0,0)
    \figdrawbegin{}
      \figdrawline [100,101]
    \figdrawend
    \figvisu{\figBoxA}{}{%
    }
    \centerline{\box\figBoxA}
  \end{marginfigure}
\fi 

\begin{solution}[\mcq]
  The total number of combinations can be found by multiplying the number of
  correct answer options for a particular question, with the number of
  questions. Since each question can have at least one correct option we
  would count the combinations using cases as follows:
  
  \begin{align}
    \text{Exactly 1 correct option}  
    	\Rightarrow& \encr{\vbtwo}{1} = \vbfour \\
    \text{Exactly 2 correct options} 
    	\Rightarrow& \encr{\vbtwo}{2} = \vbfive \\
    \ifnumequal{\value{rolldice}}{0} {
      \text{Exactly 3 correct options} 
        \Rightarrow& \encr{\vbtwo}{3} = \vbsix \\
      \text{All 4 correct options}     
        \Rightarrow& \encr{\vbtwo}{4} = \vbseven
    } {
      \ifnumequal{\value{rolldice}}{1} {
        \text{All 3 correct options}     
          \Rightarrow& \encr{\vbtwo}{3} = \vbsix
      } {
        \ifnumequal{\value{rolldice}}{2} {
	      \text{Exactly 3 correct options}
	        \Rightarrow& \encr{\vbtwo}{3} = \vbsix \\
    	  \text{All 4 correct options}     
    	    \Rightarrow& \encr{\vbtwo}{4} = \vbseven
        } {
          \text{All 3 correct options}     
            \Rightarrow& \encr{\vbtwo}{3} = \vbsix
        }
      }
    }
  \end{align}
  \begin{align}
  	\text{Total combinations} &= \vbthree\times\vbone \\
  						      &= \total
  \end{align}
\end{solution}
