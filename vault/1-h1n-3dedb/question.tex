% This is an empty shell file placed for you by the 'examiner' script.
% You can now fill in the TeX for your question here.

% Now, down to brasstacks. ** Writing good solutions is an Art **. 
% Eventually, you will find your own style. But here are some thoughts 
% to get you started: 
%
%   1. Write the solution as if you are writing it for your favorite
%      14-17 year old to help him/her understand. Could be your nephew, 
%      your niece, a cousin perhaps or probably even you when you 
%      were that age. Just write for them.
%
%   2. Use margin-notes to "talk" to students about the critical insights
%      in the question. The tone can be - in fact, should be - informal
%
%   3. Don't shy away from creating margin-figures you think will help
%      students understand. Yes, it is a little more work per question. 
%      But the question & solution will be written only once. Make that
%      attempt at writing a solution count.
%
%   4. At the same time, do not be too verbose. A long solution can
%      - at first sight - make the student think, "God, that is a lot to know".
%      Our aim is not to scare students. Rather, our aim should be to 
%      create many "Aha!" moments everyday in classrooms around the world
% 
%   5. Ensure that there are *no spelling mistakes anywhere*. We are an 
%      education company. Bad spellings suggest that we ourselves 
%      don't have any education. Also, use American spellings by default
% 
%   6. If a question has multiple parts, then first delete lines 40-41
%   7. If a question does not have parts, then first delete lines 43-69

\question Sand is pouring from a pipe at the rate of \SI{12}{\centi\metre^3\per\second}.
The falling sand forms a cone on the ground in such a way that the height of the
cone is always one-sixth the radius of the base. How fast is the height of the
sand cone increasing when its height is \SI{4}{\centi\metre}?

\insertQR{}

\ifprintanswers
  % stuff to be shown only in the answer key - like explanatory margin figures
\fi 

\begin{solution}
  If $V(t)$ be the volume of the sand cone at any given time, then
  \begin{align}
     V(t) &= \dfrac{1}{3}\pi R(t)^2\cdot h(t) \\
          &= \dfrac{1}{3}\pi (6h(t))^2\cdot h(t) \\
          &= 12\pi\cdot h(t)^3 \\
    \Rightarrow \dfrac{\ud V(t)}{\ud t} &= 36\pi \times h(t)^2\times \dfrac{\ud h(t)}{\ud t} \\
    \Rightarrow \SI{12}{\centi\meter^3\per\second} &= 36\pi\times(\SI{4}{\centi\meter})^2\times
    \left[\dfrac{\ud h(t)}{\ud t}\right]_{h=4} \\
    \Rightarrow \left[\dfrac{\ud h(t)}{\ud t}\right]_{h=4} &= 
         \dfrac{1}{48\pi}\si{\centi\meter\per\second}
  \end{align}
\end{solution}
