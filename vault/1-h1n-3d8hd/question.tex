% This is an empty shell file placed for you by the 'examiner' script.
% You can now fill in the TeX for your question here.

% Now, down to brasstacks. ** Writing good solutions is an Art **. 
% Eventually, you will find your own style. But here are some thoughts 
% to get you started: 
%
%   1. Write the solution as if you are writing it for your favorite
%      14-17 year old to help him/her understand. Could be your nephew, 
%      your niece, a cousin perhaps or probably even you when you 
%      were that age. Just write for them.
%
%   2. Use margin-notes to "talk" to students about the critical insights
%      in the question. The tone can be - in fact, should be - informal
%
%   3. Don't shy away from creating margin-figures you think will help
%      students understand. Yes, it is a little more work per question. 
%      But the question & solution will be written only once. Make that
%      attempt at writing a solution count.
%
%   4. At the same time, do not be too verbose. A long solution can
%      - at first sight - make the student think, "God, that is a lot to know".
%      Our aim is not to scare students. Rather, our aim should be to 
%      create many "Aha!" moments everyday in classrooms around the world
% 
%   5. Ensure that there are *no spelling mistakes anywhere*. We are an 
%      education company. Bad spellings suggest that we ourselves 
%      don't have any education. Also, use American spellings by default
% 
%   6. If a question has multiple parts, then first delete lines 40-41
%   7. If a question does not have parts, then first delete lines 43-69

\question At what angles do the parabolas $y=x^2$ and $y=x^2$ - \asif - intersect?

\calculator{ \tan^{-1}2 = \ang{63.44}, \tan^{-1}\frac{1}{2} = \ang{26.57}}

\insertQR{}

\ifprintanswers
  % stuff to be shown only in the answer key - like explanatory margin figures
\fi 
\begin{marginfigure}
% 1. Definition of characteristic points
\figinit{pt}
\def\Xmin{0}
\def\Ymin{0}
\def\Xmax{50.00000}
\def\Ymax{69.99999}
\def\Xori{0}
\def\Yori{0}
\figpt0:(\Xori,\Yori)
\figpt 100: (21,17)
% 2. Creation of the graphical file
\figdrawbegin{}
\def\Xmaxx{\Xmax} % To customize the position
\def\Ymaxx{\Ymax} % of the arrow-heads of the axes.
\figset arrowhead(length=4, fillmode=yes) % styling the arrowheads
\figdrawaxes 0(\Xmin, \Xmaxx, \Ymin, \Ymaxx)
\figdrawlineC(
0 0,
1.72413 .08323,
3.44827 .33293,
5.17241 .74910,
6.89655 1.33174,
8.62068 2.08085,
10.34482 2.99643,
12.06896 4.07847,
13.79310 5.32699,
15.51724 6.74197,
17.24137 8.32342,
18.96551 10.07134,
20.68965 11.98573,
22.41379 14.06658,
24.13793 16.31391,
25.86206 18.72770,
27.58620 21.30796,
29.31034 24.05469,
31.03448 26.96789,
32.75862 30.04756,
34.48275 33.29369,
36.20689 36.70630,
37.93103 40.28537,
39.65517 44.03091,
41.37931 47.94292,
43.10344 52.02140,
44.82758 56.26634,
46.55172 60.67776,
48.27586 65.25564,
49.99999 69.99999
)
\figdrawlineC(
0 0,
1.72413 4.59572,
3.44827 6.49933,
5.17241 7.96002,
6.89655 9.19145,
8.62068 10.27635,
10.34482 11.25718,
12.06896 12.15914,
13.79310 12.99867,
15.51724 13.78717,
17.24137 14.53295,
18.96551 15.24229,
20.68965 15.92005,
22.41379 16.57012,
24.13793 17.19562,
25.86206 17.79916,
27.58620 18.38290,
29.31034 18.94866,
31.03448 19.49801,
32.75862 20.03230,
34.48275 20.55270,
36.20689 21.06025,
37.93103 21.55586,
39.65517 22.04032,
41.37931 22.51436,
43.10344 22.97862,
44.82758 23.43369,
46.55172 23.88008,
48.27586 24.31829,
49.99999 24.74873
)
\figdrawlineC(
0 0,
1.72413 -4.59572,
3.44827 -6.49933,
5.17241 -7.96002,
6.89655 -9.19145,
8.62068 -10.27635,
10.34482 -11.25718,
12.06896 -12.15914,
13.79310 -12.99867,
15.51724 -13.78717,
17.24137 -14.53295,
18.96551 -15.24229,
20.68965 -15.92005,
22.41379 -16.57012,
24.13793 -17.19562,
25.86206 -17.79916,
27.58620 -18.38290,
29.31034 -18.94866,
31.03448 -19.49801,
32.75862 -20.03230,
34.48275 -20.55270,
36.20689 -21.06025,
37.93103 -21.55586,
39.65517 -22.04032,
41.37931 -22.51436,
43.10344 -22.97862,
44.82758 -23.43369,
46.55172 -23.88008,
48.27586 -24.31829,
49.99999 -24.74873
)
\figdrawend
% 3. Writing text on the figure
\figvisu{\figBoxA}{}{%
\figptsaxes 1:0(\Xmin, \Xmaxx, \Ymin, \Ymaxx)
% Points 1 and 2 are the end points of the arrows
\figwritee 1:(5pt)     \figwriten 2:(5pt)
\figptsaxes 1:0(\Xmin, \Xmax, \Ymin, \Ymax)
\figwriten 100:$A$(2)
}
\centerline{\box\figBoxA}


\end{marginfigure}

\begin{solution}
	The parabolas intersect at points where, 
	\begin{align}
		y = x^2 &= (x^2)^2 = x^4 \\
		\Rightarrow x^4-x &= 0 \text{ or } x\cdot(x^3-1) = 0 \\
		\Rightarrow x &= 0,1 \\
		\Rightarrow y &= 0^2, 1^2 \text{ or } (x,y) = (0,0) \text{ and } (1,1)
	\end{align}
	
	The angles at which the parabolas intersect are the angles at which 
	the \textit{tangents} of each at the intersection points intersect
	
	Hence, at $(x,y) = (0,0)$
	\begin{align}
		\left[\dydx\right]_{x=0} = (2x)_{x=0} = 0 &\Rightarrow \psi_1 = \tan^{-1}0 = \ang{0} \\
		\left[\dydx\right]_{x=0} = (\dfrac{1}{2\sqrt{x}})_{x=0} = \infty &\Rightarrow \psi_2 
		= \tan^{-1}\infty = \ang{90} \\
		\therefore\text{ Angle of intersection at origin } &= \psi_2-\psi_1 = \ang{90}
	\end{align}
	
	Similarly, at $(x,y) = (1,1)$
	\begin{align}
		\left[\dydx\right]_{x=1} = (2x)_{x=1} = 2 &\Rightarrow \psi_1 = \tan^{-1}2 = \ang{63.44} \\
		\left[\dydx\right]_{x=1} = (\dfrac{1}{2\sqrt{x}})_{x=1} = \frac{1}{2} &\Rightarrow \psi_2 
		= \tan^{-1}\frac{1}{2} = \ang{26.57} \\
		\therefore\text{ Angle of intersection at }(1,1) &= \psi_1-\psi_2 = \ang{36.87}
	\end{align}
\end{solution}
