% This is an empty shell file placed for you by the 'examiner' script.
% You can now fill in the TeX for your question here.

% Now, down to brasstacks. ** Writing good solutions is an Art **. 
% Eventually, you will find your own style. But here are some thoughts 
% to get you started: 
%
%   1. Write the solution as if you are writing it for your favorite
%      14-17 year old to help him/her understand. Could be your nephew, 
%      your niece, a cousin perhaps or probably even you when you 
%      were that age. Just write for them.
%
%   2. Use margin-notes to "talk" to students about the critical insights
%      in the question. The tone can be - in fact, should be - informal
%
%   3. Don't shy away from creating margin-figures you think will help
%      students understand. Yes, it is a little more work per question. 
%      But the question & solution will be written only once. Make that
%      attempt at writing a solution count.
%
%   4. At the same time, do not be too verbose. A long solution can
%      - at first sight - make the student think, "God, that is a lot to know".
%      Our aim is not to scare students. Rather, our aim should be to 
%      create many "Aha!" moments everyday in classrooms around the world
% 
%   5. Ensure that there are *no spelling mistakes anywhere*. We are an 
%      education company. Bad spellings suggest that we ourselves 
%      don't have any education. And, use American spellings

\question[3]   If the difference between the simple interest earned over one year
and the compound interest - payed semi-annually - over the same year is \texteuro 180,
then what is the initial investment amount. The annualized interest rate in both cases is 10\% 

\insertQR{QRC}

\ifprintanswers
  % stuff to be shown only in the answer key - like explanatory margin figures
\fi 

\begin{solution}[\fullpage]
	If $I_1$ be the interest earned from \emph{compounding} in one year, $P_0$ the 
	initial investment amount and $P_1$ the amount at the end of the first year, then
	
	\begin{align}
		I_1 &= P_1 - P_0 \\ 
		    &= P_0\cdot\left[ 1+\dfrac{R}{N}\right]^N - P_0 \\
		    &= P_0\cdot\left[ (1+\dfrac{R}{N})^N - 1 \right]
	\end{align}
	where $N$ = number of compounding periods = 2
	
	Over the same period, the \emph{simple} interest earned - $I_2$ - is equal to
	\begin{align}
		I_2 &= P_0\cdot R\cdot T(=1)
    \end{align}
    
    And therefore, if $I_1-I_2 = \text{\texteuro 180}$, then 
    \begin{align}
    	\text{\texteuro 180} &= P_0\cdot\left[ (1+\dfrac{0.1}{2})^2 - 1 - 0.1\right] \\
    	                     &= 0.0025\cdot P_0 \\
    	\Rightarrow P_0 &= \text{\texteuro 72,000}
    \end{align}
\end{solution}
