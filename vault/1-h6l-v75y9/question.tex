% This is an empty shell file placed for you by the 'examiner' script.
% You can now fill in the TeX for your question here.

% Now, down to brasstacks. ** Writing good solutions is an Art **. 
% Eventually, you will find your own style. But here are some thoughts 
% to get you started: 
%
%   1. Write the solution as if you are writing it for your favorite
%      14-17 year old to help him/her understand. Could be your nephew, 
%      your niece, a cousin perhaps or probably even you when you 
%      were that age. Just write for them.
%
%   2. Use margin-notes to "talk" to students about the critical insights
%      in the question. The tone can be - in fact, should be - informal
%
%   3. Don't shy away from creating margin-figures you think will help
%      students understand. Yes, it is a little more work per question. 
%      But the question & solution will be written only once. Make that
%      attempt at writing a solution count.
%
%   4. At the same time, do not be too verbose. A long solution can
%      - at first sight - make the student think, "God, that is a lot to know".
%      Our aim is not to scare students. Rather, our aim should be to 
%      create many "Aha!" moments everyday in classrooms around the world
% 
%   5. Ensure that there are *no spelling mistakes anywhere*. We are an 
%      education company. Bad spellings suggest that we ourselves 
%      don't have any education. Also, use American spellings by default
% 
%   6. If a question has multiple parts, then first delete lines 40-41
%   7. If a question does not have parts, then first delete lines 43-69

\question There are an \textit{even} number of terms in a geometric progression. The sum 
of all terms of the progression is thrice as large as the sum of its odd terms. What is 
the common ratio of the progression? 

\insertQR{}

\ifprintanswers
\fi 

\begin{solution}
	If the number of terms - $n$ - is even, then $n = 2k$ for some $k$. Moreover, because
	there are an even number of terms, $\frac{n}{2} = k$ terms would be in odd positions
	and the remaining $k$ terms in even positions
	
	Now, if $s = \lbrace a, ar, ar^2, ar^3 \ldots \rbrace$ be the original sequence, then 
	the sequence of odd terms is $s_{\text{odd}} = \lbrace a, ar^2, ar^4 \ldots  \rbrace$. Notice
	that $s_{\text{odd}}$ has the same first term but a different common ratio - $r^2$. And so, if 
	
	\begin{align}
		%\dfrac{S}{S_{\text{odd}} &= \dfrac{\eSumOfGP{r}{n}}{\eSumOfGP{r}{n}}
		\dfrac{S}{S_{\text{odd}}} &= \dfrac{\eSumOfGP{r}{n}}
		{\dfrac{a\cdot(r^{2\cdot\frac{n}{2}} - 1)}{r^2-1}} = 3 \\
		\text{ then, } \dfrac{r^2 - 1}{r - 1} &= 3 \text{ or } r = 2
	\end{align}
\end{solution}
