% This is an empty shell file placed for you by the 'examiner' script.
% You can now fill in the TeX for your question here.

% Now, down to brasstacks. ** Writing good solutions is an Art **. 
% Eventually, you will find your own style. But here are some thoughts 
% to get you started: 
%
%   1. Write the solution as if you are writing it for your favorite
%      14-17 year old to help him/her understand. Could be your nephew, 
%      your niece, a cousin perhaps or probably even you when you 
%      were that age. Just write for them.
%
%   2. Use margin-notes to "talk" to students about the critical insights
%      in the question. The tone can be - in fact, should be - informal
%
%   3. Don't shy away from creating margin-figures you think will help
%      students understand. Yes, it is a little more work per question. 
%      But the question & solution will be written only once. Make that
%      attempt at writing a solution count.
%
%   4. At the same time, do not be too verbose. A long solution can
%      - at first sight - make the student think, "God, that is a lot to know".
%      Our aim is not to scare students. Rather, our aim should be to 
%      create many "Aha!" moments everyday in classrooms around the world
% 
%   5. Ensure that there are *no spelling mistakes anywhere*. We are an 
%      education company. Bad spellings suggest that we ourselves 
%      don't have any education. And, use American spellings

\question[5]   Find the ratio in which the line $3x+y-9=0$ divides the line segment
\insertQR{QRC}
joining the points $(1,3)$ and $(2,7)$

\ifprintanswers
  % stuff to be shown only in the answer key - like explanatory margin figures
\fi 

\begin{solution}[\fullpage]
	Equation of the line joining $(1,3)$ and $(2,7)$ - $L_1$ - is given by 
	\begin{align}
		\dfrac{y-7}{x-2} &= \dfrac{7-3}{2-1} = 4 \\
		\Rightarrow y &= 4x - 1
	\end{align}
	The two lines intersect at a point - $C$ - for which
	\begin{align}
		y = 4x - 1 &= 9-3x \Rightarrow x = \dfrac{10}{7} \\
		\therefore y &= 4\cdot\dfrac{10}{7}-1 = \dfrac{33}{7}
	\end{align}
	Now, we are ready to calculate what is asked. We first need the length
	of the given segment and then the distance from one of its end-points to
	the point of intersection we just calculated 
	
	\begin{align}
		\texttt{$L_1$} &= \sqrt{(7-3)^2+(2-1)^2} = \sqrt{17} \\
		\texttt{AC} &= \sqrt{(\dfrac{33}{7}-3)^2+(\dfrac{10}{7}-1)^2} \\
		&= \sqrt{\dfrac{153}{7^2}} = \dfrac{3}{7}\sqrt{17} \\
		\therefore\dfrac{\texttt{AC}}{L_1} &= \dfrac{3}{7} \\
		\Rightarrow\dfrac{\texttt{BC}}{\texttt{AC}} &= \dfrac{(1-\frac{3}{7})\cdot L_1}{\frac{3}{7}\cdot L_1}\\
		&= \dfrac{4}{3}
	\end{align}
	
\end{solution}
