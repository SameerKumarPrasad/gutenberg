% This is an empty shell file placed for you by the 'examiner' script.
% You can now fill in the TeX for your question here.

% Now, down to brasstacks. ** Writing good solutions is an Art **. 
% Eventually, you will find your own style. But here are some thoughts 
% to get you started: 
%
%   1. Write to be understood - but be crisp. Your own solution should not take 
%      more space than you will give to the student. Hence, if you take more than 
%      a half-page to write a solution, then give the student a full-page and so on...
%
%   2. Use margin-notes to "talk" to students about the critical insights
%      in the question. The tone can be - in fact, should be - informal
%
%   3. Don't shy away from creating margin-figures you think will help
%      students understand. Yes, it is a little more work per question. 
%      But the question & solution will be written only once. Make that
%      attempt at writing a solution count.
%      
%      3b. Use bc_to_fig.tex. Its an easier way to generate plots & graphs 
% 
%   4. Ensure that there are *no spelling mistakes anywhere*. We are an 
%      education company. Bad spellings suggest that we ourselves 
%      don't have any education. Also, use American spellings by default
% 
%   5. If a question has multiple parts, then first delete lines 40-41
%   6. If a question does not have parts, then first delete lines 43-69
%   
%   7. Create versions of the question when possible. Use commands defined in 
%      tufte-tweaks.sty to do so. Its easier than you think

%\printrubric
%\noprintanswers
%\setcounter{rolldice}{3}

\ifnumequal{\value{rolldice}}{0}{
  % variables 
  \renewcommand{\vbone}{1} % x-avg
  \renewcommand{\vbtwo}{5} % y-avg
  \renewcommand{\vbthree}{2} % m-1
  \renewcommand{\vbfour}{3} % m-2
  \renewcommand{\vbfive}{4} % c-1
  \renewcommand{\vbsix}{2} % c-2
}{
  \ifnumequal{\value{rolldice}}{1}{
    % variables 
    \renewcommand{\vbone}{2}
    \renewcommand{\vbtwo}{5}
    \renewcommand{\vbthree}{3}
    \renewcommand{\vbfour}{4}
    \renewcommand{\vbfive}{5}
    \renewcommand{\vbsix}{3}
  }{
    \ifnumequal{\value{rolldice}}{2}{
      % variables 
      \renewcommand{\vbone}{4}
      \renewcommand{\vbtwo}{7}
      \renewcommand{\vbthree}{1}
      \renewcommand{\vbfour}{2}
      \renewcommand{\vbfive}{5}
      \renewcommand{\vbsix}{3}
    }{
      % variables 
      \renewcommand{\vbone}{3}
      \renewcommand{\vbtwo}{-5}
      \renewcommand{\vbthree}{2}
      \renewcommand{\vbfour}{3}
      \renewcommand{\vbfive}{7}
      \renewcommand{\vbsix}{9}
    }
  }
}


\MULTIPLY\vbone{2}\va
\MULTIPLY\vbtwo{2}\vb
\gcalcexpr[0]\vc{\vb - (\vbfour * \va) - \vbsix}
\SOLVELINEARSYSTEM(\vbthree, -1 ; -\vbfour, 1)(-\vbfive, \vc)(\ra,\rb)
\SUBTRACT\vbtwo\rb\n
\SUBTRACT\vbone\ra\d
\FRACTIONSIMPLIFY\n\d\an\ad
\gcalcexpr[0]\c{(\an * \vbone) - (\ad * \vbtwo)}


\question[4] The segment of a line between two other lines $L_1 = \vbthree x - y + \vbfive = 0$ and $L_2 = \vbfour x -y + \vbsix = 0$ is bisected at the point $(\vbone, \vbtwo)$. Find the equation of the line

\insertQR[-5pt]{QRC}

\watchout

\ifprintanswers
 
\fi 

\begin{solution}[\halfpage]
	If the required line intersects $L_1$ at $(a,b)$ and $L_2$ at $(c,d)$, then 
	\begin{align}
		\dfrac{a + c}{2} &= \vbone \Rightarrow c = \va - a \\
		\dfrac{b + d}{2} &= \vbtwo \Rightarrow d = \vb - b
	\end{align}
	Moreover, as $(c,d)$ lies on $L_2$ and $(a,b)$ lies on $L_1$
	\begin{align}
		L_1: \vbthree\cdot a - b &= -\vbfive \\
		L_2: \vbfour\cdot(\va - a) - (\vb - b) + \vbsix &= 0 \Rightarrow 
    -\vbfour a + b = \vc
	\end{align}
	Solving the two equations, we get 
	\begin{align}
		a &= \ra \text{ and } b = \rb
	\end{align}
	And a line that passes through both $(\vbone, \vbtwo)$ and $(\ra,\rb)$ has the equation
	\begin{align}
		\dfrac{y - \vbtwo}{x - \vbone} &= \WRITEFRAC\n\d \\ 
		\Rightarrow \ad y &= \an x \invgsign\c 
	\end{align}
\end{solution}

\ifprintrubric
  \begin{table}
  	\begin{tabular}{ p{5cm}p{5cm} }
  		\toprule % in brief (4-6 words), what should a grader be looking for for insights & formulations
  		  \sc{\textcolor{blue}{Insight}} & \sc{\textcolor{blue}{Formulation}} \\ 
  		\midrule % ***** Insights & formulations ******
        Mid-point coordinates in terms of $(a,b)$ and $(c,d)$ & \\ 
        $(a,b)$ and $(c,d)$ satisfy $L_1$ and $L_2$ respectively & \\
  		\toprule % final numerical answers for the various versions
        \sc{\textcolor{blue}{If question has $\ldots$}} & \sc{\textcolor{blue}{Final answer}} \\
  		\midrule % ***** Numerical answers (below) **********
        $(1,5)$ & $y = 3x + 2$ \\
        $(2,5)$ & $2y = 7x - 4$ \\
        $(4,7)$ & $3y = 4x + 5$\\
        $(3,-5)$ & $ 41y = 100x - 505$\\
  		\bottomrule
  	\end{tabular}
  \end{table}
\fi
