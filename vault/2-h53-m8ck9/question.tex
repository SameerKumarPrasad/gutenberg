% This is an empty shell file placed for you by the 'examiner' script.
% You can now fill in the TeX for your question here.

% Now, down to brasstacks. ** Writing good solutions is an Art **. 
% Eventually, you will find your own style. But here are some thoughts 
% to get you started: 
%
%   1. Write to be understood - but be crisp. Your own solution should not take 
%      more space than you will give to the student. Hence, if you take more than 
%      a half-page to write a solution, then give the student a full-page and so on...
%
%   2. Use margin-notes to "talk" to students about the critical insights
%      in the question. The tone can be - in fact, should be - informal
%
%   3. Don't shy away from creating margin-figures you think will help
%      students understand. Yes, it is a little more work per question. 
%      But the question & solution will be written only once. Make that
%      attempt at writing a solution count.
%      
%      3b. Use bc_to_fig.tex. Its an easier way to generate plots & graphs 
% 
%   4. Ensure that there are *no spelling mistakes anywhere*. We are an 
%      education company. Bad spellings suggest that we ourselves 
%      don't have any education. Also, use American spellings by default
% 
%   5. If a question has multiple parts, then first delete lines 40-41
%   6. If a question does not have parts, then first delete lines 43-69
%   
%   7. Create versions of the question when possible. Use commands defined in 
%      tufte-tweaks.sty to do so. Its easier than you think

% \noprintanswers
% \setcounter{rolldice}{0}
  % variables 
  \renewcommand{\vbone}{5}
  \renewcommand{\vbtwo}{-}
  \renewcommand{\vbthree}{+4}
  \renewcommand{\vbfour}{3}
  \renewcommand{\vbfive}{+4}
  \renewcommand{\vbsix}{-4}

\question  A line is such that the segment between the lines 
$\vbone x\vbtwo y\vbthree=0$ and $\vbfour x\vbfive y\vbsix=0$ is 
bisected at the point $(1,5)$. Find its equation.

\insertQR{}

\ifprintanswers
  % stuff to be shown only in the answer key - like explanatory margin figures
  \begin{marginfigure}
    \figinit{pt}
      \figpt 100:(0,0)
      \figpt 101:(0,0)
    \figdrawbegin{}
      \figdrawline [100,101]
    \figdrawend
    \figvisu{\figBoxA}{}{%
    }
    \centerline{\box\figBoxA}
  \end{marginfigure}
\fi 

\begin{solution}
  Let the points at which the line segment intersects the two given lines
  be $P_1(x_1, y_1)$ and $P_2(x_2, y_2)$ respectively. The coordinates can be
  written as:
  \begin{align}
    P_1 &= (x_1, 5x_1+4) \\
    P_2 &= (x_2, -\dfrac{3}{4}x_2+1)    
  \end{align}
  $(1,5)$ bisects $P_1$ and $P_2$ therefore,
  \begin{align}
    \dfrac{x_1+x_2}{2}                     &= 1 \\
    \dfrac{5x_1+4-\dfrac{3}{4}x_2+1}{2}    &= 5  
  \end{align}
  Use (3) and (4) to derive $(x_2, y_2)$,
  \begin{align}
    5(2-x_2)-\dfrac{3}{4}x_2+1 &= 10 \nonumber \\
                           x_2 &= \dfrac{20}{23} \nonumber \\
                           y_2 &= \dfrac{8}{23} \nonumber
  \end{align}
  We can now substitute $(1,5)$ and $(x_2, y_2)$ in the equation for 
  the line, 
  \begin{align}
    y-5 &= \dfrac{\frac{8}{23}-5}{\frac{20}{23}-1}(x-1) \nonumber \\
    3y  &= 107x - 92									\nonumber    
  \end{align}
\end{solution}
