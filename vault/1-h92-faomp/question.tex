% This is an empty shell file placed for you by the 'examiner' script.
% You can now fill in the TeX for your question here.

% Now, down to brasstacks. ** Writing good solutions is an Art **. 
% Eventually, you will find your own style. But here are some thoughts 
% to get you started: 
%
%   1. Write to be understood - but be crisp. Your own solution should not take 
%      more space than you will give to the student. Hence, if you take more than 
%      a half-page to write a solution, then give the student a full-page and so on...
%
%   2. Use margin-notes to "talk" to students about the critical insights
%      in the question. The tone can be - in fact, should be - informal
%
%   3. Don't shy away from creating margin-figures you think will help
%      students understand. Yes, it is a little more work per question. 
%      But the question & solution will be written only once. Make that
%      attempt at writing a solution count.
%      
%      3b. Use bc_to_fig.tex. Its an easier way to generate plots & graphs 
% 
%   4. Ensure that there are *no spelling mistakes anywhere*. We are an 
%      education company. Bad spellings suggest that we ourselves 
%      don't have any education. Also, use American spellings by default
% 
%   5. If a question has multiple parts, then first delete lines 40-41
%   6. If a question does not have parts, then first delete lines 43-69
%   
%   7. Create versions of the question when possible. Use commands defined in 
%      tufte-tweaks.sty to do so. Its easier than you think

% \noprintanswers
% \setcounter{rolldice}{1}

\ifnumequal{\value{rolldice}}{0}{
  % variables 
  \renewcommand{\vbone}{SUCCESS }
  \renewcommand{\vbtwo}{S } % do not begin with
  \renewcommand{\vbthree}{C } % end with
  \renewcommand{\vbfour}{3} % no-begin count 
  \renewcommand{\vbfive}{2} % end count
  \renewcommand{\vbsix}{7} % word-length
  \renewcommand{\vbseven}{60}
  \renewcommand{\vbeight}{}
  \renewcommand{\vbnine}{}
  \renewcommand{\vbten}{}
}{
  \ifnumequal{\value{rolldice}}{1}{
    % variables 
    \renewcommand{\vbone}{ALBATROSS }
    \renewcommand{\vbtwo}{A }
    \renewcommand{\vbthree}{S }
    \renewcommand{\vbfour}{2}
    \renewcommand{\vbfive}{2}
    \renewcommand{\vbsix}{9}
    \renewcommand{\vbseven}{15,120}
    \renewcommand{\vbeight}{}
    \renewcommand{\vbnine}{}
    \renewcommand{\vbten}{}
  }{
    \ifnumequal{\value{rolldice}}{2}{
      % variables 
      \renewcommand{\vbone}{DAMASCUS }
      \renewcommand{\vbtwo}{S }
      \renewcommand{\vbthree}{A }
      \renewcommand{\vbfour}{2}
      \renewcommand{\vbfive}{2}
      \renewcommand{\vbsix}{8}
      \renewcommand{\vbseven}{1800}
      \renewcommand{\vbeight}{}
      \renewcommand{\vbnine}{}
      \renewcommand{\vbten}{}
    }{
      % variables 
      \renewcommand{\vbone}{PROPERTY }
      \renewcommand{\vbtwo}{P }
      \renewcommand{\vbthree}{R }
      \renewcommand{\vbfour}{2}
      \renewcommand{\vbfive}{2}
      \renewcommand{\vbsix}{8}
      \renewcommand{\vbseven}{1800}
      \renewcommand{\vbeight}{}
      \renewcommand{\vbnine}{}
      \renewcommand{\vbten}{}
    }
  }
}

\gcalcexpr[0]\thfirst{\vbsix - 1}
\gcalcexpr[0]\nfirst{\vbsix - 1 - \vbfour}
\gcalcexpr[0]\nsecond{\vbsix - 2}

\question How many \textit{distinct} words can be made by jumbling \vbone so that none of the words
begins with a \vbtwo but all end with a \vbthree

\insertQR[-20pt]{abc}

\watchout[-10pt]

\ifprintanswers
  \marginnote[20pt]{We start out by treating all characters as different - even if they are actually the same. Later on we 
  will adjust the answer to account for the "same-ness" of characters}
\fi 

\begin{solution}
	The word is $\vbsix$ characters long - with $\vbfour$ \vbtwo 's and $\vbfive$ \vbthree 's. We therefore have $\vbfive$
	 options for the \textit{last} character
	 
	 Once we fix the last character, we would ordinarily have $\thfirst$ options for the first character. 
	 But we cannot use the $\vbfour$ \vbtwo 's. And so, we are down to $\nfirst$ options for the first character
	 
	 For the middle characters, however, there are no constraints. The remaining $\nsecond = (\vbsix - 2)$ characters 
	 can some in any permutation
	 
	 Moreover, given that we have more than one occurrence of the \vbtwo and \vbthree, we must discount 
	 permutations that are essentially the same word. And so, the number of distinct words is 
	 
	 \begin{align}
	 	N &= \dfrac{\nfirst \times \nsecond ! \times \vbfive}{\vbfour ! \times \vbfive !} = \vbseven
	 \end{align}
\end{solution}
