
%\noprintanswers
%\setcounter{rolldice}{3}
%\printrubric

\ifnumequal{\value{rolldice}}{0}{
  % variables 
  \renewcommand{\vbone}{5}
  \renewcommand{\vbtwo}{169}
  \renewcommand{\vbthree}{10}
}{
  \ifnumequal{\value{rolldice}}{1}{
    % variables 
    \renewcommand{\vbone}{8}
    \renewcommand{\vbtwo}{225}
    \renewcommand{\vbthree}{14}
  }{
    \ifnumequal{\value{rolldice}}{2}{
      % variables 
      \renewcommand{\vbone}{6}
      \renewcommand{\vbtwo}{121}
      \renewcommand{\vbthree}{10}
    }{
      % variables 
      \renewcommand{\vbone}{7}
      \renewcommand{\vbtwo}{225}
      \renewcommand{\vbthree}{13}
    }
  }
}

\EXPR[0]\a{(2 * \vbone) - 3}
\SQUARE\a\b
\SUBTRACT\vbone{1}\i
\SUBTRACT\vbone{2}\j
\MULTIPLY\i\j\c
\EXPR[0]\d{(\c + ((\vbtwo - \b)/4))}
\SUBTRACT\vbone{2}\z
\SUBTRACT\c\d\m

\SQRT\vbtwo\dsc
\SUBTRACT\a\dsc\p
\ADD\a\dsc\q
\MULTIPLY\p{-1}\r

\question[3] Find the \textit{smallest} valid $N$ for which $\enpr{N}{\vbone}\geq \d\cdot\enpr{N}{\z}$

\insertQR[-20pt]{qrc}

\watchout

\ifprintanswers
\fi 

\begin{solution}[\halfpage]
	\begin{align}
		&\enpr{N}\vbone\geq\d\cdot\enpr{N}\z \Rightarrow \fnpr{N}\vbone\geq\d\cdot\fnpr{N}\z \\
		&\Rightarrow \dfrac{(N-\j)\cdot(N-\i)\cdot(N-\vbone)!}{(N-\vbone)!}\geq\d \\
		&\Rightarrow N^2 - \a N\m\geq 0 \text{ or } \left(N + \WRITEFRAC\r{2}\right)\cdot\left(N - \WRITEFRAC\q{2}\right)\geq 0
	\end{align}
	For the above to be true, either $N\leq\WRITEFRAC\p{2}$ or $N\geq\WRITEFRAC\q{2}$. But then, $N\geq\vbone$ 
	and $N$ has to be an integer. And therefore, $N=\vbthree$ is the only valid solution
\end{solution}

\ifprintrubric
  \begin{table}
  	\begin{tabular}{ p{5cm}p{5cm} }
  		\toprule % in brief (4-6 words), what should a grader be looking for for insights & formulations
  		  \sc{\textcolor{blue}{Look for the following}} & \\ 
  		\midrule % ***** Insights & formulations ******
  			$N$ is an integer & $N > 0$ \\
  		\toprule % final numerical answers for the various versions
        \sc{\textcolor{blue}{If question has $\ldots$}} & \sc{\textcolor{blue}{Final answer}} \\
  		\midrule % ***** Numerical answers (below) **********
  			$\enpr{N}{5}\geq 42\cdot\enpr{N}{3}$ & $N = 10$ \\
  			$\enpr{N}{6}\geq 30\cdot\enpr{N}{4}$ & $N = 10$ \\
  			$\enpr{N}{7}\geq 56\cdot\enpr{N}{5}$ & $N = 13$ \\
  			$\enpr{N}{8}\geq 56\cdot\enpr{N}{6}$ & $N = 14$ \\
  		\bottomrule
  	\end{tabular}
  \end{table}
\fi
