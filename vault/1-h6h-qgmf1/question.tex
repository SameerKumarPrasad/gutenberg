% This is an empty shell file placed for you by the 'examiner' script.
% You can now fill in the TeX for your question here.

% Now, down to brasstacks. ** Writing good solutions is an Art **. 
% Eventually, you will find your own style. But here are some thoughts 
% to get you started: 
%
%   1. Write the solution as if you are writing it for your favorite
%      14-17 year old to help him/her understand. Could be your nephew, 
%      your niece, a cousin perhaps or probably even you when you 
%      were that age. Just write for them.
%
%   2. Use margin-notes to "talk" to students about the critical insights
%      in the question. The tone can be - in fact, should be - informal
%
%   3. Don't shy away from creating margin-figures you think will help
%      students understand. Yes, it is a little more work per question. 
%      But the question & solution will be written only once. Make that
%      attempt at writing a solution count.
%
%   4. At the same time, do not be too verbose. A long solution can
%      - at first sight - make the student think, "God, that is a lot to know".
%      Our aim is not to scare students. Rather, our aim should be to 
%      create many "Aha!" moments everyday in classrooms around the world
% 
%   5. Ensure that there are *no spelling mistakes anywhere*. We are an 
%      education company. Bad spellings suggest that we ourselves 
%      don't have any education. Also, use American spellings by default
% 
%   6. If a question has multiple parts, then first delete lines 40-41
%   7. If a question does not have parts, then first delete lines 43-69

\question[3] If the product of three numbers in geometric progression is $216$ and 
their sum is $19$, then what are the three numbers? 

\insertQR{QRC}

\ifprintanswers
\fi 

\begin{solution}[\halfpage]
	Let the three numbers be $\dfrac{a}{r}$, $a$ and $ar$. And so, 
	\begin{align}
		\dfrac{a}{r}\cdot a \cdot ar &= 216 \Rightarrow a^3 = 216 \Rightarrow a = 6
	\end{align}
	The three numbers, therefore, are $\dfrac{6}{r}$, $6$ and $6r$. Moreover, 
	\begin{align}
		\dfrac{6}{r} + 6 + 6r &= 19 \Rightarrow 6r^2-13r + 6 = 0 \\
		\Rightarrow r &= \dfrac{13 \pm \sqrt{13^2-4\cdot 6\cdot 6}}{2\times 6} = \frac{18}{12} \text{ or } \frac{8}{12} \\
		&= \frac{3}{2} \text{ or } \frac{2}{3}
	\end{align}
	
	The numbers could therefore be either $(4,6,9)$ if $r = \frac{3}{2}$ or $(9,6,4)$ if $r=\frac{2}{3}$
\end{solution}
