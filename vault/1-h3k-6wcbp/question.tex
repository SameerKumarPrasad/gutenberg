% This is an empty shell file placed for you by the 'examiner' script.
% You can now fill in the TeX for your question here.

% Now, down to brasstacks. ** Writing good solutions is an Art **. 
% Eventually, you will find your own style. But here are some thoughts 
% to get you started: 
%
%   1. Write the solution as if you are writing it for your favorite
%      14-17 year old to help him/her understand. Could be your nephew, 
%      your niece, a cousin perhaps or probably even you when you 
%      were that age. Just write for them.
%
%   2. Use margin-notes to "talk" to students about the critical insights
%      in the question. The tone can be - in fact, should be - informal
%
%   3. Don't shy away from creating margin-figures you think will help
%      students understand. Yes, it is a little more work per question. 
%      But the question & solution will be written only once. Make that
%      attempt at writing a solution count.
%
%   4. At the same time, do not be too verbose. A long solution can
%      - at first sight - make the student think, "God, that is a lot to know".
%      Our aim is not to scare students. Rather, our aim should be to 
%      create many "Aha!" moments everyday in classrooms around the world
% 
%   5. Ensure that there are *no spelling mistakes anywhere*. We are an 
%      education company. Bad spellings suggest that we ourselves 
%      don't have any education. Also, use American spellings by default
% 
%   6. If a question has multiple parts, then first delete lines 40-41
%   7. If a question does not have parts, then first delete lines 43-69

\question Two men - $M$ and $N$ - are playing a rather dangerous game. They are taking
turns firing at each other from a revolver that has 2 bullets with space for six (6). 
The revolver's bullet chamber is circular and the bullets are in \textit{adjacent} chambers.
$M$ took the first shot and it was blank. Now its $N's$ turn. Should $N$ fire the gun as-is
or should he rotate the chamber first? 

\insertQR{}

\ifprintanswers
  % stuff to be shown only in the answer key - like explanatory margin figures
  \begin{marginfigure}
    \figinit{pt}
      \figpt 100: (50,50)
      \figptcirc 200 :$A$: 100; 50(0)
      \figptcirc 201 :$B$: 100; 50(60)
      \figptcirc 202 :$C$: 100; 50(120)
      \figptcirc 203 :$D$: 100; 50(180)
      \figptcirc 204 :$E$: 100; 50(240)
      \figptcirc 205 :$F$: 100; 50(300)
    \figdrawbegin{}
      \figdrawcirc 200 (5)
      \figdrawcirc 201 (5)
      \figdrawcirc 202 (5)
      \figdrawcirc 203 (5)
      \figset (fillmode=yes)
      \figdrawcirc 204 (5)
      \figdrawcirc 205 (5)
    \figdrawend
    \figvisu{\figBoxA}{}{%
      \figwritee 200:(7)
      \figwritee 201:(7)
      \figwritee 202:(7)
      \figwritee 203:(7)
      \figwritee 204:(7)
      \figwritee 205:(7)
    }
    \centerline{\box\figBoxA}
  \end{marginfigure}
\fi 

\begin{solution}
	The two men obviously don't like each other. Otherwise they wouldn't be playing such 
	a dangerous game. Be that as it may, it is obvious that $N$ would want to fire a bullet - and not a 
	blank - at $M$ when his turn comes
	
	The question for $N$ then is - is the probability that there is a bullet in the current
	chamber higher if he rotates the chamber or if he sticks with the chamber he got when $M$
	passed him the gun. The situation is \asif where chambers $E$ and $F$ have the bullets whilst 
	others are blank. Lets assume that the chamber rotates clockwise after each shot
	
	And so, if $X$ be the event that the first shot is blank and $Y$ the event the second
	is blank, then we are seeking $\bayesp{\textoverline{Y}}{X} = \dfrac{1}{4}$. Why? Because 
	only if $M$ fired chamber $A$ (a blank) could the next shot - $F$ - be a non-blank. And $A$ 
	is one of \text{four} blank chambers
	
	If, however, $N$ were to rotate the chamber, then his probability of firing a \textit{non-blank} 
	chamber would be $\dfrac{1}{3}$ which is more than $\dfrac{1}{4}$
	
	And so, $N$ should rotate the chamber before firing
	
\end{solution}
