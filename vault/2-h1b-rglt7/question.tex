% This is an empty shell file placed for you by the 'examiner' script.
% You can now fill in the TeX for your question here.

% Now, down to brasstacks. ** Writing good solutions is an Art **. 
% Eventually, you will find your own style. But here are some thoughts 
% to get you started: 
%
%   1. Write the solution as if you are writing it for your favorite
%      14-17 year old to help him/her understand. Could be your nephew, 
%      your niece, a cousin perhaps or probably even you when you 
%      were that age. Just write for them.
%
%   2. Use margin-notes to "talk" to students about the critical insights
%      in the question. The tone can be - in fact, should be - informal
%
%   3. Don't shy away from creating margin-figures you think will help
%      students understand. Yes, it is a little more work per question. 
%      But the question & solution will be written only once. Make that
%      attempt at writing a solution count.
%
%   4. At the same time, do not be too verbose. A long solution can
%      - at first sight - make the student think, "God, that is a lot to know".
%      Our aim is not to scare students. Rather, our aim should be to 
%      create many "Aha!" moments everyday in classrooms around the world
% 
%   5. Ensure that there are *no spelling mistakes anywhere*. We are an 
%      education company. Bad spellings suggest that we ourselves 
%      don't have any education. Also, use American spellings by default
% 
%   6. If a question has multiple parts, then first delete lines 40-41
%   7. If a question does not have parts, then first delete lines 43-69

\question Let $f$ be a function that is twice differentiable for all real numbers. The table below gives the values of $f$ for selected points in the closed interval $2 \leq x \leq 13$.

\begin{center}
  \begin{tabular}{|c|c|c|c|c|c|}
    \hline
    $x$ & 2 &3 & 5 & 8 & 13 \\ 
    \hline
    $f\left(x\right)$ & 1 & 4 & -2 & 3 & 6 \\ 
    \hline
  \end{tabular}
\end{center}

\ifprintanswers
  % stuff to be shown only in the answer key - like explanatory margin figures
\fi 

\begin{parts}
  \part[3] Estimate $f'\left(4\right)$. Show the work that leads to your answer.
  \insertQR{QRC}
\begin{solution}[\halfpage]

    Definition of $\dydx = f'\left(x\right)$ is,
    \begin{align}
      f'\left(x\right) &= \dfrac{f\left(x_2\right)-f\left(x_1\right)}{x_2-x_1}
    \end{align}
    where $x_1 \rightarrow  x_2$. However, in this scenario since $x_2-x_1 \gg 0$ we can only get an approximation. Therefore,
    \begin{align}
      f'\left(x\right) &\approx \dfrac{f\left(5\right)-f\left(3\right)}{5-3} \\
                       &= -3
    \end{align}
  \end{solution}

  \part[3] Evaluate $\int_2^{13} \left(3-5f'\left(x\right)\right) \ud x$. Show the work that leads to your answer.
  \insertQR{QRC}
\begin{solution}[\halfpage]
    \begin{align}
      I &= \int_2^{13} \left(3-5'\left(x\right)\right) \ud x \\
        &= 3\int_2^{13}  \ud x - 5\int_2^{13} f'\left(x\right) \ud x \\
        &= 3\left[x\right]_2^{13} - 5\left[f\left(x\right)\right]_2^{13} \\
        &= 3\left(13 - 2\right) - 5\left(f\left(13\right) - f\left(2\right)\right) \\
        &= 8
    \end{align}
  \end{solution}

\newpage
  \part[4] Use a left Riemann sum with subintervals indicated by the data in the table to approximate $\int_2^{13} f\left(x\right) \ud x$. Show the work that leads to your answer.
  \insertQR{QRC}
\begin{solution}[\halfpage]
    Left Reimann Sum for a function $f$ over an interval $I\left(1,n\right)$ is defined as,
    \begin{align}
      S &= \sum_{i=1}^{n}f\left(x_{i-1}\right)\left(x_i-x_{i-1}\right)
    \end{align}
    Applying this for the given function we get,
    \begin{align}
      S &= f\left(2\right)\left(3-2\right) + f\left(3\right)\left(5-3\right) + \nonumber \\
        &\qquad f\left(5\right)\left(8-5\right) + f\left(8\right)\left(13-8\right) \\
        &= 18
    \end{align}

  \end{solution}

  \part[4] Suppose $f'\left(5\right) = 3$ and $f''\left(x\right) \leq 0$ for all $x$ in the closed interval $5 \leq x \leq 8$. Use the line tangent to the graph of $f$ at $x=5$ to show that $f\left(7\right) \leq 4$. Use the secant line for the graph of $f$ on $5 \leq x \leq 8$ to show that $f\left(7\right) \geq \dfrac{4}{3}$.
  \insertQR{QRC}
  \ifprintanswers
    % stuff to be shown only in the answer key - like explanatory margin figures
    \marginnote[50pt] {$y-y_1 = m\left(x-x_1\right)$}
    \marginnote[165pt] {$y-y_1 = \dfrac{y_2 - y_1}{x_2 - x_1}\left(x-x_1\right)$}
  \fi 
\begin{solution}[\halfpage]
    The equation for the tangent line at $f'\left(5\right)$ is,
    \begin{align}
      y - \left(-2\right) &= 3\left(x - 5\right) \\
                        y &= 3x - 17
    \end{align}
    Since $f''\left(x\right) <0$ in the interval $5\leq x\leq 8$, the slope of the function is steadily decreasing. Therefore the tangent line lies above the function for all $x$ in the interval $5 \leq x \leq 8$. Now, at $x=7$, the tangent line is at $4$, therefore,
    \begin{align}
      f\left(7\right) &\leq 4
    \end{align}
    The equation for the secant line of the graph in the interval $5 \leq x \leq 8$ is,
    \begin{align}
      y-3 &= \dfrac{3-\left(-2\right)}{8-5}\left(x-8\right) \\
        y &= \dfrac{5}{3}\left(x-8\right)+3
    \end{align}
    Since the secant line connecting $(5, f\left(5\right)$ and $\left(8, f\left(8\right)\right)$ lies below the graph of $y=f\left(x\right)$ for all $x$ in the interval $5 \leq x \leq 8$, therefore,
    \begin{align}
      f\left(7\right) &\geq \dfrac{5}{3}\left(7-8\right)+3 \\
                      &= \dfrac{4}{3} 
    \end{align}

  \end{solution}

\end{parts}
