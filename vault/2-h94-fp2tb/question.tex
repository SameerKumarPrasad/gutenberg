% This is an empty shell file placed for you by the 'examiner' script.
% You can now fill in the TeX for your question here.

% Now, down to brasstacks. ** Writing good solutions is an Art **. 
% Eventually, you will find your own style. But here are some thoughts 
% to get you started: 
%
%   1. Write to be understood - but be crisp. Your own solution should not take 
%      more space than you will give to the student. Hence, if you take more than 
%      a half-page to write a solution, then give the student a full-page and so on...
%
%   2. Use margin-notes to "talk" to students about the critical insights
%      in the question. The tone can be - in fact, should be - informal
%
%   3. Don't shy away from creating margin-figures you think will help
%      students understand. Yes, it is a little more work per question. 
%      But the question & solution will be written only once. Make that
%      attempt at writing a solution count.
%      
%      3b. Use bc_to_fig.tex. Its an easier way to generate plots & graphs 
% 
%   4. Ensure that there are *no spelling mistakes anywhere*. We are an 
%      education company. Bad spellings suggest that we ourselves 
%      don't have any education. Also, use American spellings by default
% 
%   5. If a question has multiple parts, then first delete lines 40-41
%   6. If a question does not have parts, then first delete lines 43-69
%   
%   7. Create versions of the question when possible. Use commands defined in 
%      tufte-tweaks.sty to do so. Its easier than you think

% \noprintanswers
% \setcounter{rolldice}{0}

\ifnumequal{\value{rolldice}}{0}{
  % variables 
  \renewcommand{\vbone}{}
  \renewcommand{\vbtwo}{}
  \renewcommand{\vbthree}{}
  \renewcommand{\vbfour}{}
  \renewcommand{\vbfive}{}
  \renewcommand{\vbsix}{}
  \renewcommand{\vbseven}{}
  \renewcommand{\vbeight}{}
  \renewcommand{\vbnine}{}
  \renewcommand{\vbten}{}
}{
  \ifnumequal{\value{rolldice}}{1}{
    % variables 
    \renewcommand{\vbone}{}
    \renewcommand{\vbtwo}{}
    \renewcommand{\vbthree}{}
    \renewcommand{\vbfour}{}
    \renewcommand{\vbfive}{}
    \renewcommand{\vbsix}{}
    \renewcommand{\vbseven}{}
    \renewcommand{\vbeight}{}
    \renewcommand{\vbnine}{}
    \renewcommand{\vbten}{}
  }{
    \ifnumequal{\value{rolldice}}{2}{
      % variables 
      \renewcommand{\vbone}{}
      \renewcommand{\vbtwo}{}
      \renewcommand{\vbthree}{}
      \renewcommand{\vbfour}{}
      \renewcommand{\vbfive}{}
      \renewcommand{\vbsix}{}
      \renewcommand{\vbseven}{}
      \renewcommand{\vbeight}{}
      \renewcommand{\vbnine}{}
      \renewcommand{\vbten}{}
    }{
      % variables 
      \renewcommand{\vbone}{}
      \renewcommand{\vbtwo}{}
      \renewcommand{\vbthree}{}
      \renewcommand{\vbfour}{}
      \renewcommand{\vbfive}{}
      \renewcommand{\vbsix}{}
      \renewcommand{\vbseven}{}
      \renewcommand{\vbeight}{}
      \renewcommand{\vbnine}{}
      \renewcommand{\vbten}{}
    }
  }
}

\question The graph of the velocity $v(t)$, in $ft/sec$, of a car travelling 
on a straight road, for $0\leq 50\leq$ is shown above. A table of values for
$v(t)$, at $5$ second intervals of time $t$, is shown below the graph.
  \insertQR{QRC}

  \begin{marginfigure}
    \figinit{pt}
      % the curve
      \figpt 0:$O$(0,0)
      \figpt 101:(5,12)
      \figpt 102:(10,20)
      \figpt 103:(15,30)
      \figpt 104:(20,55)
      \figpt 105:(25,70)
      \figpt 106:(30,78)
      \figpt 107:(35,81)
      \figpt 108:(40,75)
      \figpt 109:(45,60)
      \figpt 110:(50,72)
      % extremeties
      \def\Xmax{80}
      \def\Ymax{100}
      \def\XmaxGrid{70}
      \def\YmaxGrid{90}
      \figpt 10:$\textit{t}$(\Xmax,0)
      \figpt 20:$\textit{v(t)}$(0,\Ymax)
      \figpt 30:(\XmaxGrid,0)
      \figpt 40:(0,\YmaxGrid)
      \figpt 50:(\XmaxGrid,\YmaxGrid)
      % pts for legend
      \figpt 60:(35,-10) 
      \figpt 70:(-20,40)
      \figpt 80:(-20,30)
      % pts for numbering the axes (5, 10, 15...)
      \figpt 201:$\tiny\text{10}$(10,0)
      \figpt 202:$\tiny\text{20}$(20,0)
      \figpt 203:$\tiny\text{30}$(30,0)
      \figpt 204:$\tiny\text{40}$(40,0)
      \figpt 205:$\tiny\text{50}$(50,0)
      \figpt 206:$\tiny\text{60}$(60,0)
      \figpt 207:$\tiny\text{10}$(0,20)
      \figpt 208:$\tiny\text{20}$(0,40)
      \figpt 209:$\tiny\text{30}$(0,60)
      \figpt 210:$\tiny\text{40}$(0,80)
    \figdrawbegin{}
      \figdrawline [0,101]
      \figdrawline [101,102]
      \figdrawline [102,103]
      \figdrawline [103,104]
      \figdrawline [104,105]
      \figdrawline [105,106]
      \figdrawline [106,107]
      \figdrawline [107,108]
      \figdrawline [108,109]
      \figdrawline [109,110]
      \figdrawmesh 14,9 [0, 30, 50, 40]
      \figset arrowhead(length=4, fillmode=yes)
      \figdrawaxes 0(0, \Xmax, 0, \Ymax)
    \figdrawend
    \figvisu{\figBoxA}{}{%
      \figwritee 10:(5pt)
      \figwriten 20:(5pt)
      \figwrites 201,202,203,204,205,206 :(2pt)
      \figwritew 207,208,209,210 :(2pt)
      \figset write(mark=$\bullet$)
      \figwritep[101,102,103,104,105,106,107,108,109,110]
      \figwritec[60]{\scriptsize\textit{Time (sec)}}
      \figwritec[70]{\scriptsize\textit{Velocity}}
      \figwritec[80]{\scriptsize\textit{(ft/sec)}}
    }
    \centerline{\box\figBoxA}
  \end{marginfigure}
  \begin{marginfigure}[10pt]
    \hfill{\begin{tabular}{|c|c|}
      \hline
      \textit{t (sec)}&\textit{v(t) (ft/sec)} \\
      \hline
      0&0   \\
      5&12  \\
      10&20 \\
      15&30 \\
      20&55 \\
      25&70 \\
      30&78 \\
      35&81 \\
      40&75 \\
      45&60 \\
      50&72 \\
      \hline
    \end{tabular}}
  \end{marginfigure}
\ifprintanswers
  % stuff to be shown only in the answer key - like explanatory margin figures
\fi 

\begin{parts}
  \part[2] During what intervals of time is the acceleration of the car positive?
  Give a reason for your answer.

\begin{solution}[\mcq]
    The acceleration is positive in the intervals where the change of
    velocity is positive, in other words the velocity is increasing. This 
    happens in the time ranges:
    \begin{align}
         (0,35)&\Rightarrow 0\textit{(ft/sec)}\text{ to }81\textit{(ft/sec)} \\
         (45,50)&\Rightarrow 60\textit{(ft/sec)}\text{ to }72\textit{(ft/sec)} 
    \end{align}
  \end{solution}

  \part[2] Find the average acceleration of the car, in $ft/sec^2$ over the 
  interval $0\leq 50$.

\begin{solution}[\mcq]
    Average acceleration is give by,
    \begin{align}
      \bar{a}_{\text{avg}} &= \dfrac{v(t_2) - v(t_1)}{t_2-t_1} \\
                           &= \dfrac{v(50) - v(0)}{50 - 0} \\
                           &= \dfrac{72}{50}\text{ft/sec}^2
    \end{align}  
  \end{solution}

  \part[2] Find one approximation for the acceleration of the car in $ft/sec^2$ at
  $t=40\text{sec}$. Show the computations you used to arrive at your answer.

\begin{solution}[\mcq]
    A fair approximation for the acceleration at $t=40$(sec) could be made 
    using any one of the following time intervals $(35,40), (40, 45), 
    (35, 45)$. \\
    Let us approximate the value for acceleration using each of the three
    intervals.
    \begin{align}
      \bar{a}_\textit{35-40} 
              &= \dfrac{v(40)-v(35)}{40-35}\dfrac{(ft/sec)}{(sec)} \\
              &= \dfrac{75-81}{5}\dfrac{(ft/sec)}{(sec)} = 
                 \dfrac{-6}{5}(ft/sec^2)
    \end{align}
    \begin{align}
      \bar{a}_\textit{40-45} 
              &= \dfrac{v(45)-v(40)}{45-40}\dfrac{(ft/sec)}{(sec)} \\
              &= \dfrac{60-75}{5}\dfrac{(ft/sec)}{(sec)} = 
                 -3\text{(ft/sec$^2$)}
    \end{align}
    \begin{align}
      \bar{a}_\textit{35-45} 
              &= \dfrac{v(45)-v(35)}{45-35}\dfrac{(ft/sec)}{(sec)} \\
              &= \dfrac{60-81}{10}\dfrac{(ft/sec)}{(sec)} = 
                 \dfrac{-21}{10}\text{(ft/sec$^2$)}
    \end{align}
  \end{solution}
\end{parts}

