% This is an empty shell file placed for you by the 'examiner' script.
% You can now fill in the TeX for your question here.

% Now, down to brasstacks. ** Writing good solutions is an Art **. 
% Eventually, you will find your own style. But here are some thoughts 
% to get you started: 
%
%   1. Write to be understood - but be crisp. Your own solution should not take 
%      more space than you will give to the student. Hence, if you take more than 
%      a half-page to write a solution, then give the student a full-page and so on...
%
%   2. Use margin-notes to "talk" to students about the critical insights
%      in the question. The tone can be - in fact, should be - informal
%
%   3. Don't shy away from creating margin-figures you think will help
%      students understand. Yes, it is a little more work per question. 
%      But the question & solution will be written only once. Make that
%      attempt at writing a solution count.
%      
%      3b. Use bc_to_fig.tex. Its an easier way to generate plots & graphs 
% 
%   4. Ensure that there are *no spelling mistakes anywhere*. We are an 
%      education company. Bad spellings suggest that we ourselves 
%      don't have any education. Also, use American spellings by default
% 
%   5. If a question has multiple parts, then first delete lines 40-41
%   6. If a question does not have parts, then first delete lines 43-69
%   
%   7. Create versions of the question when possible. Use commands defined in 
%      tufte-tweaks.sty to do so. Its easier than you think

% \noprintanswers
% \setcounter{rolldice}{2}

\ifnumequal{\value{rolldice}}{0}{
  % variables 
  \renewcommand{\vbone}{3}
  \renewcommand{\vbtwo}{4}
}{
  \ifnumequal{\value{rolldice}}{1}{
    % variables 
    \renewcommand{\vbone}{1}
    \renewcommand{\vbtwo}{-2}
  }{
    \ifnumequal{\value{rolldice}}{2}{
      % variables 
      \renewcommand{\vbone}{2}
      \renewcommand{\vbtwo}{3}
    }{
      % variables 
      \renewcommand{\vbone}{-4}
      \renewcommand{\vbtwo}{2}
  }
  }
}

\gcalcexpr[0]{\vbthree}{\vbone + \vbtwo}

\question[4] Find the equation of the straight line that cuts off equal $x$ and $y$ intercepts 
and passes through the point $(\vbone, \vbtwo)$

\insertQR{QRC}

\watchout

\ifprintanswers
\fi 

\begin{solution}[\halfpage]
   If $Ax + By + C = 0$ be the equation of the line, then the $x$ and $y$ intercepts are given by
   \begin{align}
   	  x_{\text{int}} &= -\dfrac{C}{A} \\ 
   	  y_{\text{int}} &= -\dfrac{C}{B} \\
   \end{align}
   Which means that if they are equal, then $A = B \Rightarrow$ equation of the line is $Ax + Ay + C = 0$. 
   Now, if this line also passes through $(\vbone, \vbtwo)$, then 
   \begin{align}
      \vbone\cdot A + \vbtwo\cdot A + C &= 0 \Rightarrow C = -(\vbthree\cdot A) \\
      \Rightarrow Ax + Ay - (\vbthree\cdot A) &= 0 \\
      \text{ or } x + y - (\vbthree) &= 0
   \end{align}
   
\end{solution}
