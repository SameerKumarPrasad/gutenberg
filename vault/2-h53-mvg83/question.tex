% This is an empty shell file placed for you by the 'examiner' script.
% You can now fill in the TeX for your question here.

% Now, down to brasstacks. ** Writing good solutions is an Art **. 
% Eventually, you will find your own style. But here are some thoughts 
% to get you started: 
%
%   1. Write to be understood - but be crisp. Your own solution should not take 
%      more space than you will give to the student. Hence, if you take more than 
%      a half-page to write a solution, then give the student a full-page and so on...
%
%   2. Use margin-notes to "talk" to students about the critical insights
%      in the question. The tone can be - in fact, should be - informal
%
%   3. Don't shy away from creating margin-figures you think will help
%      students understand. Yes, it is a little more work per question. 
%      But the question & solution will be written only once. Make that
%      attempt at writing a solution count.
%      
%      3b. Use bc_to_fig.tex. Its an easier way to generate plots & graphs 
% 
%   4. Ensure that there are *no spelling mistakes anywhere*. We are an 
%      education company. Bad spellings suggest that we ourselves 
%      don't have any education. Also, use American spellings by default
% 
%   5. If a question has multiple parts, then first delete lines 40-41
%   6. If a question does not have parts, then first delete lines 43-69
%   
%   7. Create versions of the question when possible. Use commands defined in 
%      tufte-tweaks.sty to do so. Its easier than you think

% \noprintanswers
% \setcounter{rolldice}{3}

\ifnumequal{\value{rolldice}}{0}{
  % variables 
  \renewcommand{\vbone}{odd} % type 
  \renewcommand{\vbtwo}{0,1,4,7} % available 
  \renewcommand{\vbthree}{350} % less than 
  \renewcommand{\vbfour}{1,7} % units 
  \renewcommand{\vbfive}{0,1,4,7} % tens 
  \renewcommand{\vbsix}{1} % hundreds 
}{
  \ifnumequal{\value{rolldice}}{1}{
    % variables 
    \renewcommand{\vbone}{even}
    \renewcommand{\vbtwo}{3,6,8,9}
    \renewcommand{\vbthree}{500}
    \renewcommand{\vbfour}{6,8}
    \renewcommand{\vbfive}{3,6,8,9}
    \renewcommand{\vbsix}{3}
  }{
    \ifnumequal{\value{rolldice}}{2}{
      % variables 
      \renewcommand{\vbone}{odd}
      \renewcommand{\vbtwo}{2,5,7,8}
      \renewcommand{\vbthree}{495}
      \renewcommand{\vbfour}{5,7}
      \renewcommand{\vbfive}{2,5,7,8}
      \renewcommand{\vbsix}{2}
    }{
      % variables 
      \renewcommand{\vbone}{even}
      \renewcommand{\vbtwo}{6,7,8,9}
      \renewcommand{\vbthree}{700}
      \renewcommand{\vbfour}{6,8}
      \renewcommand{\vbfive}{6,7,8,9}
      \renewcommand{\vbsix}{6}
    }
  }
}

\question[2] How many 3-digit $\vbone$ numbers less than $\vbthree$ can be 
 formed using the digits $\vbtwo$ - if repetition of digits is allowed

\insertQR[-15pt]{QRC}

\watchout

\ifprintanswers
   \begin{table}
		\begin{tabular}{ccc}
			\toprule
			Units & Tens & Hundreds \\
			\midrule
			$\vbfour$ & $\vbfive$ & $\vbsix$ \\
			\bottomrule
		\end{tabular}
	\end{table}
\fi 

\begin{solution}[\mcq]
	The table above summarizes the digits that can appear in the units, tens and hundreds 
	places - with repetition - to form a number that meets our criterion 
	
	The answer, therefore, is - $ N = 2 \times 4 \times 1 = 8$
	
	
\end{solution}