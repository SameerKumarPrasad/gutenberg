% This is an empty shell file placed for you by the 'examiner' script.
% You can now fill in the TeX for your question here.

% Now, down to brasstacks. ** Writing good solutions is an Art **. 
% Eventually, you will find your own style. But here are some thoughts 
% to get you started: 
%
%   1. Write to be understood - but be crisp. Your own solution should not take 
%      more space than you will give to the student. Hence, if you take more than 
%      a half-page to write a solution, then give the student a full-page and so on...
%
%   2. Use margin-notes to "talk" to students about the critical insights
%      in the question. The tone can be - in fact, should be - informal
%
%   3. Don't shy away from creating margin-figures you think will help
%      students understand. Yes, it is a little more work per question. 
%      But the question & solution will be written only once. Make that
%      attempt at writing a solution count.
%      
%      3b. Use bc_to_fig.tex. Its an easier way to generate plots & graphs 
% 
%   4. Ensure that there are *no spelling mistakes anywhere*. We are an 
%      education company. Bad spellings suggest that we ourselves 
%      don't have any education. Also, use American spellings by default
% 
%   5. If a question has multiple parts, then first delete lines 40-41
%   6. If a question does not have parts, then first delete lines 43-69
%   
%   7. Create versions of the question when possible. Use commands defined in 
%      tufte-tweaks.sty to do so. Its easier than you think

% \noprintanswers
% \setcounter{rolldice}{0}

\ifnumequal{\value{rolldice}}{0}{
  % variables 
  \renewcommand{\vbone}{0}
  \renewcommand{\vbtwo}{1}
  \renewcommand{\vbthree}{4}
  \renewcommand{\vbfour}{7}
  \renewcommand{\vbfive}{350}
  \renewcommand{\vbsix}{1}
  \renewcommand{\vbseven}{odd}
  \renewcommand{\vbeight}{2}
  \renewcommand{\vbnine}{}
  \renewcommand{\vbten}{}
}{
  \ifnumequal{\value{rolldice}}{1}{
    % variables 
    \renewcommand{\vbone}{2}
    \renewcommand{\vbtwo}{0}
    \renewcommand{\vbthree}{4}
    \renewcommand{\vbfour}{9}
    \renewcommand{\vbfive}{500}
    \renewcommand{\vbsix}{2}
    \renewcommand{\vbseven}{even}
    \renewcommand{\vbeight}{3}
    \renewcommand{\vbnine}{}
    \renewcommand{\vbten}{}
  }{
    \ifnumequal{\value{rolldice}}{2}{
      % variables 
      \renewcommand{\vbone}{1}
      \renewcommand{\vbtwo}{3}
      \renewcommand{\vbthree}{0}
      \renewcommand{\vbfour}{8}
      \renewcommand{\vbfive}{495}
      \renewcommand{\vbsix}{1}
      \renewcommand{\vbseven}{odd}
      \renewcommand{\vbeight}{2}
      \renewcommand{\vbnine}{}
      \renewcommand{\vbten}{}
    }{
      % variables 
      \renewcommand{\vbone}{5}
      \renewcommand{\vbtwo}{2}
      \renewcommand{\vbthree}{6}
      \renewcommand{\vbfour}{0}
      \renewcommand{\vbfive}{700}
      \renewcommand{\vbsix}{3}
      \renewcommand{\vbseven}{even}
      \renewcommand{\vbeight}{3}
      \renewcommand{\vbnine}{}
      \renewcommand{\vbten}{}
    }
  }
}

\gcalcexpr[0]{\total}{\vbeight * \vbsix}
\gcalcexpr[0]{\total}{\total * 4}

\question[2] Find how many $\vbseven$ numbers less than $\vbfive$ can be 
 formed using the digits $\vbone$, $\vbtwo$, $\vbthree$, $\vbfour$ if
 repetition of digits is allowed.

\insertQR{QRC}

\watchout

\ifprintanswers
  % stuff to be shown only in the answer key - like explanatory margin figures
  \begin{marginfigure}
    \figinit{pt}
      \figpt 100:(0,0)
      \figpt 101:(0,0)
    \figdrawbegin{}
      \figdrawline [100,101]
    \figdrawend
    \figvisu{\figBoxA}{}{%
    }
    \centerline{\box\figBoxA}
  \end{marginfigure}
\fi 

\begin{solution}[\mcq]
  Since we are considering 3 digit numbers, let us count the ways in which
  we can fill the units, tens and hundreds places. Multiplying each of these
  will give us the total number of combinations we are looking for. \\
  Number of ways we can fill,
  \begin{align}
    &\text{units place}    &= \vbeight \\
    &\text{tens place}     &= 4 \\
    &\text{hundreds place} &= \vbsix
  \end{align}
  \begin{align}
    \text{Total Combinations} &= \vbeight\times 4\times \vbsix \\
                              &= \total
  \end{align}
\end{solution}
  \end{questions} 
\end{document}
