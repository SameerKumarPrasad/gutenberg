% This is an empty shell file placed for you by the 'examiner' script.
% You can now fill in the TeX for your question here.

% Now, down to brasstacks. ** Writing good solutions is an Art **. 
% Eventually, you will find your own style. But here are some thoughts 
% to get you started: 
%
%   1. Write the solution as if you are writing it for your favorite
%      14-17 year old to help him/her understand. Could be your nephew, 
%      your niece, a cousin perhaps or probably even you when you 
%      were that age. Just write for them.
%
%   2. Use margin-notes to "talk" to students about the critical insights
%      in the question. The tone can be - in fact, should be - informal
%
%   3. Don't shy away from creating margin-figures you think will help
%      students understand. Yes, it is a little more work per question. 
%      But the question & solution will be written only once. Make that
%      attempt at writing a solution count.
%
%   4. At the same time, do not be too verbose. A long solution can
%      - at first sight - make the student think, "God, that is a lot to know".
%      Our aim is not to scare students. Rather, our aim should be to 
%      create many "Aha!" moments everyday in classrooms around the world
% 
%   5. Ensure that there are *no spelling mistakes anywhere*. We are an 
%      education company. Bad spellings suggest that we ourselves 
%      don't have any education. Also, use American spellings by default
% 
%   6. If a question has multiple parts, then first delete lines 40-41
%   7. If a question does not have parts, then first delete lines 43-69

\printrubric

\question[4] Find an expression for the sum of the first $n$ terms of the series 
$\dfrac{1}{1\cdot 2\cdot 3} + \dfrac{1}{2\cdot 3\cdot 4} + \dfrac{1}{3\cdot 4\cdot 5} + \ldots$

\insertQR[-15pt]{QRC}

\ifprintanswers
\fi 

\begin{solution}[\fullpage]
	\begin{fullwidth}
	Each term in the series is of the form $a_n = \dfrac{1}{n\cdot(n+1)\cdot(n+2)}$. And therefore,
	\begin{align}
		S_n &= \sum_{k=1}^{n}\dfrac{1}{k\cdot(k+1)\cdot(k+2)} 
		= \sum_{k=1}^{n}\dfrac{k+1-k}{k\cdot(k+1)\cdot(k+2)} \\
		&= \sum_{k=1}^{n}\dfrac{1}{k\cdot(k+2)} - \sum_{k=1}^{n}\dfrac{1}{(k+1)\cdot(k+2)} \\
		&= \dfrac{1}{2}\sum_{k=1}^{n}\dfrac{k+2-k}{(k+2)\cdot k} - \sum_{k=1}^{n}\dfrac{k+1-k}{(k+1)\cdot k} \\
		&= \dfrac{1}{2}\left[\sum_{k=1}^{n}\dfrac{1}{k} - \sum_{k=1}^{n}\dfrac{1}{k+2} \right] 
		- \left[ \sum_{k=1}^{n}\dfrac{1}{k} - \sum_{k=1}^{n}\dfrac{1}{k+1} \right] \\
		&= \dfrac{1}{2}\left[\left(1 + \frac{1}{2} + \frac{1}{3} + \ldots + \frac{1}{n} \right) 
		- \left( \frac{1}{3} + \frac{1}{4} + \ldots + \frac{1}{n} + \frac{1}{n+1} + \frac{1}{n+2}\right) 
		\right] \nonumber \\
		&- \left[\left( \frac{1}{2} + \frac{1}{3} + \ldots + \frac{1}{n+1}\right) 
		- \left( \frac{1}{3} + \frac{1}{4} + \ldots + \frac{1}{n+1} + \frac{1}{n+2} \right) \right]  \\
		&= \dfrac{1}{2}\left[ \dfrac{3}{2} - \dfrac{1}{n+1} - \dfrac{1}{n+2}\right] 
		- \left[\dfrac{1}{2} - \dfrac{1}{n+2} \right] \\
		&= \dfrac{1}{4} - \dfrac{1}{2}\left( \dfrac{1}{n+1} - \dfrac{1}{n+2}\right) \\
		&= \dfrac{1}{4} - \dfrac{1}{2}\cdot\dfrac{1}{(n+1)\cdot(n+2)}
	\end{align}
	\end{fullwidth}
\end{solution}

\ifprintrubric
  \begin{table}
  	\begin{tabular}{ p{5cm}p{5cm} }
  		\toprule % in brief (4-6 words), what should a grader be looking for for insights & formulations
  		  \sc{\textcolor{blue}{Insight}} & \sc{\textcolor{blue}{Formulation}} \\ 
  		\midrule % ***** Insights & formulations ******
        $a_n = \dfrac{1}{n\cdot(n+1)\cdot(n+2)}$ & \\
        Break expression into partial sums & \\ 
        Middle terms cancel out & \\
  		\toprule % final numerical answers for the various versions
        \sc{\textcolor{blue}{If question has $\ldots$}} & \sc{\textcolor{blue}{Final answer}} \\
  		\midrule % ***** Numerical answers (below) **********
  		\bottomrule
  	\end{tabular}
  \end{table}
\fi
