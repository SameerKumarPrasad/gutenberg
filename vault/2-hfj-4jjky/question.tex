% This is an empty shell file placed for you by the 'examiner' script.
% You can now fill in the TeX for your question here.

% Now, down to brasstacks. ** Writing good solutions is an Art **. 
% Eventually, you will find your own style. But here are some thoughts 
% to get you started: 
%
%   1. Write to be understood - but be crisp. Your own solution should not take 
%      more space than you will give to the student. Hence, if you take more than 
%      a half-page to write a solution, then give the student a full-page and so on...
%
%   2. Use margin-notes to "talk" to students about the critical insights
%      in the question. The tone can be - in fact, should be - informal
%
%   3. Don't shy away from creating margin-figures you think will help
%      students understand. Yes, it is a little more work per question. 
%      But the question & solution will be written only once. Make that
%      attempt at writing a solution count.
%      
%      3b. Use bc_to_fig.tex. Its an easier way to generate plots & graphs 
% 
%   4. Ensure that there are *no spelling mistakes anywhere*. We are an 
%      education company. Bad spellings suggest that we ourselves 
%      don't have any education. Also, use American spellings by default
% 
%   5. If a question has multiple parts, then first delete lines 40-41
%   6. If a question does not have parts, then first delete lines 43-69
%   
%   7. Create versions of the question when possible. Use commands defined in 
%      tufte-tweaks.sty to do so. Its easier than you think

% \noprintanswers
% \setcounter{rolldice}{0}
% \printrubric

\ifnumequal{\value{rolldice}}{0}{
  % variables 
  \renewcommand{\vbone}{}
  \renewcommand{\vbtwo}{}
  \renewcommand{\vbthree}{}
  \renewcommand{\vbfour}{}
  \renewcommand{\vbfive}{}
  \renewcommand{\vbsix}{}
  \renewcommand{\vbseven}{}
  \renewcommand{\vbeight}{}
  \renewcommand{\vbnine}{}
  \renewcommand{\vbten}{}
}{
  \ifnumequal{\value{rolldice}}{1}{
    % variables 
    \renewcommand{\vbone}{}
    \renewcommand{\vbtwo}{}
    \renewcommand{\vbthree}{}
    \renewcommand{\vbfour}{}
    \renewcommand{\vbfive}{}
    \renewcommand{\vbsix}{}
    \renewcommand{\vbseven}{}
    \renewcommand{\vbeight}{}
    \renewcommand{\vbnine}{}
    \renewcommand{\vbten}{}
  }{
    \ifnumequal{\value{rolldice}}{2}{
      % variables 
      \renewcommand{\vbone}{}
      \renewcommand{\vbtwo}{}
      \renewcommand{\vbthree}{}
      \renewcommand{\vbfour}{}
      \renewcommand{\vbfive}{}
      \renewcommand{\vbsix}{}
      \renewcommand{\vbseven}{}
      \renewcommand{\vbeight}{}
      \renewcommand{\vbnine}{}
      \renewcommand{\vbten}{}
    }{
      % variables 
      \renewcommand{\vbone}{}
      \renewcommand{\vbtwo}{}
      \renewcommand{\vbthree}{}
      \renewcommand{\vbfour}{}
      \renewcommand{\vbfive}{}
      \renewcommand{\vbsix}{}
      \renewcommand{\vbseven}{}
      \renewcommand{\vbeight}{}
      \renewcommand{\vbnine}{}
      \renewcommand{\vbten}{}
    }
  }
}

\question Let $R$ be the region in the first quadrant enclosed by the graphs $f(x)=8x^3$ and 
$g(x)= \sin(\pi x)$ as shown in the adjoining figure.	

\insertQR{QRC}

%\watchout

%\ifprintanswers
  \begin{marginfigure}
    \figinit{pt}
      \figpt 1:(-10,-10)
      \figpt 2:(-10,0)
      \figpt 10:(0,0)
      \figpt 20:(20,5)
      \figpt 30:(30,20)
      \figpt 40:(40,60)
      \figpt 50:(20,32)
      \figpt 60:(40,48)
      \figpt 70:(50,38)
      \figpt 80:(60,20)
      \figpt 90:(50,90)
      % extremeties
      \def\Xmax{100}
      \def\Ymax{80}
      \def\Xmin{-100}
      \def\Ymin{-40}
      % pts for numbering the axes (5, 10, 15...)
      \figpt 204:$\tiny\text{O}$(0,0)
      \figpt 205:$\tiny\text{1}$(0,48)
      \figpt 206:$\tiny\text{R}$(15,20)
      % label graph
      \figpt 100:(\Xmax,0)
      \figpt 101:(0,\Ymax)
      \figpt 102:(0,\Ymin)
    \figdrawbegin{}
      \figset arrowhead(length=4, fillmode=yes)
      \figdrawaxes 10(\Xmin, \Xmax, \Ymin, \Ymax)
      \figdrawcurve [1,10,50,60,70,80]
      \figdrawcurve [2,10,20,30,40,90]
    \figdrawend
    \figvisu{\figBoxA}{}{%
      \figwritee 100: $x$(2 pt)
      \figwritesw 204: (2 pt)
      \figwritee	206: $R$(1 pt)
      \figsetmark {$-$}
      \figwritew 205: $1$(2 pt)
    }
    \centerline{\box\figBoxA}
  \end{marginfigure}
%\fi 

\begin{parts}
  \part[2] Write an equation for the line tangent to the graph of $f$ at $x=\dfrac{1}{2}$.

  \insertQR{QRC}
\begin{solution}[\mcq]
    \begin{align}
      f(x) = 8x^3 \Rightarrow f(\dfrac{1}{2}) = 1 
    \end{align}    
    To find the tangent at a we must first find the derivative of $f(x)$ to get slope.
    \begin{align}
      f'(x) = 24x^2 \Rightarrow f'(\dfrac{1}{2}) = 6 
    \end{align}    
    Therefore equation of the tangent at $x=\dfrac{1}{2}$ is
    \begin{align}
      y=1+6(x-\dfrac{1}{2}) 
    \end{align}
  \end{solution}

  \part[4] Find the area of $R$.
\begin{solution}[\mcq]
    Let the area of $R$ be $A_R$. Since $R$ extends from $x=0$ to $x=\dfrac{1}{2}$ area 
    can be calculated as follows,
    \begin{align}
      A_R &= \int_0^{1/2} (g(x)-f(x))\ud x \\ \nonumber
          &= \int_0^{1/2} (\sin(\pi x)-8x^3)\ud x \\ \nonumber
          &= \left[-\dfrac{1}{\pi}\cos(\pi x)-2x^4\right]_0^{1/2} \\ \nonumber
          &= -\dfrac{1}{8}+\dfrac{1}{\pi} \nonumber
    \end{align}
  \end{solution}

  \part[3] Write, but do not, an integral expression for the volume of the solid generated
  when $R$ is rotated about the horizontal line $y=1$.
\begin{solution}[\mcq]
    Let Volume of generated solid be $V$. To calculate the volume $V$ we must sum up the
    areas of all the discs of thickness $\ud x$ from $x=0$ to $x=1/2$. The area $dA$ of each
    disc being the $\pi((1-f(x))^2-(1-g(x))^2)\ud x$. Therefore expression for Volume is,
    \begin{align}
      V = \pi((1-8x^3)^2-(1-\sin(\pi x))^2)\ud x \\ \nonumber 
    \end{align}
  \end{solution}

\end{parts}

\ifprintrubric
  \begin{table}
  	\begin{tabular}{ p{5cm}p{5cm} }
  		\toprule % in brief (4-6 words), what should a grader be looking for for insights & formulations
  		  \sc{\textcolor{blue}{Insight}} & \sc{\textcolor{blue}{Formulation}} \\ 
  		\midrule % ***** Insights & formulations ******
  		\toprule % final numerical answers for the various versions
        \sc{\textcolor{blue}{If question has $\ldots$}} & \sc{\textcolor{blue}{Final answer}} \\
  		\midrule % ***** Numerical answers (below) **********
  		\bottomrule
  	\end{tabular}
  \end{table}
\fi
