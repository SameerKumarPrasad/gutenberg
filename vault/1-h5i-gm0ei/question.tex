% This is an empty shell file placed for you by the 'examiner' script.
% You can now fill in the TeX for your question here.

% Now, down to brasstacks. ** Writing good solutions is an Art **. 
% Eventually, you will find your own style. But here are some thoughts 
% to get you started: 
%
%   1. Write the solution as if you are writing it for your favorite
%      14-17 year old to help him/her understand. Could be your nephew, 
%      your niece, a cousin perhaps or probably even you when you 
%      were that age. Just write for them.
%
%   2. Use margin-notes to "talk" to students about the critical insights
%      in the question. The tone can be - in fact, should be - informal
%
%   3. Don't shy away from creating margin-figures you think will help
%      students understand. Yes, it is a little more work per question. 
%      But the question & solution will be written only once. Make that
%      attempt at writing a solution count.
%
%   4. At the same time, do not be too verbose. A long solution can
%      - at first sight - make the student think, "God, that is a lot to know".
%      Our aim is not to scare students. Rather, our aim should be to 
%      create many "Aha!" moments everyday in classrooms around the world
% 
%   5. Ensure that there are *no spelling mistakes anywhere*. We are an 
%      education company. Bad spellings suggest that we ourselves 
%      don't have any education. Also, use American spellings by default
% 
%   6. If a question has multiple parts, then first delete lines 40-41
%   7. If a question does not have parts, then first delete lines 43-69

% \left( 1 + r + \frac{1}{r} \right)
% \left( 1 + r^2 + \frac{1}{r^2} \right)

\question The sum of three numbers in geometric progression in $21$ and the sum of their
squares is $189$. Find the numbers

\insertQR{}

\ifprintanswers
	\marginnote[5.5cm]{ $(a^2+b^2+c^2) = (a+b+c)^2 - 2\cdot(ab+bc+ca)$ }
\fi

\begin{solution}
	Let the three numbers be $a$, $\frac{a}{r}$ and $ar$. And so, given that
	
	\begin{align}
		a\cdot\left( 1 + r + \frac{1}{r} \right) &= 21 \\
		a^2\cdot\left( 1 + r^2 + \frac{1}{r^2} \right) &= 189
	\end{align}
	, we can infer 
	\begin{align}
		&a^2\cdot\left[ \left( 1 + r + \frac{1}{r} \right)^2 - 
		2\cdot\left( r\cdot1 + 1\cdot\frac{1}{r} + r\cdot\frac{1}{r} \right)\right] = 189 \\
		&\Rightarrow (21)^2 - 2a^2\cdot\left( 1 + r + \frac{1}{r} \right) = 189 \\
		&\Rightarrow a^2\cdot\left( 1 + r + \frac{1}{r} \right) = \dfrac{441-189}{2} = 126 \\
		&\Rightarrow \dfrac{a^2\cdot\left( 1 + r + \frac{1}{r} \right)}
		{a\cdot\left( 1 + r + \frac{1}{r} \right)} = a = \dfrac{126}{21} = 6 \\
		&\Rightarrow \left( 1 + r + \frac{1}{r} \right) = \dfrac{21}{6} \\
		&\text{ or, } 2r^2 - 5r + 2 = 0 \Rightarrow r = 2, \frac{1}{2}
	\end{align}
	Hence, the three numbers are either $(12,6,3)$ if $r=\frac{1}{2}$ or $(3,6,12)$ if $r=2$
\end{solution}
