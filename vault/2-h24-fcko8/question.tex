% This is an empty shell file placed for you by the 'examiner' script.
% You can now fill in the TeX for your question here.

% Now, down to brasstacks. ** Writing good solutions is an Art **. 
% Eventually, you will find your own style. But here are some thoughts 
% to get you started: 
%
%   1. Write the solution as if you are writing it for your favorite
%      14-17 year old to help him/her understand. Could be your nephew, 
%      your niece, a cousin perhaps or probably even you when you 
%      were that age. Just write for them.
%
%   2. Use margin-notes to "talk" to students about the critical insights
%      in the question. The tone can be - in fact, should be - informal
%
%   3. Don't shy away from creating margin-figures you think will help
%      students understand. Yes, it is a little more work per question. 
%      But the question & solution will be written only once. Make that
%      attempt at writing a solution count.
%
%   4. At the same time, do not be too verbose. A long solution can
%      - at first sight - make the student think, "God, that is a lot to know".
%      Our aim is not to scare students. Rather, our aim should be to 
%      create many "Aha!" moments everyday in classrooms around the world
% 
%   5. Ensure that there are *no spelling mistakes anywhere*. We are an 
%      education company. Bad spellings suggest that we ourselves 
%      don't have any education. Also, use American spellings by default
% 
%   6. If a question has multiple parts, then first delete lines 40-41
%   7. If a question does not have parts, then first delete lines 43-69

\question[4] Three numbers form an arithmetic progression. The sum of the numbers is equal to 3 and 
the sum of their cubes is equal to 4. Find the numbers

\insertQR{QRC}

\ifprintanswers
  % stuff to be shown only in the answer key - like explanatory margin figures
  \marginnote[3cm] {Setting $a=1$ in $(2)$}
\fi 

\begin{solution}[\fullpage]
  Let the numbers be $a-d$, $a$ and $a+d$. And here is what we know 
  \begin{align}
     (a-d) + a + (a + d) &= 3 \Rightarrow a = 1 \\
     (1-d)^3 + 1^3 + (1+d)^3 &= 4 \\
     \underbrace{(1-3d+3d^2-d^3)}_{(1-d)^3} + 1 +
            \underbrace{(1+3d+3d^2+d^3)}_{(1+d)^3} &= 4 \\
     \Rightarrow d &= \dfrac{1}{\sqrt{6}}
  \end{align}
  
  The numbers, therefore, are $\left(1-\frac{1}{\sqrt{6}}\right)$, $1$ and 
  $\left(1+\frac{1}{\sqrt{6}}\right)$
\end{solution}
