
\ifnumequal{\value{rolldice}}{0}{
  \renewcommand{\va}{3}
  \renewcommand{\vb}{2}
  \renewcommand{\vc}{5}
}{
  \ifnumequal{\value{rolldice}}{1}{
    \renewcommand{\va}{4}
    \renewcommand{\vb}{3}
    \renewcommand{\vc}{7}
  }{
    \ifnumequal{\value{rolldice}}{2}{
      \renewcommand{\va}{5}
      \renewcommand{\vb}{4}
      \renewcommand{\vc}{9}
    }{
      \renewcommand{\va}{6}
      \renewcommand{\vb}{5}
      \renewcommand{\vc}{11}
    }
  }
}

\question[3] Find the value of $M$ for which the following identity is true.
\begin{align}
  \dfrac{\sin\theta+\sin \vb\theta}{\cos\theta+\cos \vb\theta} 
    = \tan M\theta \nonumber
\end{align}

\watchout

\begin{solution}[\halfpage]
  Rewrite the Left Hand Side (L.H.S) as follows,
  \begin{align}
    \text{L.H.S} &= \dfrac{\sin(\va\theta-\vb\theta)+\sin(\va\theta+\vb\theta)}
                      {\cos(\va\theta-\vb\theta)+\cos(\va\theta+\vb\theta)} \\
                 &= \dfrac{2\sin \va\theta\cos \vb\theta}
                      {2\cos \va\theta\cos \vb\theta} \\
                 &= \tan \vc\theta
  \end{align}
  Therefore, the identity is true for $M=\vc$.
\end{solution}

\ifprintanswers\begin{codex}$M=\vc$\end{codex}\fi
