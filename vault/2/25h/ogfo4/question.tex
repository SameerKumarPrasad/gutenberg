
\ifnumequal{\value{rolldice}}{0}{
  \renewcommand{\va}{3}
  \renewcommand{\vb}{5}
}{
  \ifnumequal{\value{rolldice}}{1}{
    \renewcommand{\va}{5}
    \renewcommand{\vb}{7}
  }{
    \ifnumequal{\value{rolldice}}{2}{
      \renewcommand{\va}{7}
      \renewcommand{\vb}{9}
    }{
      \renewcommand{\va}{4}
      \renewcommand{\vb}{7}
    }
  }
}
\SUBTRACT\vb\va\d

\question[3] If $\dfrac{\sin\theta + \sin \va\theta + \sin \vb\theta}
  	{\cos \theta + \cos \va\theta + \cos \vb\theta} = \tan M\theta $, then
  	find the value of $M$.

\begin{solution}[\halfpage]
  Let us rewrite the Left Hand Side (L.H.S) as follows,
  \begin{align}
    \text{L.H.S} &= \dfrac{\sin(\va\theta - \d\theta ) + \sin \va\theta + 
    						\sin (\va\theta + \d\theta)}
  						  {\cos (\va\theta - \d\theta) + \cos \va\theta + 
  							\cos (\va\theta + \d\theta)} \\
				 &= \dfrac{2\sin \va\theta \cos \d\theta + \sin \va\theta}
				  		  {2\cos \va\theta \cos \d\theta + \cos \va\theta} \\
				 &= \dfrac{\sin \va\theta(2\cos \d\theta + 1)}
				  		  {\cos \va\theta(2\cos \d\theta + 1)} \\
				 &= \tan \va\theta
  \end{align}
  Therefore $M=\va$
\end{solution}

\ifprintanswers\begin{codex}$M=\va$\end{codex}\fi
