
\ifnumequal{\value{rolldice}}{0}{
  % variables 
  \renewcommand{\va}{2}
  \renewcommand{\vb}{1}
  \renewcommand{\vc}{1}
  \renewcommand{\vd}{2}
  \renewcommand{\ve}{5}
  \renewcommand{\vf}{1}
  \renewcommand{\vg}{3}
  \renewcommand{\vh}{3}
  \renewcommand{\vi}{6}
  \renewcommand{\vj}{3}
}{
  \ifnumequal{\value{rolldice}}{1}{
    % variables 
    \renewcommand{\va}{3}
    \renewcommand{\vb}{2}
    \renewcommand{\vc}{2}
    \renewcommand{\vd}{2}
    \renewcommand{\ve}{4}
    \renewcommand{\vf}{1}
    \renewcommand{\vg}{4}
    \renewcommand{\vh}{3}
    \renewcommand{\vi}{5}
    \renewcommand{\vj}{2}
  }{
    \ifnumequal{\value{rolldice}}{2}{
      % variables 
      \renewcommand{\va}{3}
      \renewcommand{\vb}{2}
      \renewcommand{\vc}{4}
      \renewcommand{\vd}{2}
      \renewcommand{\ve}{5}
      \renewcommand{\vf}{1}
      \renewcommand{\vg}{9}
      \renewcommand{\vh}{1}
      \renewcommand{\vi}{8}
      \renewcommand{\vj}{1}
    }{
      % variables 
      \renewcommand{\va}{2}
      \renewcommand{\vb}{1}
      \renewcommand{\vc}{1}
      \renewcommand{\vd}{2}
      \renewcommand{\ve}{3}
      \renewcommand{\vf}{1}
      \renewcommand{\vg}{7}
      \renewcommand{\vh}{1}
      \renewcommand{\vi}{6}
      \renewcommand{\vj}{1}
    }
  }
}

\ADD\vi{0}\a
\ADD\vj{0}\b
\SUBTRACT\vd\a\distx
\SUBTRACT\ve\b\disty
\gcalcexpr[0]\rsq{(\distx * \distx) + (\disty * \disty)}
\gcalcexpr[0]\tp{(\va * \ve) + (\vb * \vd)}

\question[4] The line $l_1(\va x -\vb y =-\vc)$ touches a circle 
with center $O$ at point $P(\vd, \ve)$. Point $O$ lies on the line
$l_2(\vf x-\vg y =-\vh)$. What is the equation of the circle?


\watchout

\ifprintanswers
  % stuff to be shown only in the answer key - like explanatory margin figures
  %\begin{marginfigure}
    \figinit{pt}
      \figpt 100:(0,0)
      \figpt 101:(-20,20)
      \figpt 102:(-40,0)
      \figpt 103:(0,40)
      \figpt 104:(-30,-10)
      \figpt 105:(30,10)
    \figdrawbegin{}
      \figdrawaltitude 5 [100,102,103]
      \figdrawcirc 100 (28)
      \figdrawline [102,103]
      \figdrawline [104,105]
    \figdrawend
    \figvisu{\figBoxA}{}{%
      \figwritese 100:$O\text{($a$,$b$)}$(2pt)
      \figwritenw 101:$P\text{($\vd$, $\ve$)}$(2pt)
      \figwritesw 102:$l_1$(3pt)
      \figwritesw 104:$l_2$(3pt)
    }
    \centerline{\box\figBoxA}
  %\end{marginfigure}
\fi 

\begin{solution}[\halfpage]
  Let the co-ordinates of the center of the circle $O$ be $(a,b)$
  and it's radius be $r$. Since $l_1$ touches the circle at $P$ 
  therefore $OP \perp l_1$,
  \begin{align}
    \left(\dfrac{b-\ve}{a-\vd}\right)\cdot
      \left(\dfrac{\va}{\vb}\right)&=-1 \\
                     \vb a + \va b &= \tp
  \end{align}
  Since $(a,b)$ lies on $l_2$,
  \begin{align}
    \vf a -\vg b = -\vh
  \end{align}
  Using equations (2) and (3) we get $a=\a$ and $b=\b$. Now
  we can use distance formula to get $r$, the distance
  between $O$ and $P$.
  \begin{align}
    r^2 &= (\vd-\a)^2+(\ve-\b)^2 \\
        &= \rsq
  \end{align}
  Therefore, equation of the circle is,
  \begin{align}
    (x-\a^2)+(y-\b^2)=\rsq
  \end{align}  
\end{solution}

\ifprintanswers\begin{codex}
  $(x-\a^2)+(y-\b^2)=\rsq$
\end{codex}\fi

