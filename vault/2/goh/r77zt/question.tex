% \noprintanswers
% \setcounter{rolldice}{0}
% \printrubric

\ifnumequal{\value{rolldice}}{0}{
  % variables 
  \renewcommand{\vbone}{2}
  \renewcommand{\vbtwo}{1}
  \renewcommand{\vbthree}{1}
  \renewcommand{\vbfour}{2}
  \renewcommand{\vbfive}{5}
  \renewcommand{\vbsix}{1}
  \renewcommand{\vbseven}{3}
  \renewcommand{\vbeight}{3}
  \renewcommand{\vbnine}{6}
  \renewcommand{\vbten}{3}
}{
  \ifnumequal{\value{rolldice}}{1}{
    % variables 
    \renewcommand{\vbone}{3}
    \renewcommand{\vbtwo}{2}
    \renewcommand{\vbthree}{2}
    \renewcommand{\vbfour}{2}
    \renewcommand{\vbfive}{4}
    \renewcommand{\vbsix}{1}
    \renewcommand{\vbseven}{4}
    \renewcommand{\vbeight}{3}
    \renewcommand{\vbnine}{5}
    \renewcommand{\vbten}{2}
  }{
    \ifnumequal{\value{rolldice}}{2}{
      % variables 
      \renewcommand{\vbone}{3}
      \renewcommand{\vbtwo}{2}
      \renewcommand{\vbthree}{4}
      \renewcommand{\vbfour}{2}
      \renewcommand{\vbfive}{5}
      \renewcommand{\vbsix}{1}
      \renewcommand{\vbseven}{9}
      \renewcommand{\vbeight}{1}
      \renewcommand{\vbnine}{8}
      \renewcommand{\vbten}{1}
    }{
      % variables 
      \renewcommand{\vbone}{2}
      \renewcommand{\vbtwo}{1}
      \renewcommand{\vbthree}{1}
      \renewcommand{\vbfour}{2}
      \renewcommand{\vbfive}{3}
      \renewcommand{\vbsix}{1}
      \renewcommand{\vbseven}{7}
      \renewcommand{\vbeight}{1}
      \renewcommand{\vbnine}{6}
      \renewcommand{\vbten}{1}
    }
  }
}

\ADD\vbnine{0}\a
\ADD\vbten{0}\b
\SUBTRACT\vbfour\a\distx
\SUBTRACT\vbfive\b\disty
\gcalcexpr[0]\rsq{(\distx * \distx) + (\disty * \disty)}
\gcalcexpr[0]\tp{(\vbone * \vbfive) + (\vbtwo * \vbfour)}

\question The line $l_1(\vbone x -\vbtwo y =-\vbthree)$ touches a circle 
with center $O$ at point $P(\vbfour, \vbfive)$. Point $O$ lies on the line
$l_2(\vbsix x-\vbseven y =-\vbeight)$. What is the equation of the circle?

\insertQR{}

\watchout

\ifprintanswers
  % stuff to be shown only in the answer key - like explanatory margin figures
  \begin{marginfigure}
    \figinit{pt}
      \figpt 100:(0,0)
      \figpt 101:(-20,20)
      \figpt 102:(-40,0)
      \figpt 103:(0,40)
      \figpt 104:(-30,-10)
      \figpt 105:(30,10)
    \figdrawbegin{}
      \figdrawaltitude 5 [100,102,103]
      \figdrawcirc 100 (28)
      \figdrawline [102,103]
      \figdrawline [104,105]
    \figdrawend
    \figvisu{\figBoxA}{}{%
      \figwritese 100:$O\text{($a$,$b$)}$(2pt)
      \figwritenw 101:$P\text{($\vbfour$, $\vbfive$)}$(2pt)
      \figwritesw 102:$l_1$(3pt)
      \figwritesw 104:$l_2$(3pt)
    }
    \centerline{\box\figBoxA}
  \end{marginfigure}
\fi 

\begin{solution}
  Let the co-ordinates of the center of the circle $O$ be $(a,b)$
  and it's radius be $r$. Since $l_1$ touches the circle at $P$ 
  therefore $OP \perp l_1$,
  \begin{align}
    \left(\dfrac{b-\vbfive}{a-\vbfour}\right)\cdot
      \left(\dfrac{\vbone}{\vbtwo}\right)&=-1 \\
                     \vbtwo a + \vbone b &= \tp
  \end{align}
  Since $(a,b)$ lies on $l_2$,
  \begin{align}
    \vbsix a -\vbseven b = -\vbeight
  \end{align}
  Using equations (2) and (3) we get $a=\a$ and $b=\b$. Now
  we can use distance formula to get $r$, the distance
  between $O$ and $P$.
  \begin{align}
    r^2 &= (\vbfour-\a)^2+(\vbfive-\b)^2 \\
        &= \rsq
  \end{align}
  Therefore, equation of the circle is,
  \begin{align}
    (x-\a^2)+(y-\b^2)=\rsq
  \end{align}  
\end{solution}