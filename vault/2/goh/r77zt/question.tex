
\ifnumequal{\value{rolldice}}{0}{
  % variables 
  \renewcommand{\va}{2}
  \renewcommand{\vb}{1}
  \renewcommand{\vc}{1}
  \renewcommand{\vd}{2}
  \renewcommand{\ve}{5}
  \renewcommand{\vf}{1}
  \renewcommand{\vg}{3}
  \renewcommand{\vh}{3}
  \renewcommand{\vi}{6}
  \renewcommand{\vj}{3}
}{
  \ifnumequal{\value{rolldice}}{1}{
    % variables 
    \renewcommand{\va}{3}
    \renewcommand{\vb}{2}
    \renewcommand{\vc}{2}
    \renewcommand{\vd}{2}
    \renewcommand{\ve}{4}
    \renewcommand{\vf}{1}
    \renewcommand{\vg}{4}
    \renewcommand{\vh}{3}
    \renewcommand{\vi}{5}
    \renewcommand{\vj}{2}
  }{
    \ifnumequal{\value{rolldice}}{2}{
      % variables 
      \renewcommand{\va}{3}
      \renewcommand{\vb}{2}
      \renewcommand{\vc}{4}
      \renewcommand{\vd}{2}
      \renewcommand{\ve}{5}
      \renewcommand{\vf}{1}
      \renewcommand{\vg}{9}
      \renewcommand{\vh}{1}
      \renewcommand{\vi}{8}
      \renewcommand{\vj}{1}
    }{
      % variables 
      \renewcommand{\va}{2}
      \renewcommand{\vb}{1}
      \renewcommand{\vc}{1}
      \renewcommand{\vd}{2}
      \renewcommand{\ve}{3}
      \renewcommand{\vf}{1}
      \renewcommand{\vg}{7}
      \renewcommand{\vh}{1}
      \renewcommand{\vi}{6}
      \renewcommand{\vj}{1}
    }
  }
}

\ADD\vi{0}\a
\ADD\vj{0}\b
\SUBTRACT\vd\a\distx
\SUBTRACT\ve\b\disty
\gcalcexpr[0]\rsq{(\distx * \distx) + (\disty * \disty)}
\gcalcexpr[0]\tp{(\va * \ve) + (\vb * \vd)}

\question[4] The line $L_1: \WRITELINEEQUALSZERO\va\vb\vc\vb$ touches a circle 
with center $O$ at point $P = (\vd, \ve)$. Point $O$ lies on the line
$L_2: \WRITELINEEQUALSZERO\vf\vg\vh\vg$. What is the equation of the circle?

\watchout

\ifprintanswers
  % stuff to be shown only in the answer key - like explanatory margin figures
  \figinit{pt}
    \def\rad{50}
    \figpt 0:$O$(0,0)
    \figptcirc 1 :$P$: 0;\rad (135)
    \figvectN 500 [0,1]
    \figpttra 2 ::= 1 /1,500/
    \figpttra 3 ::= 1 /-1,500/
    \figptcirc 7 :: 0;\rad (60)
    \figvectP 501 [0,7]
    \figpttra 8 ::= 0 /1.5, 501/
    \figpttra 9 ::= 0 /-1.5, 501/
  \figdrawbegin{}
    \figdrawcirc 0 (\rad)
    \figdrawline [2,1,3]
    \figdrawline [0,1]
    \figdrawline [9,0,8]
  \figdrawend
  \figvisu{\figBoxA}{}{%
    \Large
    \figwritene 3:$L_1$ (4)
    \figwritene 8:$L_2$(4)
    \figset write(mark=$\bullet$)
    \figwritese 0:(4)
    \figwritenw 1:(4)
  }
  \vspace{0.7cm}
  \centerline{\box\figBoxA}
\fi 

\begin{solution}[\halfpage]
  \textbf{Insight \#1:$L_1\perp OP$} 

  $OP$ is the radius of the circle. And as $L_1$ only touches the circle 
  at $P$, $L_1$ is tangent to the circle at $P$. And the tangent and radius at a point 
  are perpendicular to  each other.

  If the center $O=(a,b)$, then 
  \begin{align}
    m_{OP}\times m_{L_1} &= -1 \\
    \underbrace{\dfrac{b-\ve}{a-\vd}}_{m_{OP}}\times\underbrace{\WRITEFRAC\va\vb}_{m_{L_1}} &= -1 \\
    \implies\startpoly\WRITEPOLYTERM[a]\vb{1}\WRITEPOLYTERM[b]\va{1} &= \tp
  \end{align}

  Since $O = (a,b)$ also lies on $L_2$,
  \begin{align}
    \underbrace{\startpoly\WRITEPOLYTERM[a]\vf{1}\WRITEPOLYTERM[b]{-\vg}{1} + \vh}_{x=a,y=b} = 0\implies
    \startpoly\WRITEPOLYTERM[a]\vf{1}\WRITEPOLYTERM[b]{-\vg}{1} = -\vh
  \end{align}
  Solving equations (3) and (4), we get 
  \begin{align}
    a &= \a\text{ and } b=\b 
  \end{align}

  \textbf{Step \#2: Find $R^2$}

  This bit is easy. 
  \[ R^2 = (\a-\vd)^2 + (\b-\ve)^2 = \rsq \] 
  Which means, the \textbf{equation of the circle} is 
  \[ (x-\a)^2 + (y-\b)^2 = \rsq \]
\end{solution}

\ifprintanswers\begin{codex}
  $(x-\a)^2+(y-\b)^2=\rsq$
\end{codex}\fi

