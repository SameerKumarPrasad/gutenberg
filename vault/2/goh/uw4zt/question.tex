
\ifnumequal{\value{rolldice}}{0}{
  % variables 
  \renewcommand{\va}{positive}
  \renewcommand{\vb}{positive}
  \renewcommand{\vc}{3}
  \renewcommand{\vd}{3}
  \renewcommand{\ve}{-3}
  \renewcommand{\vf}{-3}
  \renewcommand{\vg}{-}
}{
  \ifnumequal{\value{rolldice}}{1}{
    % variables 
    \renewcommand{\va}{positive}
    \renewcommand{\vb}{negative}
    \renewcommand{\vc}{3}
    \renewcommand{\vd}{-3}
    \renewcommand{\ve}{-3}
    \renewcommand{\vf}{+3}
    \renewcommand{\vg}{-}
    \renewcommand{\vh}{-3}
  }{
    \ifnumequal{\value{rolldice}}{2}{
      % variables 
      \renewcommand{\va}{negative}
      \renewcommand{\vb}{positive}
      \renewcommand{\vc}{-3}
      \renewcommand{\vd}{3}
      \renewcommand{\ve}{+3}
      \renewcommand{\vf}{-3}
      \renewcommand{\vg}{-}
      \renewcommand{\vh}{+3}
    }{
      % variables 
      \renewcommand{\va}{negative}
      \renewcommand{\vb}{negative}
      \renewcommand{\vc}{-3}
      \renewcommand{\vd}{-3}
      \renewcommand{\ve}{+3}
      \renewcommand{\vf}{+3}
      \renewcommand{\vh}{-3}
    }
  }
}
\ABSVALUE\vc\r
\SQUARE\vc\rsq

\LINESLOPE\vc{0}{0}\vd\vm\vn
\LINEINTERCEPT\vc{0}\vm\vn\vx\vy

\question A circle with center at point $C$ touches the \textbf{\va} x-axis 
at point $A$ and the \textbf{\vb} y-axis at point $B$. Points $A$ and $B$ are 
both $\r$ units from the origin.

\watchout

\ifprintanswers  
  \MULTIPLY\vc{10}\xcept
  \MULTIPLY\vd{10}\ycept
  \MULTIPLY\r{10}\RAD
  % stuff to be shown only in the answer key - like explanatory margin figures
    \figinit{pt}
      \def\min{-60}
      \def\max{60}      
      \figpt 100:(0,0)
      \figpt 101:(\xcept,0)
      \figpt 102:(0,\ycept)
      \figpt 103:(\xcept, \ycept)
    \figdrawbegin{}
      %\figdrawaxes 100(\min, \max, \min, \max)
      \drawAxes {100} \min \max \min \max
      \figdrawcirc 103(\RAD)
      \figdrawline [101,102]
    \figdrawend
    \figvisu{\figBoxA}{}{%
      \Large
      \figset write(mark=$\bullet$)      
      \figwrites 101:$A$(3)
      \figwritew 102:$B$(3)
      \figwrites 103:$C$(3)
    }
    \vspace{0.7cm}
    \centerline{\box\figBoxA}
\fi 


\begin{parts}

\part[2] Find the equation of the circle?
\begin{solution}[\mcq]
  Given that both $A$ and $B$ are $\r$ units from the origin \textbf{and} given that 
  they lie on the \textbf{\va} x and \textbf{\vb} y-axis respectively, \textbf{we can infer that}
  \begin{align}
    A &= (\vc, 0) \\
    B &= (0,\vd)
  \end{align} 

  Also, given that the circle \textbf{only touches} the axes at $A$ and $B$ means 
  that both the axes are \textbf{tangents} to the circle. Which, in turn, means 
  \begin{align}
    CA\perp\text{ x-axis and }&CB\perp\text{ y-axis} \\
     \implies C = (\vc,\vd)&\text{ and radius } = \r\text{ units}
  \end{align}

  The \textbf{general equation} of a circle with center at $(a, b)$ 
  and radius $R$ is,
  \[ (x-a)^2 + (y-b)^2 = R^2 \]
  Our circle would therefore be 
  \[ (x\ve)^2 +(y\vf)^2 = \rsq \] 
\end{solution}

\part[1] Find the equation of the line that passes through the points
$A$ and $B$?

\begin{solution}[\mcq]
  A line passing through $A$ and $B$ would be given by 
  \begin{align}
    \dfrac{y-0}{x-\vc} &= \dfrac{\vd - 0}{0-\vc} \\
    \implies y &= \WRITELINE\vm\vn\vx\vy
  \end{align}
\end{solution}

\end{parts}

\ifprintanswers
\begin{codex}
  $(a)\,(x\ve)^2 +(y\vf)^2=\rsq\qquad (b)\, y=\WRITELINE\vm\vn\vx\vy$
\end{codex}
\fi
