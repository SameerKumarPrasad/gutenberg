
\ifnumequal{\value{rolldice}}{0}{
  % variables 
  \renewcommand{\va}{positive}
  \renewcommand{\vb}{positive}
  \renewcommand{\vc}{3}
  \renewcommand{\vd}{3}
  \renewcommand{\ve}{-3}
  \renewcommand{\vf}{-3}
  \renewcommand{\vh}{+3}
}{
  \ifnumequal{\value{rolldice}}{1}{
    % variables 
    \renewcommand{\va}{positive}
    \renewcommand{\vb}{negative}
    \renewcommand{\vc}{3}
    \renewcommand{\vd}{-3}
    \renewcommand{\ve}{-3}
    \renewcommand{\vf}{+3}
    \renewcommand{\vg}{-}
    \renewcommand{\vh}{-3}
  }{
    \ifnumequal{\value{rolldice}}{2}{
      % variables 
      \renewcommand{\va}{negative}
      \renewcommand{\vb}{positive}
      \renewcommand{\vc}{-3}
      \renewcommand{\vd}{3}
      \renewcommand{\ve}{+3}
      \renewcommand{\vf}{-3}
      \renewcommand{\vg}{-}
      \renewcommand{\vh}{+3}
    }{
      % variables 
      \renewcommand{\va}{negative}
      \renewcommand{\vb}{negative}
      \renewcommand{\vc}{-3}
      \renewcommand{\vd}{-3}
      \renewcommand{\ve}{+3}
      \renewcommand{\vf}{+3}
      \renewcommand{\vh}{-3}
    }
  }
}
\ABSVALUE\vc\r
\SQUARE\vc\rsq

\question A circle with center at point $C$ touches the $\va$ x-axis 
at point $A$ and the $\vb$ y-axis at point $B$. Points $A$ and $B$ are 
both $\r$ units from the origin.


\watchout

\ifprintanswers  
  \MULTIPLY\vc{10}\xcept
  \MULTIPLY\vd{10}\ycept
  \MULTIPLY\r{10}\RAD
  % stuff to be shown only in the answer key - like explanatory margin figures
  \begin{marginfigure}
    \figinit{pt}
      \def\min{-60}
      \def\max{60}      
      \figpt 100:(0,0)
      \figpt 101:(\xcept,0)
      \figpt 102:(0,\ycept)
      \figpt 103:(\xcept, \ycept)
    \figdrawbegin{}
      %\figdrawaxes 100(\min, \max, \min, \max)
      \drawAxes {100} \min \max \min \max
      \figdrawcirc 103(\RAD)
      \figdrawline [101,102]
    \figdrawend
    \figvisu{\figBoxA}{}{%
      \figwritese 100:$O$(2pt)      
      \figwrites 101:$A$(2pt)
      \figwritew 102:$B$(2pt)
      \figset write(mark=$\bullet$)      
      \figwrites 103:$C\text{($\xcept$, $\ycept$)}$(2pt)
    }
    \centerline{\box\figBoxA}
  \end{marginfigure}
\fi 


\begin{parts}

\part[2] Find the equation of the circle?
\begin{solution}[\mcq]
  The general equation of a circle with its center at $(a, b)$ 
  and radius $r$ is,
  \begin{align}
                &(x-a)^2 + (y-b)^2 = r^2
  \end{align}
  Our circle touches the x and y axes therefore it's radius is
  $\r$ units and its center is at $(\vc,\vd)$. The equation
  for this circle would be,
  \begin{align}
    (x\ve)^2+(y\vf)^2&=\rsq 
  \end{align}
\end{solution}

\part[1] Find the equation of the line that passes through the points
$A$ and $B$?

\begin{solution}[\mcq]
  The line passing through the points of contact $(\vc,0)$
  and $(0,\vd)$ is given by,
  \begin{align}
    y = \vg x\vh
  \end{align}
\end{solution}

\end{parts}

\ifprintanswers\begin{codex}
  (a) $(x\ve)^2+(y\vf)^2=\rsq$ \\
  (b) $y = \vg x\vh$
\end{codex}\fi

