

\ifnumequal{\value{rolldice}}{0}{
  % variables 
  \renewcommand{\vbone}{4}
  \renewcommand{\vbtwo}{5}
  \renewcommand{\vbthree}{20}
}{
  \ifnumequal{\value{rolldice}}{1}{
    % variables 
    \renewcommand{\vbone}{6}
    \renewcommand{\vbtwo}{3}
    \renewcommand{\vbthree}{11}
  }{
    \ifnumequal{\value{rolldice}}{2}{
      % variables 
      \renewcommand{\vbone}{5}
      \renewcommand{\vbtwo}{7}
      \renewcommand{\vbthree}{8}
    }{
      % variables 
      \renewcommand{\vbone}{8}
      \renewcommand{\vbtwo}{3}
      \renewcommand{\vbthree}{9}
    }
  }
}

\SQUARE\vbone\aa
\SQUARE\vbtwo\bb
\SQUARE\vbthree\cc

\MIN\vbone\vbtwo\da
\MAX\vbone\vbtwo\db

\SQUARE\da\daa
\SQUARE\db\dbb
\FRACMINUS{1}{1}\daa\dbb\j\k
\SQRT\j\p
\SQRT\k\q
\DIVIDE\p\q\e
\MULTIPLY\e\vbthree\nf
\DIVIDE\nf\da\tempf
\ROUND[2]\tempf\f

\renewcommand\vbfour{\dfrac{x^2}{a^2} + \dfrac{y^2}{b^2}}
\renewcommand\vbfive{\dfrac{x^2}{b^2} + \dfrac{y^2}{a^2}}

\ifnumodd{\value{rolldice}}{ % vertical
  \renewcommand\vbsix\vbfive
  \renewcommand\vbseven{A}
  \renewcommand\vbeight{\left(0, \pm a\right)}
  \renewcommand\vbnine{\left(0, \pm\frac\vbthree\da\right)}
  \renewcommand\vbten{\left(0, \pm ae\right)}
  \providecommand\vbeleven{\left(0,\pm\f \right)}
}{
  \renewcommand\vbsix\vbfour
  \renewcommand\vbseven{B}
  \renewcommand\vbeight{\left(\pm a, 0\right)}
  \renewcommand\vbnine{\left(\pm\frac\vbthree\da, 0\right)}
  \renewcommand\vbten{\left(\pm ae, 0\right)}
  \providecommand\vbeleven{\left(\pm\f,0\right)}
}

\question The figure alongside shows two ellipses - $A$ and $B$. One of them has the  
equation $\aa x^2 + \bb y^2 = \cc$

\watchout
\insertQR[-20pt]{}

\begin{marginfigure}[15pt]
  \figinit{pt}
    \def\Xori{45}
    \def\Yori{45}
    \def\rxa{30} % A:x-radius
    \def\rya{40} % A:y-radius
    \def\rxb{40} % B:x-radius
    \def\ryb{30} % B:y-radius
    \def\Xmin{0}
    \def\Xmax{60}
    \def\Ymin{0}
    \def\Ymax{60}
    \figpt 100:$C$(\Xori,\Yori) % center of ellipse
    \ifnumodd{\value{rolldice}}{
      \figptell 101:$V_2$: 100;\rxa,\rya (90,0)
      \figptell 102:$V_1$: 100;\rxa,\rya (270,0)
    }{
      \figptell 101:$V_2$: 100;\rxb,\ryb (0,0)
      \figptell 102:$V_1$: 100;\rxb,\ryb (180,0)
    }
    \figptell 500:$A$: 100;\rxa,\rya (60,0)
    \figptell 501:$B$: 100;\rxb,\ryb (23,0)
    \figvectP 200 [100,101]
    \figvectP 201 [100,102]
    \figpttra 103:$F_2$= 100 /0.6,200/
    \figpttra 104:$F_1$= 100 /0.6,201/
  \figdrawbegin{}
    %\figset arrowhead(fillmode=yes,angle=15,length=6)
    %\figset general(color=0.7)
    %\figdrawaxes 100(-\Xmax,\Xmax,-\Ymax,\Ymax)
    \drawAxes{100}{-\Xmax}\Xmax{-\Ymax}\Ymax
    %\figreset{general}
    \figdrawarcell 100;\rxa,\rya (0,360,0)
    \figdrawarcell 100;\rxb,\ryb (0,360,0)
  \figdrawend
  \figvisu{\figBoxA}{}{%
    \figsetmark{$\bullet$}
    \figwritese 100:(3)
    \ifprintanswers
      \ifnumodd{\value{rolldice}}{
        \figwritene 101:(3)
        \figwritese 102:(3)
        \figwritew 103:(3)
        \figwritew 104:(3)
      }{
        \figwritese 101:(3)
        \figwritesw 102:(3)
        \figwritenw 103:(3)
        \figwritene 104:(3)
      }
    \fi
    \figsetmark{}
    \figwritene 500:(3)
    \figwritene 501:(3)
  }
  \centerline{\box\figBoxA}
\end{marginfigure}

\begin{parts}
  \part[2] Which is it - $A$ or $B$? Justify your answer 

\begin{solution}[\mcq]
    The given equation can be re-written as 
    \begin{align}
      \aa x^2 + \bb y^2 = \cc &\implies 
      \dfrac{x^2}{\left( \dfrac{\vbthree}{\vbone}\right)^2} + 
      \dfrac{y^2}{\left( \dfrac\vbthree\vbtwo \right)^2} = 1
    \end{align}
    
    Now, in both the standard forms of an ellipse
    \begin{align}
      \vbfour = 1 &\implies \text{horizontally aligned} \\
      \text{ and } \vbfive = 1 &\implies\text{ vertically aligned }
    \end{align}
    it is understood that $a \geq b$
    
    Hence $a = \max\left(\frac{\vbthree}{\vbone},\frac{\vbthree}{\vbtwo}\right) = \frac{\vbthree}{\da}$ and $b=\min\left(\frac{\vbthree}{\vbone},\frac{\vbthree}{\vbtwo}\right) = \frac{\vbthree}{\db}$

    This means that the original equation is of the form $\vbsix = 1\implies\vbseven$
  \end{solution}

  \part[1] Mark the vertices of the ellipse and find their coordinates. Call the vertices $V_1$ and $V_2$
\begin{solution}[\mcq]
    The vertices of ellipse $\vbseven$ are simply $\vbeight = \vbnine$. These are the points where the ellipse 
    makes the sharpest turns 
  \end{solution}

  \part[1] Find the eccentricity ($e$) of the ellipse 

\begin{solution}[\mcq]
    \begin{align}
      b^2 &= a^2\cdot (1-e^2) \implies e = \sqrt{1-\frac{b^2}{a^2}} \\
      &= \sqrt{\dfrac{\j}{\k}} = \e
    \end{align}
  \end{solution}


  \part[1] Mark the focii and find their coordinates. Call the focii $F_1$ and $F_2$

\begin{solution}[\mcq]
    The focii of the ellipse would be $\vbten = \vbeleven$
  \end{solution}

\end{parts}

