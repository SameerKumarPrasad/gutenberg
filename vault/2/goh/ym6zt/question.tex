
\ifnumequal{\value{rolldice}}{0}{
  % variables 
  \renewcommand{\va}{4}
  \renewcommand{\vb}{6}
  \renewcommand{\vc}{5}
  \renewcommand{\vd}{3\sqrt{2}}
}{
  \ifnumequal{\value{rolldice}}{1}{
    % variables 
    \renewcommand{\va}{6}
    \renewcommand{\vb}{4}
    \renewcommand{\vc}{3}
    \renewcommand{\vd}{4}
  }{
    \ifnumequal{\value{rolldice}}{2}{
      % variables 
      \renewcommand{\va}{4}
      \renewcommand{\vb}{4}
      \renewcommand{\vc}{4}
      \renewcommand{\vd}{2\sqrt{3}}
    }{
      % variables 
      \renewcommand{\va}{6}
      \renewcommand{\vb}{6}
      \renewcommand{\vc}{2}
      \renewcommand{\vd}{2\sqrt{5}}
    }
  }
}

\DIVIDE\va{2}\a
\DIVIDE\vb{2}\b

\question[2] Find the center and radius of the circle represented 
by the following equation
\begin{align}
  x^2 + y^2 - \va x + \vb y - \vc = 0 \nonumber
\end{align}


\watchout

\ifprintanswers
  % stuff to be shown only in the answer key - like explanatory margin figures
  \begin{marginfigure}
    \figinit{pt}
      \figpt 100:(0,0)
      \figpt 101:(0,0)
    \figdrawbegin{}
      \figdrawline [100,101]
    \figdrawend
    \figvisu{\figBoxA}{}{%
    }
    \centerline{\box\figBoxA}
  \end{marginfigure}
\fi 

\begin{solution}[\halfpage]
  Let us begin by expanding the general equation of a circle with 
  its center at $(a, b)$ and a radius $r$,
  \begin{align}
                &(x-a)^2 + (y-b)^2 = r^2 \\
    \implies &x^2 + y^2 - 2a -2b + (a^2 + b^2 -r^2) = 0
  \end{align}
  On comparing coefficients with the equation given in the question 
  we get,
  \begin{align}
      2a = \va &\implies a=\a \\
      2b = -\vb  &\implies b=-\b \\
      a^2+b^2-c^2 = -\vc &\implies c=\vd
  \end{align}
  Therefore, the given circle has $(\a, -\b)$ as center and a radius
  of $\vd$ units.
\end{solution}
\ifprintanswers\begin{codex}
    $C(\a,-\b)$, $R=\vd$ units.
\end{codex}\fi

