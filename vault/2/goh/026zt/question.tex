


\ifnumequal{\value{rolldice}}{0}{
  % variables 
  \renewcommand{\vbone}{-4} % open left  
  \renewcommand{\vbthree}{y^2 = \vbtwo x}
}{
  \ifnumequal{\value{rolldice}}{1}{
    % variables 
    \renewcommand{\vbone}{3} % open up
    \renewcommand{\vbthree}{y = \vbtwo x^2}
  }{
    \ifnumequal{\value{rolldice}}{2}{
      % variables 
      \renewcommand{\vbone}{-5} %open left
      \renewcommand{\vbthree}{y^2 = \vbtwo x}
    }{
      % variables 
      \renewcommand{\vbone}{-6} %open down
      \renewcommand{\vbthree}{y = \vbtwo x^2}
    }
  }
}

\ifnumodd{\value{rolldice}}{
  \renewcommand\vbfour{x^2 = 4py}
  \renewcommand\vbfive{ F = (0,\vbone) }
}{
  \renewcommand\vbfour{y^2 = 4px}
  \renewcommand\vbfive{ F = (\vbone, 0) }
}

\MULTIPLY\vbone{4}\vbtwo
\ABSVALUE\vbtwo\vbsix

\question Given the equation $\vbthree$ of a parabola

\watchout

  % stuff to be shown only in the answer key - like explanatory margin figures
  \begin{marginfigure}
    \figinit{pt}
      \def\Xori{49}
      \def\Yori{45}
      \figpt 1:$O$(\Xori,\Yori)
      \figpt 2:(55,55)
      \figpt 3:(55,35)
      \figpt 4:(100,75)
      \figpt 5:$A$(100,15)
      \figptscontrol 10[4,2,3,5]
      \ifprintanswers
        \ifnumodd{\value{rolldice}}{ % x = 4y^2
          \ifnumequal{\value{rolldice}}{1}{ % y = 4x^2
            \figpt 20:(40,49)
            \figpt 21:(58,49)
            \figpt 22:(15,90)
            \figpt 23:(75,90)
          }{ % y = -4x^2
            \figpt 20:(40,40)
            \figpt 21:(58,40)
            \figpt 22:(20,5)
            \figpt 23:(85,5)
          }
        }{
          \figpt 20:(43,35)
          \figpt 21:(43,55)
          \figpt 22:(0,15)
          \figpt 23:(0,75)
        }
        \figptscontrol 30[22,20,21,23]
      \fi
    \figdrawbegin{}
      \figdrawBezier 1[4,10,11,5]
      \ifprintanswers
        \figdrawBezier 1[22,30,31,23]
      \fi
      \drawAxes{1}{-5}{60}{-5}{60}
    \figdrawend
    \figvisu{\figBoxA}{}{%
      \figwritesw 1:(3)
      \figwritese 5:(3)
      \ifprintanswers
        \figwritew 22:$B$(2) 
      \fi
    }
    \centerline{\box\figBoxA}
  \end{marginfigure}

\begin{parts}
  \part[1] Can curve $A$ - \asif - be the plot of the given parabola? If not, then draw the correct plot

\begin{solution}[\mcq]
    The given equation - $\vbthree$ - is of the form $\vbfour$, where $p = \vbone$. 
    Hence, the curve cannot look like $A$ but should look like $B$
  \end{solution}

  \part[1] Find coordinates of the focus($F$) and the vertex ($V$). Mark both in the corrected plot 

\begin{solution}[\mcq]
    The vertex of the parabola is simply at $(0,0)$

    And given the form of the equation - $\vbfour$ - the focus is $\vbfive$
  \end{solution}

  \part[1] Find the length of the latus rectum of the parabola 

\begin{solution}[\mcq]
    The length of the latus rectum is simply $= \vert 4\times\vbone\vert = \vbsix$
  \end{solution}

\end{parts}

