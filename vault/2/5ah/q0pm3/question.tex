


\ifnumequal{\value{rolldice}}{0}{
  % variables 
  \renewcommand{\vbone}{3}
  \renewcommand{\vbtwo}{5}
  \renewcommand{\vbthree}{2}
  \renewcommand{\vbfour}{3}
  \renewcommand{\vbfive}{7}
}{
  \ifnumequal{\value{rolldice}}{1}{
    % variables 
    \renewcommand{\vbone}{4}
    \renewcommand{\vbtwo}{7}
    \renewcommand{\vbthree}{3}
    \renewcommand{\vbfour}{\dfrac{4}{3}}
    \renewcommand{\vbfive}{\dfrac{-2}{9}}
  }{
    \ifnumequal{\value{rolldice}}{2}{
      % variables 
      \renewcommand{\vbone}{5}
      \renewcommand{\vbtwo}{21}
      \renewcommand{\vbthree}{2}
      \renewcommand{\vbfour}{5}
      \renewcommand{\vbfive}{23}
    }{
      % variables 
      \renewcommand{\vbone}{2}
      \renewcommand{\vbtwo}{3}
      \renewcommand{\vbthree}{1}
      \renewcommand{\vbfour}{4}
      \renewcommand{\vbfive}{14}
    }
  }
}

\question[3] If $a=\dfrac{\vbone+\sqrt{\vbtwo}}{\vbthree}$, then find the value of
$a^2+\dfrac{1}{a^2}$.


\watchout

\ifprintanswers
  % stuff to be shown only in the answer key - like explanatory margin figures
  \begin{marginfigure}
    \figinit{pt}
      \figpt 100:(0,0)
      \figpt 101:(0,0)
    \figdrawbegin{}
      \figdrawline [100,101]
    \figdrawend
    \figvisu{\figBoxA}{}{%
    }
    \centerline{\box\figBoxA}
  \end{marginfigure}
\fi 

\begin{solution}[\halfpage]

  \begin{align}
    a^2 + \dfrac{1}{a^2} &= (a + \dfrac{1}{a})^2 - 2 
  \end{align}
  Let us compute $(a+\dfrac{1}{a})$ first,
  \begin{align}
    a+\dfrac{1}{a} &= \dfrac{\vbone + \sqrt{\vbtwo}}{\vbthree} + \dfrac{\vbthree}{\vbone + \sqrt{\vbtwo}} \\   
                   &= \vbfour
  \end{align}
  Therefore, 
  \begin{align}
    (a+\dfrac{1}{a})^2 - 2 &= (\vbfour)^2 - 2 \\
                           &= \vbfive
  \end{align}

\end{solution}

