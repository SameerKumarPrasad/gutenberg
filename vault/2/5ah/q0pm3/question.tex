


\ifnumequal{\value{rolldice}}{0}{
  % variables 
  \renewcommand{\va}{3}
  \renewcommand{\vb}{5}
}{
  \ifnumequal{\value{rolldice}}{1}{
    % variables 
    \renewcommand{\va}{4}
    \renewcommand{\vb}{7}
  }{
    \ifnumequal{\value{rolldice}}{2}{
      % variables 
      \renewcommand{\va}{5}
      \renewcommand{\vb}{21}
    }{
      % variables 
      \renewcommand{\va}{6}
      \renewcommand{\vb}{11}
    }
  }
}

\SQUARE\va\a
\SUBTRACT\a\vb\c
\SQRT\c\vc
\MULTIPLY\va{2}\vg

\FRACPOWER\vg\vc{2}\vx\vy
\FRACMINUS\vx\vy{2}{1}\vm\vn

\question[3] If $a=\dfrac{\va+\sqrt{\vb}}{\vc}$, then find the value of
$a^2+\dfrac{1}{a^2}$.

\watchout

\begin{solution}[\halfpage]
  Given that 
  \[ x^2 + y^2 = (x+y)^2 - 2xy \]
  we can infer that 
  \[ a^2+\dfrac{1}{a^2} = \left( a + \dfrac{1}{a} \right)^2 - 2\cdot a\cdot\dfrac{1}{a} = \left( a+\dfrac{1}{a}\right)^2 - 2\] 
  where
  \begin{align}
    a + \dfrac{1}{a} &= \underbrace{\dfrac{\va + \sqrt\vb}\vc}_{a} + 
    \underbrace{\dfrac\vc{\va + \sqrt\vb}}_{\frac{1}{a}} \\
    &= \dfrac{(\va^2 + \vb + 2\cdot\va\cdot\sqrt\vb) + \vc^2}{\vc\cdot(\va + \sqrt\vb)} = \WRITEFRAC[false]\vg\vc \\
    \therefore\left( a^2 + \dfrac{1}{a^2} \right) &= 
    \left(\WRITEFRAC[false]\vg\vc\right)^2 - 2 = \WRITEFRAC[false]\vx\vy - 2 = \WRITEFRAC[false]\vm\vn
  \end{align}
\end{solution}

\ifprintanswers\begin{codex}$\WRITEFRAC[false]\vm\vn$\end{codex}\fi
