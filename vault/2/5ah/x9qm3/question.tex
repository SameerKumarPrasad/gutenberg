


\ifnumequal{\value{rolldice}}{0}{
  % variables 
}{
  \ifnumequal{\value{rolldice}}{1}{
    % variables 
  }{
    \ifnumequal{\value{rolldice}}{2}{
      % variables 
    }{
      % variables 
    }
  }
}

\question There is no snow on Janet's driveway when snow begins to fall at midnight.
From midnight to $9$ A.M., snow accumulates on the driveway at a rate modeled by
$f(t) = 7te^{\cos t}$ cubic feet per hour, where $t$ is measured in hours since midnight.
Janet starts removing snow at $6$ A.M. $(t=6)$. The rate $g(t)$ in cubic feet per hour
at which Janet removes the snow from the driveway at time $t$ hours after midnight is
modelled by \\
\textit{Use of calculator is permitted for this questions}
\begin{align}
  g(t) = \left\{
  \begin{tabular}{cc}
    $0$   &$\text{for }0 \leq \text{t} < 6$ \\
    $125$ &$\text{for }6 \leq \text{t} < 7$ \\ 
    $108$ &$\text{for }7 \leq \text{t} \leq 9$
  \end{tabular} \nonumber 
  \right.
\end{align}



\ifprintanswers
  % stuff to be shown only in the answer key - like explanatory margin figures
\fi 

\begin{parts}
  \part[2] How many cubic feet of snow have accumulated on the driveway by $6$ A.M.?
\begin{solution}[\mcq]
    To find volume $V$ of snow accumulated we integrate rate function $f(t)$ over time
    from $t=0$ to $t=6$.
    \begin{align}
      \text{V} = \int_0^6 f(t)\ud t = 142.274
    \end{align}  
  \end{solution}

  \part[1] Find the rate of change of the Volume of snow on the driveway at $8$ A.M.?
\begin{solution}[\mcq]
    To find rate of change of Volume of snow we subtract rate at which snow is being 
    cleared from the rate at which snow is falling,
    \begin{align}
      f(8) - g(8) = -59.58 \textit{ (cubic feet per hour)} \nonumber
    \end{align}
    This implies that the snow is being cleared faster than it is falling!
  \end{solution}

  \part[3] Let $h(t)$ represent the total amount of snow in cubic feet that Janet has
  removed from the driveway at time $t$ hours after midnight. Express $h$ as a piecewise
  defined function with domain $0 \leq t \leq 9$.
\begin{solution}[\mcq]
    Let us define $h$ as a piecewise function with domains that align with the given 
    model.\\
    For $0 < t \leq 6$,
    \begin{align}
      h(t) = h(0) + \int_0^t g(s)\ud s = 0 + \int_0^t 0\ud s = 0
    \end{align}
    For $6 < t \leq 7$,
    \begin{align}
      h(t) = h(6) + \int_6^t g(s)\ud s = 0 + \int_6^t 125 \ud s = 125(t - 6)
    \end{align}
    For $7 < t \leq 9$,
    \begin{align}
      h(t) = h(7) + \int_7^t g(s)\ud s = 125 + \int_6^t 108 \ud s = 125 + 108(t - 7)
    \end{align}
    Therefore,
    \begin{align}
      h(t) = \left\{
      \begin{tabular}{ll}
        $0$            &for $0 \leq t \leq 6$ \\
        $125(t-6)$     &for $6 < t \leq 7$ \\
        $125+108(t-7)$ &for $7 < t \leq 9$
      \end{tabular}
      \right.
    \end{align}
  \end{solution}

  \part[3] How many cubic feet of snow are on the driveway at $9$ A.M.?
\begin{solution}[\mcq]
    Amount A, in \textit{cubic feet} of snow on the driveway $A$ at $9$ A.M.,
    \begin{align}
      \text{A} &= \int_0^9 f(t)\ud t - h(9) \\ \nonumber
               &= 367.334 - 341 = 26.334 \textit{ (cubic feet)}
    \end{align}
  \end{solution}
\end{parts}

\ifprintanswers\begin{codex}\end{codex}\fi
