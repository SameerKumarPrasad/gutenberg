


\ifnumequal{\value{rolldice}}{0}{
  % variables 
}{
  \ifnumequal{\value{rolldice}}{1}{
    % variables 
  }{
    \ifnumequal{\value{rolldice}}{2}{
      % variables 
    }{
      % variables 
    }
  }
}

\question The radius r of a sphere is increasing at a constant rate of 
 $0.04$ centimetres per second. \textit{note: The Volume $V$ of a sphere with
 radius $r$ is given by $\dfrac{4}{3}\pi r^3$.}


\ifprintanswers
\fi 

\begin{parts}
  \part[3] At the time when the radius of the sphere is 10 centimetres, what is 
  the rate of  increase of its volume?

\begin{solution}[\mcq]
    Rate of increase of volume $\dfrac{\ud V}{\ud t}$ is given by,
    \begin{align}
      &\dfrac{\ud V}{\ud t} = \dfrac{4\pi}{3}\cdot3r^2\cdot\dfrac{\ud r}{\ud t} 
                            = 4\pi r^2\dfrac{\ud r}{\ud t} \\
      &\dfrac{\ud V}{\ud t} \vert_{r=10} = 4\pi (100)(0.04) = 16\pi
    \end{align}
    $V$ is changing at the rate of $16\pi (cm^3/sec)$ when $r=10(cm)$.
  \end{solution}

  \part[3] At the time when the volume of the sphere is $36\pi$ cubic centimetres,
  what is the rate of increase of the area of a cross section through the center 
  of the sphere?

\begin{solution}[\mcq]
    When $Volume$ is $36\pi (cm^3)$,
    \begin{align}
      \dfrac{4}{3}\pi r^3 = 36\pi \Rightarrow r^3 = 27 \Rightarrow r=3      
    \end{align}
    Rate of increase of area $\dfrac{\ud A}{\ud t}$ is given by,
    \begin{align}
      &\dfrac{\ud A}{\ud t} = \pi\cdot\dfrac{\ud r^2}{\ud t} 
                            = 2\pi r\dfrac{\ud r}{\ud t} \\
      &\dfrac{\ud A}{\ud t} \vert_{r=3} = 2\pi(3)(0.04) = 0.24\pi
    \end{align}
    $A$ is changing at the rate of $0.24\pi (cm^2/sec)$ when $r=3(cm)$.
  \end{solution}

  \part[3] At the time when the volume and the radius of the sphere are increasing 
  at the same \textit{numerical} rate, what is the radius?

\begin{solution}[\mcq]
    To find $r$ when $\dfrac{\ud V}{\ud t}$ and $\dfrac{\ud r}{\ud t}$ are 
    numerically equal
    \begin{align}
      &4\pi r^2\cdot\dfrac{\ud r}{\ud t} = \dfrac{\ud r}{\ud t} \\
      &4\pi r^2 = 1 \Rightarrow r=\sqrt{\dfrac{1}{4\pi}}
    \end{align}
    $r$ equals $\sqrt{\dfrac{1}{4\pi}}(cm)$ when $\dfrac{\ud V}{\ud t}$ and
    $\dfrac{\ud r}{\ud t}$ are numerically equal.
  \end{solution}

\end{parts}

