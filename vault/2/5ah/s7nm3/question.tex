
\ifnumequal{\value{rolldice}}{0}{
  \renewcommand\va{2}
  \renewcommand\vb{3}
}{
  \ifnumequal{\value{rolldice}}{1}{
    \renewcommand\va{3}
    \renewcommand\vb{4}
  }{
    \ifnumequal{\value{rolldice}}{2}{
      \renewcommand\va{5}
      \renewcommand\vb{6}
    }{
      \renewcommand\va{4}
      \renewcommand\vb{5}
    }
  }
}

\MULTIPLY\va\vb\vc
\MULTIPLY\vc{-4}\vd

\EXPR[0]\vm{-\va*(3+\va)}
\EXPR[0]\vn{-2*\va - 1}

\EXPR[0]\vp{\vb*(3-\vb)}
\EXPR[0]\vq{2*\vb - 1}

\LINEINTERCEPT{1}{-\va}\vm\vn\a\b
\LINEINTERCEPT{1}{\vb}\vp\vq\m\n

\question Given the curve 
  \[ xy^2-x^3y=\vc \]
\watchout

\begin{parts}
  \part[2] Find $\dfrac{\ud y}{\ud x}$. 

\begin{solution}[\mcq]
  \begin{align}
    xy^2 - x^3y &= \vc \\ 
    \implies\dfrac\ud{\ud x}(xy^2 - x^3y) &= \dfrac\ud{\ud x} \vc = 0 \\
    \implies\left(y^2\dfrac\ud{\ud x}x + x\dfrac\ud{\ud x}y^2\right) 
    &- \left( y\dfrac\ud{\ud x}x^3 + x^2\dfrac\ud{\ud x} y\right) = 0 \\
    \implies \left( y^2 + 2xy\dfrac{\ud y}{\ud x} \right) &=
    \left( 3x^2y + x^3\dfrac{\ud y}{\ud x} \right) \\
    \implies \dfrac{\ud y}{\ud x} &= \dfrac{3x^2y - y^2}{2xy-x^3} 
  \end{align}
\end{solution}

  \part[3] Find all the points on the curve whose \textbf{x-coordinate} is $1$ and 
  write an equation for the tangent at each of these points.

\begin{solution}[\mcq]
  When $x=1$, then 
  \begin{align}
    1\cdot y^2 - 1^3\cdot y &= \vc \implies y^2-y-\vc = 0 \\
    \implies y &= -\va,\vb
  \end{align}
  Hence, the two points \textbf{on the curve} with $x=1$ are 
  \[ A = (1,-\va)\text{ and }B=(1,\vb) \]
  The slope of the tangent is simply the $\dfrac{\ud y}{\ud x}$ we found in part (a). 
  \begin{align}
    m_A &= \left[ \dfrac{\ud y}{\ud x} \right]_{(1,-\va)} = \WRITEFRAC[false]\vm\vn  \\
    m_B &= \left[ \dfrac{\ud y}{\ud x} \right]_{(1,\vb)} = \WRITEFRAC[false]\vp\vq
  \end{align}
  And as these lines with slopes $m_A$ and $m_B$ pass through $A$ and $B$ respectively, therefore 
  \begin{align}
    \underbrace{\dfrac{y+\va}{x-1}}_{m_A} &= \WRITEFRAC[false]\vm\vn \implies y = \WRITELINE\vm\vn\a\b \\
    \underbrace{\dfrac{y-\vb}{x-1}}_{m_B} &= \WRITEFRAC[false]\vp\vq \implies y = \WRITELINE\vp\vq\m\n
  \end{align}
\end{solution}

  \part[3] For what values of the $x-$coordinate is the tangent vertical?
\begin{solution}[\mcq]
  For the tangent to be vertical 
  \begin{align}
    \dfrac{\ud y}{\ud x} &= \dfrac{3x^2y-y^2}{2xy-x^3} = \infty \\
    \implies 2xy-x^3 &= 0\implies y = \dfrac{x^2}{2}
  \end{align}
  Moreover, as all such points are also on the curve, it also means that 
  \begin{align}
    xy^2 - x^3 y &= \vc \\
    \implies x\cdot\left(\dfrac{x^2}{2} \right)^2 - x^3\cdot\dfrac{x^2}{2} &= \vc \\
    \implies x^5\cdot\left(\dfrac{1}{4}-\dfrac{1}{2}\right)  &= \vc \\
    \implies x^5 = \vd \implies x &= \sqrt[5]{\vd}
  \end{align}
\end{solution}
\end{parts}

\ifprintanswers
  \begin{codex}
    \begin{tabular}{l l l} 
      $(a)\,\dfrac{3x^2y - y^2}{2xy-x^3}$ & $(b)\,y=\WRITELINE\vm\vn\a\b\text{ and } y = \WRITELINE\vp\vq\m\n\qquad$ 
      & $(c)\,x = \sqrt[5]{\vd}$
    \end{tabular}
  \end{codex}
\fi
