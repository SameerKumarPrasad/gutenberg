% This is an empty shell file placed for you by the 'examiner' script.
% You can now fill in the TeX for your question here.

% Now, down to brasstacks. ** Writing good solutions is an Art **. 
% Eventually, you will find your own style. But here are some thoughts 
% to get you started: 
%
%   1. Write the solution as if you are writing it for your favorite
%      14-17 year old to help him/her understand. Could be your nephew, 
%      your niece, a cousin perhaps or probably even you when you 
%      were that age. Just write for them.
%
%   2. Use margin-notes to "talk" to students about the critical insights
%      in the question. The tone can be - in fact, should be - informal
%
%   3. Don't shy away from creating margin-figures you think will help
%      students understand. Yes, it is a little more work per question. 
%      But the question & solution will be written only once. Make that
%      attempt at writing a solution count.
%
%   4. At the same time, do not be too verbose. A long solution can
%      - at first sight - make the student think, "God, that is a lot to know".
%      Our aim is not to scare students. Rather, our aim should be to 
%      create many "Aha!" moments everyday in classrooms around the world
% 
%   5. Ensure that there are *no spelling mistakes anywhere*. We are an 
%      education company. Bad spellings suggest that we ourselves 
%      don't have any education. And, use American spellings

\question[4]  Without using a calculator, find the value of the following expression:

$2\cdot\left( \dfrac{\cos\ang{58}}{\sin\ang{32}}\right) - 
\sqrt{3}\cdot\left( \dfrac{\cos\ang{38}\csc\ang{52}}{\tan\ang{15}\tan\ang{60}\tan\ang{75}}\right)$
\insertQR{QRC}
\ifprintanswers
  % stuff to be shown only in the answer key - like explanatory margin figures
\fi 
\begin{solution}[\fullpage]
  The first thing to note is that 
  $\ang{52} + \ang{38} = \ang{58} + \ang{32} = \ang{75} + \ang{15} = \ang{90}$.
  
  This is the key to the problem because we can now re-write the expression as
  
  \begin{align}
    & 2\cdot\dfrac{\cos(\ang{90}-\ang{32})}{\sin\ang{32}} 
    - \sqrt{3}\cdot\dfrac{\cos\ang{38}\csc(\ang{90}-\ang{38})}{\tan\ang{15}\tan\ang{60}
    \tan(\ang{90}-\ang{15})} \\
    &= 2\cdot\dfrac{\sin\ang{32}}{\sin\ang{32}} - 
    \sqrt{3}\cdot\left( \dfrac{\cos\ang{38}\sec\ang{38}}{\cot\ang{15}\tan\ang{60}\tan\ang{15}} \right) \\
    &= 2 - \sqrt{3}\cdot\dfrac{1}{\sqrt{3}} \\
    &= 1
  \end{align}
  
\end{solution}
