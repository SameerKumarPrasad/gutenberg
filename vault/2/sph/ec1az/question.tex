
\ifnumequal{\value{rolldice}}{0}{
  % variables 
  \renewcommand{\va}{m}
}{
  \ifnumequal{\value{rolldice}}{1}{
    % variables 
    \renewcommand{\va}{n}
  }{
    \ifnumequal{\value{rolldice}}{2}{
      % variables 
      \renewcommand{\va}{a}
    }{
      % variables 
      \renewcommand{\va}{b}
    }
  }
}

\question Prove that expression $E(\va) = \va^3+3\va^2+5\va+3$ 
is divisible by $3$ for any $\va\in\mathbb{N}$.

\watchout

\ifprintanswers
  % stuff to be shown only in the answer key - like explanatory margin figures
\fi

\begin{solution}
  For $\va=1$, the given expression appears as follows,
  \begin{align}
    E(1) &= 1^3 + 3\times 1^2 + 5\times 1 + 3 \\
         &= 12
  \end{align}
  which we know to be divisible by $3$.\\
  Now let us assume $E(\va)$ is divisible by $3$ for some 
  $\va=k$. This implies,
  \begin{align}
    E(k) = k^3+3k^2+5k+3 = 3\times C
  \end{align}
  where $C$ is some integer.\\
  To prove that $E(\va)$ is divisible by $3$ for $\va=k+1$.
  \begin{align}
    E(k+1) &= (k+1)^3+2(k+1)^2+5(k+1)+3 \\
           &= k^3+3k^2+3k+1+3k^2+3+6k+5k+5+3 \\
           &= (k^3+3k^2+5k+3)+(3k^2+9k+9) \\
           &= (3\times C)+3(k^2+3k+3) \\
           &= 3\times (C+k^2+3k+3)
  \end{align}
  Therefore, by the Principle of Mathematical Induction, $E(\va)$ is divisible by $3$.
  
\end{solution}

