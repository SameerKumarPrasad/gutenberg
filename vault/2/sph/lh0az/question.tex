
\ifnumequal{\value{rolldice}}{0}{
  % variables 
  \renewcommand{\va}{3}
}{
  \ifnumequal{\value{rolldice}}{1}{
    % variables 
    \renewcommand{\va}{5}
  }{
    \ifnumequal{\value{rolldice}}{2}{
      % variables 
      \renewcommand{\va}{2}
    }{
      % variables 
      \renewcommand{\va}{9}
    }
  }
}

\question If $x = \va + \dfrac{f(x)}{\va + \frac{f(x)}{\va + ...}}$. What
is the domain of $f(x)$ for the range $f(x) < 0$.

\watchout

\begin{solution}
  To begin with, let us express $f(x)$ in terms of $x$. We can re-write
  the given equation as,
  
  \begin{align}
    x        &= \va + \dfrac{f(x)}{x} \\
    x-\va    &= \dfrac{f(x)}{x} \\
    x(x-\va) &= f(x)
  \end{align}
  
  As per given condition $f(x) < 0 $,
  \begin{align}
    \underbrace{(x)}_{T_1} \cdot \underbrace{(x-\va)}_{T_2} < 0    
  \end{align}
  For this condition to be satisfied, $T_1$ and $T_2$ need to have
  opposite signs.\\
  
  Case I: 
  \begin{align}
    T_1 > 0 \text{ and } T_2 < 0 \Rightarrow x > 0 \text{ and } x < \va
  \end{align}
  Conditions in $(5)$ are contradictory, therefore not possible.\\
  
  Case II:
  \begin{align}
    T_1 < 0 \text{ and } T_2 > 0 \Rightarrow \mathbf{x < 0 \text{ and } x > \va}
  \end{align}
  
  From $(6)$ we get the domain for the condition $f(x)>0$ as
  $\mathbf{0 < x < \va}$.
  
\end{solution}

