


\ifnumequal{\value{rolldice}}{0}{
  % variables 
  \renewcommand{\vbone}{All 4 cards are of the same suit}
  \renewcommand{\vbtwo}{All 4 cards have the same number}
  \renewcommand{\vbthree}{All 4 cards are in succession and of the same suit}
  \renewcommand{\vbfour}{All 4 cards are face cards (Jack, Queen, King)}
}{
  \ifnumequal{\value{rolldice}}{1}{
    % variables 
    \renewcommand{\vbone}{All 4 cards have the same number}
    \renewcommand{\vbtwo}{All 4 cards are of the same suit}
    \renewcommand{\vbthree}{All 4 cards are face cards (Jack, Queen, King)}
    \renewcommand{\vbfour}{All 4 cards are in succession and of the same suit}
  }{
    \ifnumequal{\value{rolldice}}{2}{
      % variables 
      \renewcommand{\vbone}{All 4 cards are face cards (Jack, Queen, King)}
      \renewcommand{\vbtwo}{All 4 cards are in succession and of the same suit}
      \renewcommand{\vbthree}{All 4 cards are of the same suit}
      \renewcommand{\vbfour}{All 4 cards have the same number}
    }{
      % variables 
      \renewcommand{\vbone}{All 4 cards are in succession and of the same suit}
      \renewcommand{\vbtwo}{All 4 cards are face cards (Jack, Queen, King)}
      \renewcommand{\vbthree}{All 4 cards have the same number}
      \renewcommand{\vbfour}{All 4 cards are of the same suit}
    }
  }
}

\renewcommand{\vbfive}{$\encr{13}{4}=2860$}%same suit
\renewcommand{\vbsix}{$13$}%same number
\renewcommand{\vbseven}{$\encr{12}{4}=495$}%face cards
\renewcommand{\vbeight}{$52\times2\times2\times2=416$}%succession

\question In how any ways can you choose 4 cards from a complete deck of
 52 cards such that (consider the cards to be in continuation Ace follows
 King):


\watchout

\ifprintanswers
  % stuff to be shown only in the answer key - like explanatory margin figures
  \begin{marginfigure}
    \figinit{pt}
      \figpt 100:(0,0)
      \figpt 101:(0,0)
    \figdrawbegin{}
      \figdrawline [100,101]
    \figdrawend
    \figvisu{\figBoxA}{}{%
    }
    \centerline{\box\figBoxA}
  \end{marginfigure}
\fi 

\begin{parts}
  \part[2] \vbone

\begin{solution}[\mcq]
	\ifnumequal{\value{rolldice}}{0}{
	  \vbfive
	}{
	  \ifnumequal{\value{rolldice}}{1}{
	    \vbsix
	  }{
	    \ifnumequal{\value{rolldice}}{2}{
	      \vbseven
	    }{
	      \vbeight
	    }
	  }
	}  
  \end{solution}

  \part[2] \vbtwo

\begin{solution}[\mcq]
	\ifnumequal{\value{rolldice}}{0}{
	  \vbsix
	}{
	  \ifnumequal{\value{rolldice}}{1}{
	    \vbfive
	  }{
	    \ifnumequal{\value{rolldice}}{2}{
	      \vbeight
	    }{
	      \vbseven
	    }
	  }
	}  
  \end{solution}

  \part[2] \vbthree

\begin{solution}[\mcq]
	\ifnumequal{\value{rolldice}}{0}{
	  \vbeight
	}{
	  \ifnumequal{\value{rolldice}}{1}{
	    \vbseven
	  }{
	    \ifnumequal{\value{rolldice}}{2}{
	      \vbfive
	    }{
	      \vbsix
	    }
	  }
	}  
  \end{solution}

  \part[2] \vbfour

\begin{solution}[\mcq]
	\ifnumequal{\value{rolldice}}{0}{
	  \vbseven
	}{
	  \ifnumequal{\value{rolldice}}{1}{
	    \vbeight
	  }{
	    \ifnumequal{\value{rolldice}}{2}{
	      \vbsix
	    }{
	      \vbfive
	    }
	  }
	}  
  \end{solution}

\end{parts}
