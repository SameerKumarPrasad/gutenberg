


\ifnumequal{\value{rolldice}}{0}{
  % variables 
  \renewcommand{\vbone}{Heptagon}
  \renewcommand{\vbtwo}{7}
  \renewcommand{\vbthree}{14}
}{
  \ifnumequal{\value{rolldice}}{1}{
    % variables 
    \renewcommand{\vbone}{Decagon}
    \renewcommand{\vbtwo}{10}
    \renewcommand{\vbthree}{35}
  }{
    \ifnumequal{\value{rolldice}}{2}{
      % variables 
      \renewcommand{\vbone}{Octagon}
      \renewcommand{\vbtwo}{8}
      \renewcommand{\vbthree}{20}
    }{
      % variables 
      \renewcommand{\vbone}{Nonagon}
      \renewcommand{\vbtwo}{9}
      \renewcommand{\vbthree}{27}
    }
  }
}

\question[2] How many diagonals does a $\vbone$ ($\vbtwo$ sided figure) have?


\watchout

\ifprintanswers
  % stuff to be shown only in the answer key - like explanatory margin figures
  \begin{marginfigure}
    \figinit{pt}
      \figpt 100:(0,0)
      \figpt 101:(0,0)
    \figdrawbegin{}
      \figdrawline [100,101]
    \figdrawend
    \figvisu{\figBoxA}{}{%
    }
    \centerline{\box\figBoxA}
  \end{marginfigure}
\fi 

\begin{solution}[\mcq]
  To compute the diagonal we subtract the number of sides of the polygon from
  the total number of edges (lines connecting a pair of points taken together)
  \begin{align}
     &\encr{\vbtwo}{2}-\vbtwo \\
    =\quad&\vbthree
  \end{align}

\end{solution}
