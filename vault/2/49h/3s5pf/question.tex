% This is an empty shell file placed for you by the 'examiner' script.
% You can now fill in the TeX for your question here.

% Now, down to brasstacks. ** Writing good solutions is an Art **. 
% Eventually, you will find your own style. But here are some thoughts 
% to get you started: 
%
%   1. Write to be understood - but be crisp. Your own solution should not take 
%      more space than you will give to the student. Hence, if you take more than 
%      a half-page to write a solution, then give the student a full-page and so on...
%
%   2. Use margin-notes to "talk" to students about the critical insights
%      in the question. The tone can be - in fact, should be - informal
%
%   3. Don't shy away from creating margin-figures you think will help
%      students understand. Yes, it is a little more work per question. 
%      But the question & solution will be written only once. Make that
%      attempt at writing a solution count.
%      
%      3b. Use bc_to_fig.tex. Its an easier way to generate plots & graphs 
% 
%   4. Ensure that there are *no spelling mistakes anywhere*. We are an 
%      education company. Bad spellings suggest that we ourselves 
%      don't have any education. Also, use American spellings by default
% 
%   5. If a question has multiple parts, then first delete lines 40-41
%   6. If a question does not have parts, then first delete lines 43-69
%   
%   7. Create versions of the question when possible. Use commands defined in 
%      tufte-tweaks.sty to do so. Its easier than you think

% \noprintanswers
% \setcounter{rolldice}{0}

\ifnumequal{\value{rolldice}}{0}{
  % variables 
  \renewcommand{\vbone}{12}
  \renewcommand{\vbtwo}{three}
  \renewcommand{\vbthree}{3}
  \renewcommand{\vbfour}{207900}
  \renewcommand{\vbfive}{}
  \renewcommand{\vbsix}{}
  \renewcommand{\vbseven}{}
  \renewcommand{\vbeight}{}
  \renewcommand{\vbnine}{}
  \renewcommand{\vbten}{}
}{
  \ifnumequal{\value{rolldice}}{1}{
    % variables 
    \renewcommand{\vbone}{12}
    \renewcommand{\vbtwo}{four}
    \renewcommand{\vbthree}{4}
    \renewcommand{\vbfour}{443520}
    \renewcommand{\vbfive}{}
    \renewcommand{\vbsix}{}
    \renewcommand{\vbseven}{}
    \renewcommand{\vbeight}{}
    \renewcommand{\vbnine}{}
    \renewcommand{\vbten}{}
  }{
    \ifnumequal{\value{rolldice}}{2}{
      % variables 
      \renewcommand{\vbone}{15}
      \renewcommand{\vbtwo}{three}
      \renewcommand{\vbthree}{3}
      \renewcommand{\vbfour}{756756}
      \renewcommand{\vbfive}{}
      \renewcommand{\vbsix}{}
      \renewcommand{\vbseven}{}
      \renewcommand{\vbeight}{}
      \renewcommand{\vbnine}{}
      \renewcommand{\vbten}{}
    }{
      % variables 
      \renewcommand{\vbone}{15}
      \renewcommand{\vbtwo}{five}
      \renewcommand{\vbthree}{5}
      \renewcommand{\vbfour}{8408400}
      \renewcommand{\vbfive}{}
      \renewcommand{\vbsix}{}
      \renewcommand{\vbseven}{}
      \renewcommand{\vbeight}{}
      \renewcommand{\vbnine}{}
      \renewcommand{\vbten}{}
    }
  }
}

\gcalcexpr[0]{\setSize}{\vbone / \vbthree}
\gcalcexpr[0]{\second}{\vbone - \setSize}
\gcalcexpr[0]{\third}{\second - \setSize}
\gcalcexpr[0]{\fourth}{\third - \setSize}
\gcalcexpr[0]{\fifth}{\fourth - \setSize}

\question[4] In how many ways could we distribute $\vbone$ distinct items among
$\vbtwo$ people such that each person gets the same number of items and all
the items get distributed.

\insertQR{QRC}

\watchout

\ifprintanswers
  % stuff to be shown only in the answer key - like explanatory margin figures
  \begin{marginfigure}
    \figinit{pt}
      \figpt 100:(0,0)
      \figpt 101:(0,0)
    \figdrawbegin{}
      \figdrawline [100,101]
    \figdrawend
    \figvisu{\figBoxA}{}{%
    }
    \centerline{\box\figBoxA}
  \end{marginfigure}
\fi 

\begin{solution}[\halfpage]
  In order to distribute the items such that each person gets the same number,
  we need to create $\vbtwo$ sets of $\setSize$ items each. Number of ways in
  which we can do this is
  \begin{align}
    \encr{\vbone}{\setSize}\times
    \encr{\second}{\setSize}\times
    \encr{\third}{\setSize}
    \ifnumequal{\value{rolldice}}{1}{
  	  \times\encr{\fourth}{\setSize}
    }{
      \ifnumequal{\value{rolldice}}{3}{
 	    \times\encr{\fourth}{\setSize}
        \times\encr{\fifth}{\setSize}
      }{
      }
    }
  \end{align}
  The number of ways in which we can distribute three sets of equal size
  ($\setSize$) among $\vbtwo$ people is $\vbthree!$ Therefore the total 
  number of distributions is
  \begin{align}
    &\vbthree!\times
      \encr{\vbone}{\setSize}\times
      \encr{\second}{\setSize}\times
      \encr{\third}{\setSize}
    \ifnumequal{\value{rolldice}}{1}{
      \times\encr{\fourth}{\setSize}
    }{
      \ifnumequal{\value{rolldice}}{3}{
        \times\encr{\fourth}{\setSize}
        \times\encr{\fifth}{\setSize}
      }{
      }
    }
    \\
    =\quad&\vbfour
  \end{align}
\end{solution}
