% This is an empty shell file placed for you by the 'examiner' script.
% You can now fill in the TeX for your question here.

% Now, down to brasstacks. ** Writing good solutions is an Art **. 
% Eventually, you will find your own style. But here are some thoughts 
% to get you started: 
%
%   1. Write the solution as if you are writing it for your favorite
%      14-17 year old to help him/her understand. Could be your nephew, 
%      your niece, a cousin perhaps or probably even you when you 
%      were that age. Just write for them.
%
%   2. Use margin-notes to "talk" to students about the critical insights
%      in the question. The tone can be - in fact, should be - informal
%
%   3. Don't shy away from creating margin-figures you think will help
%      students understand. Yes, it is a little more work per question. 
%      But the question & solution will be written only once. Make that
%      attempt at writing a solution count.
%
%   4. At the same time, do not be too verbose. A long solution can
%      - at first sight - make the student think, "God, that is a lot to know".
%      Our aim is not to scare students. Rather, our aim should be to 
%      create many "Aha!" moments everyday in classrooms around the world
% 
%   5. Ensure that there are *no spelling mistakes anywhere*. We are an 
%      education company. Bad spellings suggest that we ourselves 
%      don't have any education. Also, use American spellings by default
% 
%   6. If a question has multiple parts, then first delete lines 40-41
%   7. If a question does not have parts, then first delete lines 43-69

\question[3] Three terms of a series are in arithmetic progression. If $8$ is added to the first term, the three terms now form a geometric progression, whose sum is $26$. Find the terms?

\insertQR{QRC}

\ifprintanswers
  % stuff to be shown only in the answer key - like explanatory margin figures
\fi 

\begin{solution}[\halfpage]
  Let the terms of the series be $a-d$, $a$ and $a+d$. \\
  Using the condition that if $8$ is added to the first term it forms a geometric progression whose sum is $26$ we get,
  \begin{align}
    (a-d+8) + a + (a+d) &= 26 \\
    3a + 8              &= 26 \\
    a                   &= 6
  \end{align}  
  Using the condition that the three terms are in geometric progression,
  \begin{align}
    (a-d+8)(a+d) &= a^2 \\
    (14-d)(6+d)  &= 36 \\
    d = 12, -4
  \end{align} 
  There are two sets of series that satisfy the given conditions. With a common difference of $12$ the progression is ${-6,6,18}$, whereas with a common difference of $-4$, the progression is ${10, 6, 2}$.

\end{solution}
