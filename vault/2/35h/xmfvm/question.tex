


\ifnumequal{\value{rolldice}}{0}{
  % variables 
  \renewcommand{\vbone}{5}
  \renewcommand{\vbtwo}{4}
  \renewcommand{\vbthree}{15}
  \renewcommand{\vbfour}{4}
  \renewcommand{\vbfive}{6}
  \renewcommand{\vbsix}{4}
  \renewcommand{\vbseven}{1}
}{
  \ifnumequal{\value{rolldice}}{1}{
    % variables 
    \renewcommand{\vbone}{5}
    \renewcommand{\vbtwo}{3}
    \renewcommand{\vbthree}{7}
    \renewcommand{\vbfour}{3}
    \renewcommand{\vbfive}{3}
    \renewcommand{\vbsix}{1}
  }{
    \ifnumequal{\value{rolldice}}{2}{
      % variables 
      \renewcommand{\vbone}{4}
      \renewcommand{\vbtwo}{4}
      \renewcommand{\vbthree}{15}
      \renewcommand{\vbfour}{4}
      \renewcommand{\vbfive}{6}
      \renewcommand{\vbsix}{4}
      \renewcommand{\vbseven}{1}
    }{
      % variables 
      \renewcommand{\vbone}{4}
      \renewcommand{\vbtwo}{3}
      \renewcommand{\vbthree}{7}
      \renewcommand{\vbfour}{3}
      \renewcommand{\vbfive}{3}
      \renewcommand{\vbsix}{1}
    }
  }
}

\gcalcexpr[0]{\total}{\vbthree ^ \vbone}
\gcalcexpr[0]{\hint}{\vbthree ^ 3}

\question[3] There are $\vbone$ multiple-choice type questions in an assignment. 
Each question presents $\vbtwo$ possible options. How many unique answer 
combinations are possible for the assignment, provided each question has 
\textbf{at least} one correct option.\\
\textit{Calculation hint: $\vbthree^3=\hint$}


\watchout

\ifprintanswers
  % stuff to be shown only in the answer key - like explanatory margin figures
  \begin{marginfigure}
    \figinit{pt}
      \figpt 100:(0,0)
      \figpt 101:(0,0)
    \figdrawbegin{}
      \figdrawline [100,101]
    \figdrawend
    \figvisu{\figBoxA}{}{%
    }
    \centerline{\box\figBoxA}
  \end{marginfigure}
\fi 

\begin{solution}[\mcq]
  The total number of combinations can be found by multiplying the number of
  correct answer options for a particular question, with the number of
  questions. Since each question can have at least one correct option we
  would count the combinations using cases as follows:
  
  \begin{align}
    \text{Exactly 1 correct option}  
    	\Rightarrow& \encr{\vbtwo}{1} = \vbfour \\
    \text{Exactly 2 correct options} 
    	\Rightarrow& \encr{\vbtwo}{2} = \vbfive \\
    \ifnumequal{\value{rolldice}}{0} {
      \text{Exactly 3 correct options} 
        \Rightarrow& \encr{\vbtwo}{3} = \vbsix \\
      \text{All 4 correct options}     
        \Rightarrow& \encr{\vbtwo}{4} = \vbseven
    } {
      \ifnumequal{\value{rolldice}}{1} {
        \text{All 3 correct options}     
          \Rightarrow& \encr{\vbtwo}{3} = \vbsix
      } {
        \ifnumequal{\value{rolldice}}{2} {
	      \text{Exactly 3 correct options}
	        \Rightarrow& \encr{\vbtwo}{3} = \vbsix \\
    	  \text{All 4 correct options}     
    	    \Rightarrow& \encr{\vbtwo}{4} = \vbseven
        } {
          \text{All 3 correct options}     
            \Rightarrow& \encr{\vbtwo}{3} = \vbsix
        }
      }
    }
  \end{align}
  \begin{align}
  	\text{Total combinations} &= \vbthree ^ \vbone \\
  						      &= \total
  \end{align}
\end{solution}
