
\ifnumequal{\value{rolldice}}{0}{
  % variables 
  \renewcommand\va{20}
  \renewcommand\vg{1.732}
}{
  \ifnumequal{\value{rolldice}}{1}{
    % variables 
    \renewcommand\va{13}
    \renewcommand\vg{0.809}
  }{
    \ifnumequal{\value{rolldice}}{2}{
      % variables 
      \renewcommand\va{11}
      \renewcommand\vg{0.649}
    }{
      % variables 
      \renewcommand\va{18}
      \renewcommand\vg{1.376}
    }
  }
}
\SUBTRACT{60}\va\vc
\ADD{60}\va\vd
\MULTIPLY{3}{\va}\vf

\question[4] Using just the values provide alongside, find the value of  
\[ \tan\va^\circ\cdot\tan\vc^\circ\cdot\tan\vd^\circ \]

\marginnote[-2cm]{
  \\$\tan 27^\circ=0.5095$ \\
  $\tan 33^\circ=0.649$ \\
  $\tan 39^\circ=0.809$ \\
  $\tan 54^\circ=1.376$ \\
  $\tan 60^\circ=1.732$ \\ 
  $\tan 64^\circ=2.05$ 
}

\watchout
\begin{solution}[\halfpage]
  If we let $\theta=\va^\circ$, then note that 
  \begin{align}
    \vc^\circ &= 60^\circ-\theta \\
    \vd^\circ &= 60^\circ+\theta
  \end{align}
  And therefore, the original expression can be re-written as 
  \begin{align}
    &\tan\theta\cdot\tan(60^\circ-\theta)\cdot\tan(60^\circ+\theta) \\
    &=\tan\theta\cdot\eTanOfDiff{60^\circ}{\theta}\cdot\eTanOfSum{60^\circ}{\theta} \\
    &= \tan\theta\cdot\left[ \dfrac{\tan^260^\circ-\tan^2\theta}{1-\tan^260^\circ\cdot\tan^2\theta} \right] \\
    &= \tan\theta\cdot\left( \dfrac{3-\tan^2\theta}{1-3\tan^2\theta}\right) 
    = \underbrace{\dfrac{3\tan\theta-\tan^3\theta}{1-3\tan^2\theta}}_{=\tan 3\theta} \\
    &= \tan{\vf^\circ}=\vg
  \end{align}
\end{solution}
