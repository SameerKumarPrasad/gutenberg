

\question[4] Without resorting to calculator or tables calculate \\
$\tan20^\circ\tan40^\circ\tan80^\circ$.


\ifprintanswers
  % stuff to be shown only in the answer key - like explanatory margin figures
  \begin{marginfigure}
    \figinit{pt}
      \figpt 100:(0,0)
      \figpt 101:(0,0)
    \figdrawbegin{}
      \figdrawline [100,101]
    \figdrawend
    \figvisu{\figBoxA}{}{%
    }
    \centerline{\box\figBoxA}
  \end{marginfigure}
\fi 

\marginnote[200pt]
           {$\tan 3\theta = \tan (\theta + 2\theta) \\
            = \dfrac{\tan\theta +\tan 2\theta}
                    {1 -\tan\theta\tan 2\theta} \\
            = \dfrac{\tan\theta(1 - \tan ^2\theta) + 2\tan \theta}
                    {(1 - \tan ^2\theta) - 2\tan ^2 \theta} \\
            = \dfrac{3\tan\theta-\tan^3\theta}
                    {1 - 3\tan^2\theta}$}

\begin{solution}[\halfpage]
  Let $20^\circ=\theta$. Then $40^\circ$ can be written as $60^\circ -\theta$ and $80^\circ$ can be written as $60^\circ +\theta$. Now, we can rewrite the expression as,
  \begin{align}
    P &= \tan\theta\tan(\dfrac{\pi}{3}-\theta)
    	   \tan(\dfrac{\pi}{3}+\theta) \\
      &= \tan\theta
           \left(\dfrac{tan(\dfrac{\pi}{3})-\tan\theta}
                       {1+\tan(\dfrac{\pi}{3})\tan\theta}\right)
           \left(\dfrac{tan(\dfrac{\pi}{3})+\tan\theta}
                       {1-\tan(\dfrac{\pi}{3})\tan\theta}\right) \\
      &= \tan\theta
           \left(\dfrac{tan^2(\dfrac{\pi}{3})-\tan^2\theta}
                       {1-\tan^2(\dfrac{\pi}{3})\tan^2\theta}\right) \\
      &= \tan\theta
           \left(\dfrac{3-\tan^2\theta}
                       {1-3\tan^2\theta}\right) \\
      &= \dfrac{3\tan\theta-\tan^3\theta}{1-3\tan^2\theta} \\
      &= \tan 3\theta \\
      &= \sqrt{3}
  \end{align}
  
\end{solution}
