


\ifnumequal{\value{rolldice}}{0}{
  % variables 
  \renewcommand{\va}{-1}
  \renewcommand{\vb}{0}
  \renewcommand{\vc}{7}
  \renewcommand{\vd}{2}
  \renewcommand{\ve}{5}
  \renewcommand{\vf}{-2}
  \renewcommand{\vg}{\frac{20}{7}}
  \renewcommand{\vh}{\frac{11}{7}}
}{
  \ifnumequal{\value{rolldice}}{1}{
    % variables 
    \renewcommand{\va}{1}
    \renewcommand{\vb}{0}
    \renewcommand{\vc}{-7}
    \renewcommand{\vd}{2}
    \renewcommand{\ve}{-5}
    \renewcommand{\vf}{-2}
    \renewcommand{\vg}{-\frac{20}{7}}
    \renewcommand{\vh}{\frac{11}{7}}
  }{
    \ifnumequal{\value{rolldice}}{2}{
      % variables 
      \renewcommand{\va}{-1}
      \renewcommand{\vb}{0}
      \renewcommand{\vc}{7}
      \renewcommand{\vd}{-2}
      \renewcommand{\ve}{5}
      \renewcommand{\vf}{2}
      \renewcommand{\vg}{\frac{20}{7}}
      \renewcommand{\vh}{-\frac{11}{7}}
    }{
      % variables 
      \renewcommand{\va}{1}
      \renewcommand{\vb}{0}
      \renewcommand{\vc}{-7}
      \renewcommand{\vd}{-2}
      \renewcommand{\ve}{-5}
      \renewcommand{\vf}{2}
      \renewcommand{\vg}{-\frac{20}{7}}
      \renewcommand{\vh}{-\frac{11}{7}}
    }
  }
}

\EXPR[0]{\mxone}{(\va + \vc) / 2}
\EXPR[0]{\myone}{(\vb + \vd) / 2}
\EXPR[0]{\mxtwo}{(\vc + \ve) / 2}
\EXPR[0]{\mytwo}{(\vd + \vf) / 2}
\EXPR[0]{\mxthree}{(\va + \ve) / 2}
\EXPR[0]{\mythree}{(\vb + \vf) / 2}

\EXPR[0]{\mab}{-(\vc - \va)/(\vd - \vb)}
\EXPR[2]{\mbc}{-(\ve - \vc)/(\vf - \vd) }
\EXPR[0]{\mca}{-(\va - \ve)/(\vb - \vf)} 

\EXPR[0]{\cone}{\myone - (\mab * \mxone)}
\EXPR[0]{\ctwo}{\mytwo - (\mbc * \mxtwo)}
\EXPR[0]{\cthree}{\mythree - (\mca * \mxthree)}

\question[3] A triangle has vertices $A = (\va, \vb)$, $B = (\vc, \vd)$ and 
$C = (\ve, \vf)$ (see figure). $X$, $Y$ and $Z$ are the mid-points of the three sides.
Show that the \textbf{perpendicular} bisectors of the sides are concurrent \textbf{and}
find the point of concurrence


\figinit{pt}
  \figpt 10:$A$(-10, 0)
  \figpt 20:$B$(70, 20)
  \figpt 30:$C$(50, -20)
  \figpt 11: $X$(30,10)
  \figpt 21: $Y$(60,0)
  \figpt 31: $Z$(20,-10)
\figdrawbegin{}
  \figdrawline [10, 20]
  \figdrawline [20, 30]
  \figdrawline [30, 10]
\figdrawend
\figvisu{\figBoxA}{}{%
  \Large
  \figwritesw 10:(5pt)
  \figwritene 20:(5pt)
  \figwritese 30:(5pt)
  \figset write(mark=$\bullet$)
  \figwriten 11:(5)
  \figwritee 21:(5)
  \figwrites 31:(5)
}

\vspace{0.5cm}
\centerline{\box\figBoxA}

\watchout

\begin{solution}[\fullpage]
	
		The mid-point of a line-segment joining two points $(x_1, y_1)$ and $(x_2, y_2)$ is given 
		by $\left( \frac{x_1 + x_2}{2}, \frac{y_1 + y_2}{2}\right)$
		
		And so, the coordinates of $X$, $Y$ and $Z$ would be 
		\begin{align}
			X &= \left( \frac{\va + \vc}{2}, \frac{\vb + \vd}{2} \right) = (\mxone, \myone) \\
			Y &= \left( \frac{\vc + \ve}{2}, \frac{\vd + \vf}{2} \right) = (\mxtwo, \mytwo) \\
			Z &= \left( \frac{\va + \ve}{2}, \frac{\vb + \vf}{2} \right) = (\mxthree, \mythree)
		\end{align}
	
		If $AB_\perp$, $BC_\perp$ and $CA_\perp$ be the perpendicular bisectors of $AB$, $BC$ and $CA$ respectively, then 
		their slopes are given by
		\begin{align}
			m_{AB,\perp} &= -\frac{1}{m_{AB}} = -\dfrac{\vc - \va}{\vd - \vb} = \mab \\
			m_{BC,\perp} &= -\frac{1}{m_{BC}} = -\dfrac{\ve - \vc}{\vf - \vd} = \mbc \\
			m_{CA,\perp} &= -\frac{1}{m_{CA}} = -\dfrac{\va - \ve}{\vb - \vf} = \mca
		\end{align}
		And their equations, therefore, would be 
		\begin{align}
			AB_\perp &= \dfrac{y-\myone}{x-\mxone} = \mab \implies y = \mab x + \cone \\
			BC_\perp &= \dfrac{y-\mytwo}{x-\mxtwo} = \mbc \implies y = \mbc x + \ctwo \\
			CA_\perp &= \dfrac{y-\mythree}{x-\mxthree} = \mca \implies y = \mca x + \cthree					
		\end{align}
	
		$AB_\perp$ and $BC_\perp$ intersect when $ y = \mab x + \cone = \mbc x + \ctwo \implies x = \vg$ 
		and $y = \vh$
		
		$(\vg, \vh)$ also satisfies the equation of $CA_\perp$. You can check this by plugging in 
		$(x,y) = (\vg, \vh)$ into the equation for $CA_\perp$. Which means that \textbf{all three} 
		perpendicular bisectors intersect at one point \textbf{and} that point is $(\vg, \vh)$
\end{solution}

\ifprintanswers\begin{codex}$(\vg, \vh)$\end{codex}\fi

