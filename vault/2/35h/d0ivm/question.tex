


\ifnumequal{\value{rolldice}}{0}{
  % variables 
  \renewcommand{\vbone}{5}
  \renewcommand{\vbtwo}{3}
  \renewcommand{\vbthree}{56}
}{
  \ifnumequal{\value{rolldice}}{1}{
    % variables 
    \renewcommand{\vbone}{4}
    \renewcommand{\vbtwo}{3}
    \renewcommand{\vbthree}{35}
  }{
    \ifnumequal{\value{rolldice}}{2}{
      % variables 
      \renewcommand{\vbone}{3}
      \renewcommand{\vbtwo}{3}
      \renewcommand{\vbthree}{20}
    }{
      % variables 
      \renewcommand{\vbone}{3}
      \renewcommand{\vbtwo}{4}
      \renewcommand{\vbthree}{35}
    }
  }
}

\gcalcexpr[0]{\vbfour}{\vbone+1}
\gcalcexpr[0]{\ans}{\vbthree-\vbfour}
\question[3] In how many ways can $\vbone$ chairs and $\vbtwo$ vases sit so that
not all the vases are sitting together.


\watchout

\ifprintanswers
  % stuff to be shown only in the answer key - like explanatory margin figures
  \begin{marginfigure}
    \figinit{pt}
      \figpt 100:(0,0)
      \figpt 101:(0,0)
    \figdrawbegin{}
      \figdrawline [100,101]
    \figdrawend
    \figvisu{\figBoxA}{}{%
    }
    \centerline{\box\figBoxA}
  \end{marginfigure}
\fi 

\begin{solution}[\mcq]
  First we compute the total number of permutations of seating arrangements
  with no restriction. Next we subtract from this the number of permutations 
  where all the vases sit together. 
  \begin{align}
        \text{N}_{\text{no restriction}} 
            &= \dfrac{(\vbone+\vbtwo)!}{\vbone!\times\vbtwo!} \\
  	\text{N}_{\text{vases together}} 
            &= \dfrac{(\vbone+1)!}{\vbone!} \\ 
        \text{N}_{\text{difference}}     
            &= \vbthree - \vbfour = \ans 
  \end{align}
\end{solution}
