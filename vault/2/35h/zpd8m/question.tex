% This is an empty shell file placed for you by the 'examiner' script.
% You can now fill in the TeX for your question here.

% Now, down to brasstacks. ** Writing good solutions is an Art **. 
% Eventually, you will find your own style. But here are some thoughts 
% to get you started: 
%
%   1. Write to be understood - but be crisp. Your own solution should not take 
%      more space than you will give to the student. Hence, if you take more than 
%      a half-page to write a solution, then give the student a full-page and so on...
%
%   2. Use margin-notes to "talk" to students about the critical insights
%      in the question. The tone can be - in fact, should be - informal
%
%   3. Don't shy away from creating margin-figures you think will help
%      students understand. Yes, it is a little more work per question. 
%      But the question & solution will be written only once. Make that
%      attempt at writing a solution count.
%      
%      3b. Use bc_to_fig.tex. Its an easier way to generate plots & graphs 
% 
%   4. Ensure that there are *no spelling mistakes anywhere*. We are an 
%      education company. Bad spellings suggest that we ourselves 
%      don't have any education. Also, use American spellings by default
% 
%   5. If a question has multiple parts, then first delete lines 40-41
%   6. If a question does not have parts, then first delete lines 43-69
%   
%   7. Create versions of the question when possible. Use commands defined in 
%      tufte-tweaks.sty to do so. Its easier than you think

%\printrubric
% \noprintanswers
%\setcounter{rolldice}{3}

\ifnumequal{\value{rolldice}}{0}{
  % variables 
  \renewcommand{\vbone}{2}
  \renewcommand{\vbtwo}{-3}
  \renewcommand{\vbthree}{3}
  \renewcommand{\vbfour}{4}
  \renewcommand{\vbfive}{-3}
  \renewcommand{\vbsix}{1}
  \renewcommand{\vbseven}{5}
  \renewcommand{\vbeight}{3}
}{
  \ifnumequal{\value{rolldice}}{1}{
    % variables 
    \renewcommand{\vbone}{3}
    \renewcommand{\vbtwo}{2}
    \renewcommand{\vbthree}{2}
    \renewcommand{\vbfour}{7}
    \renewcommand{\vbfive}{2}
    \renewcommand{\vbsix}{9}
    \renewcommand{\vbsix}{-5}
    \renewcommand{\vbseven}{-4}
    \renewcommand{\vbeight}{3}
  }{
    \ifnumequal{\value{rolldice}}{2}{
      % variables 
      \renewcommand{\vbone}{4}
      \renewcommand{\vbtwo}{-5}
      \renewcommand{\vbthree}{3}
      \renewcommand{\vbfour}{8}
      \renewcommand{\vbfive}{-15}
      \renewcommand{\vbsix}{1}
      \renewcommand{\vbseven}{-3}
      \renewcommand{\vbeight}{3}
    }{
      % variables 
      \renewcommand{\vbone}{2}
      \renewcommand{\vbtwo}{8}
      \renewcommand{\vbthree}{5}
      \renewcommand{\vbfour}{3}
      \renewcommand{\vbfive}{7}
      \renewcommand{\vbsix}{1}
      \renewcommand{\vbseven}{4}
      \renewcommand{\vbeight}{3}
    }
  }
}

\FRACMINUS\vbfour\vbfive\vbone\vbtwo\tp\tq
\FRACMINUS\vbsix\vbfive\vbthree\vbtwo\tr\ts
\FRACDIV\tr\ts\tp\tq\ixn\ixd % x-intersect 
\FRACL{-\vbone}{\vbthree}\ixn\ixd\iyn\iyd
\FRACDIV\iyn\iyd\vbtwo{1}\iyn\iyd
\MULTIPLY\ixn\iyd\tp
\MULTIPLY{\ixd}\iyn\tq
\MULTIPLY\ixd\iyd\tr
\MULTIPLY\vbseven\tp\ga
\MULTIPLY\tr\vbseven\gb
\SUBTRACT\ga\tq\aq
\FRACDIV\vbseven{1}\aq\gb\bq\cq


\question[2] If the straight line $L_1: bx - ay = ab$ passes through the point of intersection of 
$L_2: \gsign\vbone x \gsign\vbtwo y = \vbthree$ and $L_3: \gsign\vbfour x \gsign\vbfive y = \vbsix$ 
\textit{and} is parallel to $L_4: y = \gsign\vbseven x - \vbeight$, then find $a$ and $b$

\insertQR{QRC}

\watchout

\ifprintanswers
\fi 

\begin{solution}[\halfpage]
	\begin{align}
		bx - ay &= ab \Rightarrow y = \dfrac{b}{a}x -b \\
		\text{And }\therefore L_1 \parallel L_4 &\Rightarrow \dfrac{b}{a} = \vbseven 
	\end{align}
	Now, $L_2$ and $L_3$ intersect when 
	\begin{align}
		y &= \underbrace{\WRITEFRAC{-\vbone}\vbtwo x  + \WRITEFRAC\vbthree\vbtwo}_{L_2} 
		 = \underbrace{\WRITEFRAC{-\vbfour}\vbfive x + \WRITEFRAC\vbsix\vbfive}_{L_3} \\
		 \Rightarrow x &= \WRITEFRAC\ixn\ixd \text{ and } y = \WRITEFRAC\iyn\iyd
	\end{align}
	And as $L_1$ passes through this point of intersection 
	$ = M\left( \WRITEFRAC\ixn\ixd,\,\WRITEFRAC\iyn\iyd\right)$, 
	\begin{align}
		\left(\WRITEFRAC\ixn\ixd b\right) - \left( \WRITEFRAC\iyn\iyd a \right)&= ab \\
		\Rightarrow (\tp b) - (\tq a) &= \gsign\tr ab \\
		\text{And given that } \dfrac{b}{a} &= \vbseven, \text{ the above becomes } \\
		(\ga a) - (\tq a) &= (\gb a^2) \Rightarrow a = \WRITEFRAC\aq\gb \\
		\therefore b &= \frac{\vbseven}{a} = \WRITEFRAC\bq\cq
	\end{align}
\end{solution}

\ifprintrubric
  \begin{table}
  	\begin{tabular}{ p{5cm}p{5cm} }
  		\toprule % in brief (4-6 words), what should a grader be looking for for insights & formulations
  		  \sc{\textcolor{blue}{Insight}} & \sc{\textcolor{blue}{Formulation}} \\ 
  		\midrule % ***** Insights & formulations ******
        $\frac{b}{a} = $ slope of $L_4$ & Correctly finds intersection of $L_2$ and $L_3$ \\
         & Uses slope of $L_4$ to find $a$ and $b$ \\
  		\toprule % final numerical answers for the various versions
        \sc{\textcolor{blue}{If question has $\ldots$}} & \sc{\textcolor{blue}{Final answer}} \\
  		\midrule % ***** Numerical answers (below) **********
        $L_4: 5x- 8$ & $a=-\frac{2}{3}$ and $b = -\frac{15}{2}$ \\ 
        $L_4: -4x-3$ & $a= -\frac{27}{32}$ and $b = \frac{128}{27}$ \\ 
        $L_4: -3x-3$ & $a=\frac{7}{3}$ and $b = -\frac{9}{7}$ \\ 
        $L_4: 4x-3$ & $a=-\frac{121}{40}$ and $b = -\frac{160}{121}$ \\ 
  		\bottomrule
  	\end{tabular}
  \end{table}
\fi
