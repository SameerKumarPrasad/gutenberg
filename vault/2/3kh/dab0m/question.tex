


\ifnumequal{\value{rolldice}}{0}{
  \newcommand{\cost}{755}  % variables 
  \newcommand{\dep}{18}  % variables 
  \newcommand{\yrs}{5}  % variables 
  \newcommand{\calcaid}{2.5}
  \newcommand{\p}{302}
}{
  \ifnumequal{\value{rolldice}}{1}{
    \newcommand{\cost}{855}  % variables 
    \newcommand{\dep}{15}  % variables 
    \newcommand{\yrs}{4}  % variables 
    \newcommand{\calcaid}{1.8}
    \newcommand{\p}{475}
  }{
    \ifnumequal{\value{rolldice}}{2}{
      \newcommand{\cost}{690}  % variables 
      \newcommand{\dep}{16}  % variables 
      \newcommand{\yrs}{5}  % variables 
      \newcommand{\calcaid}{2.3}
      \newcommand{\p}{300}
    }{
      \newcommand{\cost}{800}  % variables 
      \newcommand{\dep}{14}  % variables 
      \newcommand{\yrs}{5}  % variables 
      \newcommand{\calcaid}{2.0}
      \newcommand{\p}{400}
    }
  }
}

\MULTIPLY\dep\yrs\m
\gcalcexpr[2]\final{\p * \dep / 100.00}

\question[2] A freezer depreciates at the rate of $\dep$\% per year. It
was purchased new for \$$\cost$. How fast is it depreciating in dollar
terms (\textit{in \$\$/year}) when it is exactly $\yrs$ years old.\\
\textit{Calculation Aid: $e^{0.\m}=\calcaid$}

\watchout

\begin{solution}[\halfpage]
  Let the price of the refrigerator as a function of time be $p(t)$.
  Now, we know that,
  \begin{align}
    \dfrac{dp}{dt}    &= -0.\dep p \\
    \dfrac{dp}{p}     &= -0.\dep dt \\
    \int\dfrac{dp}{p} &= \int -0.\dep dt \\
    \ln p(t)          &= -0.\dep t + C
  \end{align}
  Using the fact that at $t=0$, the original price of the refrigerator 
  was $\cost$, we can use the equation in step (4) to get the value of 
  the constant of integration $C$ to be $\ln\cost$. \\
  Substituting this value of $C$ back in the equation we get,
  \begin{align}
    \ln p(t)                &= -0.\dep t + \ln \cost \\
    \ln \dfrac{p(t)}{\cost} &= -0.\dep t \\
    \dfrac{p(t)}{\cost}     &= e^{-0.\dep t} \\
    p(t)                    &= \dfrac{\cost}{e^{0.\dep t}}
  \end{align}
  At $t=\yrs$ years, the price of the refrigerator $p(\yrs)$ can be
  calculated as shown,
  \begin{align}
    p(\yrs) &= \dfrac{\cost}{e^{0.\dep\times\yrs}}
             = \dfrac{\cost}{e^{0.\m}} = \dfrac{\cost}{\calcaid}
             = \p
  \end{align}
  Therefore depreciation rate at the end of the $\yrs_{th}$ year is
  \begin{align}
    0.\dep\times\p = \$\final\,\textit{per year}
  \end{align}

\end{solution}

