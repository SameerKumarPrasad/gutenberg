


\ifnumequal{\value{rolldice}}{0}{
  % variables 
}{
  \ifnumequal{\value{rolldice}}{1}{
    % variables 
  }{
    \ifnumequal{\value{rolldice}}{2}{
      % variables 
    }{
      % variables 
    }
  }
}

\question Consider a database of social groups that allows people to become
members of a group: a person can be a member of several groups and each group
maintains a list of pictures that are accessible to all members. In addition
to the groups, the database also maintains a list of friends. The schema is\\
\begin{tabular}{ll}
  Table   &(field1, field2,...) \\
  \hline
  \texttt{Member}  &(personName, groupName) \\  
  \texttt{Picture} &(groupName, pictureId$^{*}$) \\
  \texttt{Friend}  &(personName1, personName2) \\
  \hline
\end{tabular}\\
\small{* indicates field is a primary key}\\
\normalsize
\texttt{Picture} stores for each picture the name of the group that owns the 
picture. The \texttt{Friend} table is symmetrical, i.e. if X is friends with 
Y then Y is also friends with X. Every person is a member of at least one 
group. Given this, please answer the following:

\begin{parts}
  \part Write an SQL query that computes for every person the total number of
  pictures that they can access through their group memberships. That is, a
  person X can access a picture Y if X is a member of some group Z and the group
  Z owns the \texttt{Picture} Y. The query should return sets like this 
  ('Fred' $12$,'Joe' $7$, 'Sue' $0$, 'Rick' $9$,...)
  
  \begin{solution}
    Since the query is person centric, we can start with \texttt{Member} and
    \textit{join} with \texttt{Picture} on the common column groupName. In order 
    not to lose those members who do cannot have access to any pictures, we 
    need to use a \textit{left outer} instead of a normal \textit{join}.\\
    \begin{tabular}{ll}
      SELECT   &personName, COUNT(*) \\
      FROM     &\texttt{Member} LEFT OUTER JOIN \texttt{Picture} \\
      ON       &\texttt{Member}.groupName = \texttt{Picture}.groupName \\
      GROUP BY &personName \\
    \end{tabular}
  \end{solution}

  \part A cool person is one that has at least $40$ friends. Write an SQL query 
  that returns all the cool people in the database. You need to turn in an 
  SQL query that computes a list of names.

  \begin{solution}
    For this we need to group \texttt{Friend} by personName and select only those
    groups that have a count $> 40$.\\
    \begin{tabular}{ll}
      SELECT   &personName1 \\
      FROM     &\texttt{Friend} \\
      GROUP BY &personName1 \\
      HAVING   &(COUNT(*) > $40$)\\
    \end{tabular}
  \end{solution}

\end{parts}

