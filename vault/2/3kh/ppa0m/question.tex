\newcommand{\vblven}{}
\newcommand{\vbtwlv}{}
\newcommand{\vbthrtn}{}
\newcommand{\vbfrtn}{}
\newcommand{\vbfftn}{}
\newcommand{\vbsxtn}{}
\newcommand{\vbsvntn}{}
\newcommand{\vbeghtn}{}
\newcommand{\vbnntn}{}
\newcommand{\vbtwnty}{}
\newcommand{\vbtone}{}
\newcommand{\vbttwo}{}

\ifnumequal{\value{rolldice}}{0}{
  % variables 
  \renewcommand{\vbone}{ASSASSINATION}
  \renewcommand{\vbtwo}{S}
  \renewcommand{\vbthree}{SSSS}
  \renewcommand{\vbfour}{AAA}
  \renewcommand{\vbfive}{AA}
  \renewcommand{\vbsix}{A}
  \renewcommand{\vbseven}{I,N}
  \renewcommand{\vbeight}{ASSASS\textbf{ININ}ATO}
  \renewcommand{\vbnine}{13}
  \renewcommand{\vbten}{4!3!2!2!}
  \renewcommand{\vblven}{3!2!2!}
  \renewcommand{\vbtwlv}{\dfrac{2}{143}}
  \renewcommand{\vbthrtn}{\dfrac{2}{143}}
  \renewcommand{\vbfrtn}{\textbf{AA}SSASSINTION}
  \renewcommand{\vbfftn}{4!2!2!}
  \renewcommand{\vbsxtn}{\dfrac{1}{26}}
  \renewcommand{\vbsvntn}{\dfrac{25}{26}}
  \renewcommand{\vbeghtn}{\dfrac{5}{13}}
  \renewcommand{\vbnntn}{\dfrac{15}{26}}
  \renewcommand{\vbtwnty}{\dfrac{4!}{2!2!}}
  \renewcommand{\vbtone}{4!3!}
  \renewcommand{\vbttwo}{10}
}{
  \ifnumequal{\value{rolldice}}{1}{
    % variables 
    \renewcommand{\vbone}{OSTEOPOROSIS}
    \renewcommand{\vbtwo}{O}
    \renewcommand{\vbthree}{OOOO}
    \renewcommand{\vbfour}{SSS}
    \renewcommand{\vbfive}{SS}
    \renewcommand{\vbsix}{S}
    \renewcommand{\vbseven}{E,P,R}
    \renewcommand{\vbeight}{OST\textbf{EPR}OSOIOS}
    \renewcommand{\vbnine}{12}
    \renewcommand{\vbten}{4!3!}
    \renewcommand{\vblven}{3!}
    \renewcommand{\vbtwlv}{\dfrac{1}{55}}
    \renewcommand{\vbthrtn}{\dfrac{1}{22}}
    \renewcommand{\vbfrtn}{O\textbf{SS}TEOPOROIS}
    \renewcommand{\vbfftn}{4!}
    \renewcommand{\vbsxtn}{\dfrac{1}{22}}
    \renewcommand{\vbsvntn}{\dfrac{21}{22}}
    \renewcommand{\vbeghtn}{\dfrac{9}{22}}
    \renewcommand{\vbnntn}{\dfrac{6}{11}}
    \renewcommand{\vbtwnty}{3!}
    \renewcommand{\vbtone}{4!3!}
    \renewcommand{\vbttwo}{10}
  }{
    \ifnumequal{\value{rolldice}}{2}{
      % variables 
      \renewcommand{\vbone}{INATTENTIVITY}
      \renewcommand{\vbtwo}{T}
      \renewcommand{\vbthree}{TTTT}
      \renewcommand{\vbfour}{III}
      \renewcommand{\vbfive}{II}
      \renewcommand{\vbsix}{I}
      \renewcommand{\vbseven}{E,N,V}
      \renewcommand{\vbeight}{ATT\textbf{ENNV}TTIIIY}
      \renewcommand{\vbnine}{13}
      \renewcommand{\vbten}{4!3!2!}
      \renewcommand{\vblven}{3!2!}
      \renewcommand{\vbtwlv}{\dfrac{2}{143}}
      \renewcommand{\vbthrtn}{\dfrac{2}{143}}
      \renewcommand{\vbfrtn}{\textbf{II}NATTENTVITY}
      \renewcommand{\vbfftn}{4!2!}
      \renewcommand{\vbsxtn}{\dfrac{1}{26}}
      \renewcommand{\vbsvntn}{\dfrac{25}{26}}
      \renewcommand{\vbeghtn}{\dfrac{5}{13}}
      \renewcommand{\vbnntn}{\dfrac{15}{26}}
      \renewcommand{\vbtwnty}{\dfrac{4!}{2!}}
      \renewcommand{\vbtone}{4!3!}
      \renewcommand{\vbttwo}{10}
    }{
      % variables 
      \renewcommand{\vbone}{MISSISSIPPY}
      \renewcommand{\vbtwo}{S}
      \renewcommand{\vbthree}{SSSS}
      \renewcommand{\vbfour}{III}
      \renewcommand{\vbfive}{II}
      \renewcommand{\vbsix}{I}
      \renewcommand{\vbseven}{M,P,Y}
      \renewcommand{\vbeight}{ISSI\textbf{MPPY}ISS}
      \renewcommand{\vbnine}{11}
      \renewcommand{\vbten}{4!3!2!}
      \renewcommand{\vblven}{3!2!}
      \renewcommand{\vbtwlv}{\dfrac{4}{165}}
      \renewcommand{\vbthrtn}{\dfrac{4}{165}}
      \renewcommand{\vbfrtn}{M\textbf{II}SSISIPPY}
      \renewcommand{\vbfftn}{4!2!}
      \renewcommand{\vbsxtn}{\dfrac{3}{55}}
      \renewcommand{\vbsvntn}{\dfrac{52}{55}}
      \renewcommand{\vbeghtn}{\dfrac{12}{55}}
      \renewcommand{\vbnntn}{\dfrac{40}{55}}
      \renewcommand{\vbtwnty}{\dfrac{4!}{2!}}
      \renewcommand{\vbtone}{4!3!}
      \renewcommand{\vbttwo}{8}
    }
  }
}
\SUBTRACT\vbnine{3}\a
\SUBTRACT\vbnine{2}\b
\SUBTRACT\vbnine{4}\c

\question If the letters of the word $\vbone$ are arranged at 
random. Find the probability that,

\begin{parts}
  \part All $\vbtwo$s come consecutively in a word.

  \begin{solution}
    Let the number of arrangements possible with all
    4 $\vbtwo$s together be $N_{\vbthree}$.
    \begin{align}
      N_{\vbthree} &= \dfrac{\a!}{\vblven}
    \end{align}
    Total number of arrangements of the word,
    \begin{align}
      N_{total} &= \dfrac{\vbnine!}{\vbten}
    \end{align}
    Probability that 4 $\vbtwo$s come together,
    \begin{align}
      P_{\vbthree} &= \dfrac{N_{\vbthree}}{N_{total}} \\
                   &= \dfrac{\a!}{\vblven}\times\dfrac{\vbten}{\vbnine!} 
                    = \vbtwlv
    \end{align}
  \end{solution}

  \part All the $\vbseven$ are together \textit{e.g. \vbeight}

  \begin{solution}
    Let the number of arrangements possible with the $\vbseven$
    together be $N_{\vbseven}$. We will calculate this by finding
    the number of arrangements of $\vbseven$ in of themselves and
    multiply this by the number of arrangements of the rest of the
    letters but with the $\vbseven$ treated as a single letter,
    \begin{align}      
      N_{\vbseven} &= \underbrace{\vbtwnty}_\text{\vbseven only}\times
                      \overbrace{\dfrac{\vbttwo!}{\vbtone}}^\text{rest of the letters}
    \end{align}
    Probability that $\vbseven$ come together,
    \begin{align}
      P_{\vbseven} &= \dfrac{N_{\vbseven}}{N_{total}} \\
                   &= \vbtwnty\times\dfrac{\a!}{\vbtone}
                      \times\dfrac{\vbten}{\vbnine!}
                    = \vbthrtn
    \end{align}
  \end{solution}

  \part Not all $\vbsix$ come together. \textit{(includes cases
  where two $\vbsix$ are together but the third is not e.g.\vbfrtn)}

  \begin{solution}
    To find the probability that all $\vbsix$ are not together,
    find the probability that all $\vbsix$ are together and take
    it's complement.\\
    Let the number of arrangements with all $\vbsix$ together be
    $N_{\vbfour}$.
    \begin{align}
      N_{\vbfour} &= \dfrac{\b}{\vbfftn}
    \end{align}
    Probability that all $\vbsix$ are together,
    \begin{align}
      P_{\vbfour} &= \dfrac{N_{\vbfour}}{N_{total}} \\
                  &= \dfrac{\b}{\vbfftn}\times\dfrac{\vbten}{\vbnine!}
                   = \vbsxtn
    \end{align}
    Therefore, probability that not all $\vbsix$ are together,
    \begin{align}
      P'_{\vbfour} &= 1 - P_{\vbfour} \\
                   &= \vbsvntn
    \end{align}
  \end{solution}

  \part No two $\vbsix$ come together i.e. no $\vbsix$ in the word 
  has another $\vbsix$ adjacent to it.\textit{e.g. \vbone}

  \begin{solution}
    Let probability that exactly two $\vbsix$ occur together in a
    word be $P_{\vbfive}$ and $none$ of the $\vbsix$ occur together 
    in a word be $P_\vbsix$. Using this notation and $P_{\vbfour}$ 
    from c) we can declare that,
    \begin{align}
      P_{\vbfour} + P_{\vbfive} + P_{\vbsix} = 1
    \end{align}
    We already know $P_{\vbfour}$ from c). Next let us find
    $P_{\vbfive}$. The two $\vbsix$ may occur on the edge 
    (beginning or end) of the word, or somewhere in the middle. 
    Each of these need to be counted separately ($N_{edge}$, 
    $N_{middle}$).
    \begin{align}
      N_{\vbfive} &= N_{edge} + N_{middle} \\
                  &= \dfrac{\a\times 2\times \a!}{\vbfftn} +
                     \dfrac{\a\times \c\times \a!}{\vbfftn} \\
                  &= \dfrac{\a\times \b!}{\vbfftn}
    \end{align}
    Therefore, probability that exactly two $\vbsix$ are together,
    \begin{align}
      P_{\vbfive} &= \dfrac{N_{\vbfive}}{N_{total}} \\
                  &= \dfrac{\a\times \b!}{\vbfftn}
                     \times\dfrac{\vbten}{\vbnine!}
                   = \vbeghtn
    \end{align}
    Therefore, probability that each of the three $\vbsix$ are by
    themselves,
    \begin{align}
      P_{\vbsix} &= 1 - P_{\vbfour} - P{\vbfive} \\
                 &= 1 - \vbsxtn - \vbeghtn \\
                 &= \vbnntn
    \end{align}  
  \end{solution}

\end{parts}