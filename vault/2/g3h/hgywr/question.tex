% This is an empty shell file placed for you by the 'examiner' script.
% You can now fill in the TeX for your question here.

% Now, down to brasstacks. ** Writing good solutions is an Art **. 
% Eventually, you will find your own style. But here are some thoughts 
% to get you started: 
%
%   1. Write the solution as if you are writing it for your favorite
%      14-17 year old to help him/her understand. Could be your nephew, 
%      your niece, a cousin perhaps or probably even you when you 
%      were that age. Just write for them.
%
%   2. Use margin-notes to "talk" to students about the critical insights
%      in the question. The tone can be - in fact, should be - informal
%
%   3. Don't shy away from creating margin-figures you think will help
%      students understand. Yes, it is a little more work per question. 
%      But the question & solution will be written only once. Make that
%      attempt at writing a solution count.
%
%   4. At the same time, do not be too verbose. A long solution can
%      - at first sight - make the student think, "God, that is a lot to know".
%      Our aim is not to scare students. Rather, our aim should be to 
%      create many "Aha!" moments everyday in classrooms around the world
% 
%   5. Ensure that there are *no spelling mistakes anywhere*. We are an 
%      education company. Bad spellings suggest that we ourselves 
%      don't have any education. Also, use American spellings by default
% 
%   6. If a question has multiple parts, then first delete lines 40-41
%   7. If a question does not have parts, then first delete lines 43-69

\question[4] Two dots zip along the circumference of a circle. If they move in opposite directions they meet every $15sec$ whereas if they move in the same direction then meet every $60sec$. If the circumference of the circle is $1.2m$, calculate the speed of the dots? \textit{(Use speed = distance/time)}

\insertQR{QRC}

\ifprintanswers
  % stuff to be shown only in the answer key - like explanatory margin figures
\fi 

\begin{solution}[\halfpage]
  Let the speeds of the dots be $v_1(m/sec)$ and $v_2(m/sec)$. To meet the dots have to traverse the circumference of the circle. Therefore,
  \begin{align}
    \dfrac{1.2(m)}{15(sec)} &= v_1(m/sec)+v_2(m/sec) 
	\quad\quad\textit{(opp. directions)} \\
    \dfrac{1.2(m)}{60(sec)} &= v_1(m/sec)-v_2(m/sec) 
	\quad\quad\textit{(same direction)}
  \end{align}
  Add equation (1) and (2), 
  \begin{align}
    \dfrac{1.2}{15}+\dfrac{1.2}{60} &= 2v_1 \\
    v_1                             &= 0.05 \\
    v_2                             &= 0.03
  \end{align}
  The speeds of the dots are $0.05(m/sec)$ and $0.03(m/sec)$.
\end{solution}

