

\question[4] A tank of capacity $2400m^3$ is full of water. The rate at which water discharges from the tank is $10m^3/min$ higher than the rate at which it fills. As a result it takes $8min$ more to fill the tank than it takes to empty it. Find the rate at which the tank fills?


\ifprintanswers
  % stuff to be shown only in the answer key - like explanatory margin figures
\fi 

\begin{solution}[\halfpage]
  If $t_f(min)$ is the time it takes to fill the tank at rate $r_f(m^3/min)$ and $t_e(min)$ is the time it takes to empty the tank at rate $r_e(m^3/min)$, then as per the question,
  \begin{align}
    r_f(m^3/min) &= r_e(m^3/min) - 10(m^3/min) \\
    t_f(min)     &= t_e(min)+8(min)
  \end{align}
  
  We know that $(rate=volume/time)$. Substituting $times$ with the corresponding $rates$ in equation (2) we get,
  \begin{align}
    \dfrac{2400(m^3)}{r_f(m^3/min)} &= 
    	\dfrac{2400(m^3)}{r_e(m^3/min)}+8(min) 
  \end{align}
  
  Combining equations (1) and (3),
  \begin{align}      
    \dfrac{2400}{r_f}               &= \dfrac{2400}{r_f+10}+8 \\
    r_f^2 + 10r_f - 3000            &= 0 \\
    r_f^2 + 60r_f -50r_f - 3000     &= 0 \\
    r_f                             &= 50, -60
  \end{align}
  Negative value is rejected, rate at which the tank fills is $50(m^3/min)$.
\end{solution}

