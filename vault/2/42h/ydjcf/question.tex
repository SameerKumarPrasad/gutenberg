
\ifnumequal{\value{rolldice}}{0}{
  % variables 
  \renewcommand{\va}{19}
  \renewcommand{\vd}{3}
  \renewcommand{\vb}{16}
  \renewcommand{\ve}{5}
}{
  \ifnumequal{\value{rolldice}}{1}{
    % variables 
    \renewcommand{\va}{12}
    \renewcommand{\vd}{3}
    \renewcommand{\vb}{9}
    \renewcommand{\ve}{5}
  }{
    \ifnumequal{\value{rolldice}}{2}{
      % variables 
      \renewcommand{\va}{11}
      \renewcommand{\vd}{5}
      \renewcommand{\vb}{13}
      \renewcommand{\ve}{7}
    }{
      % variables 
      \renewcommand{\va}{14}
      \renewcommand{\vd}{3}
      \renewcommand{\vb}{11}
      \renewcommand{\ve}{7}
    }
  }
}

\ADD\va\vd\a
\ADD\a\vd\b

\ADD\vb\ve\m
\ADD\m\ve\n

\SUBTRACT\va\vb\c
\SUBTRACT\ve\vd\d
\ADD\c\d\e

\SUBTRACT\va\vd\vx
\MULTIPLY\vd\ve\vy
\EXPR[0]\vz{ 5*(2*\vx + 9*\vy)}

\question[4] Given two arithmetic progressions 
\[ A = \lbrace\va,\a,\b\ldots\rbrace\text{ and }B = \lbrace\vb,\m,\n\ldots\rbrace \] 
find the sum of the first ten terms \textbf{common} to $A$ and $B$.
% $A = 17,21,25,29 ...$ and $B = 16,21,26,31 ...$

\watchout

\begin{solution}[\halfpage]
  A term in $A$ is of the form 
  \[ a_m = \va + (m - 1)\cdot\vd \]
  Similarly, a term in $B$ is of the form 
  \[ b_n = \vb + (n-1)\cdot\ve \]
  Note, however, that 
  \[ m \neq n\text{ and } m,n\in\mathbb{N}\implies m,n\text{ are positive integers}\] 
  Now, for $a_m = b_n$, 
  \begin{align}
    \va + (m-1)\cdot\vd &= \vb + (n-1)\cdot\ve \\
    \implies \ve n&=\left[ \vd\cdot m + (\va-\vb) + (\ve-\vd) \right] \\
    &= (\vd\cdot m + \e)
  \end{align}
  \textbf{Now the catch!}
  \[ n\in\mathbb{N}\implies \ve n\in\mathbb{N}\implies (\vd\cdot m + \e)\text{ is a multiple of }\ve \]
  And the above is possible \textbf{only if}
  \[ m=\ve k,\, k\in\mathbb{N} \] 
  Only for these values of $m$ would a term $(C_k)$ be \textbf{common to both series}.
  \begin{align}
    C_k &= \va + [\ve\cdot k - 1]\cdot \vd = \underbrace{\vx + \vy\cdot k}_{a=\vx,\, d=\vy}
  \end{align}
  And hence, the sum of the first $10$ terms is
  \begin{align}
    S_{10} &= \dfrac{10}{2}\cdot\left[ 2\cdot\vx + (10-1)\cdot\vy \right] = \vz 
  \end{align}
\end{solution} 
\ifprintanswers\begin{codex}$\vz$\end{codex}\fi
