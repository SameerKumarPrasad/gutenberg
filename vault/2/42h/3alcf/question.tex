
\ifnumequal{\value{rolldice}}{0}{
  \renewcommand\vs{13}
  \renewcommand\vq{91}
  \renewcommand\vk{3}
}{
  \ifnumequal{\value{rolldice}}{1}{
    \renewcommand\vs{21}
    \renewcommand\vq{273}
    \renewcommand\vk{4}
  }{
    \ifnumequal{\value{rolldice}}{2}{
      \renewcommand\vs{14}
      \renewcommand\vq{84}
      \renewcommand\vk{2}
    }{
      \renewcommand\vs{21}
      \renewcommand\vq{189}
      \renewcommand\vk{2}
    }
  }
}

\SQUARE\vs\va
\MULTIPLY\vs{2}\vb
\SUBTRACT\va\vq\vc
\FRACTIONSIMPLIFY\vc\vb\vm\a % a = 1
\SUBTRACT\vs\vm\vn

\DIVIDE\vm\vk\vx
\MULTIPLY\vm\vk\vy

\question[3] The sum of three consecutive terms in a geometric progression is $\vs$ 
and the sum of their squares is $\vq$. What are the numbers? 

\watchout

\begin{solution}[\halfpage]
  If the three consecutive terms be $a, ar$ and $ar^2$, then 
  \begin{align}
    a + ar + ar^2 &= a\cdot (1+r +r^2) = \vs \\
    a^2 + a^2r^2 + a^2r^4 &= a^2\cdot(1 + r^2 + r^4) = \vq
  \end{align} 
  \textbf{Insight \#1}
  \begin{align}
    1+r^2+r^4 &= (1+r+r^2)^2 - 2r - 2r^ - 2r^3 \\
    &= (1+r+r^2)^2 - 2r\cdot (1+r+r^2)
  \end{align}
  And therefore, 
  \begin{align}
    \underbrace{a^2\cdot (1+r^2 + r^4)}_{=\vq} &= 
    \underbrace{a^2\cdot(1+r+r^2)^2}_{=\vs^2} - 2\cdot ar\cdot\underbrace{a\cdot (1+r+r^2)}_{=\vs} \\
    \implies\vb\cdot ar &= \va - \vq = \vc \\
    \implies\underbrace{(a\cdot r)}_{\text{middle term}} &= \vm\implies a = \dfrac\vm{r} \\
    \therefore a\cdot (1+r+r^2) &= \dfrac\vm{r}\cdot (1+r+r^2) = \vs \\
    \implies \vm r^2 -\vn r + \vm &= 0 
  \end{align}
  Solving this last quadratic equation for $r$ gives 

  \begin{tabular}{c c c} 
    \toprule
      $r$ & $a=\frac\vm{r}$ & Numbers \\
    \midrule
      $\frac{1}\vk$ & $\vy$ & $(\vy,\vm,\vx)$ \\ 
      $\vk$ & $\vx$ & $(\vx,\vm,\vy)$ \\ 
    \bottomrule
  \end{tabular}

  Either way, the three numbers are 
  \[ \vx,\vm\text{ and }\vy \] 
\end{solution}

\ifprintanswers\begin{codex}$\vx,\vm\text{ and }\vy$\end{codex}\fi
