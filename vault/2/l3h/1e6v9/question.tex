\ifnumequal{\value{rolldice}}{0}{
  \renewcommand{\vbone}{4}
  \renewcommand{\vbtwo}{6}
  \renewcommand{\vbthree}{24}
}{
  \ifnumequal{\value{rolldice}}{1}{
    \renewcommand{\vbone}{5}
    \renewcommand{\vbtwo}{24}
    \renewcommand{\vbthree}{120}
  }{
    \ifnumequal{\value{rolldice}}{2}{
      \renewcommand{\vbone}{6}
      \renewcommand{\vbtwo}{120}
      \renewcommand{\vbthree}{720}
    }{
      \renewcommand{\vbone}{3}
      \renewcommand{\vbtwo}{2}
      \renewcommand{\vbthree}{6}
    }
  }
}

\gcalcexpr[0]\final{\vbthree * \vbtwo}

\question[4] How many ways can $\vbone$ boys and $\vbone$ girls sit alternating around a circular table (Boy-Girl-Boy...)?

\insertQR{QRC}

\watchout

\ifprintanswers
  % stuff to be shown only in the answer key - like explanatory margin figures
  \marginnote[80pt]{When the girls are being seated in a circle, the seats themselves are indistinguishable from each other. It is only after the first girl is seated that we actually have a point of reference at the table. Hence the $-1$ in counting the total permutations}
\fi 

\begin{solution}[\halfpage]
  The only way to satisfy this condition is to seat the girls (or boys) in alternating seats leaving a gap between each one, and then seating the others in those gaps. Number of ways we can seat the $\vbone$ girls on around the circular table such that there is one empty spot between them is,
  \begin{align}
    N_{girls} = (\vbone-1)! = \vbtwo 
  \end{align}
  The number of ways we can seat the four boys in these four spots is,
  \begin{align}
    N_{boys} = \vbone! = \vbthree
  \end{align}

  Therefore, the total number of ways $N_{total}$, in which they can all be seated is,
    \begin{align}
      N_{total} &= N_{girls} \times N_{boys} \\
                &= \vbtwo \times \vbthree \\
                &= \final 
    \end{align}

\end{solution}

