% This is ringor you by the 'examiner' script.
% You can now fill in the TeX for your question here.

% Now, down to brasstacks. ** Writing good solutions is an Art **. 
% Eventually, you will find your own style. But here are some thoughts 
% to get you started: 
%
%   1. Write the solution as if you are writing it for your favorite
%      14-17 year old to help him/her understand. Could be your nephew, 
%      your niece, a cousin perhaps or probably even you when you 
%      were that age. Just write for them.
%
%   2. Use margin-notes to "talk" to students about the critical insights
%      in the question. The tone can be - in fact, should be - informal
%
%   3. Don't shy away from creating margin-figures you think will help
%      students understand. Yes, it is a little more work per question. 
%      But the question & solution will be written only once. Make that
%      attempt at writing a solution count.
%
%   4. At the same time, do not be too verbose. A long solution can
%      - at first sight - make the student think, "God, that is a lot to know".
%      Our aim is not to scare students. Rather, our aim should be to 
%      create many "Aha!" moments everyday in classrooms around the world
% 
%   5. Ensure that there are *no spelling mistakes anywhere*. We are an 
%      education company. Bad spellings suggest that we ourselves 
%      don't have any education. Also, use American spellings by default
% 
%   6. If a question has multiple parts, then first delete lines 40-41
%   7. If a question does not have parts, then first delete lines 43-69

%\noprintanswers

\question[6] In how many ways can the six faces of a cube be numbered - using 1 through 6 
so that no two cubes that have been numbered differently look the same in any orientation?
\texttt{Hint:} Unroll the cube and lay its faces flat out on a table

\insertQR[15pt]{QRC}

\ifprintanswers
  % stuff to be shown only in the answer key - like explanatory margin figures
  \begin{marginfigure}
    \figinit{pt}
      \figpt 100:(20,0)
      \figpt 101:(40,0)
      \figpt 102:(40,20)
      \figpt 103:(40,40)
      \figpt 104:(60,40)
      \figpt 105:(60,60)
      \figpt 106:(40,60)
      \figpt 107:(40,80)
      \figpt 108:(20,80) 
      \figpt 109:(20,60)
      \figpt 110:(0,60)
      \figpt 111:(0,40)
      \figpt 112:(20,40)
      \figpt 113:(20,20)
      \figpt 200:$1$(30,70)
      \figpt 201:$2$(50,50)
      \figpt 202:$3$(30,30)
      \figpt 203:$4$(10,50)
    \figdrawbegin{}
      \figdrawline[100,101,102,103,104,105,106,107,108,109,110,111,112,113,100]
      \figset (dash=8)
      \figdrawline[103,106,109,112,103]
      \figset (fillmode=yes, color=0.7)
      \figdrawline[100,101,102,113,100]
    \figdrawend
    \figvisu{\figBoxA}{}{%
      \figwriten 200:(0)
      \figwritee 201:(0)
      \figwrites 202:(0)
      \figwritew 203:(0)
    }
    \centerline{\box\figBoxA}
  \end{marginfigure}
  
  \marginnote[10pt]{These $30$ cubes are called MacMohan cubes after the British mathematician Percy Alexander MacMohan}
\fi 

\begin{solution}[\halfpage]
  There are two ways to go about this problem.\\
  First method:\\
  Let us fix a point of reference on the cube by numbering the \textbf{top}
  face $1$. Now, we can number all of the other faces with reference to it.
  The bottom face can be numbered in,
  \begin{align}
     N_{bottom} = 6 - 1 = 5 \textit{ ways}
  \end{align}
  The four side faces are in a closed loop. The number of ways in which
  $n$ distinct objects can be arranged in a closed loop is $(n-1)!$, 
  therefore the side faces can be numbered in,
  \begin{align}
     N_{sides} = (4 - 1) = 3! = 6 \textit{ ways}
  \end{align}
  Therefore the total unique ways to number the cube are,
  \begin{align}
    N_{unique} &= N_{bottom} \times N_{sides} \\
               &= 5 \times 3! = 30 
  \end{align}
  \\
  Second method:\\
  The maximum ways in which six unique numbers can be assigned to six
  places (or sides of a cube) is,
  \begin{align}
    N_{maximum} = 6! 
  \end{align}
  However this includes patterns which really are the same cube, just
  oriented differently. The number of such orientations possible for 
  a given six sided cube numbered $1$ through $6$ is,
  \begin{align}
  N_{orientations} = 6 \times 4 = 24 \quad\text{(4 with each digit as base)}
  \end{align}
  By factoring out the number of orientations of the same cube from the 
  \textit{maximum} number we get,
  \begin{align}
    N_{unique} &= \dfrac{N_{maximum}}{N_{orientations}} \\
               &= \dfrac{6!}{24} = 30 \textit{ ways}
  \end{align}   
\end{solution}

