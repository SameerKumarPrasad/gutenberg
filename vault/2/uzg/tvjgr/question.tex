% This is an empty shell file placed for you by the 'examiner' script.
% You can now fill in the TeX for your question here.

% Now, down to brasstacks. ** Writing good solutions is an Art **. 
% Eventually, you will find your own style. But here are some thoughts 
% to get you started: 
%
%   1. Write the solution as if you are writing it for your favorite
%      14-17 year old to help him/her understand. Could be your nephew, 
%      your niece, a cousin perhaps or probably even you when you 
%      were that age. Just write for them.
%
%   2. Use margin-notes to "talk" to students about the critical insights
%      in the question. The tone can be - in fact, should be - informal
%
%   3. Don't shy away from creating margin-figures you think will help
%      students understand. Yes, it is a little more work per question. 
%      But the question & solution will be written only once. Make that
%      attempt at writing a solution count.
%
%   4. At the same time, do not be too verbose. A long solution can
%      - at first sight - make the student think, "God, that is a lot to know".
%      Our aim is not to scare students. Rather, our aim should be to 
%      create many "Aha!" moments everyday in classrooms around the world
% 
%   5. Ensure that there are *no spelling mistakes anywhere*. We are an 
%      education company. Bad spellings suggest that we ourselves 
%      don't have any education. And, use American spellings
\question[4]  An amateur investor has \texteuro 83,000 that he would like to invest in the
stock-market. But rather than invest all the money in one go, he decides to invest 
a certain amount in the first year and in every subsequent year, \texteuro 300 more than
in the preceding year - over a total period of 20 years. How much money would the investor 
have invested in the $13^{th}$ year?

\insertQR{QRC}

\ifprintanswers
  % stuff to be shown only in the answer key - like explanatory margin figures
  \marginnote[2cm] {$\text{S}_{n} = \dfrac{n}{2}
  \left(2\text{a}+\left(\text{n}-1\right)\text{d}\right)$}
  \marginnote[3cm] {$\text{a}_{n} = \text{a} + \left(\text{n}-1\right)\text{d}$} 
\fi 

\begin{solution}[\halfpage]
  If $A$ be the amount the investor invests in the first year, then over 20 years $(N=20)$, his
  total investment $(S_N)$ is given by, 

  \begin{align}
    S_N &= \dfrac{N}{2}\cdot(2A + (N-1)\cdot d)
  \end{align}
  where $d=\text{\texteuro 300}$

  \begin{align}
    \Rightarrow \text{\texteuro 83,000} &= \dfrac{20}{2}\cdot(2A + (20-1)\cdot 300) \\
    \Rightarrow A &= \text{\texteuro 1,300}
  \end{align}

  And therefore, the amount the investor invests in the $13^{th}$ year $(=A_{13})$ would be
  \begin{align}
    A_{13} &= A + (13-1)\cdot d \\
           &= \text{\texteuro 1,300} + 12\cdot \text{\texteuro 300} \\
           &= \text{\texteuro 4,900}
  \end{align}
\end{solution}
