% This is an empty shell file placed for you by the 'examiner' script.
% You can now fill in the TeX for your question here.

% Now, down to brasstacks. ** Writing good solutions is an Art **. 
% Eventually, you will find your own style. But here are some thoughts 
% to get you started: 
%
%   1. Write the solution as if you are writing it for your favorite
%      14-17 year old to help him/her understand. Could be your nephew, 
%      your niece, a cousin perhaps or probably even you when you 
%      were that age. Just write for them.
%
%   2. Use margin-notes to "talk" to students about the critical insights
%      in the question. The tone can be - in fact, should be - informal
%
%   3. Don't shy away from creating margin-figures you think will help
%      students understand. Yes, it is a little more work per question. 
%      But the question & solution will be written only once. Make that
%      attempt at writing a solution count.
%
%   4. At the same time, do not be too verbose. A long solution can
%      - at first sight - make the student think, "God, that is a lot to know".
%      Our aim is not to scare students. Rather, our aim should be to 
%      create many "Aha!" moments everyday in classrooms around the world
% 
%   5. Ensure that there are *no spelling mistakes anywhere*. We are an 
%      education company. Bad spellings suggest that we ourselves 
%      don't have any education. Also, use American spellings by default
% 
%   6. If a question has multiple parts, then first delete lines 40-41
%   7. If a question does not have parts, then first delete lines 43-69

\question The rate at which people enter an auditorium for a rock concert is modeled by the function $R$ given by $R(t)=1380t^2-675t^3$ for $0 \leq t \leq 2$ hours; $R(t)$ is measured in people per hour. No one is in the auditorium at time $t=0$, when the doors open. The doors close and the concert begins at time $t=2$. 

\ifprintanswers
  % stuff to be shown only in the answer key - like explanatory margin figures
\fi 

\begin{parts}
  \part[2] How many people are in the auditorium when the concert begins?
  \insertQR{QRC}
\begin{solution}[\halfpage] 
    The function we have been given $R(t)$ is actually $\dfrac{\ud}{\ud t}N(t)$, where
    $N(t)$ is the number of people inside the auditorium at any given time $t$. And therefore,
    \begin{align}
      \text{N(2)} &= \int_0^2 R(t)\ud t \\
            &= \int_0^2 1380t^2-675t^3 \ud t \\
			&= \left[1380(\dfrac{t^3}{3})-675(\dfrac{t^4}{4})\right]_0^2 \\
			&= 980
    \end{align}
  \end{solution}

  \part[2] Find the time when the rate at which people enter the auditorium is maximum. Justify your answer
  \insertQR{QRC}
\begin{solution}[\halfpage]
    The rate at which people enter - $R(t)$ - itself changes with time. And it is maximum 
    or minimum - when 
    \begin{align}
        \dfrac{\ud}{\ud t}R(t) = \dfrac{\ud}{\ud t}(1380t^2-675t^3) &= 0 \\
        \Rightarrow 2760t - 2025t^2 = t\cdot(2760 - 2025t) &= 0 \\
        \text{that is, when } t = 0, 1.363 \text{ hours} 
    \end{align}
    
    Moreover,
    \begin{align}
       \dfrac{\ud^2}{\ud t^2}(2760t - 2025t^2) &= 2760 - 4050t \\
                                   &> 0 \text{ when } t= 0 \Rightarrow \text{ minima } \\
                                   &< 0 \text { when } t = 1.363 \Rightarrow \text{ maxima }
    \end{align}
    Hence, the rate at which people enter the auditorium is \textit{lowest} when $t=0$ and
    \textit{highest} when $t=1.363$ hours
   
  \end{solution}

\nextpg
  \part[2] The total wait time for all the people in the auditorium is found by adding the time each person waits, starting at the time the person enters the auditorium and ending when the concert begins. The function $w$ models the total wait time for all the people who enter the auditorium before time $t$. The derivative of $w$ is given by $w'(t)=(2-t)R(t)$. Find $w(2)-w(1)$, the total wait time for those who enter the auditorium after time $t=1$. 
  \insertQR{QRC}
\begin{solution}[\halfpage]
    Let us begin by expanding $w'(t)$,
    \begin{align}
      w'(t) &= (2-t)R(t) \\
                  &= (2-t)(1380t^2-675t^3) \\
                  &= 2760t^2-2730t^3+675t^4
    \end{align}
    To compute $w(2)-w(1)$, we can solve the definite integral of $w'(t)$ from $t=1$ to $t=2$. 
    \begin{align}
      w(2)-w(1) &= \int_1^2 2760t^2-2730t^3+675t^4 \ud t \\
                            &= \left[2760(\dfrac{t^3}{3})-2730(\dfrac{t^4}{4})+675(\dfrac{t^5}{5})\right]_1^2 \\
                            &= 387.5               
    \end{align}
    The total wait time for those entering the auditorium after $t=1$ hour is $387.5$ hours.    
  \end{solution}

  \part[2] On average, how long does a person wait in the auditorium for the concert to begin? Consider all people who enter the auditorium after the doors open, and use the model for total wait time from part(c).
  \insertQR{QRC}
  \ifprintanswers
     \marginnote[5cm]{Borrowing result from part(c) Step 6}
  \fi
\begin{solution}[\halfpage]
    To find the average wait time we need to calculate the total wait time and divide it by the total number of people entering the auditorium. Let total wait time for all the people entering the auditorium be $T$.
    \begin{align}
      T &= \left[2760(\dfrac{t^3}{3})-2730(\dfrac{t^4}{4})+675(\dfrac{t^5}{5})\right]_0^2 \\
      	  &= 760
    \end{align}
    To find the total number of people who entered the auditorium, we use the result from part(a). Combining these two results, the average wait time would be,
    \begin{align}
     T_{avg} &= \dfrac{760}{980} \\
                        &= 0.77551
    \end{align}
   $0.77551$ hours or $46.53$ minutes.
  \end{solution}

\end{parts}
