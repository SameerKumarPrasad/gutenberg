% This is an empty shell file placed for you by the 'examiner' script.
% You can now fill in the TeX for your question here.

% Now, down to brasstacks. ** Writing good solutions is an Art **. 
% Eventually, you will find your own style. But here are some thoughts 
% to get you started: 
%
%   1. Write the solution as if you are writing it for your favorite
%      14-17 year old to help him/her understand. Could be your nephew, 
%      your niece, a cousin perhaps or probably even you when you 
%      were that age. Just write for them.
%
%   2. Use margin-notes to "talk" to students about the critical insights
%      in the question. The tone can be - in fact, should be - informal
%
%   3. Don't shy away from creating margin-figures you think will help
%      students understand. Yes, it is a little more work per question. 
%      But the question & solution will be written only once. Make that
%      attempt at writing a solution count.
%
%   4. At the same time, do not be too verbose. A long solution can
%      - at first sight - make the student think, "God, that is a lot to know".
%      Our aim is not to scare students. Rather, our aim should be to 
%      create many "Aha!" moments everyday in classrooms around the world
% 
%   5. Ensure that there are *no spelling mistakes anywhere*. We are an 
%      education company. Bad spellings suggest that we ourselves 
%      don't have any education. Also, use American spellings by default
% 
%   6. If a question has multiple parts, then first delete lines 40-41
%   7. If a question does not have parts, then first delete lines 43-69

\question  Let $R$ be the region in the first quadrant enclosed by the graphs
$f(x) = 8x^3$ and $g(x)=sin(\pi x)$, as shown in the figure

\ifprintanswers
  % stuff to be shown only in the answer key - like explanatory margin figures
\fi 


\begin{parts}
\part[2] \part Write the equation for the line tangent to $f(x)$ at $x=\dfrac{1}{2}$
  \insertQR{}
\begin{solution}[\mcq]
  	The said tangent would be tangent at the point $(\dfrac{1}{2}, f(\dfrac{1}{2}))$.  	
  	Which means, at $\left(\dfrac{1}{2}, 8\cdot\left( \dfrac{1}{2}\right)^3\right)$ = $(1,1)$
  	
  	Also,
  	\begin{align}
  		\text{Slope of the tangent} &= \dfrac{\ud f(x)}{\ud x} \\
  		&= \dfrac{\ud}{\ud x}8x^3 = 24x^2 \\
  		&= 24\cdot\left(\dfrac{1}{2}\right)^2 = 6 \text{ when } x=\dfrac{1}{2}
  	\end{align}
  	
  	The equation of the line, therefore, is
  	
  	\begin{align}
  		\dfrac{y-1}{x-1} &= 6 \\
  		\Rightarrow y &= 6x - 5
  	\end{align}
  	
  \end{solution}

\part[3] \part Find the area of $R$
  \insertQR{}
\begin{solution}[\halfpage]
  	From the figure, we can see that the curves $f(x)$ and $g(x)$ intersect when $x=\dfrac{1}{2}$
  	
  	And therefore, the area $R$ contained between them is given by
  	\begin{align}
  		R &= \int_0^{\frac{1}{2}}\sin(\pi x)\ud x - \int_0^{\frac{1}{2}}8x^3\ud x \\
  		  &= \left[ \dfrac{-\cos(\pi x)}{\pi}\right]_0^{\frac{1}{2}} -
  		     \left[ \dfrac{8}{4}x^4\right]_0^{\frac{1}{2}} \\
  		  &= \dfrac{1}{\pi} - \dfrac{1}{8} = 0.19
  	\end{align}
  \end{solution}

\newpage
\part[3] \part Write, but do not evaluate, an integral expression for the volume of the solid
  generated when $R$ is rotated about the horizontal line $y=1$
  \insertQR{}
\begin{solution}[\halfpage]
  	This is a tricky one because you now have to switch axis. Thus far, we were 
  	expressing $y$ as a function of $x$. But now, we need to travel along the y-axis.
  	And therefore, we need to express $x$ as a function of $y$
  	
  	Which means, that if 
  	\begin{align}
		y = 8x^3 \text{ then } x &= \dfrac{1}{3}\sqrt[3]{y} \\
		y = \sin(\pi x) \text{ then } x &= \dfrac{1}{\pi}\sin^{-1}y
  	\end{align}
  	
  	For any given $x$, the volume generated by an infinitisimal strip of $R$ is given by
  	\begin{align}
  		\ud V &= \dfrac{1}{3}\sqrt[3]{y}\cdot(2\pi\cdot(1-y))\ud y -
  		         \dfrac{1}{\pi}\sin^{-1}y\cdot(2\pi\cdot(1-y))\ud y \\
  		      &= 2\pi\cdot\left[ \dfrac{1}{3}\sqrt[3]{y}(1-y)\ud y -
  		         \dfrac{1}{\pi}(1-y)\sin^{-1}y\ud y \right]
  	\end{align}
  	
  	And therefore, the total volume generated is
  	\begin{align}
  		V &= 2\pi\cdot\left[ \dfrac{1}{3}\int_0^1\sqrt[3]{y}(1-y)\ud y -
  		         \dfrac{1}{\pi}\int_0^1(1-y)\sin^{-1}y\ud y \right]
  	\end{align}
  \end{solution}

\end{parts}
