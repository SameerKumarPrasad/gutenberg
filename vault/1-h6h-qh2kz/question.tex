% This is an empty shell file placed for you by the 'examiner' script.
% You can now fill in the TeX for your question here.

% Now, down to brasstacks. ** Writing good solutions is an Art **. 
% Eventually, you will find your own style. But here are some thoughts 
% to get you started: 
%
%   1. Write the solution as if you are writing it for your favorite
%      14-17 year old to help him/her understand. Could be your nephew, 
%      your niece, a cousin perhaps or probably even you when you 
%      were that age. Just write for them.
%
%   2. Use margin-notes to "talk" to students about the critical insights
%      in the question. The tone can be - in fact, should be - informal
%
%   3. Don't shy away from creating margin-figures you think will help
%      students understand. Yes, it is a little more work per question. 
%      But the question & solution will be written only once. Make that
%      attempt at writing a solution count.
%
%   4. At the same time, do not be too verbose. A long solution can
%      - at first sight - make the student think, "God, that is a lot to know".
%      Our aim is not to scare students. Rather, our aim should be to 
%      create many "Aha!" moments everyday in classrooms around the world
% 
%   5. Ensure that there are *no spelling mistakes anywhere*. We are an 
%      education company. Bad spellings suggest that we ourselves 
%      don't have any education. Also, use American spellings by default
% 
%   6. If a question has multiple parts, then first delete lines 40-41
%   7. If a question does not have parts, then first delete lines 43-69

\question Find an expression for the sum of the first $n$ terms of the series 
$3.12 + 5.22 + 7.32 + 9.42 + \ldots$

\insertQR{}

\ifprintanswers
\fi 

\begin{solution}
	First, do we agree that
	\begin{align}
		3.12 &= 3 + 0.12 = 3 + \frac{12}{100} = 3 + \frac{6}{50} \\
		5.22 &= 5 + 0.22 = 5 + \frac{22}{100} = 5 + \frac{11}{50} \\
		7.12 &= 7 + 0.32 = 3 + \frac{32}{100} = 7 + \frac{16}{50} \\
		9.12 &= 9 + 0.42 = 3 + \frac{42}{100} = 9 + \frac{21}{50}
	\end{align}
	
	Looks like we are onto something. Looks like the $nth$ term of the series
	is given by $a_n = (2n+1) + \frac{5n+1}{50} = \dfrac{21}{20}n + \dfrac{51}{50}\, \forall n \geq 1$
	
	The sum of the first $n$ terms therefore is
	\begin{align}
		S_n &= \dfrac{21}{20}\sum_{k=0}^{n}k + \dfrac{51}{50}\sum_{k=0}^{n}1 \\
		&= \dfrac{21}{20}\eSumOfN{n} + \dfrac{51}{50}n \\
		&= \dfrac{n}{10}\cdot\left[ \dfrac{21}{4}\cdot(n+1) + \dfrac{51}{5}\right] \\
		&= \dfrac{n}{200}\cdot(105n + 309)
	\end{align}
\end{solution}
