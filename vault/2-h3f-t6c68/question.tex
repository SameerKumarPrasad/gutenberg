% This is an empty shell file placed for you by the 'examiner' script.
% You can now fill in the TeX for your question here.

% Now, down to brasstacks. ** Writing good solutions is an Art **. 
% Eventually, you will find your own style. But here are some thoughts 
% to get you started: 
%
%   1. Write the solution as if you are writing it for your favorite
%      14-17 year old to help him/her understand. Could be your nephew, 
%      your niece, a cousin perhaps or probably even you when you 
%      were that age. Just write for them.
%
%   2. Use margin-notes to "talk" to students about the critical insights
%      in the question. The tone can be - in fact, should be - informal
%
%   3. Don't shy away from creating margin-figures you think will help
%      students understand. Yes, it is a little more work per question. 
%      But the question & solution will be written only once. Make that
%      attempt at writing a solution count.
%
%   4. At the same time, do not be too verbose. A long solution can
%      - at first sight - make the student think, "God, that is a lot to know".
%      Our aim is not to scare students. Rather, our aim should be to 
%      create many "Aha!" moments everyday in classrooms around the world
% 
%   5. Ensure that there are *no spelling mistakes anywhere*. We are an 
%      education company. Bad spellings suggest that we ourselves 
%      don't have any education. Also, use American spellings by default
% 
%   6. If a question has multiple parts, then first delete lines 40-41
%   7. If a question does not have parts, then first delete lines 43-69

\question[3] The first term of a certain infinitely decreasing geometric progression is $1$ and its sum is $S$. Find the sum of the geometric progression which is formed by the squares of the terms of the initial progression.

\insertQR{QRC}

\ifprintanswers
  % stuff to be shown only in the answer key - like explanatory margin figures
\fi 

\begin{solution}[\halfpage]
  The sum of the terms of an infinitely decreasing geometric progression starting with $a$ and having common ratio $r$ is given by,
  \begin{align}
	S = \dfrac{a}{1-r}
  \end{align}
  In our case initial term is $1$ and common ratio is $r$ therefore,
  \begin{align}
                S &= \dfrac{1}{1-r} \\
    \Rightarrow r &= 1-\dfrac{1}{S}
  \end{align}
  The progression of squares of the original geometric progression will effectively have common ration $r^2$. Therefore sum of progression of squares $S_{sq}$ is given by,
  \begin{align}
	S_{sq} &= \dfrac{1}{1-r^2} \\
		   &= \dfrac{1}{1-(1-\dfrac{1}{S})^2} \\
	       &= \dfrac{S^2}{2S-1}
  \end{align}

\end{solution}

