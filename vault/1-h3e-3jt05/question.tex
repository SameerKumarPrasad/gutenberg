
%\noprintanswers
%\setcounter{rolldice}{3}
%\printrubric

\ifnumequal{\value{rolldice}}{0}{
  % variables 
  \renewcommand{\vbone}{5}
  \renewcommand{\vbtwo}{100}
  \renewcommand{\vbthree}{19}
  \renewcommand{\vbfour}{198}
}{
  \ifnumequal{\value{rolldice}}{1}{
    % variables 
    \renewcommand{\vbone}{6}
    \renewcommand{\vbtwo}{90}
    \renewcommand{\vbthree}{56}
    \renewcommand{\vbfour}{445}
  }{
    \ifnumequal{\value{rolldice}}{2}{
      % variables 
      \renewcommand{\vbone}{7}
      \renewcommand{\vbtwo}{84}
      \renewcommand{\vbthree}{77}
      \renewcommand{\vbfour}{498}
    }{
      % variables 
      \renewcommand{\vbone}{8}
      \renewcommand{\vbtwo}{104}
      \renewcommand{\vbthree}{192}
      \renewcommand{\vbfour}{1339}
    }
  }
}

\FRACTIONSIMPLIFY\vbone\vbtwo\a\b

\SUBTRACT\vbone{1}\c
\SUBTRACT\vbtwo{1}\d
\SUBTRACT\vbtwo\vbone\x
\SUBTRACT\x{1}\y

\FRACMULT\a\b\c\d\e\f

\FRACMINUS{1}{1}\vbone\vbtwo\g\h
\FRACMINUS{1}{1}\c\d\i\j

\FRACMULT\g\h\vbone\d\k\l
\FRACMULT\a\b\i\j\m\n


\question In a shipment of $\vbtwo$ television sets, $\vbone$ are defective. If a person picks 
two (2) sets from this shipment, then

\watchout

\ifprintanswers
  % stuff to be shown only in the answer key - like explanatory margin figures
\fi 

\begin{parts}
  \part[1] What is the probability that \textit{both} sets are defective?

  \insertQR[-20pt]{QRC}
\begin{solution}[\mcq]
     \begin{align}
        P(\text{both defective}) &= P(\text{first defective})\cdot P(\text{second defective}) \\
                  &= \dfrac\vbone\vbtwo \times \dfrac\c\d = \dfrac\e\f 
     \end{align}
  \end{solution}

  \part[2] What is the probability of \textit{just one} set being defective?

  \insertQR[-10pt]{QRC}
\begin{solution}[\halfpage]
    \begin{align}
       P(\text{just one defective}) &= 
       \overbrace{\left( 1 - \dfrac\vbone\vbtwo \right)\times\dfrac\vbone\d}^{\text{first ok, second defective}}\nonumber \\
       &+ \underbrace{\dfrac\vbone\vbtwo \times \left( 1-\dfrac\c\d \right)}_{\text{first defective, second ok}} \\
       &= \dfrac\k\l + \dfrac\m\n = \dfrac\vbthree\vbfour
    \end{align}
    
    Alternatively, one could also say that,
    \begin{align}
       P(\text{one defective}) &= 1 - P(\text{none defective}) - P(\text{both defective}) \\
                       &= 1 - \dfrac\x\vbtwo\times\dfrac\y\d - \dfrac\e\f \\
                       &= \dfrac\vbthree\vbfour
    \end{align}
  \end{solution}
\end{parts}

\ifprintrubric
  \begin{table}
  	\begin{tabular}{ p{5cm}p{5cm} }
  		\toprule % in brief (4-6 words), what should a grader be looking for for insights & formulations
  		  \sc{\textcolor{blue}{Things to look for}} & \\ 
  		\midrule % ***** Insights & formulations ******
        Realized that TV sets are picked without replacement & \\
  		\toprule % final numerical answers for the various versions
        \sc{\textcolor{blue}{If question has $\ldots$}} & \sc{\textcolor{blue}{Final answer}} \\
  		\midrule % ***** Numerical answers (below) **********
        5 defective per 100 & (a) = $\frac{1}{495}$, (b) = $\frac{19}{198}$ \\
        6 defective per 90 & (a) = $\frac{1}{267}$, (b) = $\frac{56}{445}$ \\
        7 defective per 84 & (a) = $\frac{1}{166}$, (b) = $\frac{77}{498}$ \\
        8 defective per 104 & (a) = $\frac{7}{1339}$, (b) = $\frac{192}{1339}$ \\
  		\bottomrule
  	\end{tabular}
  \end{table}
\fi
