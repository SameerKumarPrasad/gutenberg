% This is an empty shell file placed for you by the 'examiner' script.
% You can now fill in the TeX for your question here.

% Now, down to brasstacks. ** Writing good solutions is an Art **. 
% Eventually, you will find your own style. But here are some thoughts 
% to get you started: 
%
%   1. Write the solution as if you are writing it for your favorite
%      14-17 year old to help him/her understand. Could be your nephew, 
%      your niece, a cousin perhaps or probably even you when you 
%      were that age. Just write for them.
%
%   2. Use margin-notes to "talk" to students about the critical insights
%      in the question. The tone can be - in fact, should be - informal
%
%   3. Don't shy away from creating margin-figures you think will help
%      students understand. Yes, it is a little more work per question. 
%      But the question & solution will be written only once. Make that
%      attempt at writing a solution count.
%
%   4. At the same time, do not be too verbose. A long solution can
%      - at first sight - make the student think, "God, that is a lot to know".
%      Our aim is not to scare students. Rather, our aim should be to 
%      create many "Aha!" moments everyday in classrooms around the world
% 
%   5. Ensure that there are *no spelling mistakes anywhere*. We are an 
%      education company. Bad spellings suggest that we ourselves 
%      don't have any education. Also, use American spellings by default
% 
%   6. If a question has multiple parts, then first delete lines 40-41
%   7. If a question does not have parts, then first delete lines 43-69

\question \begin{fullwidth} In a shipment of 100 television sets, 5 are defective. If a person picks 
\textit{two} sets from this shipment, then \end{fullwidth}

\insertQR{}

\ifprintanswers
  % stuff to be shown only in the answer key - like explanatory margin figures
\fi 

\begin{parts}
  \part What is the probability that \textit{both} sets are defective?

  \insertQR{}
  \begin{solution}
     \begin{align}
        P(\text{both defective}) &= P(\text{first defective})\cdot P(\text{second defective}) \\
                  &= \dfrac{5}{100} \cdot \dfrac{4}{99} = \dfrac{1}{495}
     \end{align}
  \end{solution}

  \part What is the probability that \textit{just one} set is defective?

  \insertQR{}
  \begin{solution}
    \begin{align}
       P(\text{just one defective}) &= 
       \underbrace{\dfrac{95}{100}\cdot\dfrac{5}{99}}_{\text{first ok, second defective}} + 
       \overbrace{\dfrac{5}{100}\cdot\dfrac{95}{99}}^{\text{first defective, second ok}} \\
       &= \dfrac{19}{198}
    \end{align}
    
    Alternatively, one could also say that,
    \begin{align}
       P(\text{none defective}) &+ P(\text{one defective}) + P(\text{both defective}) = 1\\
       \Rightarrow P(\text{one defective}) &= 1 - P(\text{none defective}) 
                                                              - P(\text{both defective}) \\
                       &= 1 - \dfrac{95}{100}\cdot\dfrac{94}{99} - \dfrac{1}{495} \\
                       &= \dfrac{19}{198}
    \end{align}
  \end{solution}


\end{parts}
