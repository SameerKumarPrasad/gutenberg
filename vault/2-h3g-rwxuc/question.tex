% This is an empty shell file placed for you by the 'examiner' script.
% You can now fill in the TeX for your question here.

% Now, down to brasstacks. ** Writing good solutions is an Art **. 
% Eventually, you will find your own style. But here are some thoughts 
% to get you started: 
%
%   1. Write the solution as if you are writing it for your favorite
%      14-17 year old to help him/her understand. Could be your nephew, 
%      your niece, a cousin perhaps or probably even you when you 
%      were that age. Just write for them.
%
%   2. Use margin-notes to "talk" to students about the critical insights
%      in the question. The tone can be - in fact, should be - informal
%
%   3. Don't shy away from creating margin-figures you think will help
%      students understand. Yes, it is a little more work per question. 
%      But the question & solution will be written only once. Make that
%      attempt at writing a solution count.
%
%   4. At the same time, do not be too verbose. A long solution can
%      - at first sight - make the student think, "God, that is a lot to know".
%      Our aim is not to scare students. Rather, our aim should be to 
%      create many "Aha!" moments everyday in classrooms around the world
% 
%   5. Ensure that there are *no spelling mistakes anywhere*. We are an 
%      education company. Bad spellings suggest that we ourselves 
%      don't have any education. Also, use American spellings by default
% 
%   6. If a question has multiple parts, then first delete lines 40-41
%   7. If a question does not have parts, then first delete lines 43-69

\question[3] A train was delayed by a semaphore for $16min$. It made up for the delay along a section of $80km$ length, traveling $10km/h$ higher than its scheduled speed. Find the scheduled speed of the train. \textit{A semaphore allows a train to enter a critical section of track where only one train can go at a time. (speed = distance/time)}.

\insertQR{QRC}

\ifprintanswers
  % stuff to be shown only in the answer key - like explanatory margin figures
\fi 

\begin{solution}[\halfpage]
  Let us assume that the scheduled speed of the train is $v(km/h)$. \\
  Since the train made up for the delay, we can equate the time it would have taken if it traveled as per schedule with the time it actually took.
  \begin{align}
    T_{scheduled}           &= T_{delay} + T_{catch up} \\
    \dfrac{80(km)}{v(km/h)} &= \dfrac{16(min)}{60(min/h)}+\dfrac{80(km)}{v+10(km/h)} \\
    &\Rightarrow \dfrac{5}{v} = \dfrac{1}{60}+\dfrac{5}{v+10} \\
    &\Rightarrow v^2+10v-3000 = 0 \\
    &\Rightarrow v            = 50,-60
  \end{align}
  Rejecting the negative value, the scheduled normal speed of the train is $50km/h$.
\end{solution}

