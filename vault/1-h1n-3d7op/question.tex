% This is an empty shell file placed for you by the 'examiner' script.
% You can now fill in the TeX for your question here.

% Now, down to brasstacks. ** Writing good solutions is an Art **. 
% Eventually, you will find your own style. But here are some thoughts 
% to get you started: 
%
%   1. Write the solution as if you are writing it for your favorite
%      14-17 year old to help him/her understand. Could be your nephew, 
%      your niece, a cousin perhaps or probably even you when you 
%      were that age. Just write for them.
%
%   2. Use margin-notes to "talk" to students about the critical insights
%      in the question. The tone can be - in fact, should be - informal
%
%   3. Don't shy away from creating margin-figures you think will help
%      students understand. Yes, it is a little more work per question. 
%      But the question & solution will be written only once. Make that
%      attempt at writing a solution count.
%
%   4. At the same time, do not be too verbose. A long solution can
%      - at first sight - make the student think, "God, that is a lot to know".
%      Our aim is not to scare students. Rather, our aim should be to 
%      create many "Aha!" moments everyday in classrooms around the world
% 
%   5. Ensure that there are *no spelling mistakes anywhere*. We are an 
%      education company. Bad spellings suggest that we ourselves 
%      don't have any education. Also, use American spellings by default
% 
%   6. If a question has multiple parts, then first delete lines 40-41
%   7. If a question does not have parts, then first delete lines 43-69

\question For a parabola $y^2=2px$ and a point $A$ on its axis, 

\insertQR{}

\ifprintanswers
  \begin{marginfigure}
  % 1. Definition of characteristic points
\figinit{pt}
\def\Xmin{0}
\def\Ymin{0}
\def\Xmax{79.99999}
\def\Ymax{49.99999}
\def\Xori{0}
\def\Yori{0}
\figpt0:(\Xori,\Yori)
% 2. Creation of the graphical file
\figdrawbegin{}
\def\Xmaxx{\Xmax} % To customize the position
\def\Ymaxx{\Ymax} % of the arrow-heads of the axes.
\figset arrowhead(length=4, fillmode=yes) % styling the arrowheads
\figdrawaxes 0(\Xmin, \Xmaxx, \Ymin, \Ymaxx)
\figpt 100: (60,0)
\figpt 101: (30,31)
\figdrawline [100,101]
\figdrawlineC(
0 0,
2.75862 -9.28476,
5.51724 -13.13064,
8.27586 -16.08168,
11.03448 -18.56953,
13.79310 -20.76136,
16.55172 -22.74294,
19.31034 -24.56518,
22.06896 -26.26128,
24.82758 -27.85430,
27.58620 -29.36101,
30.34482 -30.79408,
33.10344 -32.16337,
35.86206 -33.47670,
38.62068 -34.74041,
41.37931 -35.95974,
44.13793 -37.13906,
46.89655 -38.28207,
49.65517 -39.39192,
52.41379 -40.47136,
55.17241 -41.52273,
57.93103 -42.54814,
60.68965 -43.54941,
63.44827 -44.52817,
66.20689 -45.48588,
68.96551 -46.42383,
71.72413 -47.34320,
74.48275 -48.24506,
77.24137 -49.13036,
79.99999 -49.99999
)
\figdrawlineC(
0 0,
2.75862 9.28476,
5.51724 13.13064,
8.27586 16.08168,
11.03448 18.56953,
13.79310 20.76136,
16.55172 22.74294,
19.31034 24.56518,
22.06896 26.26128,
24.82758 27.85430,
27.58620 29.36101,
30.34482 30.79408,
33.10344 32.16337,
35.86206 33.47670,
38.62068 34.74041,
41.37931 35.95974,
44.13793 37.13906,
46.89655 38.28207,
49.65517 39.39192,
52.41379 40.47136,
55.17241 41.52273,
57.93103 42.54814,
60.68965 43.54941,
63.44827 44.52817,
66.20689 45.48588,
68.96551 46.42383,
71.72413 47.34320,
74.48275 48.24506,
77.24137 49.13036,
79.99999 49.99999
)
\figdrawend
% 3. Writing text on the figure
\figvisu{\figBoxA}{}{%
\figptsaxes 1:0(\Xmin, \Xmaxx, \Ymin, \Ymaxx)
% Points 1 and 2 are the end points of the arrows
\figwritee 1:(5pt)     \figwriten 2:(5pt)
\figptsaxes 1:0(\Xmin, \Xmax, \Ymin, \Ymax)
\figwrites 100: $A{(a,0)}$(2)
\figwritee 101: ${(x_{min}, y_{min})}$(2)
}
\centerline{\box\figBoxA}


  \end{marginfigure}
\fi 

\begin{parts}
  \part What is the $x-$coordinate of the point \textit{on the parabola} closest to $A$?

  \insertQR{}
  \begin{solution}
  	If $G$ be the distance between point $(x,y)$ on the parabola and $A$, then 
  	\begin{align}
  	   G &= \sqrt{(x-a)^2 + y^2} \\
  	   \Rightarrow G^2 &= (x-a)^2 + y^2 = (x-a)^2 + 2px \\
  	   \text{And }\therefore \dfrac{\ud G^2}{\ud x} &= 2\cdot(x-a) + 2p \\
  	   &= 0 \text{ when } x = (a-p) \\
  	   \text{Also, }\dfrac{\ud^2}{\ud x^2}G^2 &= 2 > 0 \Rightarrow \text{ minima }
  	\end{align}
  	
  	$x_{min}=(a-p)$ is an answer - but it is not the whole answer. If $a < p$, then the above
  	formula would give an $x_{min} < 0$. But the parabola \textit{requires} $x_{min} \geq 0$.
  	And so, the only possible solution is
  	\begin{align}
  	   x_{min} &= \left\lbrace
  	      \begin{array}{l c}
  	         (a-p) & \text{if } a > p \\
  	         0 & \text{otherwise}
  	      \end{array}\right.
  	\end{align}
  \end{solution}

  \part Using the result in part (a), find the point on parabola $y^2=8x$ closest to the point $(6,0)$ 

  \insertQR{}
  \begin{solution}
  	In this case $p = 4$ and $a = 6$. Therefore, $x_{min} = a-p = 6-2 = 4$ and the 
  	closest points are $(2, \sqrt{8\times 2})= (2,4)$ and $(2,-4)$
  \end{solution}

\end{parts}
