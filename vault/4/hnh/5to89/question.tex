


\ifnumequal{\value{rolldice}}{0}{
  % variables 
  \renewcommand{\vbone}{7}
}{
  \ifnumequal{\value{rolldice}}{1}{
    % variables 
    \renewcommand{\vbone}{8}
  }{
    \ifnumequal{\value{rolldice}}{2}{
      % variables 
      \renewcommand{\vbone}{6}
    }{
      % variables 
      \renewcommand{\vbone}{9}
    }
  }
}
\ADD\vbone{1}\k
\question Prove that $\Sigma_{r=1}^{n+1} \left(\vbone\right)^{r-1} \left(\encr{n}{r-1}\right) = {\k}^{n}$


\watchout

\ifprintanswers
  % stuff to be shown only in the answer key - like explanatory margin figures
  \begin{marginfigure}
    \figinit{pt}
      \figpt 100:(0,0)
      \figpt 101:(0,0)
    \figdrawbegin{}
      \figdrawline [100,101]
    \figdrawend
    \figvisu{\figBoxA}{}{%
    }
    \centerline{\box\figBoxA}
  \end{marginfigure}
\fi 

\begin{solution}
To begin with, we write $\k^{n}$ as $\left( 1 + \vbone  \right) ^ {n}$ \\
Now using binomial expansion for $\left( 1 + \vbone  \right) ^ {n}$\\ 
	\begin{align}
\Rightarrow ^{n}{C}_{0} \cdot (1)^{n} \cdot (\vbone)^{0} + ^{n}{C}_{1}\cdot (1)^{n-1} \cdot (\vbone)^{1} + \cdots + ^{n}C_{n}\cdot (1)^{0} \cdot (\vbone)^{n} 
	\end{align}
The $r^{th}$ term is \\
	\begin{align}
{T}_{r} = ^{n}{C}_{r-1} \cdot (1)^{n-r+1} \cdot (\vbone)^{r-1} 
	\end{align} 
The expansion is basically the summation of the $r^{th}$ term from $r=1$ to $n+1$\\
	\begin{align}
\left(1 + \vbone \right)^{n}= \Sigma_{r=1}{^{n+1}} \lbrace {T}_{r} \rbrace &= \Sigma _{r=1} ^{n+1} \lbrace ^{n}{C}_{n-1} \cdot (1)^{n-r+1}\cdot (\vbone)^{r-1}\rbrace 
	\end{align}
	\begin{align}
\Rightarrow \k^{n} &= \Sigma_{r=1} ^{n+1} \left( \encr{n}{r-1} \right) (\vbone)^{r-1}
	\end{align}
\end{solution}


