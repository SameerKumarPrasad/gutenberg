


\ifnumequal{\value{rolldice}}{0}{
  % variables 
  \renewcommand{\vbone}{2}
  \renewcommand{\vbtwo}{3}
  \renewcommand{\vbthree}{6}
}{
  \ifnumequal{\value{rolldice}}{1}{
    % variables 
    \renewcommand{\vbone}{2}
    \renewcommand{\vbtwo}{4}
    \renewcommand{\vbthree}{8}
  }{
    \ifnumequal{\value{rolldice}}{2}{
      % variables 
      \renewcommand{\vbone}{3}
      \renewcommand{\vbtwo}{4}
      \renewcommand{\vbthree}{12}
    }{
      % variables 
      \renewcommand{\vbone}{2}
      \renewcommand{\vbtwo}{5}
      \renewcommand{\vbthree}{10}
    }
  }
}

\question Find x in the binomial $\left( \sqrt[3]{\vbone} + \dfrac{1}{\sqrt[3]{\vbtwo}} \right)$ if the ratio of seventh term from the beginning of the binomial expansion to the seventh term from its end is $\dfrac{1}{\vbthree}$.


\watchout

\ifprintanswers
  % stuff to be shown only in the answer key - like explanatory margin figures
  \begin{marginfigure}
    \figinit{pt}
      \figpt 100:(0,0)
      \figpt 101:(0,0)
    \figdrawbegin{}
      \figdrawline [100,101]
    \figdrawend
    \figvisu{\figBoxA}{}{%
    }
    \centerline{\box\figBoxA}
  \end{marginfigure}
\fi 

\begin{solution}
The expression for the general $(r+1)^{th} term$ is \\ 
$T_{r+1} =\encr{x}{r} (\vbone)^{\frac{x-r}{3}} \cdot \left(\dfrac{1}{\vbtwo}\right)^{\frac{r}{3}}$\\
$T_{7}$ is the seventh term from the start and $T_{x-5}$ is the seventh term from the end. 
\begin{align}
T_{7} &= \encr{x}{r} (\vbone)^{\frac{x-6}{3}} \cdot (\dfrac{1}{\vbtwo})^{\frac{6}{3}} \\ 
T_{x-5} &= \encr{x}{r} (\vbone)^{\frac{6}{3}} \cdot (\dfrac{1}{\vbtwo})^{\frac{x-6}{3}} \\
Now, \quad \dfrac{T_{7}}{T_{x-5}} &= \dfrac{1}{\vbthree}\\
\dfrac{\encr{x}{r} (\vbone)^{\frac{x-6}{3}} \cdot (\dfrac{1}{\vbtwo})^{\frac{6}{3}}}{\encr{x}{r} (\vbone)^{\frac{6}{3}} \cdot (\dfrac{1}{\vbtwo})^{\frac{x-6}{3}}} &= \dfrac{1}{\vbthree} \\
\Rightarrow \vbone^{\frac{x-12}{3}} \times \vbtwo^{\frac{x-12}{3}} = \dfrac{1}{\vbthree} \\
\Rightarrow (\vbone \times \vbtwo) ^{\frac{x-12}{3}} =  \dfrac{1}{\vbthree} \\
\therefore \quad \dfrac{x-12}{3} = -1 \Rightarrow x = 9 
\end{align}
\end{solution}


