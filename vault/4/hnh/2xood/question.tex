


\ifnumequal{\value{rolldice}}{0}{
  % variables 
  \renewcommand{\vbone}{2}
  \renewcommand{\vbtwo}{-1}
  \renewcommand{\vbthree}{10}
}{
  \ifnumequal{\value{rolldice}}{1}{
    % variables 
    \renewcommand{\vbone}{3}
    \renewcommand{\vbtwo}{2}
    \renewcommand{\vbthree}{12}
  }{
    \ifnumequal{\value{rolldice}}{2}{
      % variables 
      \renewcommand{\vbone}{3}
      \renewcommand{\vbtwo}{4}
      \renewcommand{\vbthree}{10}
    }{
      % variables 
      \renewcommand{\vbone}{3}
      \renewcommand{\vbtwo}{1}
      \renewcommand{\vbthree}{12}
    }
  }
}
\DIVIDE\vbthree{2}\k 
\ADD\k{1}\l
\SUBTRACT\vbthree\k\m

\question[4] Find the term independent of $x$ in the expansion of 
 \[\left({\vbone x + \dfrac{\vbtwo}{x}}\right)^{\vbthree}\]

\watchout

\begin{solution}[\halfpage]
For the expression $\left(\vbone x + \dfrac{\vbtwo}{x}\right)^{\vbthree}$, $(r+1)^{th}$ term is 
	\begin{align}
	T_{r+1} &= \encr{n}{r} \cdot (\vbone)^{\vbthree - r}\cdot \left( \vbtwo \right)^{r} \cdot {x}^{\vbthree - r}  \left(\dfrac{1}{x}\right)^{r}\\
	\Rightarrow T_{r+1} &= \encr{n}{r} \cdot \left( \vbone \right)^{\vbthree - r}\cdot \left( \vbtwo \right)^{r} \cdot {x}^{\vbthree - 2r} \\
	\end{align}
For the term to be independent of x the power of x in the above expression should be zero.\\
Hence, we equate $\vbthree - 2r$ to $0$\\
$\Rightarrow r = \k $\\
Therefore ${\l}^{th}$ term is independent of x.\\
Moreover, 
	\begin{align}
	{T}_{\l} = \encr{\vbthree}{\k} \cdot \left( \vbone \right)^{\m}\cdot \left( \vbtwo \right)^{\k}
	 \end{align}
\end{solution}

