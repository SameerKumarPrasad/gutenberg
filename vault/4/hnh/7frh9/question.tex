\ifnumequal{\value{rolldice}}{0}{
  % variables 
  \renewcommand{\vbone}{3}
  \renewcommand{\vbtwo}{559}
  \renewcommand{\vbthree}{12}
  \renewcommand{\vbfour}{3}
}{
  \ifnumequal{\value{rolldice}}{1}{
    % variables 
    \renewcommand{\vbone}{3}
    \renewcommand{\vbtwo}{376}
    \renewcommand{\vbthree}{10}
    \renewcommand{\vbfour}{4}
  }{
    \ifnumequal{\value{rolldice}}{2}{
      % variables 
      \renewcommand{\vbone}{4}
      \renewcommand{\vbtwo}{681}
      \renewcommand{\vbthree}{10}
      \renewcommand{\vbfour}{4}
    }{
      % variables 
      \renewcommand{\vbone}{3}
      \renewcommand{\vbtwo}{229}
      \renewcommand{\vbthree}{8}
      \renewcommand{\vbfour}{2}
    }
  }
}

\SQUARE\vbone\a
\EXPR[0]\b{(\a + 2*\vbone)}
\EXPR[0]\c{(2*(\vbtwo - 1))}
\SUBTRACT\vbthree\vbfour\d
\DIVIDE\d{3}\vbfive

\question[5] The sum of coefficients of first three terms of the expansion 
$\left( x- \dfrac{\vbone}{x^{2}} \right) ^{m},\, m \in \mathbb{N}$ is $\vbtwo$. 
Given this, what will be the term containing $x^{\vbfour}$?
\watchout[-30pt]
\marginnote{

	$\sqrt{4489} = 67$

	$\sqrt{1849}=43$

	$\sqrt{3025}=55$

	$\sqrt{1369}=37$
}

\begin{solution}[\fullpage]
If $a_{1}, a_{2},a_{3}$ be the coefficients of the first, second and third terms respectively, then 
	\begin{align}
		a_{1} &= \binom{m}{0} = 1 \\ 
		a_{2} &= -\binom{m}{1} \cdot \vbone \\
		a_{3} &= \binom{m}{2} \cdot \vbone^{2}  \\ 
		\text{And if } a_1 + a_2 + a_3 &= \vbtwo \nonumber\\
		\text{then }1 - \vbone m + \dfrac\a{2}\cdot m\cdot (m-1) &= \vbtwo \\
		\implies \a m^2 - \b m - \c &= 0 
	\end{align}

The only $m\in\mathbb{N}$ that satisfies the above is $m=\vbthree$. Which means, 
the original binomial expression is $\left( x - \dfrac\vbone{x^2} \right)^{\vbthree}$

\textbf{Find the term containing $x^\vbfour$}

If $T_{r+1}$ be the $(r+1)^{th}$ term in the expansion, $r\in [0,\vbthree]$, then 
\begin{align}
	T_{r+1} = \binom\vbthree{r}x^{\vbthree - r}\cdot\left( -\dfrac\vbone{x^2} \right)^{r}
	        &= (-\vbone)^r\binom\vbthree{r}\cdot x^{\vbthree-r-2r} \\
  \text{And so, }x^{\vbthree -r - 2r} = x^{\vbthree - 3r} &= x^{\vbfour} \implies r = \vbfive
\end{align}

Hence, the term we are looking for is 
\begin{align}
	T_{\vbfive + 1} &= (-\vbone)^\vbfive\binom\vbthree\vbfive\cdot x^\vbfour
\end{align}
\end{solution}

