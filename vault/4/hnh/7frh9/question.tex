% This is an empty shell file placed for you by the 'examiner' script.
% You can now fill in the TeX for your question here.

% Now, down to brasstacks. ** Writing good solutions is an Art **. 
% Eventually, you will find your own style. But here are some thoughts 
% to get you started: 
%
%   1. Write to be understood - but be crisp. Your own solution should not take 
%      more space than you will give to the student. Hence, if you take more than 
%      a half-page to write a solution, then give the student a full-page and so on...
%
%   2. Use margin-notes to "talk" to students about the critical insights
%      in the question. The tone can be - in fact, should be - informal
%
%   3. Don't shy away from creating margin-figures you think will help
%      students understand. Yes, it is a little more work per question. 
%      But the question & solution will be written only once. Make that
%      attempt at writing a solution count.
%      
%      3b. Use bc_to_fig.tex. Its an easier way to generate plots & graphs 
% 
%   4. Ensure that there are *no spelling mistakes anywhere*. We are an 
%      education company. Bad spellings suggest that we ourselves 
%      don't have any education. Also, use American spellings by default
% 
%   5. If a question has multiple parts, then first delete lines 40-41
%   6. If a question does not have parts, then first delete lines 43-69
%   
%   7. Create versions of the question when possible. Use commands defined in 
%      tufte-tweaks.sty to do so. Its easier than you think

% \noprintanswers
% \setcounter{rolldice}{0}
% \printrubric

\ifnumequal{\value{rolldice}}{0}{
  % variables 
  \renewcommand{\vbone}{3}
  \renewcommand{\vbtwo}{559}
  \renewcommand{\vbthree}{12}
  \renewcommand{\vbfour}{3}
  \renewcommand{\vbfive}{3}
  \renewcommand{\vbsix}{}
  \renewcommand{\vbseven}{}
  \renewcommand{\vbeight}{}
  \renewcommand{\vbnine}{}
  \renewcommand{\vbten}{}
}{
  \ifnumequal{\value{rolldice}}{1}{
    % variables 
    \renewcommand{\vbone}{3}
    \renewcommand{\vbtwo}{376}
    \renewcommand{\vbthree}{10}
    \renewcommand{\vbfour}{4}
    \renewcommand{\vbfive}{2}
    \renewcommand{\vbsix}{}
    \renewcommand{\vbseven}{}
    \renewcommand{\vbeight}{}
    \renewcommand{\vbnine}{}
    \renewcommand{\vbten}{}
  }{
    \ifnumequal{\value{rolldice}}{2}{
      % variables 
      \renewcommand{\vbone}{4}
      \renewcommand{\vbtwo}{681}
      \renewcommand{\vbthree}{10}
      \renewcommand{\vbfour}{4}
      \renewcommand{\vbfive}{2}
      \renewcommand{\vbsix}{}
      \renewcommand{\vbseven}{}
      \renewcommand{\vbeight}{}
      \renewcommand{\vbnine}{}
      \renewcommand{\vbten}{}
    }{
      % variables 
      \renewcommand{\vbone}{3}
      \renewcommand{\vbtwo}{229}
      \renewcommand{\vbthree}{8}
      \renewcommand{\vbfour}{3}
      \renewcommand{\vbfive}{3}
      \renewcommand{\vbsix}{}
      \renewcommand{\vbseven}{}
      \renewcommand{\vbeight}{}
      \renewcommand{\vbnine}{}
      \renewcommand{\vbten}{}
    }
  }
}

\question[5] The sum of coefficients of first three terms of the expansion $\left( x- \dfrac{\vbone}{x^{2}} \right) ^{m} (m \in \mathbb{N})$ is $\vbtwo$. Find the term of expansion containing $x^{\vbfour}$.

\insertQR{}

\watchout

\ifprintanswers
  % stuff to be shown only in the answer key - like explanatory margin figures
  \begin{marginfigure}
    \figinit{pt}
      \figpt 100:(0,0)
      \figpt 101:(0,0)
    \figdrawbegin{}
      \figdrawline [100,101]
    \figdrawend
    \figvisu{\figBoxA}{}{%
    }
    \centerline{\box\figBoxA}
  \end{marginfigure}
\fi 

\begin{solution}[\fullpage]
Let $a_{1}, a_{2},a_{3}$ be the coefficient of the first, second and third term of the expansion. \\
	\begin{align}
	a_{1} &= 1 \\ 
	a_{2} &= -\encr{m}{1} \cdot \vbone \\
	a_{3} &= \encr{m}{2} \cdot \vbone^{2} 
	\end{align}    
Now adding these gives $\vbtwo$\\
\begin{align}
\therefore \quad &a_{1} + a_{2} + a_{3} = \vbtwo \\
&\Rightarrow 1 - m\cdot\vbone + \dfrac{m(m-1)}{2} \cdot \vbone^{2} = \vbtwo \\
&\Rightarrow m^{2}\vbone^{2} - m(2\cdot \vbone + \vbone^{2}) + 2 - 2\cdot \vbtwo = 0 
\end{align}
Solving the quadratic for m, we get the value of m$ = \vbthree$ \\ 
Now , to find the coefficient of the term containing $x^{\vbfour}$ we first find the coefficient of x in $(r+1)^{th}$ term and then equate it to $\vbfour$. \\
\begin{align}
T_{r+1} &= \encr{m}{r} \cdot x^{m-r} \cdot {\dfrac{-\vbone}{x^{2}}}^{r} \\
&= {-\vbone}^{r} \lbrace \encr{m}{r} \cdot x^{m-3r} \rbrace \\
\Rightarrow m - 3r &=\vbfour 
\end{align} 
Now we put the value of r in ${T}_{r+1}$\\
$\Rightarrow T_{r+1} = \encr{\vbthree}{\vbfive} \cdot x^{\vbthree-\vbfive} \cdot {\left(\dfrac{-\vbone}{x^{2}}\right)}^{\vbthree}$ \\
Therefore the coefficient of $x^{\vbfour}$ in $T_{r+1} = \encr{\vbthree}{\vbfive} \cdot (-\vbone)^{\vbthree}$
\end{solution}


\ifprintrubric
  \begin{table}
  	\begin{tabular}{ p{5cm}p{5cm} }
  		\toprule % in brief (4-6 words), what should a grader be looking for for insights & formulations
  		  \sc{\textcolor{blue}{Insight}} & \sc{\textcolor{blue}{Formulation}} \\ 
  		\midrule % ***** Insights & formulations ******
  		\toprule % final numerical answers for the various versions
        \sc{\textcolor{blue}{If question has $\ldots$}} & \sc{\textcolor{blue}{Final answer}} \\
  		\midrule % ***** Numerical answers (below) **********
  		\bottomrule
  	\end{tabular}
  \end{table}
\fi
