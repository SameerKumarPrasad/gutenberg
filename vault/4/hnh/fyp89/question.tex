% This is an empty shell file placed for you by the 'examiner' script.
% You can now fill in the TeX for your question here.

% Now, down to brasstacks. ** Writing good solutions is an Art **. 
% Eventually, you will find your own style. But here are some thoughts 
% to get you started: 
%
%   1. Write to be understood - but be crisp. Your own solution should not take 
%      more space than you will give to the student. Hence, if you take more than 
%      a half-page to write a solution, then give the student a full-page and so on...
%
%   2. Use margin-notes to "talk" to students about the critical insights
%      in the question. The tone can be - in fact, should be - informal
%
%   3. Don't shy away from creating margin-figures you think will help
%      students understand. Yes, it is a little more work per question. 
%      But the question & solution will be written only once. Make that
%      attempt at writing a solution count.
%      
%      3b. Use bc_to_fig.tex. Its an easier way to generate plots & graphs 
% 
%   4. Ensure that there are *no spelling mistakes anywhere*. We are an 
%      education company. Bad spellings suggest that we ourselves 
%      don't have any education. Also, use American spellings by default
% 
%   5. If a question has multiple parts, then first delete lines 40-41
%   6. If a question does not have parts, then first delete lines 43-69
%   
%   7. Create versions of the question when possible. Use commands defined in 
%      tufte-tweaks.sty to do so. Its easier than you think

% \noprintanswers
% \setcounter{rolldice}{0}
% \printrubric

\ifnumequal{\value{rolldice}}{0}{
  % variables 
  \renewcommand{\vbone}{3}
  \renewcommand{\vbtwo}{\text{rd}}
  \renewcommand{\vbthree}{\dfrac{2}{3}}
  \renewcommand{\vbfour}{2}
  \renewcommand{\vbfive}{2048}
  \renewcommand{\vbsix}{11}
  \renewcommand{\vbseven}{8}
  \renewcommand{\vbeight}{}
  \renewcommand{\vbnine}{}
  \renewcommand{\vbten}{}
}{
  \ifnumequal{\value{rolldice}}{1}{
    % variables 
    \renewcommand{\vbone}{4}
    \renewcommand{\vbtwo}{\text{th}}
    \renewcommand{\vbthree}{2}
    \renewcommand{\vbfour}{\dfrac{1}{2}}
    \renewcommand{\vbfive}{2048}
    \renewcommand{\vbsix}{11}
    \renewcommand{\vbseven}{7}
    \renewcommand{\vbeight}{}
    \renewcommand{\vbnine}{}
    \renewcommand{\vbten}{}
  }{
    \ifnumequal{\value{rolldice}}{2}{
      % variables 
      \renewcommand{\vbone}{3}
      \renewcommand{\vbtwo}{\text{rd}}
      \renewcommand{\vbthree}{2}
      \renewcommand{\vbfour}{\dfrac{1}{4}}
      \renewcommand{\vbfive}{1024}
      \renewcommand{\vbsix}{10}
      \renewcommand{\vbseven}{7}
      \renewcommand{\vbeight}{}
      \renewcommand{\vbnine}{}
      \renewcommand{\vbten}{}
    }{
      % variables 
      \renewcommand{\vbone}{4}
      \renewcommand{\vbtwo}{\text{th}}
      \renewcommand{\vbthree}{\dfrac{1}{3}}
      \renewcommand{\vbfour}{3}
      \renewcommand{\vbfive}{1024}
      \renewcommand{\vbsix}{10}
      \renewcommand{\vbseven}{6}
      \renewcommand{\vbeight}{}
      \renewcommand{\vbnine}{}
      \renewcommand{\vbten}{}
    }
  }
}

\question Prove that the sum of all the binomial coefficients in the general expansion $(x+y)^{n}$ is  $2^{n}$. Find the $\vbone^{\vbtwo}$ 
term in the expansion $\left(  \vbthree x + \vbfour \right) ^ n $ if the sum of all binomial coefficients is $\vbfive$.

\insertQR{}

\watchout

\ifprintanswers
  % stuff to be shown only in the answer key - like explanatory margin figures
  \begin{marginfigure}
    \figinit{pt}
      \figpt 100:(0,0)
      \figpt 101:(0,0)
    \figdrawbegin{}
      \figdrawline [100,101]
    \figdrawend
    \figvisu{\figBoxA}{}{%
    }
    \centerline{\box\figBoxA}
  \end{marginfigure}
\fi 

\begin{solution}
Expanding the general binomial expression $(x+y)^{n}$\\
\\
$(x+y)^{n} = \encr{n}{0}\cdot x^{n} \cdot y^{0} + \encr{n}{1}\cdot x^{n-1} \cdot y^{1} + \cdots  +\encr{n}{n} \cdot x^{0} \cdot y^{n}$ \\
We put $x = y = 1$ \\
\begin{align}
\Rightarrow 2^{n} &= \encr{n}{0} + \encr{n}{1} + \encr{n}{2} + \encr{n}{3}  +\encr{n}{n} 
\end{align}
Thus, the sum of all the coefficients is $2^{n}$ \\
\begin{align}
Here, \quad 2^{n} = \vbfive \quad \Rightarrow n = \vbsix
\end{align} 
The $\vbone^{\vbtwo}$ term will therefore be 
\begin{align}
{T}_{\vbone} = \encr{\vbsix}{\vbone}\cdot \left(\vbthree x\right)^{\vbsix - \vbone} \cdot \left(\vbfour\right)^{\vbone}  \\
{T}_{\vbone} = \encr{\vbsix}{\vbone}\cdot \left(\vbthree x\right)^{\vbseven} \cdot \left(\vbfour\right)^{\vbone}
\end{align}
\end{solution}


\ifprintrubric
  \begin{table}
  	\begin{tabular}{ p{5cm}p{5cm} }
  		\toprule % in brief (4-6 words), what should a grader be looking for for insights & formulations
  		  \sc{\textcolor{blue}{Insight}} & \sc{\textcolor{blue}{Formulation}} \\ 
  		\midrule % ***** Insights & formulations ******
  		\toprule % final numerical answers for the various versions
        \sc{\textcolor{blue}{If question has $\ldots$}} & \sc{\textcolor{blue}{Final answer}} \\
  		\midrule % ***** Numerical answers (below) **********
  		\bottomrule
  	\end{tabular}
  \end{table}
\fi
