


\ifnumequal{\value{rolldice}}{0}{
  % variables 
  \renewcommand{\vbone}{7}
}{
  \ifnumequal{\value{rolldice}}{1}{
    % variables 
    \renewcommand{\vbone}{8}
  }{
    \ifnumequal{\value{rolldice}}{2}{
      % variables 
      \renewcommand{\vbone}{6}
    }{
      % variables 
      \renewcommand{\vbone}{9}
    }
  }
}
\SUBTRACT\vbone{2}\k
\question If a $\vbone$ digit numbers is being formed by using the digits $\lbrace1,2,3\rbrace$


\watchout

\ifprintanswers
  % stuff to be shown only in the answer key - like explanatory margin figures
  \begin{marginfigure}
    \figinit{pt}
      \figpt 100:(0,0)
      \figpt 101:(0,0)
    \figdrawbegin{}
      \figdrawline [100,101]
    \figdrawend
    \figvisu{\figBoxA}{}{%
    }
    \centerline{\box\figBoxA}
  \end{marginfigure}
\fi 

\begin{parts}
  \part How many numbers can be formed without any restrictions.

  \begin{solution}
  Every place can be filled by any these three numbers. Thus $3^\vbone = \k$ numbers can be formed.
  \end{solution}

  \part How many numbers can be formed when the digit 2 occurs only twice.

  \begin{solution}
   Since the digit 2 should occur only twice, we can select any two places form the $\vbone$ digit number and put 2 there. \\
   In $\encr\vbone{2}$ ways. Rest of the places can be filled by any of the digits $1$ or $3$ in $2^{\k}$ ways. \\
   $\therefore$ in all we have $\encr\vbone{2} \times 2^{\k}$ ways.   
  \end{solution}
\end{parts}

