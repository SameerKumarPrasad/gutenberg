


\ifnumequal{\value{rolldice}}{0}{
  % variables 
  \renewcommand{\vbone}{5}
}{
  \ifnumequal{\value{rolldice}}{1}{
    % variables 
    \renewcommand{\vbone}{6}
  }{
    \ifnumequal{\value{rolldice}}{2}{
      % variables 
      \renewcommand{\vbone}{7}
    }{
      % variables 
      \renewcommand{\vbone}{4}
    }
  }
}
\SUBTRACT\vbone{1}\k
\question How can we place $\vbone$ boys and $\vbone$ girls around a circular table so that they sit alternately.  


\watchout

\ifprintanswers
  % stuff to be shown only in the answer key - like explanatory margin figures
  \begin{marginfigure}
    \figinit{pt}
      \figpt 100:(0,0)
      \figpt 101:(0,0)
    \figdrawbegin{}
      \figdrawline [100,101]
    \figdrawend
    \figvisu{\figBoxA}{}{%
    }
    \centerline{\box\figBoxA}
  \end{marginfigure}
\fi 

\begin{solution}	
This is a problem of circular permutation, we first seat the boys at 5 alternate places on a circular table. \\
The number of ways to seat n people around a circular table is (n-1)!\\ 
$\therefore$ the first set of people are seated around the table in $(\vbone -1)! = \k!$ way.\\
Now the rest of the $\vbone$ girls can be seated in the gaps between the $\vbone$ boys in $\vbone!$ ways. \\
Here, since the boys are already seated the $\vbone$.The alternate places are now distinct and do not adhere to the concept of circular permutation.\\
Hence, the total number of arrangements are $\vbone! \cdot \k!$   
\end{solution}

