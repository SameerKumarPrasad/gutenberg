


\ifnumequal{\value{rolldice}}{0}{
  % variables 
}{
  \ifnumequal{\value{rolldice}}{1}{
    % variables 
  }{
    \ifnumequal{\value{rolldice}}{2}{
      % variables 
    }{
      % variables 
    }
  }
}

\question If the coefficients of $x, x^{2}, x^{3}$ in the binomial expansion of \\ $(1+x)^{2n}$, $(n\in \mathbb{N})$ are in A.P, then prove that $2n^{2} -9n +7 =0$


\watchout

\ifprintanswers
  % stuff to be shown only in the answer key - like explanatory margin figures
  \begin{marginfigure}
    \figinit{pt}
      \figpt 100:(0,0)
      \figpt 101:(0,0)
    \figdrawbegin{}
      \figdrawline [100,101]
    \figdrawend
    \figvisu{\figBoxA}{}{%
    }
    \centerline{\box\figBoxA}
  \end{marginfigure}
\fi 

\begin{solution}
The general $(r+1)^{th}$ term will be $T_{r+1} = \encr{2n}{r} x^{r} $ \\
Thus coefficient of x will be $\encr{2n}{1}$ in the second term when r =1 . \\
Similarly, coefficient of $x^{2}$ will be $\encr{2n}{2}$ when r=2 \\
For $x^{3}$ this will be $\encr{2n}{3}$. \\
Now as these are in A.P.
\begin{align}
&\Rightarrow 2\cdot \encr{2n}{2} = \encr{2n}{1}+\encr{2n}{3} \\
&\Rightarrow 2\cdot \fncr{2n}{2} = \fncr{2n}{1}+\fncr{2n}{3} \\
\end{align}   
Simplifying the above equation \\
\begin{align}
\dfrac{1}{2n-2} &= \dfrac{1}{(2n-1)(2n-2)} + \dfrac{1}{6}\\
\dfrac{1}{2n-2} &= \dfrac{4n^{2} + 8 - 6n}{6(2n-1)(2n-2)}\\
12n - 6 &= 4n^{2} - 6n + 8\\
\Rightarrow  2n^2 - 9n + 7 &= 0
\end{align}
\end{solution}

