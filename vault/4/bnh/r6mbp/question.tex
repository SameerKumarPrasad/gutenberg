


\ifnumequal{\value{rolldice}}{0}{
  % variables 
  \renewcommand{\vbone}{3}
  \renewcommand{\vbtwo}{2}
  \renewcommand{\vbthree}{12}
  \renewcommand{\vbfour}{5}
  \renewcommand{\vbfive}{\dfrac{63}{25}}
  \renewcommand{\vbsix}{\dfrac{16}{25}}
  \renewcommand{\vbseven}{2}
  \renewcommand{\vbeight}{3}
}{
  \ifnumequal{\value{rolldice}}{1}{
    % variables 
    \renewcommand{\vbone}{2}
    \renewcommand{\vbtwo}{3}
    \renewcommand{\vbthree}{12}
    \renewcommand{\vbfour}{-5}
    \renewcommand{\vbfive}{\dfrac{33}{25}}
    \renewcommand{\vbsix}{\dfrac{56}{25}}
    \renewcommand{\vbseven}{3}
    \renewcommand{\vbeight}{2}
  }{
    \ifnumequal{\value{rolldice}}{2}{
      % variables 
      \renewcommand{\vbone}{3}
      \renewcommand{\vbtwo}{4}
      \renewcommand{\vbthree}{18}
      \renewcommand{\vbfour}{1}
      \renewcommand{\vbfive}{3}
      \renewcommand{\vbsix}{2}
      \renewcommand{\vbseven}{4}
      \renewcommand{\vbeight}{3}
    }{
      % variables 
    }
  }
}

\question  Find the conjugate of 
\begin{align}
\dfrac{(\vbone-\vbtwo \textit{i})(\vbseven +\vbeight \textit{i})}{(1+2\textit{i})(2-\textit{i})} 
\end{align}


\watchout

\ifprintanswers
  % stuff to be shown only in the answer key - like explanatory margin figures
  \begin{marginfigure}
    \figinit{pt}
      \figpt 100:(0,0)
      \figpt 101:(0,0)
    \figdrawbegin{}
      \figdrawline [100,101]
    \figdrawend
    \figvisu{\figBoxA}{}{%
    }
    \centerline{\box\figBoxA}
  \end{marginfigure}
\fi 

\begin{solution}
Conjugate of a complex number $z = a + b\textit{i}$ is $\overline{z} = a - b\textit{i} $ \\
So we convert the above in the form of $z = a + b\textit{i}$ \\ 
	\begin{align}
  		\dfrac{(\vbone-\vbtwo \textit{i})(\vbseven + \vbeight \textit{i})}{(1+2\textit{i})(2-\textit{i})} \\
		&= \dfrac{\vbthree + \vbfour \textit{i} }{4 + 3 \textit{i}} \\
		&= \dfrac{\vbthree + \vbfour \textit{i}}{4 + 3 \textit{i}} \times \dfrac{4 - 3 \textit{i}}{4 - 3 \textit{i}} \\
		&= \vbfive - \vbsix \textit{i}  \\
		\therefore \overline{z} &= \vbfive + \vbsix \textit{i} 
	\end{align}
\end{solution}


