


\ifnumequal{\value{rolldice}}{0}{
  % variables 
  \renewcommand{\vbone}{4}
  \renewcommand{\vbtwo}{5}
  \renewcommand{\vbthree}{20}
}{
  \ifnumequal{\value{rolldice}}{1}{
    % variables 
    \renewcommand{\vbone}{4}
    \renewcommand{\vbtwo}{6}
    \renewcommand{\vbthree}{24}
  }{
    \ifnumequal{\value{rolldice}}{2}{
      % variables 
      \renewcommand{\vbone}{3}
      \renewcommand{\vbtwo}{7}
      \renewcommand{\vbthree}{21}
    }{
      % variables 
      \renewcommand{\vbone}{6}
      \renewcommand{\vbtwo}{8}
      \renewcommand{\vbthree}{48}
    }
  }
}

\question[5] A bag contains $\vbone$ mangoes and and $\vbtwo$ oranges. In how many ways can a person make a selection so as to take atleast one mango and one orange?  


\watchout

\ifprintanswers
  % stuff to be shown only in the answer key - like explanatory margin figures
  \begin{marginfigure}
    \figinit{pt}
      \figpt 100:(0,0)
      \figpt 101:(0,0)
    \figdrawbegin{}
      \figdrawline [100,101]
    \figdrawend
    \figvisu{\figBoxA}{}{%
    }
    \centerline{\box\figBoxA}
  \end{marginfigure}
\fi 

\begin{solution}[\halfpage]
Since all mangoes are identical, the number of ways of selecting r mangoes from n mangoes is 1(r$\leq$n). Thus, there is 1 way for choosing 0 mango or 1 mango or 2 and so on....\\
Total ways of choosing from $\vbone$ mangoes is $\lbrace \underbrace{1+1..}_{\vbone +1 \quad times} \rbrace = (\vbone +1)$ \\
Here 1 case will be for no mango getting picked. Now, to choose \textit{atleast} one mango we subtract this from above\\
$\therefore $ Ways for choosing \textit{atleast} one mango is $\vbone + 1 -1 = \vbone$\\
Similarly, number of ways for choosing \textit{atleast} one orange from $\vbtwo$ oranges is $\vbtwo  $.\\
Hence, total ways for selecting fruits is : $\vbone \times \vbtwo = \vbthree$     
\end{solution}


