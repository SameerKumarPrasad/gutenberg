% This is an empty shell file placed for you by the 'examiner' script.
% You can now fill in the TeX for your question here.

% Now, down to brasstacks. ** Writing good solutions is an Art **. 
% Eventually, you will find your own style. But here are some thoughts 
% to get you started: 
%
%   1. Write to be understood - but be crisp. Your own solution should not take 
%      more space than you will give to the student. Hence, if you take more than 
%      a half-page to write a solution, then give the student a full-page and so on...
%
%   2. Use margin-notes to "talk" to students about the critical insights
%      in the question. The tone can be - in fact, should be - informal
%
%   3. Don't shy away from creating margin-figures you think will help
%      students understand. Yes, it is a little more work per question. 
%      But the question & solution will be written only once. Make that
%      attempt at writing a solution count.
%      
%      3b. Use bc_to_fig.tex. Its an easier way to generate plots & graphs 
% 
%   4. Ensure that there are *no spelling mistakes anywhere*. We are an 
%      education company. Bad spellings suggest that we ourselves 
%      don't have any education. Also, use American spellings by default
% 
%   5. If a question has multiple parts, then first delete lines 40-41
%   6. If a question does not have parts, then first delete lines 43-69
%   
%   7. Create versions of the question when possible. Use commands defined in 
%      tufte-tweaks.sty to do so. Its easier than you think

% \noprintanswers
% \setcounter{rolldice}{0}
% \printrubric

\ifnumequal{\value{rolldice}}{0}{
  % variables 
  \renewcommand{\vbone}{\text{REGULATION}}
  \renewcommand{\vbtwo}{5}
  \renewcommand{\vbthree}{5}
  \renewcommand{\vbfour}{}
  \renewcommand{\vbfive}{}
  \renewcommand{\vbsix}{}
  \renewcommand{\vbseven}{}
  \renewcommand{\vbeight}{}
  \renewcommand{\vbnine}{}
  \renewcommand{\vbten}{}
}{
  \ifnumequal{\value{rolldice}}{1}{
    % variables 
    \renewcommand{\vbone}{\text{SPRINGVALE}}
    \renewcommand{\vbtwo}{3}
    \renewcommand{\vbthree}{7}
    \renewcommand{\vbfour}{}
    \renewcommand{\vbfive}{}
    \renewcommand{\vbsix}{}
    \renewcommand{\vbseven}{}
    \renewcommand{\vbeight}{}
    \renewcommand{\vbnine}{}
    \renewcommand{\vbten}{}
  }{
    \ifnumequal{\value{rolldice}}{2}{
      % variables 
      \renewcommand{\vbone}{\text{FALCONVIEW}}
      \renewcommand{\vbtwo}{4}
      \renewcommand{\vbthree}{6}
      \renewcommand{\vbfour}{}
      \renewcommand{\vbfive}{}
      \renewcommand{\vbsix}{}
      \renewcommand{\vbseven}{}
      \renewcommand{\vbeight}{}
      \renewcommand{\vbnine}{}
      \renewcommand{\vbten}{}
    }{
      % variables 
      \renewcommand{\vbone}{\text{PATHFINDER}}
      \renewcommand{\vbtwo}{3}
      \renewcommand{\vbthree}{7}
      \renewcommand{\vbfour}{}
      \renewcommand{\vbfive}{}
      \renewcommand{\vbsix}{}
      \renewcommand{\vbseven}{}
      \renewcommand{\vbeight}{}
      \renewcommand{\vbnine}{}
      \renewcommand{\vbten}{}
    }
  }
}
\ADD\vbthree{1}\k
\question How many different words can be formed from the letters of the word $\vbone$. 

\insertQR{}

\watchout

\ifprintanswers
  % stuff to be shown only in the answer key - like explanatory margin figures
  \begin{marginfigure}
    \figinit{pt}
      \figpt 100:(0,0)
      \figpt 101:(0,0)
    \figdrawbegin{}
      \figdrawline [100,101]
    \figdrawend
    \figvisu{\figBoxA}{}{%
    }
    \centerline{\box\figBoxA}
  \end{marginfigure}
\fi 
\begin{solution}
Total number of words that can be formed by rearranging the 10 letters : $10!$ 
\end{solution}
\begin{parts}
  

  \part In how many words 'E' comes first.

  \insertQR{}
  \begin{solution}
  Let's fix E at the first place and rearrange the rest of the $9$ letters.\\
  $\therefore$ Number of words with E in first place $= 9!$ 
  \end{solution}

  \part E comes first and A at the end.

  \insertQR{}
  \begin{solution}
  Fixing E at first spot and A at end. We rearrange the remaining 8 letters.\\
  $\therefore$ Number of words with E in first place and A in the end $=8!$ ways. 
  \end{solution}

  \part Vowels are together

  \insertQR{}
  \begin{solution}
  Number of vowels in the word = $\vbtwo$\\
  Number of consonants in the word = $\vbthree$\\
  Lets assume all the vowels are in a packet \\
  Now we arrange the packet and $\vbthree$ consonants in $\k!$ ways.\\
  The vowels inside the packet can be arranged in $\vbtwo!$ ways.\\
  Thus, total arrangements $=(\vbtwo!) \cdot(\k!)$
  \end{solution}

\end{parts}

\ifprintrubric
  \begin{table}
  	\begin{tabular}{ p{5cm}p{5cm} }
  		\toprule % in brief (4-6 words), what should a grader be looking for for insights & formulations
  		  \sc{\textcolor{blue}{Insight}} & \sc{\textcolor{blue}{Formulation}} \\ 
  		\midrule % ***** Insights & formulations ******
  		\toprule % final numerical answers for the various versions
        \sc{\textcolor{blue}{If question has $\ldots$}} & \sc{\textcolor{blue}{Final answer}} \\
  		\midrule % ***** Numerical answers (below) **********
  		\bottomrule
  	\end{tabular}
  \end{table}
\fi
