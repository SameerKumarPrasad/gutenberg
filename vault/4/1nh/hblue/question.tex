


\ifnumequal{\value{rolldice}}{0}{
  % variables 
  \renewcommand{\va}{1}
  \renewcommand{\vb}{2}
}{
  \ifnumequal{\value{rolldice}}{1}{
    % variables 
    \renewcommand{\va}{2}
    \renewcommand{\vb}{3}
  }{
    \ifnumequal{\value{rolldice}}{2}{
      % variables 
      \renewcommand{\va}{3}
      \renewcommand{\vb}{4}
    }{
      % variables 
      \renewcommand{\va}{1}
      \renewcommand{\vb}{3}
    }
  }
}

\SQUARE\va\aa
\SQUARE\vb\bb
\SUBTRACT\bb\aa\vc

\MULTIPLY\va{-1}\vd

\question[4] Evaluate the following limit \[ \lim_{x\to -\va}\dfrac{x + \va}{\sqrt{x^2 + \vc} - \vb} \] 

\watchout


\begin{solution}[\halfpage]
  Plugging in $x = -\va$ will turn the expression into $\frac{A}{0}$ form. Not good. So, we need to 
  do something to turn it into $\frac{C}{D},\, D\neq 0$ form. Lets try rationalizing 

  \begin{align}
    \lim_{x\to -\va}\dfrac{x + \va}{\sqrt{x^2 + \vc} - \vb} &= 
    \lim_{x\to -\va}\dfrac{x + \va}{\sqrt{x^2 + \vc} - \vb}\times\dfrac{\sqrt{x^2 + \vc} + \vb}{\sqrt{x^2 + \vc} + \vb}  \\
    &= \lim_{x\to -\va}\dfrac{(x + \va)\cdot(\sqrt{x^2 + \vc} + \vb)}
    {\underbrace{(x^2 + \vc) - \bb}_{x^2-\aa = (x-\va)\cdot(x+\va)}} \\
    &= \lim_{x\to -\va}\dfrac{(\sqrt{x^2 + \vc} + \vb)}{x-\va} = \WRITEFRAC\vb\vd
  \end{align}
\end{solution}

\ifprintanswers\begin{codex}\end{codex}\fi
