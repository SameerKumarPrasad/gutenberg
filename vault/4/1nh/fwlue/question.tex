


\ifnumequal{\value{rolldice}}{0}{
  % variables 
  \renewcommand{\va}{1}
}{
  \ifnumequal{\value{rolldice}}{1}{
    % variables 
    \renewcommand{\va}{2}
  }{
    \ifnumequal{\value{rolldice}}{2}{
      % variables 
      \renewcommand{\va}{3}
    }{
      % variables 
      \renewcommand{\va}{4}
    }
  }
}

\POWER\va{2}\p
\providecommand\e{ (1+x)^{\frac{1}{m}} - (1-x)^{\frac{1}{m}} }

\question[4] Find the limit $\limit{x}{\va}\dfrac{\sqrt[m]{1+x} - \sqrt[m]{1-x}}{\sqrt[n]{1+x} - \sqrt[n]{1-x}}$, $m\neq n$

\watchout

\ifprintanswers
  % stuff to be shown only in the answer key - like explanatory margin figures
  \begin{marginfigure}
    \figinit{pt}
      \figpt 100:(0,0)
      \figpt 101:(0,0)
    \figdrawbegin{}
      \figdrawline [100,101]
    \figdrawend
    \figvisu{\figBoxA}{}{%
    }
    \centerline{\box\figBoxA}
  \end{marginfigure}
\fi 

\begin{solution}[\halfpage]
	When $x=1$, both the numerator and the denominator are equal to $0$ - irrespective of what $m$ and $n$ are. 
	Which means, the whole expression is of the form $\dfrac{0}{0}$ - which is an inderminate form.

	Thus, L'Hospital's Rule can be applied. 
\begin{align}
&\implies \limit{x}{1}\dfrac{ \dfrac{d}{dx} \left[  \e \right] }{2} \\
&\implies \limit{x}{1}\dfrac{\dfrac{d}{dx}\sqrt[m]{1+x} - \sqrt[m]{1-x}} {\dfrac{d}{dx} \sqrt[n]{1+x} - \sqrt[n]{1-x}}\\
&\implies \limit{x}{1} \dfrac{\dfrac{1}{m} \times (1) - \dfrac{1}{m} \times (-1)}{\dfrac{1}{n}\times (1) - \dfrac{1}{n}\times(-1)}\\
&\implies \dfrac{n}{m}
\end{align}
\end{solution}


\ifprintanswers\begin{codex}\end{codex}\fi
