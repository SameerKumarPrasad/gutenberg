


\ifnumequal{\value{rolldice}}{0}{
  % variables 
}{
  \ifnumequal{\value{rolldice}}{1}{
    % variables 
  }{
    \ifnumequal{\value{rolldice}}{2}{
      % variables 
    }{
      % variables 
    }
  }
}

\question[4] Is $0$ a term of the AP: $31,28,25$ Justify your answer?  


\watchout

\ifprintanswers
  % stuff to be shown only in the answer key - like explanatory margin figures
  \begin{marginfigure}
    \figinit{pt}
      \figpt 100:(0,0)
      \figpt 101:(0,0)
    \figdrawbegin{}
      \figdrawline [100,101]
    \figdrawend
    \figvisu{\figBoxA}{}{%
    }
    \centerline{\box\figBoxA}
  \end{marginfigure}
\fi 

\begin{solution}[\halfpage]

     \begin{align}
        A_1 &= 31 \\ 
        D &= A_2 - A_1 = 28 - 31 = -3
	\end{align}
	
	 Now, for $0$ to be in this arithmetic progression, the following must hold 
	 \begin{align}
	 	0 &= A_1 + (N-1)\cdot D = 31 + (N-1)\cdot -3 \\
	 	\Rightarrow N &= \dfrac{34}{3}
	 \end{align}
	 
	 But $N \in I$ - which it is not. Therefore $0$ cannot be part of this arithmetic progression
\end{solution}



