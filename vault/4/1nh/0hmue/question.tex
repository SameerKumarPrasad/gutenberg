


\ifnumequal{\value{rolldice}}{0}{
  \renewcommand\va{31}
  \renewcommand\vd{-3}
  \renewcommand\vn{15}
}{
  \ifnumequal{\value{rolldice}}{1}{
    \renewcommand\va{25}
    \renewcommand\vd{-4}
    \renewcommand\vn{11}
  }{
    \ifnumequal{\value{rolldice}}{2}{
      \renewcommand\va{39}
      \renewcommand\vd{-3}
      \renewcommand\vn{16}
    }{
      \renewcommand\va{43}
      \renewcommand\vd{-4}
      \renewcommand\vn{14}
    }
  }
}

\ADD\va\vd\vb
\ADD\vb\vd\vc
\EXPR[0]\vz{ \va + (\vn-1)*\vd }
\ADD\vz{1}\vx

\SUBTRACT\vx\va\vp
\FRACDIV\vp{1}\vd{1}\vq\vr
\FRACADD\vq\vr{1}{1}\vs\vt

\question[4] If the first three terms of an arithmetic progression are $\va,\vb$ and $\vc$, then 
is $\vx$ a term in the progression? If not, then find the value closest to $\vx$ that is.

\watchout

\begin{solution}[\halfpage]
  \textbf{Step \#1: Find the common difference (d)}

  The common difference - $d$ - of the given progression is simply the difference 
  between any two successive terms. 
  \[ d = a_2 - a_1 = \vb - \va = \vd \] 

  For $\vx$ to be a term in the progression,
  \begin{align}
    \va + (n-1)\cdot\vd &= \vx \implies n = \WRITEFRAC[false]\vs\vt = \WRITEFRAC\vs\vt
  \end{align}

  But $n\in\mathbb{N}$ - which it is not. Hence, $\vx$\textbf{ is not} a term in the progression. 

  That said, the $k\in\mathbb{N}$ closest to $n$ is $\vn$. And hence, 
  the term in the progression closest to $\vx$ is 
  \[ a_{\vj} = a_1 + (k - 1)\cdot d = \va + (\vn - 1)\times\vd = \vz \]
\end{solution}

\ifprintanswers\begin{codex}No. Closest term = $\vz$\end{codex}\fi
