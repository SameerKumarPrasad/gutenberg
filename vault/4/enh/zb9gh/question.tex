


\ifnumequal{\value{rolldice}}{0}{
  % variables 
  \renewcommand{\va}{{1,2,3,4,3,2,1 }}
  \renewcommand{\vb}{{1,3,3,1}}
  \renewcommand{\vc}{2}
}{
  \ifnumequal{\value{rolldice}}{1}{
    % variables 
    \renewcommand{\va}{{3,4,1,2,1,4,3}}
    \renewcommand{\vb}{{3,1,1,3}}
    \renewcommand{\vc}{4}
  }{
    \ifnumequal{\value{rolldice}}{2}{
      % variables 
      \renewcommand{\va}{{3,4,5,2,5,4,3}}
      \renewcommand{\vb}{{3,5,5,3}}
      \renewcommand{\vc}{4}
    }{
      % variables 
    }
  }
}

\question[3] How many different numbers can be formed using all the digits $\lbrace\va\rbrace$ so that odd digits occupy odd places 


\watchout

\ifprintanswers
  % stuff to be shown only in the answer key - like explanatory margin figures
  \begin{marginfigure}
    \figinit{pt}
      \figpt 100:(0,0)
      \figpt 101:(0,0)
    \figdrawbegin{}
      \figdrawline [100,101]
    \figdrawend
    \figvisu{\figBoxA}{}{%
    }
    \centerline{\box\figBoxA}
  \end{marginfigure}
\fi 

\begin{solution}[\mcq]
A 7-digit number will be formed using these digits, we have four available odd places and four odd numbers $\lbrace\vb\rbrace$.\\
\\
These can be arranged at four places in $\dfrac{4!}{2!\cdot2!}$ $\left[\text{ As two numbers repeat themselves} \right]$\\
Now, the remaining 3 digits can be rearranged at 3 places in $\dfrac{3!}{2!}$ $\left[ \text{As $\vc$ repeats itself} \right]$\\
Thus,\\
total arrangements$ =\dfrac{4!}{2!\cdot2!} \times \dfrac{3!}{2!} = 18$ 
\end{solution}

\ifprintanswers\begin{codex}\end{codex}\fi
