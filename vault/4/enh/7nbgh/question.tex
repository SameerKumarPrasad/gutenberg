

\renewcommand\va{4}

\ifnumequal{\value{rolldice}}{0}{
  % variables 
  \renewcommand{\vb}{7}
}{
  \ifnumequal{\value{rolldice}}{1}{
    % variables 
    \renewcommand{\vb}{6}
  }{
    \ifnumequal{\value{rolldice}}{2}{
      % variables 
      \renewcommand{\vb}{5}
    }{
      % variables 
      \renewcommand{\vb}{2}
    }
  }
}

\SUBTRACT\va{1}\a
\POWER{10}\a\b
\MULTIPLY{9}\b\n

\POWER{9}\a\d
\MULTIPLY\d{8}\p
\SUBTRACT\n\p\vz

\question[3] How many $\va$ digit numbers are there such that atleast one of their digit is $\vb$. 

\watchout


\begin{solution}[\halfpage]
  The \textbf{left-most} digit of a $N-$ digit number \textbf{cannot} be $0$. If it is, then 
  we do not have a $N-$digit, but an $N-1$ digit number. 
  
  Other than that, there is no constraint on which digits can come in the remaining $N-1$ positions.

  Given this, the \textbf{total} number of $\va-$digit numbers that can be formed is
  \[ N_{\text{total}} = 9\times\underbrace{10\times10\times\ldots\times10}_{\a\text{ times}} = \n \] 
  The number of numbers with \textbf{atleast one} $\vb$ is simply 
  \[ N = N_{\text{total}} - N_{\text{no }\vb} \] 
  If $\vb$ can't appear anywhere, then we have a total of $9$ digits (including $0$) that we can use. 

  And of these, only $8$ can be used for the left-most digit. Hence, 
  \begin{align}
     N_{\text{no }\vb} &= 8\times\underbrace{9\times9\times\ldots\times9}_{\a\text{ times}} = \p \\
     \implies N &= N_{\text{total}} - N_{\text{no }\vb} = \n-\p = \vz
  \end{align}

\end{solution}
 
\ifprintanswers\begin{codex}$\vz$\end{codex}\fi
