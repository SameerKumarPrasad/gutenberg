


\ifnumequal{\value{rolldice}}{0}{
  % variables 
  \renewcommand{\vbone}{33}
  \renewcommand{\vbtwo}{29}
}{
  \ifnumequal{\value{rolldice}}{1}{
    % variables 
    \renewcommand{\vbone}{30}
    \renewcommand{\vbtwo}{26}
  }{
    \ifnumequal{\value{rolldice}}{2}{
      % variables 
      \renewcommand{\vbone}{18}
      \renewcommand{\vbtwo}{14}
    }{
      % variables 
      \renewcommand{\vbone}{24}
      \renewcommand{\vbtwo}{20}
    }
  }
}
\DIVIDE\vbone{3}\k
\SQUARE{\k}\l
\gcalcexpr[0]\m{\l-\k-\vbtwo} 
\SQUAREROOT{\m}\n
\ADD{\k}{\n}{\pl} 
\SUBTRACT{\k}{\n}{\mi}

\question The sum of the first three terms of an arithmetic progression is $\vbone$. 
If the product of first and the third term exceeds the second term by $\vbtwo$, then find the progression 

\watchout

\begin{solution}
Assuming the terms of the progression to be : $a-d, a, a+d $\\
Adding the terms,\\
$\Rightarrow$  3a = $\vbone \Rightarrow$ a = $\k$\\
\begin{align}
(a+d)(a-d) - a &= \vbtwo\\ % Storing a^2 in l
\Rightarrow {a}^2 - {d}^2 - a &= \vbtwo 
\end{align}
Substituting, a = $\k$\\
\begin{align}  
d^{2} &= {\k}^2 - {\k} - \vbtwo\\ 
\Rightarrow d^{2} &= \m\\ % m= l-k-\vbtwo
\Rightarrow d &= \pm \n  %n is square root of m.
\end{align}
Thus the progression becomes:
\begin{align}
    \underbrace{ \mi, \k, \pl \cdots}_{d=9}\quad\text{OR}\quad\quad\underbrace{ \pl, \k, \mi \cdots  }_{d=-9}
\end{align}

\end{solution}

