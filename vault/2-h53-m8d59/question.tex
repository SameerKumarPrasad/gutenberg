% This is an empty shell file placed for you by the 'examiner' script.
% You can now fill in the TeX for your question here.

% Now, down to brasstacks. ** Writing good solutions is an Art **. 
% Eventually, you will find your own style. But here are some thoughts 
% to get you started: 
%
%   1. Write to be understood - but be crisp. Your own solution should not take 
%      more space than you will give to the student. Hence, if you take more than 
%      a half-page to write a solution, then give the student a full-page and so on...
%
%   2. Use margin-notes to "talk" to students about the critical insights
%      in the question. The tone can be - in fact, should be - informal
%
%   3. Don't shy away from creating margin-figures you think will help
%      students understand. Yes, it is a little more work per question. 
%      But the question & solution will be written only once. Make that
%      attempt at writing a solution count.
%      
%      3b. Use bc_to_fig.tex. Its an easier way to generate plots & graphs 
% 
%   4. Ensure that there are *no spelling mistakes anywhere*. We are an 
%      education company. Bad spellings suggest that we ourselves 
%      don't have any education. Also, use American spellings by default
% 
%   5. If a question has multiple parts, then first delete lines 40-41
%   6. If a question does not have parts, then first delete lines 43-69
%   
%   7. Create versions of the question when possible. Use commands defined in 
%      tufte-tweaks.sty to do so. Its easier than you think

% \noprintanswers
% \setcounter{rolldice}{0}
\ifnumequal{\value{rolldice}}{0}{
  % variables 
  \renewcommand{\vbone}{-1}
  \renewcommand{\vbtwo}{2}
  \renewcommand{\vbthree}{}
  \renewcommand{\vbfour}{+}
  \renewcommand{\vbfive}{15}
  \renewcommand{\vbsix}{6}
  \renewcommand{\vbseven}{-8}
  \renewcommand{\vbeight}{+11}
}{
  \ifnumequal{\value{rolldice}}{1}{
    % variables 
    \renewcommand{\vbone}{2}
    \renewcommand{\vbtwo}{0}
    \renewcommand{\vbthree}{}
    \renewcommand{\vbfour}{+}
    \renewcommand{\vbfive}{15}
    \renewcommand{\vbsix}{8}
    \renewcommand{\vbseven}{-6}
    \renewcommand{\vbeight}{+13}
  }{
    \ifnumequal{\value{rolldice}}{2}{
      % variables 
      \renewcommand{\vbone}{7}
      \renewcommand{\vbtwo}{2}
      \renewcommand{\vbthree}{}
      \renewcommand{\vbfour}{-}
      \renewcommand{\vbfive}{-9}
      \renewcommand{\vbsix}{3}
      \renewcommand{\vbseven}{+4}
      \renewcommand{\vbeight}{+11}
    }{
      % variables 
      \renewcommand{\vbone}{5}
      \renewcommand{\vbtwo}{0}
      \renewcommand{\vbthree}{}
      \renewcommand{\vbfour}{-}
      \renewcommand{\vbfive}{-9}
      \renewcommand{\vbsix}{4}
      \renewcommand{\vbseven}{-3}
      \renewcommand{\vbeight}{+13}
    }
  }
}

\question Find the distance of the point $(\vbone, \vbtwo)$ from the line 
$\vbthree x \vbfour y = \vbfive$ measured parallel to the line 
$\vbsix x \vbseven y \vbeight = 0$.


\insertQR{QRC}

\watchout

\ifprintanswers
  % stuff to be shown only in the answer key - like explanatory margin figures
  \begin{marginfigure}
    \figinit{pt}
      \figpt 100:(0,0)
      \figpt 101:(0,0)
    \figdrawbegin{}
      \figdrawline [100,101]
    \figdrawend
    \figvisu{\figBoxA}{}{%
    }
    \centerline{\box\figBoxA}
  \end{marginfigure}
\fi 

\begin{solution}[\halfpage]
  Let the point from which distance is to be measured be $x_{1}$, $y_{1}$.
  Since this point lies on the line $\vbthree x + \vbfour y = \vbfive$, it 
  must satisfy its equation. Therefore,
  \begin{align}
  	\vbthree x_{1} + \vbfour y_{1} = \vbfive  	
  \end{align}  
  Since the line connecting the two points is parallel to, $\vbsix x 
  \vbseven y + \vbeight = 0$, its slope must be the same. Therefore,
  \begin{align}
  	\dfrac{y_{1}-(\vbtwo)}{x_{1}-(\vbone)} = \dfrac{3}{4}
  \end{align}
  Solving for $x_{1}$ and $y_{1}$ using (1) and (2) we get,
  \begin{align}
  	x_{1} = 7 \text{,} y_{1} = 8 \nonumber
  \end{align}
  Therefore distance between the two points as calculated using
  distance formula equals,
  \begin{align}
  	\text{D} &= \sqrt{(8-2)^2+(7-(-1))^2} \nonumber \\
  			 &= 100 \nonumber
  \end{align}
\end{solution}
