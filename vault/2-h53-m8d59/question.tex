% This is an empty shell file placed for you by the 'examiner' script.
% You can now fill in the TeX for your question here.

% Now, down to brasstacks. ** Writing good solutions is an Art **. 
% Eventually, you will find your own style. But here are some thoughts 
% to get you started: 
%
%   1. Write to be understood - but be crisp. Your own solution should not take 
%      more space than you will give to the student. Hence, if you take more than 
%      a half-page to write a solution, then give the student a full-page and so on...
%
%   2. Use margin-notes to "talk" to students about the critical insights
%      in the question. The tone can be - in fact, should be - informal
%
%   3. Don't shy away from creating margin-figures you think will help
%      students understand. Yes, it is a little more work per question. 
%      But the question & solution will be written only once. Make that
%      attempt at writing a solution count.
%      
%      3b. Use bc_to_fig.tex. Its an easier way to generate plots & graphs 
% 
%   4. Ensure that there are *no spelling mistakes anywhere*. We are an 
%      education company. Bad spellings suggest that we ourselves 
%      don't have any education. Also, use American spellings by default
% 
%   5. If a question has multiple parts, then first delete lines 40-41
%   6. If a question does not have parts, then first delete lines 43-69
%   
%   7. Create versions of the question when possible. Use commands defined in 
%      tufte-tweaks.sty to do so. Its easier than you think

%\noprintanswers
% \setcounter{rolldice}{1}
%\printrubric

\ifnumequal{\value{rolldice}}{0}{
  % variables 
  \renewcommand{\vbone}{-1}
  \renewcommand{\vbtwo}{2}
  \renewcommand{\vbthree}{1}
  \renewcommand{\vbfour}{1}
  \renewcommand{\vbfive}{15}
  \renewcommand{\vbsix}{6}
  \renewcommand{\vbseven}{-8}
  \renewcommand{\vbeight}{11}
  \renewcommand{\vbnine}{10}
}{
  \ifnumequal{\value{rolldice}}{1}{
    % variables 
    \renewcommand{\vbone}{-31}
    \renewcommand{\vbtwo}{-44}
    \renewcommand{\vbthree}{1}
    \renewcommand{\vbfour}{-1}
    \renewcommand{\vbfive}{15}
    \renewcommand{\vbsix}{8}
    \renewcommand{\vbseven}{-6}
    \renewcommand{\vbeight}{13}
    \renewcommand{\vbnine}{10}
  }{
    \ifnumequal{\value{rolldice}}{2}{
      % variables 
      \renewcommand{\vbone}{7}
      \renewcommand{\vbtwo}{2}
      \renewcommand{\vbthree}{1}
      \renewcommand{\vbfour}{-1}
      \renewcommand{\vbfive}{-9}
      \renewcommand{\vbsix}{3}
      \renewcommand{\vbseven}{4}
      \renewcommand{\vbeight}{11}
      \renewcommand{\vbnine}{10}
    }{
      % variables 
      \renewcommand{\vbone}{39}
      \renewcommand{\vbtwo}{50}
      \renewcommand{\vbthree}{1}
      \renewcommand{\vbfour}{-1}
      \renewcommand{\vbfive}{-9}
      \renewcommand{\vbsix}{4}
      \renewcommand{\vbseven}{-3}
      \renewcommand{\vbeight}{13}
      \renewcommand{\vbnine}{10}
    }
  }
}

\FRACTIONSIMPLIFY{-\vbsix}\vbseven\n\d
\gcalcexpr[0]\c{(\n * \vbone) - (\d * \vbtwo)}
\SOLVELINEARSYSTEM(\vbthree,\vbfour ; \n, -\d)(\vbfive,\c)(\p,\q)

\question Find the distance of the point $A = (\vbone, \vbtwo)$ from the line 
$L_1: \gsign\vbthree x \gsign\vbfour y = \vbfive$ measured parallel to the line 
$L_2: \gsign\vbsix x \gsign\vbseven y \gsign\vbeight = 0$.

\insertQR[-10pt]{QRC}

\watchout

\ifprintanswers
\fi 

\begin{solution}[\halfpage]
   There is a point - lets say $B$ - on $L_1$ that when joined with $A$ forms a line parallel to $L_2$. 
   We need to find the distance between $A$ and $B$. Simple $\ldots$
  
  Now, the point $B = (m,n)$ satisfies the following two conditions
  \begin{align}
  	\gsign\vbthree m \gsign\vbfour n &= \vbfive \because \text{ its on } L_1 \\
  	\dfrac{\vbtwo - n}{\vbone - m} &= \dfrac{\n}{\d} \because AB \parallel L_2 \\
  	\Rightarrow \n m - \d n &= \c
  \end{align}
  Solving (1) and (3), we get $B = (m,n) = (\p,\q)$. Now we know everything to calculate the distance 
  between $A$ and $B$
  \begin{align}
    D &= \sqrt{(\p - \vbone)^2 + (\q - \vbtwo)^2} = \vbnine
  \end{align}
\end{solution}

\ifprintrubric
  \begin{table}
  	\begin{tabular}{ p{5cm}p{5cm} }
  		\toprule % in brief (4-6 words), what should a grader be looking for for insights & formulations
  		  \sc{\textcolor{blue}{Insight}} & \sc{\textcolor{blue}{Formulation}} \\ 
  		\midrule % ***** Insights & formulations ******
        There is a pt $B$ on $L_1$ such that $AB \parallel L_2$ & \\
        Slope of $AB = $ slope of $L_2$ & \\
  		\toprule % final numerical answers for the various versions
        \sc{\textcolor{blue}{If question has $\ldots$}} & \sc{\textcolor{blue}{Final answer}} \\
  		\midrule % ***** Numerical answers (below) **********
        $A = (-1,2)$ & $10$\\
        $A = (-31,-44)$ & $10$\\
        $A = (7,2)$ & $10$\\
        $A = (39,50)$ & $10$\\
  		\bottomrule
  	\end{tabular}
  \end{table}
\fi
