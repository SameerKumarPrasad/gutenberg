% This is an empty shell file placed for you by the 'examiner' script.
% You can now fill in the TeX for your question here.

% Now, down to brasstacks. ** Writing good solutions is an Art **. 
% Eventually, you will find your own style. But here are some thoughts 
% to get you started: 
%
%   1. Write to be understood - but be crisp. Your own solution should not take 
%      more space than you will give to the student. Hence, if you take more than 
%      a half-page to write a solution, then give the student a full-page and so on...
%
%   2. Use margin-notes to "talk" to students about the critical insights
%      in the question. The tone can be - in fact, should be - informal
%
%   3. Don't shy away from creating margin-figures you think will help
%      students understand. Yes, it is a little more work per question. 
%      But the question & solution will be written only once. Make that
%      attempt at writing a solution count.
%      
%      3b. Use bc_to_fig.tex. Its an easier way to generate plots & graphs 
% 
%   4. Ensure that there are *no spelling mistakes anywhere*. We are an 
%      education company. Bad spellings suggest that we ourselves 
%      don't have any education. Also, use American spellings by default
% 
%   5. If a question has multiple parts, then first delete lines 40-41
%   6. If a question does not have parts, then first delete lines 43-69
%   
%   7. Create versions of the question when possible. Use commands defined in 
%      tufte-tweaks.sty to do so. Its easier than you think

% \noprintanswers
%\setcounter{rolldice}{0}

\ifnumequal{\value{rolldice}}{0}{
  % variables 
  \renewcommand{\vbone}{2}
  \renewcommand{\vbtwo}{3}
  \renewcommand{\vbthree}{3}
  \renewcommand{\vbfour}{-1}
  \renewcommand{\vbfive}{5}
  \renewcommand{\vbsix}{2}
}{
  \ifnumequal{\value{rolldice}}{1}{
    % variables 
    \renewcommand{\vbone}{1.5}
    \renewcommand{\vbtwo}{4}
    \renewcommand{\vbthree}{3}
    \renewcommand{\vbfour}{1}
    \renewcommand{\vbfive}{5}
    \renewcommand{\vbsix}{3}
  }{
    \ifnumequal{\value{rolldice}}{2}{
      % variables 
      \renewcommand{\vbone}{1}
      \renewcommand{\vbtwo}{4}
      \renewcommand{\vbthree}{1.5}
      \renewcommand{\vbfour}{2}
      \renewcommand{\vbfive}{3}
      \renewcommand{\vbsix}{7}
    }{
      % variables 
      \renewcommand{\vbone}{4}
      \renewcommand{\vbtwo}{2}
      \renewcommand{\vbthree}{2}
      \renewcommand{\vbfour}{6}
      \renewcommand{\vbfive}{4}
      \renewcommand{\vbsix}{5}
    }
  }
}

\gcalcexpr[2]{\vbseven}{\vbfour - \vbtwo}
\gcalcexpr[2]{\vbeight}{ \vbthree - \vbone}
\gcalcexpr[0]{\vbnine}{(\vbseven / \vbeight) }
\gcalcexpr[2]{\vbten}{ (-1 / \vbnine )}

% reuse \vbeight
\gcalcexpr[2]{\vbeight}{\vbsix - (\vbten * \vbfive)}

\question Find the equation of the line that passes through $(\vbfive,\vbsix)$ and 
is perpendicular to the line joining the points $(\vbone,\vbtwo)$ and $(\vbthree, \vbfour)$

\insertQR{}

\watchout

\ifprintanswers
\fi 

\begin{solution}
	For the line passing through $(\vbone, \vbtwo)$ and $(\vbthree, \vbfour)$, the slope is given by 
	\begin{align}
		& \fSlope{1}{2} = \dfrac{\vbfour - \vbtwo}{\vbthree - \vbone} = \vbnine \\
		& \Rightarrow m_2 = \text{ slope of the perpendicular line } = \dfrac{-1}{m_1} = \vbten
	\end{align}
	
	And hence, the equation of the line passing through $(\vbfive,\vbsix)$ and with slope $= \vbten$, is
	\begin{align}
		\dfrac{y-\vbsix}{x-\vbfive} &= \vbten \\
		\text{ or } y &= \vbten\cdot x + \vbeight
	\end{align}
\end{solution}
