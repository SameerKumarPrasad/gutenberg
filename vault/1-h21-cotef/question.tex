% This is an empty shell file placed for you by the 'examiner' script.
% You can now fill in the TeX for your question here.

% Now, down to brasstacks. ** Writing good solutions is an Art **. 
% Eventually, you will find your own style. But here are some thoughts 
% to get you started: 
%
%   1. Write the solution as if you are writing it for your favorite
%      14-17 year old to help him/her understand. Could be your nephew, 
%      your niece, a cousin perhaps or probably even you when you 
%      were that age. Just write for them.
%
%   2. Use margin-notes to "talk" to students about the critical insights
%      in the question. The tone can be - in fact, should be - informal
%
%   3. Don't shy away from creating margin-figures you think will help
%      students understand. Yes, it is a little more work per question. 
%      But the question & solution will be written only once. Make that
%      attempt at writing a solution count.
%
%   4. At the same time, do not be too verbose. A long solution can
%      - at first sight - make the student think, "God, that is a lot to know".
%      Our aim is not to scare students. Rather, our aim should be to 
%      create many "Aha!" moments everyday in classrooms around the world
% 
%   5. Ensure that there are *no spelling mistakes anywhere*. We are an 
%      education company. Bad spellings suggest that we ourselves 
%      don't have any education. Also, use American spellings by default
% 
%   6. If a question has multiple parts, then first delete lines 40-41
%   7. If a question does not have parts, then first delete lines 43-69

\question For what value of $x$ are the tangents to the curves $y=x^2$ and 
$y=x^3$ parallel?

\insertQR{}

\ifprintanswers
  % stuff to be shown only in the answer key - like explanatory margin figures
  % 1. Definition of characteristic points
  \begin{marginfigure}
\figinit{pt}
\def\Xmin{-22.85714}
\def\Ymin{.01522}
\def\Xmax{57.14285}
\def\Ymax{80.01522}
\def\Xori{22.85714}
\def\Yori{-.01522}
\figpt0:(\Xori,\Yori)
\figpt 100: (0,-6.5)
\figpt 101: (0, 13)
% 2. Creation of the graphical file
\figdrawbegin{}
\def\Xmaxx{\Xmax} % To customize the position
\def\Ymaxx{\Ymax} % of the arrow-heads of the axes.
\figset arrowhead(length=4, fillmode=yes) % styling the arrowheads
\figdrawaxes 0(\Xmin, \Xmaxx, \Ymin, \Ymaxx)
\figdrawlineC(
0 -6.41644,
2.75862 -4.36722,
5.51724 -2.80993,
8.27586 -1.67704,
11.03448 -.90103,
13.79310 -.41439,
16.55172 -.14960,
19.31034 -.03913,
22.06896 -.01548,
24.82758 -.01112,
27.58620 .04146,
30.34482 .20980,
33.10344 .56140,
35.86206 1.16379,
38.62068 2.08448,
41.37931 3.39099,
44.13793 5.15084,
46.89655 7.43155,
49.65517 10.30063,
52.41379 13.82561,
55.17241 18.07401,
57.93103 23.11333,
60.68965 29.01112,
63.44827 35.83487,
66.20689 43.65211,
68.96551 52.53035,
71.72413 62.53713,
74.48275 73.73995,
77.24137 86.20633,
79.99999 100.00380
)
\figdrawlineC(
0 12.78721,
2.75862 9.88344,
5.51724 7.35264,
8.27586 5.19480,
11.03448 3.40992,
13.79310 1.99800,
16.55172 .95904,
19.31034 .29304,
22.06896 0,
24.82758 .07992,
27.58620 .53280,
30.34482 1.35864,
33.10344 2.55744,
35.86206 4.12920,
38.62068 6.07392,
41.37931 8.39160,
44.13793 11.08225,
46.89655 14.14585,
49.65517 17.58241,
52.41379 21.39194,
55.17241 25.57442,
57.93103 30.12987,
60.68965 35.05827,
63.44827 40.35964,
66.20689 46.03396,
68.96551 52.08125,
71.72413 58.50149,
74.48275 65.29470,
77.24137 72.46087,
79.99999 79.99999
)
\figdrawend
% 3. Writing text on the figure
\figvisu{\figBoxA}{}{%
\figptsaxes 1:0(\Xmin, \Xmaxx, \Ymin, \Ymaxx)
% Points 1 and 2 are the end points of the arrows
\figwritee 1:(5pt)     \figwriten 2:(5pt)
\figptsaxes 1:0(\Xmin, \Xmax, \Ymin, \Ymax)
\figwriten 101: $y=x^2$(2)
\figwrites 100: $y=x^3$(2)
}
\centerline{\box\figBoxA}
\end{marginfigure}

\fi 

\begin{solution}
	Two lines are parallel if their slopes are equal. And the slopes of the 
	two tangents in question are given by
	
	\begin{align}
		\dfrac{\ud x^2}{\ud x} &= 2x \\
		\text{ and } \dfrac{\ud x^3}{\ud x} &= 3x^2 
	\end{align}
	
	Which means, they are parallel when 
	\begin{align}
		2x &= 3x^2 \\
		\Rightarrow x\cdot(3x-2) &= 0 \\
		\Rightarrow x &= 0, \frac{2}{3}
	\end{align}
\end{solution}
