% This is an empty shell file placed for you by the 'examiner' script.
% You can now fill in the TeX for your question here.

% Now, down to brasstacks. ** Writing good solutions is an Art **. 
% Eventually, you will find your own style. But here are some thoughts 
% to get you started: 
%
%   1. Write the solution as if you are writing it for your favorite
%      14-17 year old to help him/her understand. Could be your nephew, 
%      your niece, a cousin perhaps or probably even you when you 
%      were that age. Just write for them.
%
%   2. Use margin-notes to "talk" to students about the critical insights
%      in the question. The tone can be - in fact, should be - informal
%
%   3. Don't shy away from creating margin-figures you think will help
%      students understand. Yes, it is a little more work per question. 
%      But the question & solution will be written only once. Make that
%      attempt at writing a solution count.
%
%   4. At the same time, do not be too verbose. A long solution can
%      - at first sight - make the student think, "God, that is a lot to know".
%      Our aim is not to scare students. Rather, our aim should be to 
%      create many "Aha!" moments everyday in classrooms around the world
% 
%   5. Ensure that there are *no spelling mistakes anywhere*. We are an 
%      education company. Bad spellings suggest that we ourselves 
%      don't have any education. Also, use American spellings by default
% 
%   6. If a question has multiple parts, then first delete lines 40-41
%   7. If a question does not have parts, then first delete lines 43-69

\question In the figure alongside, $PQRS$ is the diameter of a circle with radius 6cm.
The lengths $PQ$, $QR$ and $RS$ are equal. Semi-circles are drawn on $PQ$ and $QS$ as diameters.
Find the area of the shaded region

\insertQR{}

\ifprintanswers
  % stuff to be shown only in the answer key - like explanatory margin figures
\fi 
\begin{marginfigure}

  \figinit{pt}
    \figpt 100: $O$(45,0)
    \figpt 101: $P$(0,0)
    \figpt 102: $Q$(30,0)
    \figpt 103: $R$(60,0)
    \figpt 104: $S$(90,0)
    \figpt 105: $M$(15,0)
  \figdrawbegin{}
    %\figdrawcirc 100 (45)
    \figset (fillmode=yes,color=0.7)
    \figdrawarccirc 100 ; 45 (0,180)
    \figdrawarccirc 105 ; 15 (180,360)
    \figset (fillmode=no)
    \figdrawarccirc 100 ; 45 (180,360)
    \figset (fillmode=yes,color=1)
    \figdrawline [101,102,103,104]
    \figdrawarccirc 103 ; 30 (0,180)
  \figdrawend
  % 3. Writing text on the figure
  \figvisu{\figBoxA}{}{%
    \figset write (mark=$\bullet$)
    \figwrites 100:(2)
    \figwritew 101:(2)
    \figwrites 102:(2)
    \figwrites 103:(2)
    \figwritee 104:(2)
  }
  \centerline{\box\figBoxA}

\end{marginfigure}

\begin{solution}
	\begin{align}
		PS &= 2\times OP = \text{12 cm} \\
		\therefore PQ &= QR = RS \Rightarrow PQ = QR = RS = \text{4 cm}
	\end{align}
	
	And hence, the required area $A$ is
	\begin{align}
		A &= \dfrac{1}{2}\pi\dfrac{PS^2}{4} + \dfrac{1}{2}\pi\dfrac{PQ^2}{4} 
		- \dfrac{1}{2}\pi\dfrac{QS^2}{4} \\
		&= \dfrac{\pi}{8}\cdot\left( 12^2 + 4^2 - 8^2\right) \\
		&= 12\pi \text{cm}^2
	\end{align}
\end{solution}
