% This is an empty shell file placed for you by the 'examiner' script.
% You can now fill in the TeX for your question here.

% Now, down to brasstacks. ** Writing good solutions is an Art **. 
% Eventually, you will find your own style. But here are some thoughts 
% to get you started: 
%
%   1. Write the solution as if you are writing it for your favorite
%      14-17 year old to help him/her understand. Could be your nephew, 
%      your niece, a cousin perhaps or probably even you when you 
%      were that age. Just write for them.
%
%   2. Use margin-notes to "talk" to students about the critical insights
%      in the question. The tone can be - in fact, should be - informal
%
%   3. Don't shy away from creating margin-figures you think will help
%      students understand. Yes, it is a little more work per question. 
%      But the question & solution will be written only once. Make that
%      attempt at writing a solution count.
%
%   4. At the same time, do not be too verbose. A long solution can
%      - at first sight - make the student think, "God, that is a lot to know".
%      Our aim is not to scare students. Rather, our aim should be to 
%      create many "Aha!" moments everyday in classrooms around the world
% 
%   5. Ensure that there are *no spelling mistakes anywhere*. We are an 
%      education company. Bad spellings suggest that we ourselves 
%      don't have any education. And, use American spellings

\question[3]  If an amount is compounded $N$ times a year at the rate of
\insertQR{QRC}
$R\%$ in each period, then what would be the equivalent \textit{annualized}
compounding rate - $R'$? 

\ifprintanswers
  % stuff to be shown only in the answer key - like explanatory margin figures
  \marginnote{If you are getting 1\% a month, then why are you told you are getting 12\% per annum?}
  \marginnote[0.3cm]{Its simple. People find it easier to calculate $1\%\times 12 = 12\%$ in their
  heads than they do calculating the more precise $(1+0.01)^{12} - 1 = 12.68\%$} 
  \marginnote[0.3cm]{And so, you will most often be told the nominal rate - the 12\% - and
  not the effective rate - the 12.68\%. Its your job to find out which rate it is that 
  you are being told}
\fi 

\begin{solution}[\mcq]
	The point to understand is that if you started with a principal 
	amount $P$ at the beginning of the year, then
	- at the end of the year - it should make no difference to you if you got $R\%$
	 $N$ times a year or $R'\%$ once at the end of the year
	 
	 In other words
	 \begin{align}
	 	P\cdot\left( 1 + \dfrac{R}{100}\right)^N &= P\cdot\left( 1 + \dfrac{R'}{100}\right) \\
	 	\Rightarrow R' &= 100\cdot\left[ \left( 1+\dfrac{R}{100}\right)^N - 1 \right]
	 \end{align}
\end{solution}
