% This is an empty shell file placed for you by the 'examiner' script.
% You can now fill in the TeX for your question here.

% Now, down to brasstacks. ** Writing good solutions is an Art **. 
% Eventually, you will find your own style. But here are some thoughts 
% to get you started: 
%
%   1. Write to be understood - but be crisp. Your own solution should not take 
%      more space than you will give to the student. Hence, if you take more than 
%      a half-page to write a solution, then give the student a full-page and so on...
%
%   2. Use margin-notes to "talk" to students about the critical insights
%      in the question. The tone can be - in fact, should be - informal
%
%   3. Don't shy away from creating margin-figures you think will help
%      students understand. Yes, it is a little more work per question. 
%      But the question & solution will be written only once. Make that
%      attempt at writing a solution count.
%      
%      3b. Use bc_to_fig.tex. Its an easier way to generate plots & graphs 
% 
%   4. Ensure that there are *no spelling mistakes anywhere*. We are an 
%      education company. Bad spellings suggest that we ourselves 
%      don't have any education. Also, use American spellings by default
% 
%   5. If a question has multiple parts, then first delete lines 40-41
%   6. If a question does not have parts, then first delete lines 43-69
%   
%   7. Create versions of the question when possible. Use commands defined in 
%      tufte-tweaks.sty to do so. Its easier than you think

%\noprintanswers
%\setcounter{rolldice}{2}

\ifnumequal{\value{rolldice}}{0}{
  % variables 
  \renewcommand{\vbone}{1} % p
  \renewcommand{\vbtwo}{3} % q 
  \renewcommand{\vbthree}{5} % r 
}{
  \ifnumequal{\value{rolldice}}{1}{
    % variables 
    \renewcommand{\vbone}{6}
    \renewcommand{\vbtwo}{14}
    \renewcommand{\vbthree}{21}
  }{
    \ifnumequal{\value{rolldice}}{2}{
      % variables 
      \renewcommand{\vbone}{20}
      \renewcommand{\vbtwo}{35}
      \renewcommand{\vbthree}{42}
    }{
      % variables 
      \renewcommand{\vbone}{5}
      \renewcommand{\vbtwo}{10}
      \renewcommand{\vbthree}{14}
    }
  }
}

\gcalcexpr[0]\tp{\vbone + \vbtwo}
\gcalcexpr[0]\tq{\vbtwo + \vbthree}
\gcalcexpr[2]\tr{\tq / \tp}
\gcalcexpr[0]\ts{(\tr * \vbone + \vbthree) / (\vbtwo - \tr * \vbone)}

\question[4] If some \textit{consecutive} coefficients in the expansion of $(1+x)^n$ are in the 
ratio $\vbone:\vbtwo:\vbthree$, then find $n$

\insertQR[-15pt]{QRC}

\watchout

\ifprintanswers
\fi 

\begin{solution}[\halfpage]
	Let the $(m-1)^{\text{th}}$, $m^{\text{th}}$ and $(m+1)^{\text{th}}$ terms be in the given ratio
	\begin{align}
		\dfrac{\encr{n}{m-1}}{\encr{n}{m}} &= \dfrac{\vbone}{\vbtwo} \\
		\Rightarrow \dfrac{\fncr{n}{m-1}}{\fncr{n}{m}} &= \dfrac{\vbone}{\vbtwo} \\
		\Rightarrow \dfrac{m}{n-m+1} &= \dfrac{\vbone}{\vbtwo} \Rightarrow \tp\cdot m = \vbone\cdot(n+1)
	\end{align}
	Similarly,
	\begin{align}
		\dfrac{\encr{n}{m}}{\encr{n}{m+1}} &= \dfrac{\vbtwo}{\vbthree} \Rightarrow \dfrac{m+1}{n-m} = \dfrac{\vbtwo}{\vbthree} \\
		\Rightarrow \tq\cdot m &= \vbtwo\cdot n - \vbthree \\
		\text{And therefore, } \dfrac{\tq\cdot m}{\tp\cdot m} &= \dfrac{\vbtwo\cdot n - \vbthree}{\vbone\cdot(n+1)} \\
		\Rightarrow n &= \ts
	\end{align}
\end{solution}
