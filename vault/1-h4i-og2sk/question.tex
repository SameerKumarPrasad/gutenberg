% This is an empty shell file placed for you by the 'examiner' script.
% You can now fill in the TeX for your question here.

% Now, down to brasstacks. ** Writing good solutions is an Art **. 
% Eventually, you will find your own style. But here are some thoughts 
% to get you started: 
%
%   1. Write the solution as if you are writing it for your favorite
%      14-17 year old to help him/her understand. Could be your nephew, 
%      your niece, a cousin perhaps or probably even you when you 
%      were that age. Just write for them.
%
%   2. Use margin-notes to "talk" to students about the critical insights
%      in the question. The tone can be - in fact, should be - informal
%
%   3. Don't shy away from creating margin-figures you think will help
%      students understand. Yes, it is a little more work per question. 
%      But the question & solution will be written only once. Make that
%      attempt at writing a solution count.
%
%   4. At the same time, do not be too verbose. A long solution can
%      - at first sight - make the student think, "God, that is a lot to know".
%      Our aim is not to scare students. Rather, our aim should be to 
%      create many "Aha!" moments everyday in classrooms around the world
% 
%   5. Ensure that there are *no spelling mistakes anywhere*. We are an 
%      education company. Bad spellings suggest that we ourselves 
%      don't have any education. Also, use American spellings by default
% 
%   6. If a question has multiple parts, then first delete lines 40-41
%   7. If a question does not have parts, then first delete lines 43-69

\question Find three numbers in geometric progression whose sum is $52$ and the
sum of whose products - taken in pairs - is $624$

\insertQR{}

\ifprintanswers
\fi 

\begin{solution}
	Let the three numbers be $\dfrac{a}{r}$, $a$ and $ar$
	\begin{align}
		\dfrac{a}{r} + a + ar &= 52 \\
		\text{And, } \left( \dfrac{a}{r}\cdot a\right) + \left( a\cdot ar\right) + 
		\left( \dfrac{a}{r}\cdot ar\right) &= 624 \\
		\Rightarrow \dfrac{a^2\cdot\left( \frac{1}{r} + r + 1 \right)}{a\cdot\left( \frac{1}{r} + r + 1\right)}
		&= \dfrac{624}{52} \Rightarrow a = 12 \\
		\text{Hence, } 12\cdot\left( \dfrac{1}{r} + r + 1 \right) &= 52 \Rightarrow 3r^2-10r + 3 = 0 \\
		\Rightarrow r = \frac{1}{3}, 3
	\end{align}
	
	Hence, the three numbers are either $(36,12,4)$ if $r=\frac{1}{3}$ or $(4,12,36)$ if $r=3$
\end{solution}

