% This is an empty shell file placed for you by the 'examiner' script.
% You can now fill in the TeX for your question here.

% Now, down to brasstacks. ** Writing good solutions is an Art **. 
% Eventually, you will find your own style. But here are some thoughts 
% to get you started: 
%
%   1. Write the solution as if you are writing it for your favorite
%      14-17 year old to help him/her understand. Could be your nephew, 
%      your niece, a cousin perhaps or probably even you when you 
%      were that age. Just write for them.
%
%   2. Use margin-notes to "talk" to students about the critical insights
%      in the question. The tone can be - in fact, should be - informal
%
%   3. Don't shy away from creating margin-figures you think will help
%      students understand. Yes, it is a little more work per question. 
%      But the question & solution will be written only once. Make that
%      attempt at writing a solution count.
%
%   4. At the same time, do not be too verbose. A long solution can
%      - at first sight - make the student think, "God, that is a lot to know".
%      Our aim is not to scare students. Rather, our aim should be to 
%      create many "Aha!" moments everyday in classrooms around the world
% 
%   5. Ensure that there are *no spelling mistakes anywhere*. We are an 
%      education company. Bad spellings suggest that we ourselves 
%      don't have any education. Also, use American spellings by default
% 
%   6. If a question has multiple parts, then first delete lines 40-41
%   7. If a question does not have parts, then first delete lines 43-69

\question[4] The sum terms of a converging geometric progression is $\dfrac{3}{2}$. The sum of squares of the same series is $\dfrac{1}{8}$. Find the initial term and common ratio of the progression.

\insertQR{QRC}

\ifprintanswers
  % stuff to be shown only in the answer key - like explanatory margin figures
\fi 

\begin{solution}[\fullpage]
  The sum of terms of an infinite converging geometric series with first term $a$ and common ration $r$ is given by,
  \begin{align}
    S &= \dfrac{a}{1-r}
  \end{align}
  Sum of squares of the same series would be,
  \begin{align}
    S_{sq} &= \dfrac{a^2}{1-r^2}
  \end{align}
  Substituting the given data in the above results we get,
  \begin{align}
    \dfrac{3}{2} &= \dfrac{a}{1-r} \\
    \dfrac{1}{8} &= \dfrac{a^2}{1-r^2}
  \end{align}  
  Take square on both sides of equation (3) and divide by equation (4),
  \begin{align}
    \dfrac{72}{4} &= \dfrac{a^2}{(1-r)^2}\dfrac{1-r^2}{a^2} \\
    18            &= \dfrac{1+r}{1-r} \\
    r             &=1,\dfrac{17}{19}
  \end{align}
  Ignoring result $r=1$ we get common ratio $r=\dfrac{17}{19}$ and initial term $a=\dfrac{3}{19}$.

\end{solution}

