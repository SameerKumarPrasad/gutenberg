
\ifnumequal{\value{rolldice}}{0}{
  \renewcommand\va{2}
  \renewcommand\vc{1}
  \renewcommand\vd{3}
  \renewcommand\vk{2}
}{
  \ifnumequal{\value{rolldice}}{1}{
    \renewcommand\va{3}
    \renewcommand\vc{5}
    \renewcommand\vd{3}
    \renewcommand\vk{2}
  }{
    \ifnumequal{\value{rolldice}}{2}{
      \renewcommand\va{3}
      \renewcommand\vc{2}
      \renewcommand\vd{3}
      \renewcommand\vk{2}
    }{
      \renewcommand\va{5}
      \renewcommand\vc{4}
      \renewcommand\vd{3}
      \renewcommand\vk{2}
    }
  }
}

\POWER\vk{2}\vm
\POWER\vd{2}\vn
\SUBTRACT\vm\vn\vo
\MULTIPLY\va{4}\vp
\FRACADD{\vc}{1}\vo\vp\vx\vy % b = vx / vy
\MULTIPLY\vd{-1}\ve
\ADD\ve\vk\vf
\SUBTRACT\ve\vk\vg
\MULTIPLY\va{2}\vh % roots = vf/vh, vg/vh

\question Given two functions 
\[
  f:\mathbb{R}\rightarrow\mathbb{R} = \va x^2 -\frac\vx\vy \text{ and }
  g:\mathbb{R}\rightarrow\mathbb{R} = \vc-\vd x 
\]

\begin{parts}
  \part[2] Find the domain for which the two functions are equal
\begin{solution}[\mcq]
    The two functions would be equal when 
    \begin{align}
      \va x^2 -\frac\vx\vy &= \vc-\vd x \\
      \implies \va x^2 +\vd x -\left( \frac\vx\vy - \vc\right) &= 0 \\
      \implies x &= \WRITEFRAC\vf\vh, \WRITEFRAC\vg\vh
    \end{align}

    Hence, the domain is $\left\{ \WRITEFRAC\vf\vh, \WRITEFRAC\vg\vh \right\}$
  \end{solution}
  
  \part[2] What would the same "domain of equality" be if $f(x) = f:\mathbb{N}\rightarrow\mathbb{R}$ (instead of 
  $f:\mathbb{R}\rightarrow\mathbb{R}$) and $g(x)=g:\mathbb{N}\rightarrow\mathbb{R}$ (instead of 
  $g:\mathbb{R}\rightarrow\mathbb{R}$). Provide justification for credit.
\begin{solution}[\mcq]
    As neither of the two $x$'s we found in part (a) $\in\mathbb{N}$, there is no $x\in\mathbb{N}$ for which 
    $f(x) = g(x)$.

    Hence, in this case, the domain for which the two functions are equal is $=\lbrace\rbrace=\phi$
  \end{solution}

\end{parts}

