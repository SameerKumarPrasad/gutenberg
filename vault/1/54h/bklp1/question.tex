% This is an empty shell file placed for you by the 'examiner' script.
% You can now fill in the TeX for your question here.

% Now, down to brasstacks. ** Writing good solutions is an Art **. 
% Eventually, you will find your own style. But here are some thoughts 
% to get you started: 
%
%   1. Write the solution as if you are writing it for your favorite
%      14-17 year old to help him/her understand. Could be your nephew, 
%      your niece, a cousin perhaps or probably even you when you 
%      were that age. Just write for them.
%
%   2. Use margin-notes to "talk" to students about the critical insights
%      in the question. The tone can be - in fact, should be - informal
%
%   3. Don't shy away from creating margin-figures you think will help
%      students understand. Yes, it is a little more work per question. 
%      But the question & solution will be written only once. Make that
%      attempt at writing a solution count.
%
%   4. At the same time, do not be too verbose. A long solution can
%      - at first sight - make the student think, "God, that is a lot to know".
%      Our aim is not to scare students. Rather, our aim should be to 
%      create many "Aha!" moments everyday in classrooms around the world
% 
%   5. Ensure that there are *no spelling mistakes anywhere*. We are an 
%      education company. Bad spellings suggest that we ourselves 
%      don't have any education. Also, use American spellings by default
% 
%   6. If a question has multiple parts, then first delete lines 40-41
%   7. If a question does not have parts, then first delete lines 43-69

%\noprintanswers
%\setcounter{rolldice}{0}
%\printrubric

\ifnumequal{\value{rolldice}}{0}{
  % variables 
  \renewcommand{\vbone}{2}
  \renewcommand{\vbtwo}{11}
  \renewcommand{\vbthree}{12}
  \renewcommand{\vbfour}{7}
}{
  \ifnumequal{\value{rolldice}}{1}{
    % variables 
    \renewcommand{\vbone}{-3}
    \renewcommand{\vbtwo}{8}
    \renewcommand{\vbthree}{5}
    \renewcommand{\vbfour}{11}
  }{
    \ifnumequal{\value{rolldice}}{2}{
      % variables 
      \renewcommand{\vbone}{9}
      \renewcommand{\vbtwo}{4}
      \renewcommand{\vbthree}{7}
      \renewcommand{\vbfour}{-13}
    }{
      % variables 
      \renewcommand{\vbone}{-6}
      \renewcommand{\vbtwo}{7}
      \renewcommand{\vbthree}{5}
      \renewcommand{\vbfour}{14}
    }
  }
}

\FRACADD\vbone\vbtwo\vbfour\vbthree\tp\tq
\FRACMULT\vbone\vbtwo\vbfour\vbthree\tr\ts
\FRACMINUS{1}{1}\tr\ts\ta\tb
\FRACDIV\tp\tq\ta\tb\tx\ty

\question[1] Find $A$ if $\tan^{-1} \dfrac{\vbone}{\vbtwo} + \cot^{-1}\dfrac{\vbthree}{\vbfour} = \tan^{-1} A$

\insertQR[-20pt]{QRC}
\watchout

\ifprintanswers
  % stuff to be shown only in the answer key - like explanatory margin figures
\fi 

\begin{solution}[\halfpage]
	\begin{align}
	    \tan^{-1}\frac{\vbone}{\vbtwo} + \cot^{-1}\frac{\vbthree}{\vbfour} &= 
      \tan^{-1}\frac{\vbone}{\vbtwo} + \tan^{-1}\frac{\vbfour}{\vbthree} \\
	    &= \tan^{-1}\left(\dfrac{\frac{\vbone}{\vbtwo} + 
      \frac{\vbfour}{\vbthree}}{1-\frac{\vbone}{\vbtwo}\cdot\frac{\vbfour}{\vbthree}}\right) \\
	    &= \tan^{-1}\left(\dfrac\tx\ty\right) \\
      &\Rightarrow A = \WRITEFRAC\tx\ty 
	\end{align}
\end{solution}

\ifprintrubric
  \begin{table}
  	\begin{tabular}{ p{5cm}p{5cm} }
  		\toprule % in brief (4-6 words), what should a grader be looking for for insights & formulations
  		  \sc{\textcolor{blue}{Insight}} & \sc{\textcolor{blue}{Formulation}} \\ 
  		\midrule % ***** Insights & formulations ******
        $\cot^{-1}\dfrac{x}{y} = \tan^{-1}\dfrac{y}{x}$ & $\tan^{-1}a + \tan^{-1}b = \tan^{-1}\dfrac{a+b}{1-ab}$\\
  		\toprule % final numerical answers for the various versions
        \sc{\textcolor{blue}{If question has $\ldots$}} & \sc{\textcolor{blue}{Final answer}} \\
  		\midrule % ***** Numerical answers (below) **********
        $\tan^{-1}\frac{2}{11} + \cot^{-1}\frac{12}{7}$ & $A=\frac{101}{118}$ \\     
        $\tan^{-1}\frac{-3}{8} + \cot^{-1}\frac{5}{11}$ & $A=1$ \\     
        $\tan^{-1}\frac{9}{4} + \cot^{-1}\frac{7}{-13}$ & $A=\frac{11}{145}$ \\     
        $\tan^{-1}\frac{-6}{7} + \cot^{-1}\frac{5}{14}$ & $A=\frac{4}{7}$ \\     
  		\bottomrule
  	\end{tabular}
  \end{table}
\fi
