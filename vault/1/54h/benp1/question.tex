% This is an empty shell file placed for you by the 'examiner' script.
% You can now fill in the TeX for your question here.

% Now, down to brasstacks. ** Writing good solutions is an Art **. 
% Eventually, you will find your own style. But here are some thoughts 
% to get you started: 
%
%   1. Write the solution as if you are writing it for your favorite
%      14-17 year old to help him/her understand. Could be your nephew, 
%      your niece, a cousin perhaps or probably even you when you 
%      were that age. Just write for them.
%
%   2. Use margin-notes to "talk" to students about the critical insights
%      in the question. The tone can be - in fact, should be - informal
%
%   3. Don't shy away from creating margin-figures you think will help
%      students understand. Yes, it is a little more work per question. 
%      But the question & solution will be written only once. Make that
%      attempt at writing a solution count.
%
%   4. At the same time, do not be too verbose. A long solution can
%      - at first sight - make the student think, "God, that is a lot to know".
%      Our aim is not to scare students. Rather, our aim should be to 
%      create many "Aha!" moments everyday in classrooms around the world
% 
%   5. Ensure that there are *no spelling mistakes anywhere*. We are an 
%      education company. Bad spellings suggest that we ourselves 
%      don't have any education. Also, use American spellings by default
% 
%   6. If a question has multiple parts, then first delete lines 40-41
%   7. If a question does not have parts, then first delete lines 43-69

\question[2] Find the inverse of the one-to-one function
$f:\Re-\{-5\}\rightarrow \Re-\{2\}$ defined by $f(x)=\dfrac{2x-1}{x+5}$

\insertQR{QRC}

\ifprintanswers
	\marginnote{$\Re-\{-5\}$ means all real numbers \textit{except} -5. Similarly, $\Re-\{-5,7\}$ means
	all real numbers \textit{except} -5 and 7. Getting the picture?}
	\marginnote[1cm]{If $f(x) = y$, then $f^{-1}(y) = x$. This is the definition of an inverse function }
\fi 

\begin{solution}[\mcq]
	\begin{align}
		\text{If } y &= f(x) = \dfrac{2x-1}{x+5} \\
		\text{then } x &= \dfrac{5y+1}{2-y} = f^{-1}(y)
	\end{align}
	In $y = f(x)$, $y$ is the dependent variable and $x$ the independent variable. 
	And the convention is to write the \textit{independent} variable as $x$ and the 
	dependent variable as $y$
	
	So, we could also write (2) as $f^{-1}(x) = \dfrac{5x+1}{2-x}$. But remember that 
	the $x$ here \textit{is not} the same as the $x$ in (1). We are just following a naming
	convention here
	
	Alternately, we could use the fact that $f \circ f^{-1}(y) = y$ (or $f \circ f^{-1}(x) = x$, if 
	we follow the naming convention)
	
	\begin{align}
		\Rightarrow \dfrac{2f^{-1}(x)-1}{f^{-1}(x)+5} &= x \\
		\Rightarrow f^{-1}(x) &= \dfrac{5x+1}{2-x}
	\end{align}
\end{solution}
