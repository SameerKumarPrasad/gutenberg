% This is an empty shell file placed for you by the 'examiner' script.
% You can now fill in the TeX for your question here.

% Now, down to brasstacks. ** Writing good solutions is an Art **. 
% Eventually, you will find your own style. But here are some thoughts 
% to get you started: 
%
%   1. Write the solution as if you are writing it for your favorite
%      14-17 year old to help him/her understand. Could be your nephew, 
%      your niece, a cousin perhaps or probably even you when you 
%      were that age. Just write for them.
%
%   2. Use margin-notes to "talk" to students about the critical insights
%      in the question. The tone can be - in fact, should be - informal
%
%   3. Don't shy away from creating margin-figures you think will help
%      students understand. Yes, it is a little more work per question. 
%      But the question & solution will be written only once. Make that
%      attempt at writing a solution count.
%
%   4. At the same time, do not be too verbose. A long solution can
%      - at first sight - make the student think, "God, that is a lot to know".
%      Our aim is not to scare students. Rather, our aim should be to 
%      create many "Aha!" moments everyday in classrooms around the world
% 
%   5. Ensure that there are *no spelling mistakes anywhere*. We are an 
%      education company. Bad spellings suggest that we ourselves 
%      don't have any education. Also, use American spellings by default
% 
%   6. If a question has multiple parts, then first delete lines 40-41
%   7. If a question does not have parts, then first delete lines 43-69

\question[2] Imagine a rectangular field 16 m long and 10 m wide. And now you want
to build a tiled walking track - of constant width - around it. But you have enough 
money to buy only 120 sq. meters of tile. How wide can you make the track then?

\insertQR{QRC}

\ifprintanswers
  % stuff to be shown only in the answer key - like explanatory margin figures
  \begin{marginfigure}
    \figinit{pt}
      \figpt 100: $A$(0,0)
      \figpt 101: $B$(90,0)
      \figpt 102: $C$(90,60)
      \figpt 103: $D$(0,60)
      \figpt 104: $E$(15,15)
      \figpt 105: $F$(75,15)
      \figpt 106: $G$(75,45)
      \figpt 107: $H$(15,45)
      \figpt 204: (0,15)
      \figpt 205: (90,15)
      \figpt 206: (90,45)
      \figpt 207: (0,45)
    \figdrawbegin{}
      \figset (fillmode=yes,color=0.7)
      \figdrawline [100,101,102,103,100]
      \figset (fillmode=yes,color=1)
      \figdrawline [104,105,106,107,104]
    \figdrawend
    \figvisu{\figBoxA}{}{%
      \figwrites 100:(2)
      \figwrites 101:(2)
      \figwritene 104:(1)
      \figwritenw 105:(1)
      \figwriten 102:(2)
      \figwriten 103:(2)
      \figwritesw 106:(1)
      \figwritese 107:(1)
    }
    \centerline{\box\figBoxA}
  \end{marginfigure}
\fi 

\begin{solution}[\halfpage]
  The situation is \asif. The shaded portion is the walking track and $EFGH$ is the field

  Now, if $d$ be the (constant) width of the walking track, then
  \begin{align}
  	2\times\left[\underbrace{(16+2d)\cdot d}_{\text{along the length}} + 
  	\underbrace{10\cdot d}_{\text{side strips}} \right] &= 120 \\
  	\Rightarrow 16d + 2d^2 + 10d &= 60 \\
  	\Rightarrow d^2 + 13d - 30 &= 0 \\
  	\Rightarrow d &= 2, -15
  \end{align}
  
  As the walking track  can be only of a positive width, only $d = 2$ makes sense. 
\end{solution}

