
\ifnumequal{\value{rolldice}}{0}{
  % variables 
  \renewcommand{\vb}{7} 
  \renewcommand\vx{-\frac{60}{19}}
  \renewcommand\vd{5}
}{
  \ifnumequal{\value{rolldice}}{1}{
    % variables 
    \renewcommand{\vb}{6} 
    \renewcommand\vx{-\frac{5}{2}}
    \renewcommand\vd{4}
  }{
    \ifnumequal{\value{rolldice}}{2}{
      % variables 
      \renewcommand{\vb}{8} 
      \renewcommand\vx{-\frac{9}{17}}
      \renewcommand\vd{1}
    }{
      % variables 
      \renewcommand{\vb}{5} 
      \renewcommand\vx{-\frac{7}{6}}
      \renewcommand\vd{2}
    }
  }
}

\ADD\vb\vd\va
\FRACADD{1}\va{1}\vb\p\q
\EXPR[0]\r{2*\q-\vd*\p}
\MULTIPLY\vd\q\s

\question[4] Two pipes - $A$ and $B$ - when used together can fill a tank in $\WRITEFRAC\q\p$ minutes.
Individually, however, pipe $A$ takes $\vd$ minute(s) longer than pipe $B$ to fill the same
tank. Find the time pipe $A$ would take to fill the tank on its own. 

\watchout

\begin{calcaid}
  \begin{tabular}{c c c c c}
    $\sqrt{1369}=37$ & $\sqrt{289}=17$ & $\sqrt{37,249}=193$ & $73^2=5329$ \\
    $\sqrt{21,025}=145$ & $127^2=16129$ & $263^2=69169$ & $47^2=2209$ & $161^2=25921$ 
  \end{tabular}
\begin{calcaid}

\begin{solution}[\fullpage]
	If $R_A$, $R_B$ be the rates (unit volume/minute) at which $A$ and $B$ fill individually
	and $T_A$, $T_B$ the times they take to fill a tank (individually), then we know
	\begin{align}
		T_A &= \dfrac{V}{R_A} \implies R_A = \dfrac{V}{T_A}\\
		T_B &= \dfrac{V}{R_B} \implies R_B = \dfrac{V}{T_B}\\
		T_A &= T_B + \vd\text{ minutes} \\
		\dfrac\q\p\text{ minutes} &= \dfrac{V}{R_A + R_B} \\
		\implies\dfrac\q\p &= \dfrac{V}{V\cdot\left(\dfrac{1}{T_B + \vd} + \dfrac{1}{T_B}\right)} \\
		\implies \p T_B^2 - \r T_B - \s &= 0\implies T_B = \vx,\vb
	\end{align}
	
	Ignoring $T_B = \vx$, we get 
  \[ T_B = \vb\text{ minutes} \implies T_A = T_B + \vd = \va\text{ minutes} \]

  \textbf{If you solved for $T_A$ first}, then make sure that 
  \[T_B = T_A-\vd > 0\]
\end{solution}

\ifprintanswers\begin{codex}$T_A=\va\text{ minutes}$\end{codex}\fi
