

\question[4] Two pipes - $A$ and $B$ - when used together can fill a tank in 6 minutes.
Individually, however, pipe $A$ takes 5 minutes longer than pipe $B$ to fill the same
tank. Find the times then that each pipe takes individually


\ifprintanswers
  % stuff to be shown only in the answer key - like explanatory margin figures
\fi 

\begin{solution}[\fullpage]
	If $R_A$ and $R_B$ be the rates - expressed in unit volume/minute - at which the two
	operate individually and $T_A$ and $T_B$ the times they take to fill the tank, then
	where is what we know
	\begin{align}
		T_A &= \dfrac{V_{tank}}{R_A} \Rightarrow R_A = \dfrac{V_{tank}}{T_A}\\
		T_B &= \dfrac{V_{tank}}{R_B} \Rightarrow R_B = \dfrac{V_{tank}}{T_B}\\
		T_A &= T_B + 5\text{ minutes} \\
		6\text{ minutes} &= \dfrac{V_{tank}}{R_A + R_B} \\
		\Rightarrow 6 &= \dfrac{V_{tank}}{V_{tank}\cdot\left(\dfrac{1}{T_B + 5} + \dfrac{1}{T_B}\right)} \\
		\Rightarrow T_B^2 - 7T_B - 30 &= 0 \\
		\Rightarrow T_B &= -3, 10
	\end{align}
	
	Ignoring $T_B = -3$, we get $T_B = 10\text{ minutes}$ and $T_A = T_B + 5 = 15\text{ minutes}$
\end{solution}
