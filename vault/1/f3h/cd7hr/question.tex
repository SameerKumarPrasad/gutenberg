% This is an empty shell file placed for you by the 'examiner' script.
% You can now fill in the TeX for your question here.

% Now, down to brasstacks. ** Writing good solutions is an Art **. 
% Eventually, you will find your own style. But here are some thoughts 
% to get you started: 
%
%   1. Write the solution as if you are writing it for your favorite
%      14-17 year old to help him/her understand. Could be your nephew, 
%      your niece, a cousin perhaps or probably even you when you 
%      were that age. Just write for them.
%
%   2. Use margin-notes to "talk" to students about the critical insights
%      in the question. The tone can be - in fact, should be - informal
%
%   3. Don't shy away from creating margin-figures you think will help
%      students understand. Yes, it is a little more work per question. 
%      But the question & solution will be written only once. Make that
%      attempt at writing a solution count.
%
%   4. At the same time, do not be too verbose. A long solution can
%      - at first sight - make the student think, "God, that is a lot to know".
%      Our aim is not to scare students. Rather, our aim should be to 
%      create many "Aha!" moments everyday in classrooms around the world
% 
%   5. Ensure that there are *no spelling mistakes anywhere*. We are an 
%      education company. Bad spellings suggest that we ourselves 
%      don't have any education. Also, use American spellings by default
% 
%   6. If a question has multiple parts, then first delete lines 40-41
%   7. If a question does not have parts, then first delete lines 43-69

\question[5] Two people started simultaneously towards each other from points $A$ and $B$ which 
are 50 km apart. They met 5 hours later. After their meeting, the first person, who travelled 
from $A$ to $B$ decreased his speed by 1km per hour on the way back. The second person, 
on the other hand, increased his speed by 1 km per hour on the way back. The first person 
is known to arrive back at $A$ 2 hours \textit{after} the second person arrives back at $B$. What is 
the first person's \textit{initial} speed?

\insertQR{QRC}

\ifprintanswers
  % stuff to be shown only in the answer key - like explanatory margin figures
\fi 

\begin{solution}[\fullpage]
	If $s_1$ and $s_2$ be the speeds of the first and second person respectively and they
	meet at some point $X$ in the middle, then 
	
	\begin{align}
		\dfrac{AX}{s_1} &= \dfrac{BX}{s_2} = 5\text{ hours} \\
		\Rightarrow AX &= 5\cdot s_1 \\
		\text{and, } \dfrac{\overbrace{50-AX}^{BX}}{s_2} = \dfrac{50-5\cdot s_1}{s_2} &= 5 \Rightarrow s_1 + s_2 = 10
	\end{align}
	
	On the way back, 
	\begin{align}
		\dfrac{AX}{s_1 - 1} - \dfrac{BX}{s_2 + 1} &= 2\text{ hours} \\
		\Rightarrow \dfrac{5s_1}{s_1 - 1} - \dfrac{50 - 5s_1}{s_2} &= 2 \\
		\Rightarrow \dfrac{5s_1}{s_1 - 1} - \dfrac{50 - 5s_1}{10-s_1+1} &= 2 \\
		(55s_1 - 5s_1^2) - 5\cdot(10-s_1)\cdot(s_1-1) &= 2\cdot(s_1-1)\cdot(11-s_1)
	\end{align}
	
	The above reduces to 
	\begin{align}
		2s_1^2 - 24s_1 + 72 &= 0 \\
		\text{or, }s_1^2 -12s_1 + 36 &= 0 \\
		\Rightarrow s_1 &= \dfrac{12\pm\sqrt{12^2 - 4\cdot 1 \cdot 36}}{2} \\
		                &= 6\text{ km per hour}
	\end{align}
	
\end{solution}

