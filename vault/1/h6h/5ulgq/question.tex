% This is an empty shell file placed for you by the 'examiner' script.
% You can now fill in the TeX for your question here.

% Now, down to brasstacks. ** Writing good solutions is an Art **. 
% Eventually, you will find your own style. But here are some thoughts 
% to get you started: 
%
%   1. Write the solution as if you are writing it for your favorite
%      14-17 year old to help him/her understand. Could be your nephew, 
%      your niece, a cousin perhaps or probably even you when you 
%      were that age. Just write for them.
%
%   2. Use margin-notes to "talk" to students about the critical insights
%      in the question. The tone can be - in fact, should be - informal
%
%   3. Don't shy away from creating margin-figures you think will help
%      students understand. Yes, it is a little more work per question. 
%      But the question & solution will be written only once. Make that
%      attempt at writing a solution count.
%
%   4. At the same time, do not be too verbose. A long solution can
%      - at first sight - make the student think, "God, that is a lot to know".
%      Our aim is not to scare students. Rather, our aim should be to 
%      create many "Aha!" moments everyday in classrooms around the world
% 
%   5. Ensure that there are *no spelling mistakes anywhere*. We are an 
%      education company. Bad spellings suggest that we ourselves 
%      don't have any education. Also, use American spellings by default
% 
%   6. If a question has multiple parts, then first delete lines 40-41
%   7. If a question does not have parts, then first delete lines 43-69

\question[4] Three numbers in an A.P add upto $15$. If, however, $1$ is added to the first number,
$4$ to the second and $19$ to the third, then the results are in geometric progression. What are 
the original three numbers?

\insertQR{QRC}

\ifprintanswers
  % stuff to be shown only in the answer key - like explanatory margin figures

\fi 

\begin{solution}[\halfpage]
	Let the three numbers be $a-d$, $a$ and $a+d$. And so,
	\begin{align}
		(a-d) + a + (a+d) &= 15 \Rightarrow a = 5
	\end{align}
	Hence, the 3 numbers are $5-d$, $5$ and $5+d$. Adding 1,4 and 19 to them respectively therefore gives us
	$6-d$, $9$ and $24+d$ - which are in geometric progression
	\begin{align}
		\Rightarrow 6-d &= \dfrac{9}{r} \\
		24+d &= 9r \\
		\Rightarrow (6-d) + (24+d) &= 30 = 9\cdot\left( r + \dfrac{1}{r}\right) \\
		\Rightarrow 3r^2 - 10r + 1 &= 0 \text{ or } r = 3, \dfrac{1}{3} \\
		\text{ If } r = 3, \text{ then } 24 + d &= 27 \Rightarrow d = 3 \\
		\text{ If } r = \frac{1}{3}, \text{ then } 24 + d &= 3 \Rightarrow d = -21
	\end{align}
	
	Which means, the original terms are either $(2,5,8) \text{ if } d = 3$ or $(26,5,-16) \text{ if } d = \frac{1}{3}$.	
\end{solution}
