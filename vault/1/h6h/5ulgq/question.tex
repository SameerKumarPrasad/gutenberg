

\question[4] Three numbers in an A.P add upto $15$. If, however, $1$ is added to the first number,
$4$ to the second and $19$ to the third, then the results are in geometric progression. What are 
the original three numbers?


\ifprintanswers
  % stuff to be shown only in the answer key - like explanatory margin figures

\fi 

\begin{solution}[\halfpage]
	Let the three numbers be $a-d$, $a$ and $a+d$. And so,
	\begin{align}
		(a-d) + a + (a+d) &= 15 \Rightarrow a = 5
	\end{align}
	Hence, the 3 numbers are $5-d$, $5$ and $5+d$. Adding 1,4 and 19 to them respectively therefore gives us
	$6-d$, $9$ and $24+d$ - which are in geometric progression
	\begin{align}
		\Rightarrow 6-d &= \dfrac{9}{r} \\
		24+d &= 9r \\
		\Rightarrow (6-d) + (24+d) &= 30 = 9\cdot\left( r + \dfrac{1}{r}\right) \\
		\Rightarrow 3r^2 - 10r + 1 &= 0 \text{ or } r = 3, \dfrac{1}{3} \\
		\text{ If } r = 3, \text{ then } 24 + d &= 27 \Rightarrow d = 3 \\
		\text{ If } r = \frac{1}{3}, \text{ then } 24 + d &= 3 \Rightarrow d = -21
	\end{align}
	
	Which means, the original terms are either $(2,5,8) \text{ if } d = 3$ or $(26,5,-16) \text{ if } d = \frac{1}{3}$.	
\end{solution}
