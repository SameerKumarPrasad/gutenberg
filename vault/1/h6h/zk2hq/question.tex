% This is an empty shell file placed for you by the 'examiner' script.
% You can now fill in the TeX for your question here.

% Now, down to brasstacks. ** Writing good solutions is an Art **. 
% Eventually, you will find your own style. But here are some thoughts 
% to get you started: 
%
%   1. Write the solution as if you are writing it for your favorite
%      14-17 year old to help him/her understand. Could be your nephew, 
%      your niece, a cousin perhaps or probably even you when you 
%      were that age. Just write for them.
%
%   2. Use margin-notes to "talk" to students about the critical insights
%      in the question. The tone can be - in fact, should be - informal
%
%   3. Don't shy away from creating margin-figures you think will help
%      students understand. Yes, it is a little more work per question. 
%      But the question & solution will be written only once. Make that
%      attempt at writing a solution count.
%
%   4. At the same time, do not be too verbose. A long solution can
%      - at first sight - make the student think, "God, that is a lot to know".
%      Our aim is not to scare students. Rather, our aim should be to 
%      create many "Aha!" moments everyday in classrooms around the world
% 
%   5. Ensure that there are *no spelling mistakes anywhere*. We are an 
%      education company. Bad spellings suggest that we ourselves 
%      don't have any education. Also, use American spellings by default
% 
%   6. If a question has multiple parts, then first delete lines 40-41
%   7. If a question does not have parts, then first delete lines 43-69

%\setcounter{rolldice}{10}
%\noprintanswers

\ifnumequal{\value{rolldice}}{0}{
  % variables 
  \renewcommand{\vbone}{4}
  \renewcommand{\vbtwo}{3}
  \renewcommand{\vbthree}{0.09}
  \renewcommand{\vbfour}{0.04}
}{
	\ifnumequal{\value{rolldice}}{1}{
		\renewcommand{\vbone}{3}
    \renewcommand{\vbtwo}{2}
  	\renewcommand{\vbthree}{0.12}
  	\renewcommand{\vbfour}{0.10}
	}{
	  \ifnumequal{\value{rolldice}}{2}{
      \renewcommand{\vbone}{5}
      \renewcommand{\vbtwo}{3}
      \renewcommand{\vbthree}{0.07}
      \renewcommand{\vbfour}{0.08}
	  }{
      \renewcommand{\vbone}{4}
      \renewcommand{\vbtwo}{5}
      \renewcommand{\vbthree}{0.06}
      \renewcommand{\vbfour}{0.08}
	  }
	}
}

\gcalcexpr{\vbfive}{\vbone + \vbthree}
\gcalcexpr{\vbsix}{\vbfive + \vbtwo + \vbfour}
\gcalcexpr{\vbseven}{2*(\vbone + \vbthree) - (\vbtwo + \vbfour)}
\gcalcexpr{\vbeight}{\vbtwo + \vbfour}
\gcalcexpr{\vbnine}{\vbsix + \vbtwo + \vbfour}
\gcalcexpr{\vbten}{\vbnine + \vbtwo + \vbfour}

\question[4] Find an expression for the sum of the first $n$ terms of the series 
$\vbfive + \vbsix + \vbnine + \vbten \ldots$

\watchout
\insertQR[-15pt]{QRC}

\ifprintanswers
\fi 

\begin{solution}[\halfpage]
  If you split each term in the series into its integer and fractional parts - for example - 
  $\vbfive = \vbone + \vbthree$, then you will notice that the $n^{th}$ term of the series
  is of the form $a_n = \underbrace{\vbone + (n-1)\cdot\vbtwo}_{\text{one A.P}} + 
  \underbrace{\vbthree + (n-1)\cdot\vbfour}_{\text{second A.P}} $
	
	The sum of the first $n$ terms therefore is
	\begin{align}
		S_n &= \sum_{k=1}^{n}(\vbone + (n-1)\cdot\vbtwo) + \sum_{k=1}^{n}(\vbthree + (n-1)\cdot\vbfour) \\
		&= \underbrace{\eSumOfAP[1]{n}}_{a_1=\vbone, d_1=\vbtwo} + \underbrace{\eSumOfAP[2]{n}}_{a_2=\vbthree, d_2=\vbfour} \\
		&= \dfrac{n}{2}\cdot\left[ \lbrace 2\cdot(\vbone + \vbthree) - (\vbtwo + \vbfour)\rbrace + n\cdot(\vbtwo + \vbfour) \right] \\
		&= \dfrac{n}{2}\cdot\left( \vbseven + \vbeight\cdot n\right)
	\end{align}
\end{solution}
