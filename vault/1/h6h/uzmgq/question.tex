

\question[3] If the $m^{th}$ term of an A.P is $\frac{1}{n}$ and the $n^{th}$ term is $\frac{1}{m}$, 
then prove that the sum of the first $mn$ terms is $\dfrac{1+mn}{2}$, given that $m \neq n$


\ifprintanswers
\fi 

\begin{solution}[\halfpage]
	\begin{align}
		\fTermOfAP{m} &= \frac{1}{n} \\
		\fTermOfAP{n} &= \frac{1}{m} \\
		\Rightarrow T_m - T_n = (m-n)\cdot d &= \frac{1}{n}-\frac{1}{m} = \dfrac{m-n}{mn} \\
		\Rightarrow d &= \dfrac{1}{mn}
	\end{align}
	And so,
	\begin{align}
		\fTermOfAP{m} &= a + (m-1)\cdot\dfrac{1}{mn} = \dfrac{1}{n} \\
		\Rightarrow amn + m - 1 &= mn \text{ or } a = \dfrac{1}{mn}
	\end{align}
	The sum of the first $mn$ terms therefore is,
	\begin{align}
		&\fSumOfAP{mn} = \dfrac{mn}{2}\cdot\left[ \dfrac{2}{mn} + (mn-1)\cdot\dfrac{1}{mn}\right] \\
		&= \dfrac{mn}{2}\cdot\left[ \dfrac{2 + mn - 1}{mn} \right] = \dfrac{1+mn}{2} 
	\end{align}
\end{solution}
