

\question[4] The sixth term of an arithmetic progression is 3 and the common difference is greater than 0.5. 
At what value of common difference is the \textit{product} of the \textit{first}, \textit{fourth} and \textit{fifth}
terms the greatest?


\ifprintanswers
\fi 

\begin{solution}[\fullpage]
	If the sixth term is 3 and the common difference is - say - $b$, then 
	\begin{align}
		\text{ Fifth term } &= 3-b \\
		\text{ Fourth term } &= 3-2b \\
		\text{ First term } &= 3-5b \\
		\Rightarrow \text{Product} = P &= (3-b)\cdot(3-2b)\cdot(3-5b) \\
		&= (9-21b+10b^2)\cdot(3-b) = (27-72b+51b^2-10b^3)
	\end{align}
	
	Now its a simple case of finding the maxima of $P$
	\begin{align}
		\dfrac{\ud P}{\ud b} &= -72 + 102b - 30b^2 = 0 \\
		\Rightarrow 10b^2 - 34b + 24 &= 0 \text{ or } (b-1)\cdot(10b-24) = 0 \\
		\Rightarrow b = 1,\, \frac{24}{10} = \frac{12}{5}
	\end{align}
	Also, note that for a maxima
	\begin{align}
		\dfrac{\ud^2 P}{\ud b^2} &= 102 - 60b < 0 \\
		\Rightarrow b > \frac{102}{60} = \frac{17}{10}
	\end{align}
	As $b = 1$ does \textit{not} satisfy the condition for maxima, $b = \dfrac{12}{5}$ is 
	the only acceptable solution
\end{solution}
