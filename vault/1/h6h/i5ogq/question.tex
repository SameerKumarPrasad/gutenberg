% This is an empty shell file placed for you by the 'examiner' script.
% You can now fill in the TeX for your question here.

% Now, down to brasstacks. ** Writing good solutions is an Art **. 
% Eventually, you will find your own style. But here are some thoughts 
% to get you started: 
%
%   1. Write the solution as if you are writing it for your favorite
%      14-17 year old to help him/her understand. Could be your nephew, 
%      your niece, a cousin perhaps or probably even you when you 
%      were that age. Just write for them.
%
%   2. Use margin-notes to "talk" to students about the critical insights
%      in the question. The tone can be - in fact, should be - informal
%
%   3. Don't shy away from creating margin-figures you think will help
%      students understand. Yes, it is a little more work per question. 
%      But the question & solution will be written only once. Make that
%      attempt at writing a solution count.
%
%   4. At the same time, do not be too verbose. A long solution can
%      - at first sight - make the student think, "God, that is a lot to know".
%      Our aim is not to scare students. Rather, our aim should be to 
%      create many "Aha!" moments everyday in classrooms around the world
% 
%   5. Ensure that there are *no spelling mistakes anywhere*. We are an 
%      education company. Bad spellings suggest that we ourselves 
%      don't have any education. Also, use American spellings by default
% 
%   6. If a question has multiple parts, then first delete lines 40-41
%   7. If a question does not have parts, then first delete lines 43-69

\question[4] The sixth term of an arithmetic progression is 3 and the common difference is greater than 0.5. 
At what value of common difference is the \textit{product} of the \textit{first}, \textit{fourth} and \textit{fifth}
terms the greatest?

\insertQR{QRC}

\ifprintanswers
\fi 

\begin{solution}[\fullpage]
	If the sixth term is 3 and the common difference is - say - $b$, then 
	\begin{align}
		\text{ Fifth term } &= 3-b \\
		\text{ Fourth term } &= 3-2b \\
		\text{ First term } &= 3-5b \\
		\Rightarrow \text{Product} = P &= (3-b)\cdot(3-2b)\cdot(3-5b) \\
		&= (9-21b+10b^2)\cdot(3-b) = (27-72b+51b^2-10b^3)
	\end{align}
	
	Now its a simple case of finding the maxima of $P$
	\begin{align}
		\dfrac{\ud P}{\ud b} &= -72 + 102b - 30b^2 = 0 \\
		\Rightarrow 10b^2 - 34b + 24 &= 0 \text{ or } (b-1)\cdot(10b-24) = 0 \\
		\Rightarrow b = 1,\, \frac{24}{10} = \frac{12}{5}
	\end{align}
	Also, note that for a maxima
	\begin{align}
		\dfrac{\ud^2 P}{\ud b^2} &= 102 - 60b < 0 \\
		\Rightarrow b > \frac{102}{60} = \frac{17}{10}
	\end{align}
	As $b = 1$ does \textit{not} satisfy the condition for maxima, $b = \dfrac{12}{5}$ is 
	the only acceptable solution
\end{solution}
