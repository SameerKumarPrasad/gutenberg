

\ifnumequal{\value{rolldice}}{0}{
  % variables 
  \renewcommand{\vbone}{13}
  \renewcommand{\vbtwo}{7}
  \renewcommand{\vbthree}{4}
  \renewcommand{\vbfour}{11}
}{
  \ifnumequal{\value{rolldice}}{1}{
    % variables 
    \renewcommand{\vbone}{4}
    \renewcommand{\vbtwo}{9}
    \renewcommand{\vbthree}{17}
    \renewcommand{\vbfour}{3}
  }{
    \ifnumequal{\value{rolldice}}{2}{
      % variables 
      \renewcommand{\vbone}{5}
      \renewcommand{\vbtwo}{8}
      \renewcommand{\vbthree}{10}
      \renewcommand{\vbfour}{17}
    }{
      % variables 
      \renewcommand{\vbone}{7}
      \renewcommand{\vbtwo}{17}
      \renewcommand{\vbthree}{3}
      \renewcommand{\vbfour}{11}
    }
  }
}

\FRACMULT\vbthree\vbfour{2}{1}\a\b
\FRACADD\vbone\vbtwo\vbthree\vbfour\c\d

\question[2] If the sum of the first $n$ terms of an arithmetic progression is 
$\dfrac{\vbone}{\vbtwo} n + \dfrac{\vbthree}{\vbfour} n^2$, then what is the common difference? 
\watchout
\insertQR{QRC}

\ifprintanswers
\fi 

\begin{solution}[\mcq]
  If $S_n$ be the sum of the first $n$ terms, then the $n^{th} term = a_n$ is simply 
  $S_{n+1} - S_n$
	\begin{align}
	  a_n &= \left[\dfrac{\vbone}{\vbtwo}\cdot(n+1) + \dfrac{\vbthree}{\vbfour}\cdot (n+1)^2 \right] - 
	    \left( \dfrac{\vbone}{\vbtwo} n + \dfrac{\vbthree}{\vbfour} n^2\right) \\
	    &= \left( \dfrac{\vbone}{\vbtwo} + \dfrac{\vbthree}{\vbfour}\right)
	       + 2\cdot\dfrac{\vbthree}{\vbfour} n \\
	     &= \WRITEFRAC\c\d + \WRITEFRAC\a\b n = a + n\cdot d
	\end{align}
	On comparing the terms in the last equation, we can see that 
	\begin{align}
	  a &= \WRITEFRAC\c\d \text{ and, } \\
	  d &= \WRITEFRAC\a\b
	\end{align}
\end{solution}
