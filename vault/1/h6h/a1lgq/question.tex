

\ifnumequal{\value{rolldice}}{0}{
  % variables 
  \renewcommand{\va}{13}
  \renewcommand{\vb}{7}
  \renewcommand{\vc}{4}
  \renewcommand{\vd}{11}
}{
  \ifnumequal{\value{rolldice}}{1}{
    % variables 
    \renewcommand{\va}{4}
    \renewcommand{\vb}{9}
    \renewcommand{\vc}{17}
    \renewcommand{\vd}{3}
  }{
    \ifnumequal{\value{rolldice}}{2}{
      % variables 
      \renewcommand{\va}{5}
      \renewcommand{\vb}{8}
      \renewcommand{\vc}{10}
      \renewcommand{\vd}{17}
    }{
      % variables 
      \renewcommand{\va}{7}
      \renewcommand{\vb}{17}
      \renewcommand{\vc}{3}
      \renewcommand{\vd}{11}
    }
  }
}

\FRACMULT\vc\vd{2}{1}\a\b
\FRACADD\va\vb\vc\vd\c\d

\question[2] If the sum of the first $n$ terms of an arithmetic progression is given by  
\[ S(n) = \dfrac{\va}{\vb} n + \dfrac{\vc}{\vd} n^2 \] 
then what is the common difference? 

\watchout

\ifprintanswers
\fi 

\begin{solution}[\mcq]
  If $S_n$ be the sum of the first $n$ terms, then the $n^{th} term = a_n$ is simply 
  $S_{n+1} - S_n$
	\begin{align}
	  a_n &= \left[\dfrac{\va}{\vb}\cdot(n+1) + \dfrac{\vc}{\vd}\cdot (n+1)^2 \right] - 
	    \left( \dfrac{\va}{\vb} n + \dfrac{\vc}{\vd} n^2\right) \\
	    &= \left( \dfrac{\va}{\vb} + \dfrac{\vc}{\vd}\right)
	       + 2\cdot\dfrac{\vc}{\vd} n \\
	     &= \WRITEFRAC\c\d + \WRITEFRAC\a\b n = a + n\cdot d
	\end{align}
	On comparing the terms in the last equation, we can see that 
	\begin{align}
	  a &= \WRITEFRAC\c\d \text{ and } d = \WRITEFRAC\a\b
	\end{align}
\end{solution}

\ifprintanswers\begin{codex}$\WRITEFRAC\a\b$\end{codex}\fi
