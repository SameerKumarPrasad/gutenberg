
\ifnumequal{\value{rolldice}}{0}{
  \renewcommand\va{5}
  \renewcommand\vb{2}
  \renewcommand\ve{1,2}
}{
  \ifnumequal{\value{rolldice}}{1}{
    \renewcommand\va{7}
    \renewcommand\vb{3}
    \renewcommand\ve{1,2}
  }{
    \ifnumequal{\value{rolldice}}{2}{
      \renewcommand\va{9}
      \renewcommand\vb{2}
      \renewcommand\ve{1,2,3,4}
    }{
      \renewcommand\va{11}
      \renewcommand\vb{3}
      \renewcommand\ve{1,2,3}
    }
  }
}

\question For the following function$\ldots$
\[
  f:\mathbb{R}\rightarrow\mathbb{R} = \sqrt{\va x -\vb x^2}
\]

\watchout

\begin{parts}
  \part[2] $\ldots$ find the domain 

  \insertQR{}
\begin{solution}[\mcq]
    For $f(x) = \sqrt{\va x -\vb x^2} = \sqrt{x\cdot(\va - \vb x)}\in\mathbb{R}$ (in other words, for $f(x)$ 
    to be a real number), $x\cdot(\va - \vb x)\geq 0$. 

    Which means, that either 
    \begin{align}
      &x\geq 0\textbf{ and }(\va-\vb x)\geq 0 \implies x\geq 0 \text{ and } x\leq\frac\va\vb \\
      &\textbf{ OR } x\leq 0 \textbf{ and } (\va-\vb x)\leq 0 \nonumber\\
      &\text{ not possible as } (\va -\vb x)\geq 0\text{ when } x \leq 0
    \end{align}

    Hence, the domain of $f(x)$ can only be $\left[ 0, \frac\va\vb \right]$
  \end{solution}

  \part[2] $\ldots$ find the domain \textbf{if} $f:\mathbb{N}\rightarrow\mathbb{R}$ \textbf{instead of} 
   $f:\mathbb{R}\rightarrow\mathbb{R}$

  \insertQR{}
\begin{solution}[\mcq]
    In part (a) $x$ in $f(x)$ was a real number. Now, we are saying that it can only be a natural number. 

    And hence, we are looking for $\left\{ x:x\in\mathbb{N}, x\geq 0, x\leq\frac\va\vb\right\}\implies x\in\left\{ \ve \right\}$. 

    The domain for $f:\mathbb{N}\rightarrow\mathbb{R}$ is therefore $\left\{ \ve \right\}$
  \end{solution}

\end{parts}

\ifprintanswers
  \begin{codex}
    $(a)\,\left[0,\dfrac\va\vb\right]\quad (b)\,\lbrace\ve\rbrace$
  \end{codex}
\fi
