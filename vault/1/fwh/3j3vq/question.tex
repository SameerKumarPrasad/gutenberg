
\ifnumequal{\value{rolldice}}{0}{
  \renewcommand\va{4}
  \renewcommand\vb{2}
  \renewcommand\vc{3}
  \renewcommand\vg{9}
  \renewcommand\vh{2}
}{
  \ifnumequal{\value{rolldice}}{1}{
    \renewcommand\va{3}
    \renewcommand\vb{7}
    \renewcommand\vc{9}
    \renewcommand\vg{3}
    \renewcommand\vh{7}
  }{
    \ifnumequal{\value{rolldice}}{2}{
      \renewcommand\va{5}
      \renewcommand\vb{3}
      \renewcommand\vc{7}
      \renewcommand\vg{7}
      \renewcommand\vh{4}
    }{
      \renewcommand\va{7}
      \renewcommand\vb{2}
      \renewcommand\vc{5}
      \renewcommand\vg{5}
      \renewcommand\vh{3}
    }
  }
}

\EXPR[0]\vd{\vb + (\va*\vc)}
\MULTIPLY\va\vb\ve
\EXPR[0]\vm{\vg + (\vh*\va)}
\MULTIPLY\va\vg\vn

\question Given $U=$ universal set = $\lbrace x : x > 2, x < 6, x\in\mathbb{Q}\rbrace$, 
$A=\lbrace x : \vc x^2 - \vd x + \ve = 0, x \in \mathbb{Q} \rbrace$ 
and $B=\lbrace x : \vh x^2 - \vm x + \vn = 0, x\in\mathbb{Q} \rbrace$, answer the following questions

\watchout

\begin{parts}
  \part[1] $\mathbb{Q}$ is the set of what - integers, real numbers, natural numbers or rational numbers?

\begin{solution}[\mcq]
    Rational numbers. The $\mathbb{Q}$ comes from the Italian word \textit{quoziente} - which means quotient
  \end{solution}

  \part[2] $A\cup B=?$

\begin{solution}[\halfpage]
    If we solve the two quadratic equations given in the question, then we get 
    \begin{align}
      \vc x^2-\vd x +\ve = 0 &\implies (x-\va)\cdot(x-\frac\vb\vc)= 0 \nonumber\\
      &\implies x = \va,\frac\vb\vc \\
      \vh x^2-\vm x +\vn = 0 &\implies (x-\va)\cdot(x-\frac\vg\vh)= 0 \nonumber\\
      &\implies x = \va,\frac\vg\vh
    \end{align}
    All the values of $x$ we just found, namely $\va,\frac\vb\vc,\frac\vg\vh$ $\in\mathbb{Q}$. 

    Which means
    $A = \lbrace \va,\frac\vb\vc\rbrace$ and $B=\lbrace \va,\frac\vg\vh\rbrace$

    And therefore, 
    \begin{align}
      A\cup B &= \lbrace\va, \frac\vb\vc,\frac\vg\vh \rbrace
    \end{align}
  \end{solution}

  \part[1] $A\cap B=?$

\begin{solution}[\mcq]
    Simple. $A\cap B = \lbrace\va\rbrace$
  \end{solution}

  \part[1] $A-B=?$

\begin{solution}[\mcq]
    $A-B=\lbrace x : x\in A, x\notin B \rbrace\implies A-B = \lbrace\frac\vb\vc\rbrace$
  \end{solution}

  \part[1] $B\cap A'=?$

\begin{solution}[\mcq]
    $B\cap A'$ is simply another way of writing $B-A$. 

    And $B-A=\lbrace x:x\in B, x\notin A \rbrace = \lbrace\frac\vg\vh\rbrace$
  \end{solution}

\end{parts}

\ifprintanswers
  \begin{codex}
    \begin{tabular}{l l l}
      $(a)\,$ Rational numbers & $(b)\,\lbrace\va, \frac\vb\vc,\frac\vg\vh \rbrace$ & $(c)\,\lbrace\va\rbrace$ \\
      $(d)\,\lbrace\frac\vb\vc\rbrace$ & $(e)\,\lbrace\frac\vg\vh\rbrace$
    \end{tabular}
  \end{codex}
\fi
