
\ifnumequal{\value{rolldice}}{0}{
  \renewcommand\va{5}
  \renewcommand\vb{2}
  \renewcommand\vc{20}
  \renewcommand\vd{10}
  \renewcommand\ve{4}
}{
  \ifnumequal{\value{rolldice}}{1}{
    \renewcommand\va{7}
    \renewcommand\vb{3}
    \renewcommand\vc{10}
    \renewcommand\vd{14}
    \renewcommand\ve{6}
  }{
    \ifnumequal{\value{rolldice}}{2}{
      \renewcommand\va{4}
      \renewcommand\vb{14}
      \renewcommand\vc{28}
      \renewcommand\vd{16}
      \renewcommand\ve{4}
    }{
      \renewcommand\va{6}
      \renewcommand\vb{15}
      \renewcommand\vc{20}
      \renewcommand\vd{12}
      \renewcommand\ve{9}
    }
  }
}

\MULTIPLY\va\vc\vx
\DIVIDE\vx\vd\vunique
\MULTIPLY\vunique\ve\a
\DIVIDE\a\vb\vy

\question[5] Let us suppose that there is a set $S$ and that there are two ways of building it - 
either by doing a union of $\vc$ sets of type X or by union of $N$ sets of type Y.
In other words 
    \[ S = \bigcup_{i=1}^{\vc} X_i = \bigcup_{j=1}^N Y_j\] 
This can be done \textbf{only if} for every $x\in X_i$ (for some $i$), it is also true that $x\in Y_j$ 
(for some $j$) \textbf{and vice-versa}

Now, if each element of $S$ belongs to exactly $\vd$ of the $X_i$s and to exactly 
$\ve$ of the $Y_j$s, then what is $N=?$ 

\watchout[-4cm]

\ifprintanswers
\fi 

\begin{solution}[\fullpage]
  If there were no common elements between any of the $X_R$s - that is - 
  if $X_i\cap X_j = \phi,i\neq j$ for all pairs $(i,j)$, then $S$ would have 
  $=\vc\times\va = \vx$ elements

  But, we have been told that each element in $S$ belongs to exactly $\vd$ of the $X_R$s.
  Which means, $S$ has $ =\dfrac\vx\vd = \vunique$ \textbf{unique} elements

  Similarly, if $S$ is built from the $Y_R$s, then it would have $=\dfrac{\vb\times N}{\ve}$ 
  unique elements 

  But because it is the same set $S$, 
  \begin{align}
    \dfrac{\vb\times N}{\ve} &= \vunique \\
    \implies N &= \vy
  \end{align}
\end{solution}

