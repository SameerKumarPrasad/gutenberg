
\ifnumequal{\value{rolldice}}{0}{
  % variables 
  \renewcommand{\va}{90}
  \renewcommand{\vb}{3}
}{
  \ifnumequal{\value{rolldice}}{1}{
    % variables 
    \renewcommand{\va}{60}
    \renewcommand{\vb}{4}
  }{
    \ifnumequal{\value{rolldice}}{2}{
      % variables 
      \renewcommand{\va}{110}
      \renewcommand{\vb}{3}
    }{
      % variables 
      \renewcommand{\va}{80}
      \renewcommand{\vb}{4}
    }
  }
}

\MULTIPLY\va\vb\vc
\FRACTIONSIMPLIFY\vc{875}\vx\vy
\FRACMULT\vx\vy{99}{1}\vp\vq

\question[2] The wheel of a railway carriage is $\va$ cm in diameter and makes $\vb$ revolutions 
per second. How fast is the train going - in kilometers per hour? Use $\pi=\frac{22}{7}$.

\watchout

\begin{solution}[\mcq]
  With every revolution of the wheel, the train moves forward a distance equal to the wheel's \textbf{circumference}.

  Which means, the train's speed is 
  \[ S = \vb\dfrac{\text{rev}}{\text{sec}} \times\va\pi\dfrac{\text{cm}}{\text{rev}} = \vc\pi\dfrac{\text{cm}}{\text{sec}} \]

  Expressed in \textbf{kilometers per hour}, that is 
  \begin{align}
    S &= \vc\cdot\dfrac{22}{7}\dfrac{\text{cm}}{\text{sec}}\times 3600\dfrac{\text{sec}}{\text{hour}}\times\dfrac{1}{10^5}\dfrac{\text{km}}{\text{cm}}
      = \WRITEFRAC\vp\vq\text{ km/hour}
  \end{align} 


\end{solution}
\ifprintanswers
  \begin{codex}
    $\WRITEFRAC\vp\vq\text{ km/hour}$
  \end{codex}
\fi
