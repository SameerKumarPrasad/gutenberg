% This is an empty shell file placed for you by the 'examiner' script.
% You can now fill in the TeX for your question here.

% Now, down to brasstacks. ** Writing good solutions is an Art **. 
% Eventually, you will find your own style. But here are some thoughts 
% to get you started: 
%
%   1. Write the solution as if you are writing it for your favorite
%      14-17 year old to help him/her understand. Could be your nephew, 
%      your niece, a cousin perhaps or probably even you when you 
%      were that age. Just write for them.
%
%   2. Use margin-notes to "talk" to students about the critical insights
%      in the question. The tone can be - in fact, should be - informal
%
%   3. Don't shy away from creating margin-figures you think will help
%      students understand. Yes, it is a little more work per question. 
%      But the question & solution will be written only once. Make that
%      attempt at writing a solution count.
%
%   4. At the same time, do not be too verbose. A long solution can
%      - at first sight - make the student think, "God, that is a lot to know".
%      Our aim is not to scare students. Rather, our aim should be to 
%      create many "Aha!" moments everyday in classrooms around the world
% 
%   5. Ensure that there are *no spelling mistakes anywhere*. We are an 
%      education company. Bad spellings suggest that we ourselves 
%      don't have any education. Also, use American spellings by default
% 
%   6. If a question has multiple parts, then first delete lines 40-41
%   7. If a question does not have parts, then first delete lines 43-69
\question[4] Prove that $\sin^4\dfrac{\pi}{8} + \sin^4\dfrac{3\pi}{8} + \sin^4\dfrac{5\pi}{8} + \sin^4\dfrac{7\pi}{8} = \dfrac{3}{2}$

\insertQR[-30pt]{QRC}

\ifprintanswers
\fi 

\begin{solution}[\fullpage]
  \begin{fullwidth}
    Using the fact that $\sin^2\theta = \dfrac{1-\cos 2\theta}{2}$,  we can re-write the above as
    \begin{align}
       &\dfrac{1}{2^2}\cdot\left[\left( 1-\cos\dfrac{\pi}{4}\right)^2 + 
       \left( 1-\cos\dfrac{3\pi}{4}\right)^2
       + \left( 1-\cos\dfrac{5\pi}{4}\right)^2 + 
       \left( 1-\cos\dfrac{7\pi}{4}\right)^2\right] \\
       &= \dfrac{1}{4}\cdot\left( 1 + \cos^2\dfrac{\pi}{4} - 2\cos\dfrac{\pi}{4}\right) +
          \dfrac{1}{4}\left( 1 + \cos^2\dfrac{3\pi}{4} - 2\cos\dfrac{3\pi}{4}\right) + \nonumber \\
       & \dfrac{1}{4}\left( 1 + \cos^2\dfrac{5\pi}{4} - 2\cos\dfrac{5\pi}{4}\right) + 
          \dfrac{1}{4}\left( 1 + \cos^2\dfrac{7\pi}{4} - 2\cos\dfrac{7\pi}{4}\right) 
    \end{align}
    Now, $\cos(n\pi \pm \theta) = (-1)^{n}\cos\theta$. And so, if you noticed that,
    \begin{align}
    	\dfrac{3\pi}{4} &= \pi - \frac{\pi}{4} \\
    	\dfrac{5\pi}{4} &= \pi + \frac{\pi}{4} \\
    	\dfrac{7\pi}{4} &= 2\pi - \frac{\pi}{4}
    \end{align}
    , then the expression can be re-written as 
    \begin{align}
       &= \dfrac{1}{4}\cdot\left( 1 + \cos^2\dfrac{\pi}{4} - 2\cos\dfrac{\pi}{4}\right) +
          \dfrac{1}{4}\left( 1 + \cos^2\dfrac{\pi}{4} + 2\cos\dfrac{\pi}{4}\right) + \nonumber \\
       & \dfrac{1}{4}\left( 1 + \cos^2\dfrac{\pi}{4} + 2\cos\dfrac{\pi}{4}\right) + 
          \dfrac{1}{4}\left( 1 + \cos^2\dfrac{\pi}{4} - 2\cos\dfrac{\pi}{4}\right) \\
       &= \dfrac{1}{4}\cdot\left[ 4 + 4\cdot\dfrac{1}{(\sqrt{2})^2}\right] = \dfrac{3}{2} 
    \end{align}
  \end{fullwidth}
\end{solution}
