

\question[3] The angles of a triangle are in arithmetic progression and the ratio of 
the number of \textbf{radians} in the \textbf{smallest} angle to the number of 
\textbf{degrees} of the \textbf{mean} angle is $\frac{1}{120}$. Find the angles 
of the triangle - in radians 


\ifprintanswers
\fi 

\begin{solution}[\halfpage]
   Let the three angles be $A=a$, $B=a+d$ and $C=a+2d$ - either all in degrees or all in radians. 

   Moreover, let $A_R$ be the \textbf{number of radians} in $\angle A$ and $B_D$ the
   \textbf{number of degrees} in (mean) $\angle B$
   
   And so, 
   \begin{align}
       \dfrac{a}{a+d} &= \overbrace{\dfrac{A_R}{\frac{\pi}{180}\cdot B_D}}^{\texttt{everything in radians}} \\
       \text{where } \dfrac{A_R}{B_D} &= \dfrac{1}{120} \\
       \implies \dfrac{a}{a+d} &= \dfrac{180}{\pi}\cdot\dfrac{1}{120} = \dfrac{3}{2\pi} \\
       \implies d &= \left( \dfrac{2\pi}{3}-1\right)\cdot a 
   \end{align}
   The three angles, therefore, are 
   \begin{align}
      A &= a \\
      B = a + d &= a\cdot\left(1 + \dfrac{2\pi}{3} - 1 \right) = \dfrac{2\pi}{3}a\\
      C = a + 2d &= a\cdot\left( 1 + \dfrac{4\pi}{3} - 2\right) = \left( \dfrac{4\pi}{3} - 1\right)\cdot a
   \end{align} 
  
    Moreover, $A+B+C=\pi\text{ (angles in a triangle) }$. And therefore,
   \begin{align}
      a\cdot\left[1 + \dfrac{2\pi}{3} + \dfrac{4\pi}{3} - 1 \right] &= \pi\implies 2\pi\cdot a = \pi \implies A = a=\dfrac{1}{2}\\
      \therefore B &= \dfrac{2\pi}{3}a = \dfrac{\pi}{3} \\
      \text{and }C &= 
      \left(\dfrac{4\pi}{3} - 1 \right)\cdot a = 
      \left(\dfrac{2\pi}{3} - \dfrac{1}{2} \right)
   \end{align}
\end{solution}
\ifprintanswers
  \begin{codex}
    $\dfrac{1}{2},\,\dfrac\pi{3}\text{ and }\left(\dfrac{2\pi}{3}-\dfrac{1}{2} \right)$
  \end{codex}
\fi
