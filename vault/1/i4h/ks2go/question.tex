% This is an empty shell file placed for you by the 'examiner' script.
% You can now fill in the TeX for your question here.

% Now, down to brasstacks. ** Writing good solutions is an Art **. 
% Eventually, you will find your own style. But here are some thoughts 
% to get you started: 
%
%   1. Write the solution as if you are writing it for your favorite
%      14-17 year old to help him/her understand. Could be your nephew, 
%      your niece, a cousin perhaps or probably even you when you 
%      were that age. Just write for them.
%
%   2. Use margin-notes to "talk" to students about the critical insights
%      in the question. The tone can be - in fact, should be - informal
%
%   3. Don't shy away from creating margin-figures you think will help
%      students understand. Yes, it is a little more work per question. 
%      But the question & solution will be written only once. Make that
%      attempt at writing a solution count.
%
%   4. At the same time, do not be too verbose. A long solution can
%      - at first sight - make the student think, "God, that is a lot to know".
%      Our aim is not to scare students. Rather, our aim should be to 
%      create many "Aha!" moments everyday in classrooms around the world
% 
%   5. Ensure that there are *no spelling mistakes anywhere*. We are an 
%      education company. Bad spellings suggest that we ourselves 
%      don't have any education. Also, use American spellings by default
% 
%   6. If a question has multiple parts, then first delete lines 40-41
%   7. If a question does not have parts, then first delete lines 43-69

%\noprintanswers
% \setcounter{rolldice}{3}
% \printrubric

\ifnumequal{\value{rolldice}}{0}{
  % variables 
  \renewcommand{\vbone}{52}
  \renewcommand{\vbtwo}{624}
}{
  \ifnumequal{\value{rolldice}}{1}{
    % variables 
    \renewcommand{\vbone}{38}
    \renewcommand{\vbtwo}{456}
  }{
    \ifnumequal{\value{rolldice}}{2}{
      % variables 
      \renewcommand{\vbone}{37}
      \renewcommand{\vbtwo}{444}
    }{
      % variables 
      \renewcommand{\vbone}{42}
      \renewcommand{\vbtwo}{336}
    }
  }
}

\DIVIDE\vbtwo\vbone\vbthree
\SQUARE\vbone\tp
\SUBTRACT\vbtwo\tp\tq
\FRACTIONSIMPLIFY\tq\vbtwo\tr\ts
\QUADEQNROOTS\ts\tr\ts\dnm\fnm\snm
\FRACDIV{\vbthree}{1}\fnm\dnm\anm\adnm
\FRACMULT{\vbthree}{1}\fnm\dnm\bnm\bdnm

\question[3] Find three numbers in geometric progression whose sum is $\vbone$ and the
sum of whose products - taken in pairs - is $\vbtwo$

\insertQR[-15pt]{QRC}
\watchout

\ifprintanswers
\fi 

\begin{solution}[\halfpage]
	Let the three numbers be $\dfrac{a}{r}$, $a$ and $ar$
	\begin{align}
		\dfrac{a}{r} + a + ar &= \vbone \\
		\text{And, } \left( \dfrac{a}{r}\cdot a\right) + \left( a\cdot ar\right) + 
		\left( \dfrac{a}{r}\cdot ar\right) &= \vbtwo \\
		\Rightarrow \dfrac{a^2\cdot\left( \frac{1}{r} + r + 1 \right)}{a\cdot\left( \frac{1}{r} + r + 1\right)}
		&= \dfrac{\vbtwo}{\vbone} \Rightarrow a = \vbthree \\
		\text{Hence, } \vbthree\cdot\left( \dfrac{1}{r} + r + 1 \right) &= \vbone 
    \Rightarrow \ts r^2\gsign\tr r + \ts = 0 \\
    \Rightarrow r &= \WRITEFRAC\fnm\dnm,\,\WRITEFRAC\snm\dnm
	\end{align}
	
	Hence, the three numbers are either 
    $\left( \WRITEFRAC\anm\adnm, \vbthree, \WRITEFRAC\bnm\bdnm \right)$ if $r=\WRITEFRAC\fnm\dnm$ or 
    $\left( \WRITEFRAC\bnm\bdnm, \vbthree, \WRITEFRAC\anm\adnm \right)$ if $r=\WRITEFRAC\snm\dnm$
\end{solution}

\ifprintrubric
  \begin{table}
  	\begin{tabular}{ p{5cm}p{5cm} }
  		\toprule % in brief (4-6 words), what should a grader be looking for for insights & formulations
  		  \sc{\textcolor{blue}{Insight}} & \sc{\textcolor{blue}{Formulation}} \\ 
  		\midrule % ***** Insights & formulations ******
        Divide the given sums to get first term - $a$ & Sums expressed properly \\
         & A quadratic equation formed - and then solved - properly \\ 
  		\toprule % final numerical answers for the various versions
        \sc{\textcolor{blue}{If question has $\ldots$}} & \sc{\textcolor{blue}{Final answer}} \\
  		\midrule % ***** Numerical answers (below) **********
        $(52,624)$ & $(36,12,4)$ \\
        $(38,456)$ & $(18,12,8)$ \\
        $(37,444)$ & $(16,12,9)$ \\
        $(42,336)$ & $(32,8,2)$ \\
  		\bottomrule
  	\end{tabular}
  \end{table}
\fi
