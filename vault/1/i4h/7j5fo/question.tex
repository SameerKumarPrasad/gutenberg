

\ifnumequal{\value{rolldice}}{0}{
  % variables 
  \renewcommand{\vbone}{1}
  \renewcommand{\vbtwo}{2}
  \renewcommand{\vbthree}{3}
  \renewcommand{\vbfour}{6}
  \renewcommand{\vbfive}{4}
  \renewcommand{\vbsix}{13}
}{
  \ifnumequal{\value{rolldice}}{1}{
    % variables 
  }{
    \ifnumequal{\value{rolldice}}{2}{
      % variables 
    }{
      % variables 
    }
  }
}

\SUBTRACT\vbone\vbtwo\a
\ADD\vbone\vbtwo\b
\SUBTRACT\vbthree\vbfour\c
\ADD\vbthree\vbfour\d
\SUBTRACT\vbfive\vbsix\e
\ADD\vbfive\vbsix\f

\question[4] For what value of $M$ would the following hold?
\[ \dfrac{\sin\vbone A\sin\vbtwo A + \sin\vbthree A\sin\vbfour A + \sin\vbfive A\sin\vbsix A}
{\sin\vbone A\cos\vbtwo A + \sin\vbthree A\cos\vbfour A + \sin\vbfive A\cos\vbsix A} = \tan\textbf{M}A\]

\begin{explanation}[\fullpage]
  Given that 
  \begin{align}
    \cos(x-y)-\cos(x+y) &= 2\sin x\sin y \\
    \sin(x+y) - \sin(x-y) &= 2\cos x\sin y
  \end{align}
  We can re-write given expression as 
  \begin{align}
    \dfrac{(\cos\a A - \cos\b A) + (\cos\c A - \cos\d A) + (\cos\e A -\cos\f A }
    {(\sin\b A-\sin\a A) + (\sin\d A - \sin\c A) + (\sin\f A - \sin\e A)}
  \end{align}
\end{explanation}

