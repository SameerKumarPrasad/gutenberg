
\ifnumequal{\value{rolldice}}{0}{
  % variables 
  \renewcommand{\vbone}{1}
  \renewcommand{\vbtwo}{2}
  \renewcommand{\vbthree}{3}
  \renewcommand{\vbfour}{6}
  \renewcommand{\vbfive}{4}
  \renewcommand{\vbsix}{13}
}{
  \ifnumequal{\value{rolldice}}{1}{
    % variables 
    \renewcommand{\vbone}{3}
    \renewcommand{\vbtwo}{4}
    \renewcommand{\vbthree}{5}
    \renewcommand{\vbfour}{12}
    \renewcommand{\vbfive}{4}
    \renewcommand{\vbsix}{21}
  }{
    \ifnumequal{\value{rolldice}}{2}{
      % variables 
      \renewcommand{\vbone}{4}
      \renewcommand{\vbtwo}{7}
      \renewcommand{\vbthree}{4}
      \renewcommand{\vbfour}{15}
      \renewcommand{\vbfive}{7}
      \renewcommand{\vbsix}{26}
    }{
      % variables 
      \renewcommand{\vbone}{5}
      \renewcommand{\vbtwo}{2}
      \renewcommand{\vbthree}{3}
      \renewcommand{\vbfour}{10}
      \renewcommand{\vbfive}{8}
      \renewcommand{\vbsix}{21}
    }
  }
}

\MAX\vbone\vbtwo\a
\MIN\vbone\vbtwo\b
\MAX\vbthree\vbfour\c
\MIN\vbthree\vbfour\d
\MAX\vbfive\vbsix\e
\MIN\vbfive\vbsix\f

\SUBTRACT\a\b\g
\ADD\a\b\h
\SUBTRACT\c\d\k
\ADD\c\d\m
\SUBTRACT\e\f\p
\ADD\e\f\q

\ADD\q\g\r

\question[4] For what value of $M$ would the following hold?
\[ \dfrac{\sin\vbone A\sin\vbtwo A + \sin\vbthree A\sin\vbfour A + \sin\vbfive A\sin\vbsix A}
{\sin\vbone A\cos\vbtwo A + \sin\vbthree A\cos\vbfour A + \sin\vbfive A\cos\vbsix A} = \tan\textbf{M}A\]

\watchout

\begin{explanation}[\fullpage]
  Given that 
  \begin{align}
    \cos(x-y)-\cos(x+y) &= 2\sin x\sin y \\
    \sin(x+y) - \sin(x-y) &= 2\cos x\sin y
  \end{align}
  We can re-write given expression as 
  \begin{align}
    &\dfrac{(\cos\g A - \cos\h A) + (\cos\k A - \cos\m A) + (\cos\p A - \cos\q A)}
    {(\sin\h A - \sin\g A) + (\sin\m A - \sin\k A) + (\sin\q A - \sin\p A)} \nonumber\\
    &= \dfrac{\cos\g A - \cos\q A}{\sin\q A - \sin\g A}  \\
    &= \dfrac{\eDiffOfCos{\g A}{\q A}}{\eDiffOfSin{\q A}{\g A}} \\
    &= \tan\WRITEFRAC\r{2} A \implies M = \WRITEFRAC\r{2}
  \end{align}
\end{explanation}

