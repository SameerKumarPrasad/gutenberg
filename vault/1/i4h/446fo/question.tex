% This is an empty shell file placed for you by the 'examiner' script.
% You can now fill in the TeX for your question here.

% Now, down to brasstacks. ** Writing good solutions is an Art **. 
% Eventually, you will find your own style. But here are some thoughts 
% to get you started: 
%
%   1. Write the solution as if you are writing it for your favorite
%      14-17 year old to help him/her understand. Could be your nephew, 
%      your niece, a cousin perhaps or probably even you when you 
%      were that age. Just write for them.
%
%   2. Use margin-notes to "talk" to students about the critical insights
%      in the question. The tone can be - in fact, should be - informal
%
%   3. Don't shy away from creating margin-figures you think will help
%      students understand. Yes, it is a little more work per question. 
%      But the question & solution will be written only once. Make that
%      attempt at writing a solution count.
%
%   4. At the same time, do not be too verbose. A long solution can
%      - at first sight - make the student think, "God, that is a lot to know".
%      Our aim is not to scare students. Rather, our aim should be to 
%      create many "Aha!" moments everyday in classrooms around the world
% 
%   5. Ensure that there are *no spelling mistakes anywhere*. We are an 
%      education company. Bad spellings suggest that we ourselves 
%      don't have any education. Also, use American spellings by default
% 
%   6. If a question has multiple parts, then first delete lines 40-41
%   7. If a question does not have parts, then first delete lines 43-69

% \setcounter{rolldice}{3}
% \noprintanswers

\ifnumequal{\value{rolldice}}{0}{
	\renewcommand{\vbone}{13}
	\renewcommand{\vbtwo}{9}
	\renewcommand{\vbthree}{5}
	\renewcommand{\vbfour}{3}
	\renewcommand{\vbfive}{1}
	\renewcommand{\vbsix}{4}
}{
	\ifnumequal{\value{rolldice}}{1}{
		\renewcommand{\vbone}{15}
		\renewcommand{\vbtwo}{7}
		\renewcommand{\vbthree}{11}
		\renewcommand{\vbfour}{5}
		\renewcommand{\vbfive}{3}
		\renewcommand{\vbsix}{8}
	}{
		\ifnumequal{\value{rolldice}}{2}{
		\renewcommand{\vbone}{11}
		\renewcommand{\vbtwo}{6}
		\renewcommand{\vbthree}{7}
		\renewcommand{\vbfour}{3}
		\renewcommand{\vbfive}{2}
		\renewcommand{\vbsix}{5}
	}{
		\renewcommand{\vbone}{17}
		\renewcommand{\vbtwo}{10}
		\renewcommand{\vbthree}{9}
		\renewcommand{\vbfour}{5}
		\renewcommand{\vbfive}{2}
		\renewcommand{\vbsix}{7}
	}
  }
}


\question[4] Prove that $2\cos\dfrac{\vbfive\pi}{\vbone}\cos\dfrac{\vbtwo\pi}{\vbone} + 
\cos\dfrac{\vbthree\pi}{\vbone} + \cos\dfrac{\vbfour\pi}{\vbone} = 0 $

\insertQR[-20pt]{QRC}
\watchout

\ifprintanswers
\fi 

\begin{solution}[\halfpage]
   %\begin{fullwidth}
     \begin{align}
     	&2\cos\dfrac{\vbfive\pi}{\vbone}\cos\dfrac{\vbtwo\pi}{\vbone} + 
     	\cos\dfrac{\vbthree\pi}{\vbone} + \cos\dfrac{\vbfour\pi}{\vbone} \nonumber \\
     	&= 2\cos\dfrac{\vbfive\pi}{\vbone}\cos\dfrac{\vbtwo\pi}{\vbone} + 
     	\eSumOfCos{\frac{\vbthree\pi}{\vbone}}{\frac{\vbfour\pi}{\vbone}} \\
     	&= 2\cos\dfrac{\vbfive\pi}{\vbone}\cdot\left[ \cos\dfrac{\vbtwo\pi}{\vbone} + \cos\dfrac{\vbsix\pi}{\vbone}\right] \\
     	&= 2\cos\dfrac{\vbfive\pi}{\vbone}\cdot\left[ \eSumOfCos{\frac{\vbtwo\pi}{\vbone}}{\frac{\vbsix\pi}{\vbone}} \right] \\
     	&= 2\cos\dfrac{\vbfive\pi}{\vbone}\cdot
     	\left( 2\cdot\underbrace{\cos\dfrac{\vbone\pi}{2\cdot\vbone}}_{\cos\frac{\pi}{2} = 0}
     	\cos\dfrac{\vbtwo\pi-\vbsix\pi}{2\cdot\vbone}\right) \\
        &= 0
     \end{align}
   %\end{fullwidth}
\end{solution}
