

\question[3] The angles of a triangle are in arithmetic progression and the ratio of 
the number of \textit{degrees} of the \textit{smallest} angle to the number of 
\textit{radians} of the \textit{largest} angle is $\frac{60}{\pi}$. Find the angles 
of the triangle - in degrees 


\ifprintanswers
  \marginnote[4cm]{You can also write (1) as $\dfrac{a}{a+2d} = \dfrac{A_D}{\frac{180}{\pi}\cdot C_R}$. 
  The key is to remember that when taking ratios, both the values must have the same units }
\fi 

\begin{solution}[\halfpage]
   Let the three angles be $A=a$, $B=a+d$ and $C=a+2d$ - either all in degrees or all in radians. 
   Moreover, let $A_D$ be the \textit{number of degrees} in $\angle A$ and $C_R$ the
   \textit{number of radians} in $\angle C$
   
   And so, 
   \begin{align}
       \dfrac{a}{a+2d} &= \overbrace{\dfrac{\frac{\pi}{180}\times A_D}{C_R}}^{\texttt{everything in radians}} \\
       \text{where } \dfrac{A_D}{C_R} &= \dfrac{60}{\pi} \\
       \Rightarrow \dfrac{a}{a+2d} &= \dfrac{\pi}{180}\cdot\dfrac{60}{\pi} = \dfrac{1}{3} \\
       \Rightarrow a &= d 
   \end{align}
   And therefore, $A=a$, $B = 2a$ and $C = 3a$. Given that $ A+B+C = \ang{180}$, we get 
   $A = \ang{30}$, $B=\ang{60}$ and $C=\ang{90}$
\end{solution}
