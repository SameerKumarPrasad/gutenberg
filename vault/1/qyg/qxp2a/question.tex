

\question[2]  Find the value of $k$, if $x+1$ is a factor of $4x^3+3x^2-4x+k$

\ifprintanswers
  % stuff to be shown only in the answer key - like explanatory margin figures
  \marginnote[7cm] {The expression is cubic. Therefore there have to be 3 terms
	of the form $(x+a)$ so that there are 3 $x$'s to multiply to form $x^3$}
  \marginnote[0.25cm] {Instead of $(4x+a)$ and $(x+b)$, we could have chosen
	- say - $(2x+a)$ and $(2x+b)$ as the second \& third factors}
  \marginnote[0.25cm] {What is more important, however, is that whatever we choose,
	the coefficient of $x^3$ eventually turns out to be $4$}
  \marginnote[0.25cm] {Exercise: Solve for $a$ and $b$ and find out what the other
	factors of the expression are}
  \marginnote[0.25cm] {The second approach is 'cooler' because at the end of it
	you know more about the expression - namely, its other factors}
\fi 

\begin{solution}[\mcq]
	There are two approaches to solving this problem. The first uses the fact
	that if $(x-a)$ is a factor of some polynomial equation $f(x)$, then $f(a) = 0$.
	Lets start with that
	
	If $(x+1)$ is a factor, then $4x^3+3x^2-4x+k = 0 $ for $x = -1$
	\begin{align}
		& \implies 4\cdot(-1)^3+3\cdot(-1)^2-4\cdot(-1)+k = 0 \\
		& \implies k + 3 = 0 \\
		& \implies k = -3
	\end{align}
	
	The second approach, on the other hand, is based on what it means for something  
	to be a factor of a polynomial
	
	So, if $(x+1)$ is a factor of $4x^3+3x^2-4x+k$, then 
	\begin{align}
		4x^3+3x^2-4x+k &= (x+1)\cdot(4x+a)\cdot(x+b) \\
					   &= \underbrace{(4x^2+ax+4x+a)}_\texttt{first two factors}\cdot(x+b) \\
					   &= 4x^3+(a+4b+4)x^2+(a+4b+ab)x+ab
	\end{align}	 
	Its the same expression on both sides. So, coefficients of $x^n$ should also be 
  the same on both sides
	\begin{align}
		\implies a+4b+4 &= 3 \implies a + 4b = -1\\
		            a+4b+ab &= -4 \implies ab = -3\\
		            ab &= k \implies k = -3
	\end{align}
	
\end{solution}
\ifprintanswers\begin{codex}\end{codex}\fi
