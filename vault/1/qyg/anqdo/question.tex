% This is an empty shell file placed for you by the 'examiner' script.
% You can now fill in the TeX for your question here.

% Now, down to brasstacks. ** Writing good solutions is an Art **. 
% Eventually, you will find your own style. But here are some thoughts 
% to get you started: 
%
%   1. Write the solution as if you are writing it for your favorite
%      14-17 year old to help him/her understand. Could be your nephew, 
%      your niece, a cousin perhaps or probably even you when you 
%      were that age. Just write for them.
%
%   2. Use margin-notes to "talk" to students about the critical insights
%      in the question. The tone can be - in fact, should be - informal
%
%   3. Don't shy away from creating margin-figures you think will help
%      students understand. Yes, it is a little more work per question. 
%      But the question & solution will be written only once. Make that
%      attempt at writing a solution count.
%
%   4. At the same time, do not be too verbose. A long solution can
%      - at first sight - make the student think, "God, that is a lot to know".
%      Our aim is not to scare students. Rather, our aim should be to 
%      create many "Aha!" moments everyday in classrooms around the world
% 
%   5. Ensure that there are *no spelling mistakes anywhere*. We are an 
%      education company. Bad spellings suggest that we ourselves 
%      don't have any education. And, use American spellings

\question[3] The base of a bucket - formed by truncating a right circular cone - 
has a radius of 7 inches. The open end has a radius of 15 inches and the height 
of the bucket is 16 inches. What is the bucket's volume ?

\calculator{\sqrt{245}=15.6, \sqrt{1125}=33.5}

\insertQR{QRC}

\ifprintanswers
  % stuff to be shown only in the answer key - like explanatory margin figures
\fi 

\begin{solution}[\fullpage]
	From the top to the base, the radius falls from $15$ inches to $7$ inches 
	over a height of $16$ inches
	
	This suggests that, 
	\begin{align}
		\dfrac{15-7}{16} &= \dfrac{15-0}{16+h} \\
		\Rightarrow (16+h) &= 30 \text{ or } h = 14
	\end{align}
	, where $h$ is the height that was lopped off from the full cone to form the bucket
	
	The volume of the bucket is simply the volume of the full cone 
	minus the volume of the bit that was removed
	
	\begin{align}
	   V &= \dfrac{1}{3}\pi\left[ 15^2\cdot(16+h) - 7^2\cdot h\right] \\
	     &= \dfrac{6064}{3}\pi = 2021\frac{1}{3}\pi \text{ cubic inch }
	\end{align}
	
\end{solution}
