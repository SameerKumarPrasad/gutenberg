% This is an empty shell file placed for you by the 'examiner' script.
% You can now fill in the TeX for your question here.

% Now, down to brasstacks. ** Writing good solutions is an Art **. 
% Eventually, you will find your own style. But here are some thoughts 
% to get you started: 
%
%   1. Write the solution as if you are writing it for your favorite
%      14-17 year old to help him/her understand. Could be your nephew, 
%      your niece, a cousin perhaps or probably even you when you 
%      were that age. Just write for them.
%
%   2. Use margin-notes to "talk" to students about the critical insights
%      in the question. The tone can be - in fact, should be - informal
%
%   3. Don't shy away from creating margin-figures you think will help
%      students understand. Yes, it is a little more work per question. 
%      But the question & solution will be written only once. Make that
%      attempt at writing a solution count.
%
%   4. At the same time, do not be too verbose. A long solution can
%      - at first sight - make the student think, "God, that is a lot to know".
%      Our aim is not to scare students. Rather, our aim should be to 
%      create many "Aha!" moments everyday in classrooms around the world
% 
%   5. Ensure that there are *no spelling mistakes anywhere*. We are an 
%      education company. Bad spellings suggest that we ourselves 
%      don't have any education. And, use American spellings

\question[3]  Water is leaking from a conical funnel at the rate of \SI{5}{\cubic\centi\metre\per\second}.
If the radius of the base of the funnel is \SI{5}{\centi\meter} and the funnel's height is 
\SI{10}{\centi\meter}, find the rate at which water level is dropping when it is 
\SI{2.5}{\centi\meter} from the top. Assume that the other end of the funnel is a tip

\insertQR{QRC}

\ifprintanswers
  % stuff to be shown only in the answer key - like explanatory margin figures
\fi 

\begin{solution}[\fullpage]
	From the figure alongside, we can see that if 
	$h$ be the height from the tip of the funnel to the water-level and $r$
	be the radius at the water-level, then 
	\begin{align}
		\dfrac{r}{h} &= \dfrac{R_0}{H} = \dfrac{5}{10} = \dfrac{1}{2} \\
		\Rightarrow r &= \dfrac{h}{2}
	\end{align}
	Also, we have been told that 
	\begin{align}
		\dfrac{\ud V}{\ud t} &= \dfrac{\pi}{3}\dfrac{\ud}{\ud t}r^2h = \SI{5}{\cubic\centi\meter\per\second} \\
		&= \dfrac{\pi}{3}\left[2rh\dfrac{\ud r}{\ud t} + r^2\dfrac{\ud h}{\ud t}\right] \\
		&= \dfrac{\pi}{3}\left[h^2\cdot\dfrac{1}{2}\dfrac{\ud h}{\ud t} + \dfrac{h^2}{4}\dfrac{\ud h}{\ud t} \right] \\
		\Rightarrow \dfrac{\pi h^2}{4}\dfrac{\ud h}{\ud t} &= \SI{5}{\cubic\centi\meter\per\second} \\
		\Rightarrow \dfrac{\ud h}{\ud t} &= \dfrac{\SI{5}{\cubic\centi\meter\per\second}}
		{\dfrac{\pi h^2}{4}}
	\end{align}
	When the water-level is $2.5$ cm from the base, $h = 7.5$ cm. And at that instant
	\begin{align}
		\dfrac{\ud h}{\ud t} &= \dfrac{\SI{5}{\cubic\centi\meter\per\second}}
		{\dfrac{\pi}{4}\cdot(\SI{7.5}{\centi\meter})^2} \\
		&= \SI{0.028}{\centi\meter\per\second}
	\end{align}
\end{solution}
