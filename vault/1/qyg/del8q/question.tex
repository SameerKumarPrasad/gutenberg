% This is an empty shell file placed for you by the 'examiner' script.
% You can now fill in the TeX for your question here.

% Now, down to brasstacks. ** Writing good solutions is an Art **. 
% Eventually, you will find your own style. But here are some thoughts 
% to get you started: 
%
%   1. Write the solution as if you are writing it for your favorite
%      14-17 year old to help him/her understand. Could be your nephew, 
%      your niece, a cousin perhaps or probably even you when you 
%      were that age. Just write for them.
%
%   2. Use margin-notes to "talk" to students about the critical insights
%      in the question. The tone can be - in fact, should be - informal
%
%   3. Don't shy away from creating margin-figures you think will help
%      students understand. Yes, it is a little more work per question. 
%      But the question & solution will be written only once. Make that
%      attempt at writing a solution count.
%
%   4. At the same time, do not be too verbose. A long solution can
%      - at first sight - make the student think, "God, that is a lot to know".
%      Our aim is not to scare students. Rather, our aim should be to 
%      create many "Aha!" moments everyday in classrooms around the world
% 
%   5. Ensure that there are *no spelling mistakes anywhere*. We are an 
%      education company. Bad spellings suggest that we ourselves 
%      don't have any education. And, use American spellings

\question[3]  Ice-cream is stored in a right circular cylinder of height \SI{1}{\meter}
\insertQR{QRC}
and radius of \SI{20}{\centi\meter}. A single scoop of ice-cream is a sphere that sits 
perfectly on a cone at its (the scoop's) diameter. Cones are \SI{12}{\centi\meter} tall and have a diameter of \SI{6}{\centi\meter} at their open end. Assuming that an ice-cream seller starts
the day with a full cylinder of ice-cream and assuming that he serves only 
single scoops, how long would it be before he runs out of ice-cream given that 
he averages 60 cones per hour?

\ifprintanswers
  % stuff to be shown only in the answer key - like explanatory margin figures
  \marginnote[1cm]{ Writing units as you write values is a great habit to have }
\fi 

\begin{solution}[\fullpage]
	The volume of ice-cream in a single scoop is given by
	\begin{align}
		V_{\texttt{single-scoop}} &= \dfrac{4}{3}\pi R^3 \\
		                 &= \dfrac{4}{3}\pi(\dfrac{D}{2})^3 = \pi\dfrac{D^3}{6} \\
		                 &= \pi\dfrac{(\SI{6}{\centi\meter})^3}{6} = 36\pi\si{cm^3}
	\end{align}
	The volume of ice-cream in the cylinder, on the other hand, is given by
	\begin{align}
		V_{\texttt{cylinder}} &= \pi R^2 H \\
		                      &= \pi(\SI{20}{cm})^2\times\underbrace{\SI{1}{\meter}}_{\texttt{100 cm}} \\
		                      &= 40,000\pi\si{\centi\meter}^3
	\end{align} 
	If $T$ be the time - in hours - that the ice-cream seller needs to empty the cylinder, then
	\begin{align}
		T &= \dfrac{40,000\pi\si{cm^3}}{36\pi\si[per-mode=fraction]{cm^3\per scoop}}
		     \times \dfrac{1}{\SI{60}{scoops\per hour}} \\
		  &= \SI{18.52}{hours}\approx \SI{18}{\hour}\SI{31}{\minute}
	\end{align}
\end{solution}
