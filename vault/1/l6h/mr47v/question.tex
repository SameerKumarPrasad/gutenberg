% This is an empty shell file placed for you by the 'examiner' script.
% You can now fill in the TeX for your question here.

% Now, down to brasstacks. ** Writing good solutions is an Art **. 
% Eventually, you will find your own style. But here are some thoughts 
% to get you started: 
%
%   1. Write the solution as if you are writing it for your favorite
%      14-17 year old to help him/her understand. Could be your nephew, 
%      your niece, a cousin perhaps or probably even you when you 
%      were that age. Just write for them.
%
%   2. Use margin-notes to "talk" to students about the critical insights
%      in the question. The tone can be - in fact, should be - informal
%
%   3. Don't shy away from creating margin-figures you think will help
%      students understand. Yes, it is a little more work per question. 
%      But the question & solution will be written only once. Make that
%      attempt at writing a solution count.
%
%   4. At the same time, do not be too verbose. A long solution can
%      - at first sight - make the student think, "God, that is a lot to know".
%      Our aim is not to scare students. Rather, our aim should be to 
%      create many "Aha!" moments everyday in classrooms around the world
% 
%   5. Ensure that there are *no spelling mistakes anywhere*. We are an 
%      education company. Bad spellings suggest that we ourselves 
%      don't have any education. Also, use American spellings by default
% 
%   6. If a question has multiple parts, then first delete lines 40-41
%   7. If a question does not have parts, then first delete lines 43-69

\question[5] A computer was given several problems in succession to solve. The time it took 
to solve each successive problem was the same number of times smaller than it took
to solve the preceding problem. How many problems was the computer given \text{if} it 
spent 63.5 minutes to solve all problems except the first, 127 min to solve all except
the last and 31.5 min to solve all except the first two?

\insertQR{QRC}

\ifprintanswers
	\marginnote[3cm]{ $r \in (0,1)$ because each successive problem took \text{less} time than the preceding one}
\fi 

\begin{solution}[\fullpage]
	The times the computer took to solve each problem form a geometric sequence. Let 
	that sequence be $\lbrace a, ar, ar^2 \ldots ar^{n-1}\rbrace$ where $r \in (0,1)$ and $n$
	is what we have to find out
	
	Now,
	\begin{align}
		\eSumOfGP{r}{n} - a &= 63.5 \\
		\eSumOfGP{r}{n} - ar^{n-1} &= 127 \\
		\eSumOfGP{r}{n} - (a + ar) &= 31.5 \\
		\Rightarrow \underbrace{\eSumOfGP{r}{n} - a}_{=63.5} - ar &= 31.5 \\
		\Rightarrow ar &= 32\text{ min }
	\end{align}
	Morever, 
	\begin{align}
		\dfrac{\eSumOfGP{r}{n} - ar^{n-1}}{\eSumOfGP{r}{n} - a} &= \frac{127}{63.5} = 2 \\
		\Rightarrow \dfrac{(1-r^n) - r^{n-1}\cdot(1-r)}{(1-r^n) - (1-r)} &= 2  \\
		\dfrac{1-r^n-r^{n-1}+r^n}{r\cdot(1-r^{n-1})} &= 2 \Rightarrow \dfrac{1}{r} = 2 
		\text{ or } r = \frac{1}{2} \\
		\text{And so, if } ar = 32 &\text{ and } r = \frac{1}{2}, \text{ then } a = 64 = 2^6
	\end{align}
	We now know enough to find the number of problems given. Lets substitute (8) and (9) in (2)
	\begin{align}
		\eSumOfGP[2^6]{\frac{1}{2}}{n}-\dfrac{2^6}{2^{n-1}} &= 127 \\
		\Rightarrow 2^7-127 = 2\cdot 2^{7-n} &= 2^{8-n} \\
		\Rightarrow 2^{8-n} &= 1 \text{ or } 8-n = 0 \text{ or } n = 8
	\end{align}
\end{solution}
