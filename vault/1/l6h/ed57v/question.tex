
\ifnumequal{\value{rolldice}}{0}{
  % variables 
  \renewcommand\vc{4}
  \renewcommand\vd{3}
}{
  \ifnumequal{\value{rolldice}}{1}{
    % variables 
    \renewcommand\vc{5}
    \renewcommand\vd{7}
  }{
    \ifnumequal{\value{rolldice}}{2}{
      % variables 
      \renewcommand\vc{11}
      \renewcommand\vd{7}
    }{
      % variables 
      \renewcommand\vc{9}
      \renewcommand\vd{4}
    }
  }
}

\FRACDIV\vc\vd{13}{9}\va\vb
\FRACDIV\vc\vd{4}{3}\vx\vy

\question[3] The ratio of the sum of the cubes of the terms of an \textbf{infinitely decreasing}
geometric progression to the sum of the squares of its terms is $\va:\vb$. The sum of the 
first two terms of the progression is $\frac\vc\vd$. What is the first term $(a)$ of the progression?

\watchout

\begin{solution}[\halfpage]
  If the original progression be $G$
  \[ G = \lbrace a,ar,ar^2,ar^3\ldots \rbrace \]
  then the sequence of cubic terms would be 
  \[ G_3 = \underbrace{\lbrace a^3, a^3r^3,a^3r^6\ldots \rbrace}_{\text{first term}=a^3,\text{ common ratio}=r^3}
  \implies S_3^\infty = \dfrac{a^3}{1-r^3} \]
  and the sequence of square terms would be 
  \[ G_2 = \underbrace{\lbrace a^2, a^2r^2,a^2r^4\ldots \rbrace}_{\text{first term}=a^2,\text{ common ratio}=r^2}
  \implies S_2^\infty = \dfrac{a^2}{1-r^2} \]
  And therefore, 
   \begin{align}
   		\dfrac{S_3^\infty}{S_2^\infty} = \dfrac{\frac{a^3}{1-r^3}}{\frac{a^2}{1-r^2}} = \dfrac\va\vb
   		\implies &\dfrac{a\cdot(1-r)\cdot(1+r)}{a\cdot(1-r)\cdot(1+r+r^2)} = \dfrac\va\vb \\
   		\implies\dfrac{a\cdot(1+r)}{1+r+r^2} &= \dfrac\va\vb 
   \end{align}
   Moreover, the sum of the first two terms in $G = a + ar = a\cdot (1+r) = \frac\vc\vd$. Hence, 
   \begin{align}
   		\dfrac{S_3^\infty}{S_2^\infty} = \dfrac{a\cdot (1+r)}{1+r+r^2} &= \dfrac{\frac\vc\vd}{1+r+r^2} = \dfrac\va\vb \\
      \implies 9 r^2 + 9 r - 4 &= 0\implies r = \frac{1}{3}\text{ or } -\frac{4}{3}
   \end{align}
   \textbf{Insight \#1: Infinitely decreasing $\implies 0 < r < 1$}

   And hence, only $r=\frac{1}{3}$ makes sense.
   And therefore 
   \begin{align}
   	a\cdot(1+r) &= \dfrac\vc\vd \implies a = \WRITEFRAC\vx\vy
   \end{align}
\end{solution}
\ifprintanswers\begin{codex}$\WRITEFRAC\vx\vy$\end{codex}\fi
