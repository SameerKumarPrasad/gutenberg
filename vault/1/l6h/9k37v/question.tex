% This is an empty shell file placed for you by the 'examiner' script.
% You can now fill in the TeX for your question here.

% Now, down to brasstacks. ** Writing good solutions is an Art **. 
% Eventually, you will find your own style. But here are some thoughts 
% to get you started: 
%
%   1. Write the solution as if you are writing it for your favorite
%      14-17 year old to help him/her understand. Could be your nephew, 
%      your niece, a cousin perhaps or probably even you when you 
%      were that age. Just write for them.
%
%   2. Use margin-notes to "talk" to students about the critical insights
%      in the question. The tone can be - in fact, should be - informal
%
%   3. Don't shy away from creating margin-figures you think will help
%      students understand. Yes, it is a little more work per question. 
%      But the question & solution will be written only once. Make that
%      attempt at writing a solution count.
%
%   4. At the same time, do not be too verbose. A long solution can
%      - at first sight - make the student think, "God, that is a lot to know".
%      Our aim is not to scare students. Rather, our aim should be to 
%      create many "Aha!" moments everyday in classrooms around the world
% 
%   5. Ensure that there are *no spelling mistakes anywhere*. We are an 
%      education company. Bad spellings suggest that we ourselves 
%      don't have any education. Also, use American spellings by default
% 
%   6. If a question has multiple parts, then first delete lines 40-41
%   7. If a question does not have parts, then first delete lines 43-69

\question[3] The sequence ${b_n}, \, n \geq 1$ is a geometric progression with $\dfrac{b_4}{b_6} = \dfrac{1}{4}$
and $b_2 + b_5 = 216$. Find $b_1$

\insertQR{QRC}

\ifprintanswers
\fi 

\begin{solution}[\halfpage]
	The terms of the sequence are of the form $b_n = b_1r^{n-1}$. And so, 
	\begin{align}
		\dfrac{b_4}{b_6} &= \dfrac{b_1r^3}{b_1r^5} = \dfrac{1}{r^2} = \dfrac{1}{4} \Rightarrow r = \pm 2 \\
		b_2 + b_5 = b_1r+b_1r^4 &= b_1r\cdot(1 + r^3) = 216 \\
		\Rightarrow \text{ when } r = 2 \text{ then } b_1 &= \dfrac{216}{2\times 9} = 12 \\
		\text{ And when } r = -2 \text{ then } b_1 &= \dfrac{216}{-2\times -7} = \dfrac{108}{7}
	\end{align}
\end{solution}
