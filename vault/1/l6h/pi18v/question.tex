% This is an empty shell file placed for you by the 'examiner' script.
% You can now fill in the TeX for your question here.

% Now, down to brasstacks. ** Writing good solutions is an Art **. 
% Eventually, you will find your own style. But here are some thoughts 
% to get you started: 
%
%   1. Write the solution as if you are writing it for your favorite
%      14-17 year old to help him/her understand. Could be your nephew, 
%      your niece, a cousin perhaps or probably even you when you 
%      were that age. Just write for them.
%
%   2. Use margin-notes to "talk" to students about the critical insights
%      in the question. The tone can be - in fact, should be - informal
%
%   3. Don't shy away from creating margin-figures you think will help
%      students understand. Yes, it is a little more work per question. 
%      But the question & solution will be written only once. Make that
%      attempt at writing a solution count.
%
%   4. At the same time, do not be too verbose. A long solution can
%      - at first sight - make the student think, "God, that is a lot to know".
%      Our aim is not to scare students. Rather, our aim should be to 
%      create many "Aha!" moments everyday in classrooms around the world
% 
%   5. Ensure that there are *no spelling mistakes anywhere*. We are an 
%      education company. Bad spellings suggest that we ourselves 
%      don't have any education. Also, use American spellings by default
% 
%   6. If a question has multiple parts, then first delete lines 40-41
%   7. If a question does not have parts, then first delete lines 43-69

\question[5] Three numbers form an arithmetic progression. If we add 8 to the first number,
 we get a geometric progression whose terms add upto 26. Find the numbers in the original 
 arithmetic progression

\insertQR{QRC}

\ifprintanswers
\fi 

\begin{solution}[\fullpage]
	If the original arithmetic progression is $A = \lbrace a, a + d, a+2d\rbrace$, then the
	resulting geometric progression is $G = \lbrace a + 8, a + d, a + 2d \rbrace$
	\begin{align}
		\Rightarrow a +d &= (a+8)\cdot r \\
		a + 2d &= (a+d)\cdot r = (a + 8)\cdot r^2 
	\end{align}
	Moreover, 
	\begin{align}
		(a+8) + (a+d)+(a+2d) &= 3\cdot(a+d) + 8 = 26 \\
		\Rightarrow (a+d) = 6
	\end{align}
	So, our geometric sequence - $G$ - now is $\frac{6}{r}, 6, 6r$
	\begin{align}
		\Rightarrow 6\cdot\left( \dfrac{1}{r} + 1 + r\right) &= 26 \\
		\text{ or } r + \dfrac{1}{r} &= \dfrac{10}{3} \Rightarrow 3r^2-10r+3 = 0 \\
		\Rightarrow r = 3, \frac{1}{3}
	\end{align}
	
	Which means that the \textit{geometric} progression is either $(18,6,2)$ or $(2,6,18)$.
	And that in turn means that the original \textit{arithmetic} progression is either
	$(18-8,6,2) = (10,6,2)$ or $(2-8,6,18) = (-6,6,18)$
\end{solution}
