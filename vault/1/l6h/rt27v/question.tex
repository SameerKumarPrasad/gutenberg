% This is an empty shell file placed for you by the 'examiner' script.
% You can now fill in the TeX for your question here.

% Now, down to brasstacks. ** Writing good solutions is an Art **. 
% Eventually, you will find your own style. But here are some thoughts 
% to get you started: 
%
%   1. Write the solution as if you are writing it for your favorite
%      14-17 year old to help him/her understand. Could be your nephew, 
%      your niece, a cousin perhaps or probably even you when you 
%      were that age. Just write for them.
%
%   2. Use margin-notes to "talk" to students about the critical insights
%      in the question. The tone can be - in fact, should be - informal
%
%   3. Don't shy away from creating margin-figures you think will help
%      students understand. Yes, it is a little more work per question. 
%      But the question & solution will be written only once. Make that
%      attempt at writing a solution count.
%
%   4. At the same time, do not be too verbose. A long solution can
%      - at first sight - make the student think, "God, that is a lot to know".
%      Our aim is not to scare students. Rather, our aim should be to 
%      create many "Aha!" moments everyday in classrooms around the world
% 
%   5. Ensure that there are *no spelling mistakes anywhere*. We are an 
%      education company. Bad spellings suggest that we ourselves 
%      don't have any education. Also, use American spellings by default
% 
%   6. If a question has multiple parts, then first delete lines 40-41
%   7. If a question does not have parts, then first delete lines 43-69

\question[4] Find four numbers forming a geometric progression in which the sum of the 
extreme terms is 112 and the sum of the middle terms is 48
\insertQR{QRC}

\ifprintanswers
\fi 

\begin{solution}[\halfpage]
	Let the four numbers be $\dfrac{a}{r^2}, \dfrac{a}{r}, a$ and $ar$. Now, we are told that
	\begin{align}
		\dfrac{a}{r^2} + ar = 112 \Rightarrow &a\cdot\left( \dfrac{r^3+1}{r^2}\right) = 112 \\
		\dfrac{a}{r} + a = 48 &\Rightarrow a\cdot\left( \dfrac{r+1}{r}\right) = 48 \\
		\therefore \dfrac{a\cdot\left( \dfrac{r^3+1}{r^2}\right)}
		{a\cdot\left( \dfrac{r+1}{r}\right)} &= \dfrac{112}{48} = \dfrac{7}{3} \\
		\Rightarrow \dfrac{(1+r)^3 - 3r^2 - 3r}{r\cdot(1+r)} &= \dfrac{7}{3} \\
		\Rightarrow 3\cdot\lbrace (r+1)^2 - 3r\rbrace &= 7r \\
		\Rightarrow 3r^2-10r+3 = 0 \text{ or } r = \frac{1}{3}, 3
		\text{Using } r &= \frac{1}{3}, \text{ we get } a = 12
	\end{align}
	
	Hence, the three numbers are (using $r=\frac{1}{3}$) - $(108, 36, 12, 4)$
\end{solution}
