

\ifnumodd{\value{rolldice}}{
  % equations
  % variables 
  \renewcommand{\vbone}{10}
  \renewcommand{\vbtwo}{50}
  \renewcommand{\vbthree}{30}
  \renewcommand{\vbfour}{70}
  \renewcommand{\vbfive}{\dfrac{1}{16}}
  \renewcommand{\vbsix}{\dfrac{1}{2}}
  \renewcommand{\vbseven}{110}
}{
    % equations
    % variables 
    \renewcommand{\vbone}{20}
    \renewcommand{\vbtwo}{40}
    \renewcommand{\vbthree}{60}
    \renewcommand{\vbfour}{80}
    \renewcommand{\vbfive}{\dfrac{3}{16}}
    \renewcommand{\vbsix}{\dfrac{\sqrt{3}}{2}}
    \renewcommand{\vbseven}{100}
}

\question[4] Prove that $(\sin\ang{\vbone}\times\sin\ang{\vbtwo}\times\sin\ang{\vbthree}\times\sin\ang{\vbfour}) 
= \vbfive$

\ifprintanswers
\fi 

\begin{solution}[\halfpage]
    \begin{align}
       \sin\ang{\vbone}\cdot\sin\ang{\vbtwo}&\cdot\sin\ang{\vbthree}\cdot\sin\ang{\vbfour} = 
       \dfrac{1}{4}\times\overbrace{(4\cdot\sin\ang{\vbone}\cdot\sin\ang{\vbtwo}\cdot\sin\ang{\vbfour})}
       ^{\sin 3\theta = 4\sin\theta\cdot\sin\left( \frac{\pi}{3} - \theta\right)\cdot\sin\left(\frac{\pi}{3} +
        \theta\right)}\cdot\sin\ang{\vbthree} \\
       &= \dfrac{1}{4}\cdot\sin\ang{\vbthree}\cdot\sin\ang{\vbthree} \\
       &= \dfrac{1}{4}\cdot\left(\vbsix \right)^2 = \vbfive
    \end{align}
    
    Alternatively, you can start by using the fact that $\fProdOfSin{a}{b}$ to re-write the above expression as
    \begin{align}
    	&\sin\ang{\vbthree}\cdot\dfrac{1}{2}\cdot \underbrace{2\sin\ang{\vbone}\sin\ang{\vbtwo}}_{\fProdOfSin{a}{b}}\cdot\sin\ang{\vbfour} \\
    	&= \dfrac{\sqrt{3}}{4}\cdot(\cos\ang{\vbone}-\underbrace{\cos\ang{\vbthree}}_{= \frac{1}{2}})\cdot\sin\ang{\vbfour} \\
    	&= \dfrac{\sqrt{3}}{4}\cdot\left(\dfrac{1}{2}\cdot \underbrace{2\cos\ang{\vbone}\sin\ang{\vbfour}}
    	_{\eProdCosSin{a}{b}} - \dfrac{1}{2}\sin\ang{\vbfour}\right)
    \end{align}
    
    Now, $ \ang{\vbseven} = \ang{180}-\ang{\vbfour}$ and therefore $\sin\ang{\vbseven} =
     \sin(\ang{180}-\ang{\vbfour}) = \sin\ang{\vbfour}$
    \begin{align}
    	&= \dfrac{\sqrt{3}}{4}\cdot\left( \dfrac{1}{2}\cdot(\sin\ang{\vbseven} - \sin(\ang{-\vbthree}))
    	-\dfrac{1}{2}\sin\ang{\vbfour}\right) \\
    	&= \dfrac{\sqrt{3}}{4}\cdot\dfrac{\sin\ang{\vbthree}}{2} = \vbfive
    \end{align}
\end{solution}
