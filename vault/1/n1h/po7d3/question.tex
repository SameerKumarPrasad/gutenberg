
\ifnumequal{\value{rolldice}}{0}{
  \renewcommand\vp{4}
}{
  \ifnumequal{\value{rolldice}}{1}{
    \renewcommand\vp{2}
  }{
    \ifnumequal{\value{rolldice}}{2}{
      \renewcommand\vp{3}
    }{
      \renewcommand\vp{4}
    }
  }
}

\MULTIPLY\vp{2}\vq
\ADD\vq\vp\va

\question For a parabola $y^2=2px$ and a point $A$ on its axis (does not have to be the focus), 

\watchout

\ifprintanswers
  \vspace{0.75cm}
  \figinit{pt}
  \def\Xmin{0}
  \def\Ymin{0}
  \def\Xmax{79.99999}
  \def\Ymax{49.99999}
  \def\Xori{0}
  \def\Yori{0}
  \figpt0:(\Xori,\Yori)
  \figdrawbegin{}
  \def\Xmaxx{\Xmax} % To customize the position
  \def\Ymaxx{\Ymax} % of the arrow-heads of the axes.
  \figset arrowhead(length=4, fillmode=yes) % styling the arrowheads
  \figdrawaxes 0(\Xmin, \Xmaxx, \Ymin, \Ymaxx)
  \figpt 100: (60,0)
  \figpt 101: (30,31)
  \figdrawline [100,101]
  \figdrawlineC(
  0 0,
  2.75862 -9.28476,
  5.51724 -13.13064,
  8.27586 -16.08168,
  11.03448 -18.56953,
  13.79310 -20.76136,
  16.55172 -22.74294,
  19.31034 -24.56518,
  22.06896 -26.26128,
  24.82758 -27.85430,
  27.58620 -29.36101,
  30.34482 -30.79408,
  33.10344 -32.16337,
  35.86206 -33.47670,
  38.62068 -34.74041,
  41.37931 -35.95974,
  44.13793 -37.13906,
  46.89655 -38.28207,
  49.65517 -39.39192,
  52.41379 -40.47136,
  55.17241 -41.52273,
  57.93103 -42.54814,
  60.68965 -43.54941,
  63.44827 -44.52817,
  66.20689 -45.48588,
  68.96551 -46.42383,
  71.72413 -47.34320,
  74.48275 -48.24506,
  77.24137 -49.13036,
  79.99999 -49.99999
  )
  \figdrawlineC(
  0 0,
  2.75862 9.28476,
  5.51724 13.13064,
  8.27586 16.08168,
  11.03448 18.56953,
  13.79310 20.76136,
  16.55172 22.74294,
  19.31034 24.56518,
  22.06896 26.26128,
  24.82758 27.85430,
  27.58620 29.36101,
  30.34482 30.79408,
  33.10344 32.16337,
  35.86206 33.47670,
  38.62068 34.74041,
  41.37931 35.95974,
  44.13793 37.13906,
  46.89655 38.28207,
  49.65517 39.39192,
  52.41379 40.47136,
  55.17241 41.52273,
  57.93103 42.54814,
  60.68965 43.54941,
  63.44827 44.52817,
  66.20689 45.48588,
  68.96551 46.42383,
  71.72413 47.34320,
  74.48275 48.24506,
  77.24137 49.13036,
  79.99999 49.99999
  )
  \figdrawend
  \figvisu{\figBoxA}{}{%
  \fontfamily{cmss}\selectfont\normalsize
  \figptsaxes 1:0(\Xmin, \Xmaxx, \Ymin, \Ymaxx)
  \figwritee 1:(5pt)     \figwriten 2:(5pt)
  \figptsaxes 1:0(\Xmin, \Xmax, \Ymin, \Ymax)
  \figwrites 100: $\text{A(a,0)}$(3)
  \figwritenw 101: $\text{B}$(3)
  }
  \centerline{\box\figBoxA}
\fi 

\begin{parts}
  \part[3] What is the $x-$coordinate of the point \textbf{on the parabola} closest to $A$?

\begin{solution}[\halfpage]
  	If $G$ be the distance between point $B (x,y)$ on the parabola and $A (a,0)$ (see figure), then 
  	\begin{align}
  	   G &= \sqrt{(x-a)^2 + y^2} \\
  	   \implies G^2 &= (x-a)^2 + y^2 = (x-a)^2 + 2px \\
  	   \text{And }\therefore \dfrac{\ud G^2}{\ud x} &= 2\cdot(x-a) + 2p \\
  	   &= 0 \text{ when } x = (a-p) \\
  	   \text{Also, }\dfrac{\ud^2}{\ud x^2}G^2 &= 2 > 0 \implies \text{ minima }
  	\end{align}
  	
  	$x_{min}=(a-p)$ is an answer - but it is not the whole answer. If $a < p$, then the above
  	formula would give an $x_{min} < 0$. But the parabola \textit{requires} $x_{min} \geq 0$.
  	And so, the only possible solution is
  	\begin{align}
  	   x_{min} &= \left\lbrace
  	      \begin{array}{l c}
  	         (a-p) & \text{if } a > p \\
  	         0 & \text{otherwise}
  	      \end{array}\right.
  	\end{align}
  \end{solution}

  \part[1] Using the result in part (a), find the point on parabola $y^2=\vq x$ closest to the point $(\va,0)$ 

\begin{solution}[\mcq]
  We need values for $a$ and $p$ that we used in part (a). 
  \begin{align}
    y^2 &=  \vq x = 2px \implies p = \vp \\
    a &= \va \implies a-p = \va - \vp = \vq
  \end{align}
  Hence, the closest points are 
  \[ \left(a-p,\, \sqrt{\vq\cdot (a-p)}\right) = \left(\vq,\,\sqrt{\vq\cdot\vq}\right) = (\vq,-\vq)\text{ and }(\vq,\vq) \]
  \end{solution}
\end{parts}

\ifprintanswers
  \begin{codex}
    $(a)\, (a-p)\text{ if } a > p\text{ else } 0\qquad (b)\, (\vq,-\vq)\text{ and }(\vq,\vq)$ 
  \end{codex}
\fi
