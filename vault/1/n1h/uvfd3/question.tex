% This is an empty shell file placed for you by the 'examiner' script.
% You can now fill in the TeX for your question here.

% Now, down to brasstacks. ** Writing good solutions is an Art **. 
% Eventually, you will find your own style. But here are some thoughts 
% to get you started: 
%
%   1. Write the solution as if you are writing it for your favorite
%      14-17 year old to help him/her understand. Could be your nephew, 
%      your niece, a cousin perhaps or probably even you when you 
%      were that age. Just write for them.
%
%   2. Use margin-notes to "talk" to students about the critical insights
%      in the question. The tone can be - in fact, should be - informal
%
%   3. Don't shy away from creating margin-figures you think will help
%      students understand. Yes, it is a little more work per question. 
%      But the question & solution will be written only once. Make that
%      attempt at writing a solution count.
%
%   4. At the same time, do not be too verbose. A long solution can
%      - at first sight - make the student think, "God, that is a lot to know".
%      Our aim is not to scare students. Rather, our aim should be to 
%      create many "Aha!" moments everyday in classrooms around the world
% 
%   5. Ensure that there are *no spelling mistakes anywhere*. We are an 
%      education company. Bad spellings suggest that we ourselves 
%      don't have any education. Also, use American spellings by default
% 
%   6. If a question has multiple parts, then first delete lines 40-41
%   7. If a question does not have parts, then first delete lines 43-69

\question[2] A \SI{160}{\centi\metre} tall man walks away from a source of light 
mounted on a \SI{6}{\meter} high pole at the rate of \SI{1.1}{\meter\per\second}.
How fast is the length of his shadow increasing when he is \SI{1}{\meter} away from the pole?

\insertQR{QRC}

\ifprintanswers
  \begin{marginfigure}
    % 1. Definition of characteristic points
\figinit{pt}
\def\Xmin{0}
\def\Ymin{0}
\def\Xmax{80.00000}
\def\Ymax{79.99999}
\def\Xori{0}
\def\Yori{0}
\figpt0:(\Xori,\Yori)
% 2. Creation of the graphical file
\figdrawbegin{}
\def\Xmaxx{\Xmax} % To customize the position
\def\Ymaxx{\Ymax} % of the arrow-heads of the axes.
\figset arrowhead(length=4, fillmode=yes) % styling the arrowheads
\figdrawaxes 0(\Xmin, \Xmaxx, \Ymin, \Ymaxx)
\figpt 100: (80,80)
\figpt 101: (80,0)
\figpt 102: (40,40)
\figpt 103: (40,0)
\figpt 104: (20,1)
\figpt 105: (55,1)
\figpt 200: (40,20)
\figpt 201: (80,40)
\figpt 300: (10,6)
\figdrawline [100,101]
\figdrawline [102,103]
\figdrawlineC(
0 0,
2.75862 2.75862,
5.51724 5.51724,
8.27586 8.27586,
11.03448 11.03448,
13.79310 13.79310,
16.55172 16.55172,
19.31034 19.31034,
22.06896 22.06896,
24.82758 24.82758,
27.58620 27.58620,
30.34482 30.34482,
33.10344 33.10344,
35.86206 35.86206,
38.62068 38.62068,
41.37931 41.37931,
44.13793 44.13793,
46.89655 46.89655,
49.65517 49.65517,
52.41379 52.41379,
55.17241 55.17241,
57.93103 57.93103,
60.68965 60.68965,
63.44827 63.44827,
66.20689 66.20689,
68.96551 68.96551,
71.72413 71.72413,
74.48275 74.48275,
77.24137 77.24137,
79.99999 79.99999
)
\figdrawend
% 3. Writing text on the figure
\figvisu{\figBoxA}{}{%
\figptsaxes 1:0(\Xmin, \Xmaxx, \Ymin, \Ymaxx)
% Points 1 and 2 are the end points of the arrows
\figwritee 1:(5pt)     \figwriten 2:(5pt)
\figptsaxes 1:0(\Xmin, \Xmax, \Ymin, \Ymax)
\figwrites 104:$L$(4)
\figwrites 105:$x$(4)
\figwritee 200: $\SI{1.6}{\meter}$(2)
\figwritee 201: $\SI{6}{\meter}$(2)
\figwritee 300: $\theta$(1)
}
\centerline{\box\figBoxA}

  \end{marginfigure}

\marginnote[0.3cm]{Note that while the length of the shadow depends on how 
far away from the pole the man is, the rate at which it - the shadow - grows does not! }
\fi 

\begin{solution}[\halfpage]
  In the figure alongside, $L$ is the length of the shadow and $x$
  the distance the man is away from the pole
  
  \begin{align}
    \tan\theta &= \dfrac{1.6\si{\meter}}{L} = \dfrac{6\si{\meter}}{L+x} \\
    \Rightarrow L &= \dfrac{1.6}{4.4}x \\
    \Rightarrow \dfrac{\ud L}{\ud t} &= \dfrac{1.6}{4.4}\times\dfrac{\ud x}{\ud t} \\
       &= \dfrac{1.6}{4.4}\times\SI{1.1}{\meter\per\second} \\
       &= \SI{0.4}{\meter\per\second}
  \end{align}
\end{solution}
