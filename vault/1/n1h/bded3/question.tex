

\question[2] Sand is pouring from a pipe at the rate of \SI{12}{\centi\metre^3\per\second}.
The falling sand forms a cone on the ground in such a way that the height of the
cone is always one-sixth the radius of the base. How fast is the height of the
sand cone increasing when its height is \SI{4}{\centi\metre}?


\ifprintanswers
  % stuff to be shown only in the answer key - like explanatory margin figures
\fi 

\begin{solution}[\halfpage]
  If $V(t)$ be the volume of the sand cone at any given time, then
  \begin{align}
     V(t) &= \dfrac{1}{3}\pi R(t)^2\cdot h(t) \\
          &= \dfrac{1}{3}\pi (6h(t))^2\cdot h(t) \\
          &= 12\pi\cdot h(t)^3 \\
    \Rightarrow \dfrac{\ud V(t)}{\ud t} &= 36\pi \times h(t)^2\times \dfrac{\ud h(t)}{\ud t} \\
    \Rightarrow \SI{12}{\centi\meter^3\per\second} &= 36\pi\times(\SI{4}{\centi\meter})^2\times
    \left[\dfrac{\ud h(t)}{\ud t}\right]_{h=4} \\
    \Rightarrow \left[\dfrac{\ud h(t)}{\ud t}\right]_{h=4} &= 
         \dfrac{1}{48\pi}\si{\centi\meter\per\second}
  \end{align}
\end{solution}
