


\ifnumequal{\value{rolldice}}{0}{
  % variables 
  \renewcommand{\vbone}{6}
  \renewcommand{\vbtwo}{6} % even power 
  \renewcommand{\vbthree}{9} % odd power
  \renewcommand{\vbfour}{-20} % even coefficient 
  \renewcommand{\vbfive}{-15} % odd coefficient
}{
  \ifnumequal{\value{rolldice}}{1}{
    % variables 
    \renewcommand{\vbone}{7}
    \renewcommand{\vbtwo}{4}
    \renewcommand{\vbthree}{10}
    \renewcommand{\vbfour}{21}
    \renewcommand{\vbfive}{21}
  }{
    \ifnumequal{\value{rolldice}}{2}{
      % variables 
      \renewcommand{\vbone}{5}
      \renewcommand{\vbtwo}{4}
      \renewcommand{\vbthree}{9}
      \renewcommand{\vbfour}{10}
      \renewcommand{\vbfive}{-5}
    }{
      % variables 
      \renewcommand{\vbone}{8}
      \renewcommand{\vbtwo}{8}
      \renewcommand{\vbthree}{5}
      \renewcommand{\vbfour}{70}
      \renewcommand{\vbfive}{-28}
    }
  }
}

\ADD\vbone{1}\a
\DIVIDE\vbtwo{2}\k
\INTEGERDIVISION\vbthree{2}\m\n 

\question[4] Find the \textit{coefficients} of $x^{\vbtwo}$ and $x^{\vbthree}$ in the expansion of $(1+x)^{\vbone}\cdot(1-x)^{\a}$

\insertQR{}

\watchout

\ifprintanswers
\fi 

\begin{solution}[\halfpage]
  \begin{align}
    (1+x)^\vbone\cdot (1-x)^{\a} &= 
    (1+x)^{\vbone}\cdot (1-x)^{\vbone} \cdot (1-x) \nonumber\\
    &= \left[ (1+x)\cdot (1-x) \right]^{\vbone}\cdot (1-x) \nonumber\\
    &= (1-x^2)^{\vbone}\cdot (1-x) \\
    &= \underbrace{
    \left[ \sum_{k=0}^{\vbone}\binom\vbone{k} (-1)^{k}\cdot (x^2)^{k} \right]
    }_{\texttt{A}}\cdot \underbrace{(1-x)}_{\texttt{B}}
  \end{align}
  
  Now, we can see that only even powers of $x$ can come from $A$ - terms like $x^2$, $x^4$. 
  If that even term is multiplied with the $x$ in $(1-x)$, then we can get an odd power of $x$
  
    Hence, in order to get a $x^{\vbtwo}$ term 
    \begin{align}
      x^{2k} &= x^{\vbtwo} \Rightarrow k = \k \\
      \Rightarrow T_{\k} &= \binom\vbone\k (-1)^{\k} x^{\vbtwo} \text{ or } K_{\k} = \vbfour
    \end{align}
    
    And, to get $x^{\vbthree}$
    \begin{align}
      x^{2k + 1} &= x^{\vbthree} \Rightarrow k = \m \\
      \Rightarrow T_{\m} &= \binom\vbone\m (-1)^{\m + 1} x^{\vbthree} \text{ or } K_{\m} = \vbfive
    \end{align}
\end{solution}

\ifprintrubric
  \begin{table}
  	\begin{tabular}{ p{5cm}p{5cm} }
  		\toprule % in brief (4-6 words), what should a grader be looking for for insights & formulations
  		  \sc{\textcolor{blue}{Insight}} & \sc{\textcolor{blue}{Formulation}} \\ 
  		\midrule % ***** Insights & formulations ******
  		\toprule % final numerical answers for the various versions
        \sc{\textcolor{blue}{If question has $\ldots$}} & \sc{\textcolor{blue}{Final answer}} \\
  		\midrule % ***** Numerical answers (below) **********
  		\bottomrule
  	\end{tabular}
  \end{table}
\fi
