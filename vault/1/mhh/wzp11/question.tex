
% \noprintanswers
% \setcounter{rolldice}{1}
% \printrubric

\ifnumequal{\value{rolldice}}{0}{
  % variables 
  \renewcommand{\vbone}{4}
  \renewcommand{\vbtwo}{4}
  \renewcommand{\vbfour}{8}
  \renewcommand{\vbfive}{\dfrac{1}{2}}
}{
  \ifnumequal{\value{rolldice}}{1}{
    % variables 
    \renewcommand{\vbone}{9}
    \renewcommand{\vbtwo}{3}
    \renewcommand{\vbfour}{27}
    \renewcommand{\vbfive}{\dfrac{1}{3^{\frac{2}{9}}}}
  }{
    \ifnumequal{\value{rolldice}}{2}{
      % variables 
      \renewcommand{\vbone}{4}
      \renewcommand{\vbtwo}{4}
      \renewcommand{\vbfour}{8}
      \renewcommand{\vbfive}{\dfrac{1}{2}}
    }{
      % variables 
      \renewcommand{\vbone}{9}
      \renewcommand{\vbtwo}{3}
      \renewcommand{\vbfour}{27}
      \renewcommand{\vbfive}{\dfrac{1}{3^{\frac{2}{9}}}}
    }
  }
}

\POWER\vbone\vbtwo\p
\renewcommand{\vbthree}{\left(\vbone^{-\frac{3}{2}}\right)}
\FRACTIONSIMPLIFY\vbtwo\vbfour\j\k

\question[3] Simplify $\p^{-\left(\vbone^{-\frac{3}{2}}\right)}$. \texttt{Hint:} $\p = \vbone^\vbtwo$

\insertQR[-25pt]{qrc}

\watchout

\begin{solution}[\mcq]
	\begin{align}
		\p^{-\vbthree} &= \left(\vbone^\vbtwo\right)^{-\vbthree} = 
		\vbone^{-\left( \vbtwo\times\vbthree\right)}
	\end{align}
	Now, \begin{align}
		\vbtwo\times\vbthree &= \dfrac{\vbtwo}{\vbone^{\frac{3}{2}}} = \dfrac{\vbtwo}{\vbfour} = \dfrac{\j}{\k}
	\end{align}
	And therefore \begin{align}
		\vbone^{-\left( \vbtwo\times\vbthree \right)} &= \vbone^{-\frac{\j}{\k}} 
		= \dfrac{1}{\vbone^{\frac{\j}{\k}}} = \vbfive
	\end{align}
\end{solution}

\ifprintrubric
  \begin{table}
  	\begin{tabular}{ p{5cm}p{5cm} }
  		\toprule % final numerical answers for the various versions
        \sc{\textcolor{blue}{If question has $\ldots$}} & \sc{\textcolor{blue}{Final answer}} \\
  		\midrule % ***** Numerical answers (below) **********
  			$256^{-\left( 4^{-\frac{3}{2}}\right)}$ & $\frac{1}{2}$ \\
  			$729^{-\left( 9^{-\frac{3}{2}}\right)}$ & $\dfrac{1}{3^{\frac{2}{9}}}$ \\
  		\bottomrule
  	\end{tabular}
  \end{table}
\fi
