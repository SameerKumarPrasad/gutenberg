

\ifnumequal{\value{rolldice}}{0}{
  % variables 
  \renewcommand{\va}{4}
  \renewcommand{\vb}{4}
  \renewcommand{\vd}{8}
  \renewcommand{\ve}{\dfrac{1}{2}}
}{
  \ifnumequal{\value{rolldice}}{1}{
    % variables 
    \renewcommand{\va}{9}
    \renewcommand{\vb}{3}
    \renewcommand{\vd}{27}
    \renewcommand{\ve}{\dfrac{1}{3^{\frac{2}{9}}}}
  }{
    \ifnumequal{\value{rolldice}}{2}{
      % variables 
      \renewcommand{\va}{4}
      \renewcommand{\vb}{4}
      \renewcommand{\vd}{8}
      \renewcommand{\ve}{\dfrac{1}{2}}
    }{
      % variables 
      \renewcommand{\va}{9}
      \renewcommand{\vb}{3}
      \renewcommand{\vd}{27}
      \renewcommand{\ve}{\dfrac{1}{3^{\frac{2}{9}}}}
    }
  }
}

\POWER\va\vb\p
\renewcommand{\vc}{\left(\va^{-\frac{3}{2}}\right)}
\FRACTIONSIMPLIFY\vb\vd\j\k

\question[3] Simplify 
\[ \p^{-\left(\va^{-\frac{3}{2}}\right)} \]
\begin{calcaid}$\p=\va^\vb$\end{calcaid}

\watchout

\begin{solution}[\mcq]
	\begin{align}
		\p^{-\vc} &= \left(\va^\vb\right)^{-\vc} = 
		\va^{-\left( \vb\times\vc\right)}
	\end{align}
	Now, \begin{align}
		\vb\times\vc &= \dfrac{\vb}{\va^{\frac{3}{2}}} = \dfrac{\vb}{\vd} = \dfrac{\j}{\k}
	\end{align}
	And therefore \begin{align}
		\va^{-\left( \vb\times\vc \right)} &= \va^{-\frac{\j}{\k}} 
		= \dfrac{1}{\va^{\frac{\j}{\k}}} = \ve
	\end{align}
\end{solution}

\ifprintanswers\begin{codex}$\ve$\end{codex}\fi
