% This is an empty shell file placed for you by the 'examiner' script.
% You can now fill in the TeX for your question here.

% Now, down to brasstacks. ** Writing good solutions is an Art **. 
% Eventually, you will find your own style. But here are some thoughts 
% to get you started: 
%
%   1. Write the solution as if you are writing it for your favorite
%      14-17 year old to help him/her understand. Could be your nephew, 
%      your niece, a cousin perhaps or probably even you when you 
%      were that age. Just write for them.
%
%   2. Use margin-notes to "talk" to students about the critical insights
%      in the question. The tone can be - in fact, should be - informal
%
%   3. Don't shy away from creating margin-figures you think will help
%      students understand. Yes, it is a little more work per question. 
%      But the question & solution will be written only once. Make that
%      attempt at writing a solution count.
%
%   4. At the same time, do not be too verbose. A long solution can
%      - at first sight - make the student think, "God, that is a lot to know".
%      Our aim is not to scare students. Rather, our aim should be to 
%      create many "Aha!" moments everyday in classrooms around the world
% 
%   5. Ensure that there are *no spelling mistakes anywhere*. We are an 
%      education company. Bad spellings suggest that we ourselves 
%      don't have any education. Also, use American spellings by default
% 
%   6. If a question has multiple parts, then first delete lines 40-41
%   7. If a question does not have parts, then first delete lines 43-69

%\noprintanswers
%\setcounter{rolldice}{3}

\ifnumequal{\value{rolldice}}{0}{
  % variables 
  \renewcommand{\vbone}{2} %a
  \renewcommand{\vbtwo}{1} %b
}{
	\ifnumequal{\value{rolldice}}{1}{
		\renewcommand{\vbone}{1}
    \renewcommand{\vbtwo}{3}
	}{
	  \ifnumequal{\value{rolldice}}{2}{
      \renewcommand{\vbone}{1}
      \renewcommand{\vbtwo}{2}
	  }{
      \renewcommand{\vbone}{2}
      \renewcommand{\vbtwo}{2}
	  }
	}
}

% variables for question text
\gcalcexpr[0]\af{\vbone * (\vbtwo + 1) + 1}
\gcalcexpr[0]\as{(\vbone * 2)*((\vbtwo * 2) + 1) + 1}
\gcalcexpr[0]\at{(\vbone * 3)*((\vbtwo * 3) + 1) + 1}
\gcalcexpr[0]\afo{(\vbone * 4)*((\vbtwo * 4) + 1) + 1}

% variables for solution
\gcalcexpr[0]\tn{\vbone * \vbtwo}
\gcalcexpr[0]\tp{6 / \vbone}
\gcalcexpr[0]\tq{2 * \vbtwo}
\gcalcexpr[0]\tr{3 * \vbtwo} 
\gcalcexpr[0]\ts{3 * (\vbtwo + 1)}
\gcalcexpr[0]\tu{\tp+ \vbtwo + 3}

\question[3] Find an expression for the sum of the first $n$ terms of the series 
$\af + \as + \at + \afo + \ldots$. \texttt{Hint:} Each term in the series is of the 
form $T_k = ak\cdot(bk + 1) + 1,\, k \geq 1$ and $a,b \in\aleph$

\watchout
\insertQR{QRC}

\ifprintanswers
\fi 

\begin{solution}[\halfpage]
  If you found out that $(a,b) = (\vbone, \vbtwo)$, then you are on your way. The sum of the 
  first $n$ terms is given by
	\begin{align}
		S_n &= \sum_{k=1}^{n}\left(\vbone n\cdot(\vbtwo n+1)+1\right) = \sum_{k=1}^{n}(\tn n^2+\vbone n + 1) \\
		&= \tn\cdot\sum_{k=1}^{n}n^2 + \vbone\cdot\sum_{k=1}^{n}n + \sum_{k=1}^{n}1 \\
		&= \tn\cdot\eSumOfSquares{n} + \vbone\cdot\eSumOfN{n} + n \\
		&= \dfrac{\vbone n}{6}\cdot\left[ \vbtwo\cdot(n+1)\cdot(2n+1) + 3\cdot(n+1) + \tp \right] \\
		&= \dfrac{\vbone n}{6}\cdot\left( \tq n^2 + \tr n + \vbtwo + 3n + 3 + \tp \right) \\
		&= \dfrac{\vbone n}{6}\cdot\left( \tq n^2 + \ts n + \tu \right)
	\end{align}
\end{solution}
