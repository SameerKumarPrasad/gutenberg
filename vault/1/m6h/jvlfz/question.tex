% This is an empty shell file placed for you by the 'examiner' script.
% You can now fill in the TeX for your question here.

% Now, down to brasstacks. ** Writing good solutions is an Art **. 
% Eventually, you will find your own style. But here are some thoughts 
% to get you started: 
%
%   1. Write the solution as if you are writing it for your favorite
%      14-17 year old to help him/her understand. Could be your nephew, 
%      your niece, a cousin perhaps or probably even you when you 
%      were that age. Just write for them.
%
%   2. Use margin-notes to "talk" to students about the critical insights
%      in the question. The tone can be - in fact, should be - informal
%
%   3. Don't shy away from creating margin-figures you think will help
%      students understand. Yes, it is a little more work per question. 
%      But the question & solution will be written only once. Make that
%      attempt at writing a solution count.
%
%   4. At the same time, do not be too verbose. A long solution can
%      - at first sight - make the student think, "God, that is a lot to know".
%      Our aim is not to scare students. Rather, our aim should be to 
%      create many "Aha!" moments everyday in classrooms around the world
% 
%   5. Ensure that there are *no spelling mistakes anywhere*. We are an 
%      education company. Bad spellings suggest that we ourselves 
%      don't have any education. Also, use American spellings by default
% 
%   6. If a question has multiple parts, then first delete lines 40-41
%   7. If a question does not have parts, then first delete lines 43-69

%\noprintanswers
%\setcounter{rolldice}{2}

\ifnumequal{\value{rolldice}}{0}{
  % variables 
  \renewcommand{\vbone}{2} %p
}{
	\ifnumequal{\value{rolldice}}{1}{
		\renewcommand{\vbone}{3}
	}{
	  \ifnumequal{\value{rolldice}}{2}{
      \renewcommand{\vbone}{1}
	  }{
      \renewcommand{\vbone}{2}
	  }
	}
}

\gcalcexpr[0]\tp{\vbone * \vbone}
\gcalcexpr[0]\tq{2 * \tp}
\gcalcexpr[0]\tr{3 * \tp}
\gcalcexpr[0]\ts{\tp - 6}

% variables for question text
\gcalcexpr[0]\af{\tp -1}
\gcalcexpr[0]\as{4*\tp - 1}
\gcalcexpr[0]\at{9*\tp - 1}
\gcalcexpr[0]\afo{16*\tp - 1}

\question[3] Find an expression for the sum of the first $n$ terms of the series 
$\af + \as + \at + \afo + \ldots$. \texttt{Hint:} Each term in the series is of the 
form $a_k = (pk-1)\cdot(qk+1),\, k \geq 1$ and $p,q \in\aleph$

\watchout
\insertQR{QRC}

\ifprintanswers
\fi 

\begin{solution}[\halfpage]
	The $k^{th}$ term of the series is of the form $a_k = (\vbone k-1)\cdot(\vbone k+1),\, k \geq 1$
	Hence, the sum of the first $n$ terms s given by
	\begin{align}
		S_n &= \sum_{k=1}^{n}(\vbone k-1)\cdot(\vbone k+1) = \sum_{k=1}^{n}(\tp k^2-1) \\
		&= \tp\cdot\sum_{k=1}^{n}k^2 - \sum_{k=1}^{n}1 \\
		&= \tp\cdot\eSumOfSquares{n} - n \\
		&= \dfrac{n}{6}\cdot\left[\tp\cdot(2n^2 + 3n + 1) - 6\right] = \dfrac{n}{6}\left[ \tq n^2 + \tr n + \ts \right] 
	\end{align}
\end{solution}
