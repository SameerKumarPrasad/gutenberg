
\ifnumequal{\value{rolldice}}{0}{
  \renewcommand{\va}{-2}
  \renewcommand{\vb}{3}
}{
  \ifnumequal{\value{rolldice}}{1}{
    \renewcommand{\va}{-2}
    \renewcommand{\vb}{5}
  }{
    \ifnumequal{\value{rolldice}}{2}{
      \renewcommand{\va}{-1}
      \renewcommand{\vb}{5}
    }{
      \renewcommand{\va}{-2}
      \renewcommand{\vb}{6}
    }
  }
} 

\ADD\va\vb\vc
\MULTIPLY\vb{-\va}\vd
\SQUARE\va\ve
\SQUARE\vb\vf
\MULTIPLY\ve\va\vx
\MULTIPLY\vf\vb\vy
\FRACTIONSIMPLIFY\vc{2}\vp\vq

\FRACMULT\vp\vq\vf{1}\vg\vh
\MULTIPLY\vd\vb\vi

\FRACMULT\vp\vq\ve{1}\vj\vk
\MULTIPLY\vd\va\vl

\FRACADD\vg\vh\vi{1}\a\b
\FRACMINUS\a\b\vy{3}\c\d

\FRACADD\vj\vk\vl{1}\e\f
\FRACMINUS\e\f\vx{3}\g\h

\FRACMINUS\c\d\g\h\vm\vn

\question[3] Compute the area of the parabolic segment cut by the straight line 
$y=\vc x + \vd$ off the parabola $y=x^2$

\watchout

\ifprintanswers
  \figinit{pt}
  \def\Xmin{-26.66666}
  \def\Ymin{-6.13026}
  \def\Xmax{53.33333}
  \def\Ymax{93.86973}
  \def\Xori{26.66666}
  \def\Yori{6.13026}
  \figpt0:(\Xori,\Yori)
  \figpt 100: (30,30)
  \figpt 101: (11,14)
  \figpt 102: (68,74)
  \figdrawbegin{}
  \def\Xmaxx{\Xmax} % To customize the position
  \def\Ymaxx{\Ymax} % of the arrow-heads of the axes.
  \figset arrowhead(length=4, fillmode=yes) % styling the arrowheads
  \figdrawaxes 0(\Xmin, \Xmaxx, \Ymin, \Ymaxx)
  \figdrawlineC(
  0 2.29885,
  2.75862 5.07332,
  5.51724 7.84780,
  8.27586 10.62227,
  11.03448 13.39674,
  13.79310 16.17122,
  16.55172 18.94569,
  19.31034 21.72017,
  22.06896 24.49464,
  24.82758 27.26912,
  27.58620 30.04359,
  30.34482 32.81807,
  33.10344 35.59254,
  35.86206 38.36702,
  38.62068 41.14149,
  41.37931 43.91597,
  44.13793 46.69044,
  46.89655 49.46492,
  49.65517 52.23939,
  52.41379 55.01387,
  55.17241 57.78834,
  57.93103 60.56282,
  60.68965 63.33729,
  63.44827 66.11177,
  66.20689 68.88624,
  68.96551 71.66072,
  71.72413 74.43519,
  74.48275 77.20967,
  77.24137 79.98414,
  79.99999 82.75862
  )
  \figdrawlineC(
  0 29.59770,
  2.75862 24.99350,
  5.51724 20.89159,
  8.27586 17.29194,
  11.03448 14.19458,
  13.79310 11.59949,
  16.55172 9.50667,
  19.31034 7.91613,
  22.06896 6.82787,
  24.82758 6.24188,
  27.58620 6.15817,
  30.34482 6.57673,
  33.10344 7.49757,
  35.86206 8.92068,
  38.62068 10.84607,
  41.37931 13.27374,
  44.13793 16.20368,
  46.89655 19.63590,
  49.65517 23.57039,
  52.41379 28.00716,
  55.17241 32.94620,
  57.93103 38.38752,
  60.68965 44.33111,
  63.44827 50.77698,
  66.20689 57.72513,
  68.96551 65.17555,
  71.72413 73.12825,
  74.48275 81.58322,
  77.24137 90.54047,
  79.99999 99.99999
  )
  \figdrawend
  \figvisu{\figBoxA}{}{%
  \figptsaxes 1:0(\Xmin, \Xmaxx, \Ymin, \Ymaxx)
  \figwritee 1:(5pt)     \figwriten 2:(5pt)
  \figptsaxes 1:0(\Xmin, \Xmax, \Ymin, \Ymax)
  \figwritee 100: $R$(4)
  \figwritew 101: ${(\va,\ve)}$(4)
  \figwritee 102: ${(\vb,\vf)}$(4)
  }
  \vspace{0.7cm}
  \centerline{\box\figBoxA}
\fi 

\begin{solution}[\halfpage]
   The two curves intersect at points where
   \begin{align}
      x^2 &= \vc x +\vd\implies x^2 - \vc x - \vd = 0 \\
      \text{or } x &= \va,\vb
   \end{align}
   
   And therefore, the area $(A)$ of region $R$ (see figure) is 
   \begin{align}
     A &= \int_{\va}^\vb (\vc x+\vd-x^2)\ud x \\
       &= \left[ \WRITEFRAC\vp\vq x^2+\vd x-\dfrac{x^3}{3} \right]_{\va}^\vb \\
       &= \left( \WRITEFRAC\vg\vh + \vi - \dfrac\vy{3} \right) 
       - \left( \WRITEFRAC\vj\vk + \vl - \dfrac\vx{3} \right) = \WRITEFRAC\vm\vn
   \end{align}
\end{solution}

\ifprintanswers
  \begin{codex}
    $\WRITEFRAC\vm\vn$
  \end{codex}
\fi
