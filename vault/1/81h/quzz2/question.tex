% This is an empty shell file placed for you by the 'examiner' script.
% You can now fill in the TeX for your question here.

% Now, down to brasstacks. ** Writing good solutions is an Art **. 
% Eventually, you will find your own style. But here are some thoughts 
% to get you started: 
%
%   1. Write the solution as if you are writing it for your favorite
%      14-17 year old to help him/her understand. Could be your nephew, 
%      your niece, a cousin perhaps or probably even you when you 
%      were that age. Just write for them.
%
%   2. Use margin-notes to "talk" to students about the critical insights
%      in the question. The tone can be - in fact, should be - informal
%
%   3. Don't shy away from creating margin-figures you think will help
%      students understand. Yes, it is a little more work per question. 
%      But the question & solution will be written only once. Make that
%      attempt at writing a solution count.
%
%   4. At the same time, do not be too verbose. A long solution can
%      - at first sight - make the student think, "God, that is a lot to know".
%      Our aim is not to scare students. Rather, our aim should be to 
%      create many "Aha!" moments everyday in classrooms around the world
% 
%   5. Ensure that there are *no spelling mistakes anywhere*. We are an 
%      education company. Bad spellings suggest that we ourselves 
%      don't have any education. Also, use American spellings by default
% 
%   6. If a question has multiple parts, then first delete lines 40-41
%   7. If a question does not have parts, then first delete lines 43-69

\question[3] Compute the area of the parabolic segment cut by the straight line 
$y=2x + 3$ off the parabola $y=x^2$

\insertQR{QRC}

\ifprintanswers
  \begin{marginfigure}
% 1. Definition of characteristic points
\figinit{pt}
\def\Xmin{-26.66666}
\def\Ymin{-6.13026}
\def\Xmax{53.33333}
\def\Ymax{93.86973}
\def\Xori{26.66666}
\def\Yori{6.13026}
\figpt0:(\Xori,\Yori)
\figpt 100: (30,30)
\figpt 101: (11,14)
\figpt 102: (68,74)
% 2. Creation of the graphical file
\figdrawbegin{}
\def\Xmaxx{\Xmax} % To customize the position
\def\Ymaxx{\Ymax} % of the arrow-heads of the axes.
\figset arrowhead(length=4, fillmode=yes) % styling the arrowheads
\figdrawaxes 0(\Xmin, \Xmaxx, \Ymin, \Ymaxx)
\figdrawlineC(
0 2.29885,
2.75862 5.07332,
5.51724 7.84780,
8.27586 10.62227,
11.03448 13.39674,
13.79310 16.17122,
16.55172 18.94569,
19.31034 21.72017,
22.06896 24.49464,
24.82758 27.26912,
27.58620 30.04359,
30.34482 32.81807,
33.10344 35.59254,
35.86206 38.36702,
38.62068 41.14149,
41.37931 43.91597,
44.13793 46.69044,
46.89655 49.46492,
49.65517 52.23939,
52.41379 55.01387,
55.17241 57.78834,
57.93103 60.56282,
60.68965 63.33729,
63.44827 66.11177,
66.20689 68.88624,
68.96551 71.66072,
71.72413 74.43519,
74.48275 77.20967,
77.24137 79.98414,
79.99999 82.75862
)
\figdrawlineC(
0 29.59770,
2.75862 24.99350,
5.51724 20.89159,
8.27586 17.29194,
11.03448 14.19458,
13.79310 11.59949,
16.55172 9.50667,
19.31034 7.91613,
22.06896 6.82787,
24.82758 6.24188,
27.58620 6.15817,
30.34482 6.57673,
33.10344 7.49757,
35.86206 8.92068,
38.62068 10.84607,
41.37931 13.27374,
44.13793 16.20368,
46.89655 19.63590,
49.65517 23.57039,
52.41379 28.00716,
55.17241 32.94620,
57.93103 38.38752,
60.68965 44.33111,
63.44827 50.77698,
66.20689 57.72513,
68.96551 65.17555,
71.72413 73.12825,
74.48275 81.58322,
77.24137 90.54047,
79.99999 99.99999
)
\figdrawend
% 3. Writing text on the figure
\figvisu{\figBoxA}{}{%
\figptsaxes 1:0(\Xmin, \Xmaxx, \Ymin, \Ymaxx)
% Points 1 and 2 are the end points of the arrows
\figwritee 1:(5pt)     \figwriten 2:(5pt)
\figptsaxes 1:0(\Xmin, \Xmax, \Ymin, \Ymax)
\figwritee 100: $R$(4)
\figwritew 101: ${(-1,1)}$(4)
\figwritee 102: ${(3,9)}$(4)
}
\centerline{\box\figBoxA}

  \end{marginfigure}
\fi 

\begin{solution}[\halfpage]
   The two curves intersect at points where
   \begin{align}
      x^2 &= 2x +3 \\
      \Rightarrow x^2 - 2x - 3 &= 0 \\
      \Rightarrow x &= -1, 3 
   \end{align}
   
   And therefore, the required area - of region $R$ - is 
   \begin{align}
     Area(R) &= \int_{-1}^3 (2x+3-x^2)\ud x \\
             &= \left[ x^2+3x-\dfrac{x^3}{3} \right]_{-1}^3 \\
             &= \left( 9 + 27 -9 \right) - \left( 1-3+\dfrac{1}{3}\right) \\
             &= 10\frac{2}{3}
   \end{align}
\end{solution}
