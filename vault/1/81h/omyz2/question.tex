% This is an empty shell file placed for you by the 'examiner' script.
% You can now fill in the TeX for your question here.

% Now, down to brasstacks. ** Writing good solutions is an Art **. 
% Eventually, you will find your own style. But here are some thoughts 
% to get you started: 
%
%   1. Write the solution as if you are writing it for your favorite
%      14-17 year old to help him/her understand. Could be your nephew, 
%      your niece, a cousin perhaps or probably even you when you 
%      were that age. Just write for them.
%
%   2. Use margin-notes to "talk" to students about the critical insights
%      in the question. The tone can be - in fact, should be - informal
%
%   3. Don't shy away from creating margin-figures you think will help
%      students understand. Yes, it is a little more work per question. 
%      But the question & solution will be written only once. Make that
%      attempt at writing a solution count.
%
%   4. At the same time, do not be too verbose. A long solution can
%      - at first sight - make the student think, "God, that is a lot to know".
%      Our aim is not to scare students. Rather, our aim should be to 
%      create many "Aha!" moments everyday in classrooms around the world
% 
%   5. Ensure that there are *no spelling mistakes anywhere*. We are an 
%      education company. Bad spellings suggest that we ourselves 
%      don't have any education. Also, use American spellings by default
% 
%   6. If a question has multiple parts, then first delete lines 40-41
%   7. If a question does not have parts, then first delete lines 43-69

\question  Let $R$ be the region in the first quadrant enclosed by the graphs
$f(x) = 8x^3$ and $g(x)=sin(\pi x)$, as shown in the figure

\ifprintanswers
\fi
  \begin{marginfigure}
  \def\GraphLegend{$A = \left( \frac{1}{2}, 1 \right)$}
  % 1. Definition of characteristic points
\figinit{pt}
\def\Xmin{0}
\def\Ymin{0}
\def\Xmax{79.99999}
\def\Ymax{99.99999}
\def\Xori{0}
\def\Yori{0}
\figpt0:(\Xori,\Yori)
% 2. Creation of the graphical file
\figdrawbegin{}
\def\Xmaxx{\Xmax} % To customize the position
\def\Ymaxx{\Ymax} % of the arrow-heads of the axes.
\figset arrowhead(length=4, fillmode=yes) % styling the arrowheads
\figdrawaxes 0(\Xmin, \Xmaxx, \Ymin, \Ymaxx)
\figpt 100: (25,25)
\figpt 101: (63,53)
\figdrawlineC(
0 0,
2.75862 3.75883,
5.51724 7.50180,
8.27586 11.21308,
11.03448 14.87700,
13.79310 18.47809,
16.55172 22.00115,
19.31034 25.43128,
22.06896 28.75402,
24.82758 31.95531,
27.58620 35.02165,
30.34482 37.94008,
33.10344 40.69828,
35.86206 43.28460,
38.62068 45.68811,
41.37931 47.89866,
44.13793 49.90693,
46.89655 51.70442,
49.65517 53.28355,
52.41379 54.63765,
55.17241 55.76099,
57.93103 56.64884,
60.68965 57.29744,
63.44827 57.70406,
66.20689 57.86697,
68.96551 57.78549,
71.72413 57.45997,
74.48275 56.89178,
77.24137 56.08331,
79.99999 55.03799
)
\figdrawlineC(
0 0,
2.75862 .00410,
5.51724 .03280,
8.27586 .11070,
11.03448 .26241,
13.79310 .51252,
16.55172 .88564,
19.31034 1.40637,
22.06896 2.09930,
24.82758 2.98905,
27.58620 4.10020,
30.34482 5.45737,
33.10344 7.08516,
35.86206 9.00815,
38.62068 11.25097,
41.37931 13.83820,
44.13793 16.79445,
46.89655 20.14432,
49.65517 23.91241,
52.41379 28.12333,
55.17241 32.80167,
57.93103 37.97203,
60.68965 43.65902,
63.44827 49.88724,
66.20689 56.68129,
68.96551 64.06576,
71.72413 72.06527,
74.48275 80.70441,
77.24137 90.00779,
79.99999 99.99999
)
\figdrawend
% 3. Writing text on the figure
\figvisu{\figBoxA}{\GraphLegend}{%
\figptsaxes 1:0(\Xmin, \Xmaxx, \Ymin, \Ymaxx)
% Points 1 and 2 are the end points of the arrows
\figwritee 1:(5pt)     \figwriten 2:(5pt)
\figptsaxes 1:0(\Xmin, \Xmax, \Ymin, \Ymax)
% Points 1 and 2 are the first two end points of the axes
\figwrites 1:$.6$(5pt) \figwritew 2:$1.7$(8pt)
% Points 3 and 4 are the two other end points of the axes
\figpttraC 3:=0/\Xmin,0/ \figwrites 3:$0$(5pt) 
\figpttraC 4:=0/0,\Ymin/ \figwritew 4:$0$(8pt)
\figwritee 100:$R$(8pt)
\figwriten 101:$A$(8pt)
}
\centerline{\box\figBoxA}
\end{marginfigure} 


\begin{parts}
\part[3]     Write the equation for the line tangent to $f(x)$ at $x=\frac{1}{2}$
\insertQR{QRC}
\begin{solution}[\halfpage]
  	The said tangent would be tangent at the point $(\frac{1}{2}, f(\frac{1}{2}))$.  	
  	Which means, at $\left(\frac{1}{2}, 8\cdot\left( \frac{1}{2}\right)^3\right)$ 
  	= $\left(\frac{1}{2},1\right)$
  	
  	Also,
  	\begin{align}
  		\text{Slope of the tangent} &= \dfrac{\ud f(x)}{\ud x} \\
  		&= \dfrac{\ud}{\ud x}8x^3 = 24x^2 \\
  		&= 24\cdot\left(\frac{1}{2}\right)^2 = 6 \text{ when } x=\frac{1}{2}
  	\end{align}
  	
  	The equation of the line, therefore, is
  	
  	\begin{align}
  		\dfrac{y-1}{x-\frac{1}{2}} &= 6 \\
  		\Rightarrow y &= 6x - 2
  	\end{align}
  	
  \end{solution}

\part[3]     Find the area of $R$
\insertQR{QRC}
\begin{solution}[\halfpage]
  	From the figure, we can see that the curves $f(x)$ and $g(x)$ intersect when $x=\dfrac{1}{2}$
  	
  	And therefore, the area $R$ contained between them is given by
  	\begin{align}
  		R &= \int_0^{\frac{1}{2}}\sin(\pi x)\ud x - \int_0^{\frac{1}{2}}8x^3\ud x \\
  		  &= \left[ \dfrac{-\cos(\pi x)}{\pi}\right]_0^{\frac{1}{2}} -
  		     \left[ \dfrac{8}{4}x^4\right]_0^{\frac{1}{2}} \\
  		  &= \dfrac{1}{\pi} - \dfrac{1}{8} = 0.19
  	\end{align}
  \end{solution}

\part[3]     Write, but do not evaluate, an integral expression for the volume of the solid
  generated when $R$ is rotated about the horizontal line $y=1$
\insertQR{QRC}

\begin{solution}[\halfpage]
  	The volume of the said solid is the sum of volumes generated by rotating 
  	each of the infinitesimally thin horizontal strips that make up $R$ around
  	$y = 1$
  	
  	The catch is that each of the infinitesimally thin strips is at a different
  	distance from $y=1$. And therefore, the circumfrence they cover when rotated is
  	different. And as a result, the volume they contribute towards the total is 
  	different
  	
  	Now, at any given $y$, the circumference a strip covers is given by
  	\begin{align}
  		L &= 2\cdot\pi\cdot(1-y)
	\end{align}
	
  	
  	Also, if 
  	\begin{align}
		y = 8x^3 \text{ then } x &= \dfrac{1}{2}\sqrt[3]{y} \\
		y = \sin(\pi x) \text{ then } x &= \dfrac{1}{\pi}\sin^{-1}y
  	\end{align}
  	
  	And therefore, for any given $x$, the volume generated by an infinitisimal strip of $R$ is given by
  	\begin{align}
  		\ud V &= \dfrac{1}{2}\sqrt[3]{y}\cdot(2\pi\cdot(1-y))\ud y -
  		         \dfrac{1}{\pi}\sin^{-1}y\cdot(2\pi\cdot(1-y))\ud y \\
  		      &= 2\pi\cdot\left[ \dfrac{1}{2}\sqrt[3]{y}(1-y)\ud y -
  		         \dfrac{1}{\pi}(1-y)\sin^{-1}y\ud y \right]
  	\end{align}
  	
  	The total volume generated is hence
  	\begin{align}
  		V &= 2\pi\cdot\left[ \dfrac{1}{2}\int_0^1\sqrt[3]{y}(1-y)\ud y -
  		         \dfrac{1}{\pi}\int_0^1(1-y)\sin^{-1}y\ud y \right]
  	\end{align}
  \end{solution}

\end{parts}
