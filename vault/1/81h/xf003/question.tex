
\ifnumequal{\value{rolldice}}{0}{
  \renewcommand{\va}{10}
  \renewcommand{\vb}{6}
}{
  \ifnumequal{\value{rolldice}}{1}{
    \renewcommand{\va}{8}
    \renewcommand{\vb}{10}
  }{
    \ifnumequal{\value{rolldice}}{2}{
      \renewcommand{\va}{12}
      \renewcommand{\vb}{10}
    }{
      \renewcommand{\va}{14}
      \renewcommand{\vb}{8}
    }
  }
} 

\question[3] Compute the area contained \textbf{within} the parabola (see figure) if 
the parabola's base $=a=\va$ cm and its height $=h=\vb$ cm. The base is perpendicular to the 
axis of the parabola. 

\watchout 

\figinit{pt}
\def\Xmin{-35.00000}
\def\Ymin{-10.00000}
\def\Xmax{35.00000}
\def\Ymax{70.00000}
\def\Xori{35.00000}
\def\Yori{10.00000}
\figpt0:(\Xori,\Yori)
\figpt 100: (0,70)
\figpt 101: (70,70)
\figpt 102: (70,10)
\figpt 103: (52,70)
\figpt 104: (70,40)
\figdrawbegin{}
\def\Xmaxx{\Xmax} % To customize the position
\def\Ymaxx{\Ymax} % of the arrow-heads of the axes.
\figset arrowhead(length=4, fillmode=yes) % styling the arrowheads
\figdrawaxes 0(\Xmin, \Xmaxx, \Ymin, \Ymaxx)
\figdrawline[100,101,102]
\figdrawarrowhead [100,101]
\figdrawarrowhead [101,100]
\figdrawarrowhead [101,102]
\figdrawarrowhead [102,101]
\figdrawlineC(
0 70.00000,
2.41379 62.00951,
4.82758 54.58977,
7.24137 47.74078,
9.65517 41.46254,
12.06896 35.75505,
14.48275 30.61831,
16.89655 26.05231,
19.31034 22.05707,
21.72413 18.63258,
24.13793 15.77883,
26.55172 13.49583,
28.96551 11.78359,
31.37931 10.64209,
33.79310 10.07134,
36.20689 10.07134,
38.62068 10.64209,
41.03448 11.78359,
43.44827 13.49583,
45.86206 15.77883,
48.27586 18.63258,
50.68965 22.05707,
53.10344 26.05231,
55.51724 30.61831,
57.93103 35.75505,
60.34482 41.46254,
62.75862 47.74078,
65.17241 54.58977,
67.58620 62.00951,
69.99999 69.99999
)
\figdrawend
\figvisu{\figBoxA}{}{%
\figptsaxes 1:0(\Xmin, \Xmaxx, \Ymin, \Ymaxx)
\figwritee 1:(5pt)     \figwriten 2:(5pt)
\figptsaxes 1:0(\Xmin, \Xmax, \Ymin, \Ymax)
\figwriten 103: $a$(4)
\figwritee 104: $h$(2)
}

\ifprintanswers
  \begin{marginfigure}
    \centerline{\box\figBoxA}
  \end{marginfigure}
\else
  \vspace{1cm}
  \centerline{\box\figBoxA}
\fi


\DIVIDE\va{2}\vc
\SQUARE\vc\vd
\MULTIPLY\vd\vc\ve 

\FRACTIONSIMPLIFY\vb\vd\vp\vq
\MULTIPLY\vq{3}\vr

\MULTIPLY\vb\vc\vi
\FRACMULT\vp\vr\ve{1}\vj\vk
\FRACMINUS\vi{1}\vj\vk\vl\vm
\FRACMULT\vl\vm{2}{1}\vn\vo

\begin{solution}[\halfpage]
  As the parabola is symmetrical about the $y$-axis, it spans from $x=-\vc$ to $x=\vc$.

  Moreover, this parabola is of the form 
    \[ y = k\cdot x^2 \]

  And in particular 
    \[ \vb = k\cdot \vc^2 \implies k = \WRITEFRAC\vp\vq \] 

  The required area $(A)$ is therefore 
  \begin{align}
    A &= \overbrace{\int_{-\vc}^{\vc}\vb\cdot dx}^{\text{area under }y=\vb} 
    - \underbrace{\int_{-\vc}^{\vc}\WRITEFRAC\vp\vq\cdot x^2\cdot dx}_{\text{area under parabola}} \\
    &= \underbrace{2\cdot\left[ \vb x - \WRITEFRAC\vp\vr\cdot x^3 \right]_0^{\vc}}_{\text{symmetrical about y-axis}} 
    = \WRITEFRAC\vn\vo\text{ cm}^2
  \end{align}
\end{solution}

\ifprintanswers
  \begin{codex}
    $\WRITEFRAC\vn\vo\text{ cm}^2$ 
  \end{codex}
\fi

