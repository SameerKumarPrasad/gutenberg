


\ifnumequal{\value{rolldice}}{0}{
  % variables 
  \renewcommand{\vbone}{2}
  \renewcommand{\vbtwo}{4}
  \renewcommand{\vbthree}{5}
  \renewcommand{\vbfour}{10}
}{
  \ifnumequal{\value{rolldice}}{1}{
    % variables 
    \renewcommand{\vbone}{5}
    \renewcommand{\vbtwo}{3}
    \renewcommand{\vbthree}{2}
    \renewcommand{\vbfour}{9}
  }{
    \ifnumequal{\value{rolldice}}{2}{
      % variables 
      \renewcommand{\vbone}{2}
      \renewcommand{\vbtwo}{1}
      \renewcommand{\vbthree}{3}
      \renewcommand{\vbfour}{11}
    }{
      % variables 
      \renewcommand{\vbone}{3}
      \renewcommand{\vbtwo}{2}
      \renewcommand{\vbthree}{4}
      \renewcommand{\vbfour}{8}
    }
  }
}

\SUBTRACT\vbthree{1}\p
\ADD\vbone\vbtwo\a
\SUBTRACT\vbfour\p\b
\DIVIDE\b\a\c

\question[3] In the expansion of $(1+x)^{\vbfour}$, for what value of $m$ are 
the \textit{coefficients} of the $(\vbone m + 1)$th and $(\vbtwo m + \vbthree)$th terms equal?

\insertQR{}

\watchout

\ifprintanswers
\fi 

\begin{solution}[\halfpage]
  In the expansion of $(1+x)^{\vbfour}$, where $m \in [0,\vbfour]$, $T_m$ is the $(m+1)$th term and its 
  value is given by
  \begin{align}
    T_m &= \encr\vbfour{m}\cdot 1^{m}\cdot x^{\vbfour - m}
  \end{align}
  
  But we are interested only in the coefficients. And therefore, if $K_m$ be the coefficient of the $(m+1)$th term, then $K_m = \encr\vbfour{m}$
  
  What we want is $\underbrace{K_{\vbone m}}_{\texttt{(\vbone m + 1)th}} = 
  \underbrace{K_{\vbtwo m + \p}}_{\texttt{(\vbtwo m + \vbthree)th}} \Rightarrow
  \encr\vbfour{\vbone m} = \encr\vbfour{\vbtwo m + \p}$
  
  
  And obviously, we are looking for two $distinct$ terms 
  $\Rightarrow \vbone m \neq \vbtwo m + \p$
  
  The other option is $\vbone m = \vbfour - (\vbtwo m + \p)$ because we know that 
  $\encr{N}{p} = \encr{N}{N-p}$
  
  Given this, 
  \begin{align}
    \vbone m &= \vbfour - (\vbtwo m + \p) \\
    \Rightarrow m &= \dfrac{\vbfour - \p}{\vbone + \vbtwo} = \c
  \end{align}
\end{solution}

\ifprintrubric
  \begin{table}
  	\begin{tabular}{ p{5cm}p{5cm} }
  		\toprule % in brief (4-6 words), what should a grader be looking for for insights & formulations
  		  \sc{\textcolor{blue}{Insight}} & \sc{\textcolor{blue}{Formulation}} \\ 
  		\midrule % ***** Insights & formulations ******
  		\toprule % final numerical answers for the various versions
        \sc{\textcolor{blue}{If question has $\ldots$}} & \sc{\textcolor{blue}{Final answer}} \\
  		\midrule % ***** Numerical answers (below) **********
  		\bottomrule
  	\end{tabular}
  \end{table}
\fi
