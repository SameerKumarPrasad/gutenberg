

\ifnumequal{\value{rolldice}}{0}{
  % variables 
  \renewcommand{\va}{22}
  \renewcommand{\vb}{43}
  \renewcommand{\vc}{11}
  \renewcommand{\vd}{3}
  \renewcommand{\ve}{2}
  \renewcommand{\vf}{441}
  \renewcommand{\vg}{10000}
  \renewcommand{\vh}{4.41}
  \renewcommand{\vi}{242}
  \renewcommand{\vj}{441}
}{
  \ifnumequal{\value{rolldice}}{1}{
    % variables 
    \renewcommand{\va}{15}
    \renewcommand{\vb}{39}
    \renewcommand{\vc}{12}
    \renewcommand{\vd}{4}
    \renewcommand{\ve}{3}
    \renewcommand{\vf}{237}
    \renewcommand{\vg}{5000}
    \renewcommand{\vh}{4.74}
    \renewcommand{\vi}{30}
    \renewcommand{\vj}{79}
  }{
    \ifnumequal{\value{rolldice}}{2}{
      % variables 
      \renewcommand{\va}{18}
      \renewcommand{\vb}{45}
      \renewcommand{\vc}{16}
      \renewcommand{\vd}{5}
      \renewcommand{\ve}{2}
      \renewcommand{\vf}{1019}
      \renewcommand{\vg}{10000}
      \renewcommand{\vh}{10.19}
      \renewcommand{\vi}{720}
      \renewcommand{\vj}{1019}
    }{
      % variables 
      \renewcommand{\va}{24}
      \renewcommand{\vb}{38}
      \renewcommand{\vc}{16}
      \renewcommand{\vd}{4}
      \renewcommand{\ve}{3}
      \renewcommand{\vf}{13}
      \renewcommand{\vg}{200}
      \renewcommand{\vh}{6.5}
      \renewcommand{\vi}{192}
      \renewcommand{\vj}{325}
    }
  }
}

\FRACADD\va{100}\vb{100}\a\b
\FRACMINUS{1}{1}\a\b\c\d

\question 
An insurance company charges younger drivers a higher premium because younger drivers 
as a group tend to have more accidents. The company has 3 age groups - Group A includes those
under 25 years of age, \va\% of all its policyholders. Group B includes those that are 25-39 years
old,  \vb\% of its policyholders. Group C includes those over 40. Company 
records show that in any given \textit{1 year} period, \vc\% of its Group A policyholders have 
an accident. Figures for Groups B and C are \vd\% and \ve\% respectively

\watchout[-80pt]

\ifprintanswers
  % stuff to be shown only in the answer key - like explanatory margin figures
\fi 

\begin{parts}
  \part[2] What percentage of the company's policyholders are expected to have an accident during 
  the next 12 months?

  \ifprintanswers
  	  \begin{tabular}{cccc}
  	     \toprule
  		 & A & B & C \\
  		 \midrule
  		 P($x \in X$) & $\WRITEFRAC\va{100}$ & $\WRITEFRAC\vb{100}$ & $\WRITEFRAC\c\d$ \\
  		 P(accident $\vert x \in X$) & $\WRITEFRAC\vc{100}$ & 
                                     $\WRITEFRAC\vd{100}$ & 
                                     $\WRITEFRAC\ve{100}$ \\
  		 \bottomrule
  		\end{tabular}
  \fi
\begin{solution}[\mcq]
  	If $P(E)$ be the probability of an accident occuring in the next 12 months and $P(A)$, $P(B)$
  	, $P(C)$ the probabilities that a policyholder belongs to Group A, B or C respectively, then
  	
  	\begin{align}
      P(C) &= 1 - \left( \dfrac\va{100} + \dfrac\vb{100} \right) = \dfrac\c\d
  	\end{align}

    And therefore, 
  	\begin{align}
  		P(E) &= P(E \vert A)\cdot P(A) + P(E \vert B)\cdot P(B) + P(E \vert C)\cdot P(C) \\
           &= \WRITEFRAC\vc{100}\times\WRITEFRAC\va{100} + 
              \WRITEFRAC\vd{100}\times\WRITEFRAC\vb{100} + 
              \WRITEFRAC\ve{100}\times\dfrac\c\d \\
           &= \dfrac\vf\vg = \vh\%
  	\end{align}
  \end{solution}

  \part[2] Suppose Mr. X has just had a car accident. If he is one of the company's policyholders, 
  then what is the probability that he is under 25?

\begin{solution}[\halfpage]
      The probability we are seeking is $P(A\vert E)$, which Baye's theorem tells us would be
      \begin{align}
         P(A\vert E) &= \dfrac{P(E\vert A)\cdot P(A)}{P(E)} \\
                     &= \left(\dfrac{\WRITEFRAC\vc{100} \times \WRITEFRAC\va{100} }{\frac\vf\vg}\right) = \dfrac\vi\vj 
      \end{align}
  \end{solution}

\end{parts}

\ifprintanswers
  \begin{codex}
    $(a)\quad\vh\%\qquad(b)\quad\dfrac\vi\vj$
  \end{codex}
\fi

