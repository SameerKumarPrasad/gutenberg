% This is an empty shell file placed for you by the 'examiner' script.
% You can now fill in the TeX for your question here.

% Now, down to brasstacks. ** Writing good solutions is an Art **. 
% Eventually, you will find your own style. But here are some thoughts 
% to get you started: 
%
%   1. Write the solution as if you are writing it for your favorite
%      14-17 year old to help him/her understand. Could be your nephew, 
%      your niece, a cousin perhaps or probably even you when you 
%      were that age. Just write for them.
%
%   2. Use margin-notes to "talk" to students about the critical insights
%      in the question. The tone can be - in fact, should be - informal
%
%   3. Don't shy away from creating margin-figures you think will help
%      students understand. Yes, it is a little more work per question. 
%      But the question & solution will be written only once. Make that
%      attempt at writing a solution count.
%
%   4. At the same time, do not be too verbose. A long solution can
%      - at first sight - make the student think, "God, that is a lot to know".
%      Our aim is not to scare students. Rather, our aim should be to 
%      create many "Aha!" moments everyday in classrooms around the world
% 
%   5. Ensure that there are *no spelling mistakes anywhere*. We are an 
%      education company. Bad spellings suggest that we ourselves 
%      don't have any education. Also, use American spellings by default
% 
%   6. If a question has multiple parts, then first delete lines 40-41
%   7. If a question does not have parts, then first delete lines 43-69

\question[5]
  My friend Fred has two urns - A and B - each containing colored balls. Urn A
  has 30\% brown, 20\% yellow, 20\% red, 10\% green, 10\% orange and 10\% tan 
  colored balls. Urn B has 13\% brown, 14\% yellow, 13\% red, 20\% green,
  16\% orange and 20\% blue balls. 
  
  Fred draws one ball each from each of the urns. They turn out to be yellow and
  green. What is the probability that the yellow ball was from Urn A?

\insertQR{QRC}

\ifprintanswers
  % stuff to be shown only in the answer key - like explanatory margin figures
  \begin{table}
    \begin{tabular}{ccc}
       \toprule
       & A & B \\
       \midrule
       Brown & 0.3 & 0.13 \\
       Yellow & 0.2 & 0.14 \\
       Red & 0.2 & 0.13 \\
       Green & 0.1 & 0.2 \\
       Orange & 0.1 & 0.16 \\
       Tan & 0.1 & 0 \\
       Blue & 0 & 0.2 \\
       \bottomrule
    \end{tabular}
  \end{table}
\fi 

\begin{solution}[\fullpage]
   If $Y$ be the event that a yellow ball is drawn and $G$ the event that a green ball is 
   drawn, then the probability we seek is $P(Y_A \vert YG)$, where $YG$ is the event
   that a yellow and a green ball are drawn and $Y_A$ the event that the yellow ball is 
   drawn from Urn A
   
   \begin{align}
      P(Y_A \vert YG) &= \dfrac{P(YG \vert Y_A)\cdot P(Y_A)}{P(YG \vert Y_A)\cdot P(Y_A) + P(YG \vert Y_B)\cdot P(Y_B)}
   \end{align}
   
   Now, $P(YG \vert Y_A) = P(G_B)$. Why? Because the only way to get a green and a yellow ball 
   \textit{given that} the yellow ball is drawn from Urn A is to draw a green ball from Urn B.
   
   Similarly, $P(YG \vert Y_B) = P(G_A)$. And therefore, 
   
   \begin{align}
       P(Y_A \vert YG) &= \dfrac{P(G_B)\cdot P(Y_A)}{P(G_B)\cdot P(Y_A) + P(G_A)\cdot P(Y_B)} \\
                       &= \dfrac{0.2 \times 0.2}{0.2\times 0.2 + 0.1\times 0.14} \\
                       &= 0.741 = 74.1\%
   \end{align}
   That is an unexpectedly high probability. But it is what it is.
\end{solution}

