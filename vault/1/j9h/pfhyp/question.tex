


\ifnumequal{\value{rolldice}}{0}{
  % variables 
  \renewcommand{\vbone}{1} % p
  \renewcommand{\vbtwo}{3} % q 
  \renewcommand{\vbthree}{5} % r 
}{
  \ifnumequal{\value{rolldice}}{1}{
    % variables 
    \renewcommand{\vbone}{6}
    \renewcommand{\vbtwo}{14}
    \renewcommand{\vbthree}{21}
  }{
    \ifnumequal{\value{rolldice}}{2}{
      % variables 
      \renewcommand{\vbone}{20}
      \renewcommand{\vbtwo}{35}
      \renewcommand{\vbthree}{42}
    }{
      % variables 
      \renewcommand{\vbone}{5}
      \renewcommand{\vbtwo}{10}
      \renewcommand{\vbthree}{14}
    }
  }
}

\gcalcexpr[0]\tp{\vbone + \vbtwo}
\gcalcexpr[0]\tq{\vbtwo + \vbthree}
\gcalcexpr[2]\tr{\tq / \tp}
\gcalcexpr[0]\ts{(\tr * \vbone + \vbthree) / (\vbtwo - \tr * \vbone)}

\question[4] If some \textit{consecutive} coefficients in the expansion of $(1+x)^n$ are in the 
ratio $\vbone:\vbtwo:\vbthree$, then find $n$

\insertQR[-15pt]{QRC}

\watchout

\ifprintanswers
\fi 

\begin{solution}[\halfpage]
	Let the $(m-1)^{\text{th}}$, $m^{\text{th}}$ and $(m+1)^{\text{th}}$ terms be in the given ratio
	\begin{align}
		\dfrac{\encr{n}{m-1}}{\encr{n}{m}} &= \dfrac{\vbone}{\vbtwo} \\
		\Rightarrow \dfrac{\fncr{n}{m-1}}{\fncr{n}{m}} &= \dfrac{\vbone}{\vbtwo} \\
		\Rightarrow \dfrac{m}{n-m+1} &= \dfrac{\vbone}{\vbtwo} \Rightarrow \tp\cdot m = \vbone\cdot(n+1)
	\end{align}
	Similarly,
	\begin{align}
		\dfrac{\encr{n}{m}}{\encr{n}{m+1}} &= \dfrac{\vbtwo}{\vbthree} \Rightarrow \dfrac{m+1}{n-m} = \dfrac{\vbtwo}{\vbthree} \\
		\Rightarrow \tq\cdot m &= \vbtwo\cdot n - \vbthree \\
		\text{And therefore, } \dfrac{\tq\cdot m}{\tp\cdot m} &= \dfrac{\vbtwo\cdot n - \vbthree}{\vbone\cdot(n+1)} \\
		\Rightarrow n &= \ts
	\end{align}
\end{solution}

