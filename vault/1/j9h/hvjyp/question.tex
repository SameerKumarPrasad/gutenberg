

\ifnumequal{\value{rolldice}}{0}{
  % variables 
  \renewcommand{\vbone}{ARLINGTON}
  \renewcommand{\vbtwo}{TORONTO}
  \renewcommand{\vbthree}{9}
  \renewcommand{\vbfour}{7}
  \renewcommand{\vbfive}{ON} % common
  \renewcommand{\vbsix}{1}
  \renewcommand{\vbseven}{1}
}{
  \ifnumequal{\value{rolldice}}{1}{
    % variables 
    \renewcommand{\vbone}{ISTANBUL}
    \renewcommand{\vbtwo}{MILAN}
    \renewcommand{\vbthree}{8}
    \renewcommand{\vbfour}{5}
    \renewcommand{\vbfive}{AN}
    \renewcommand{\vbsix}{1}
    \renewcommand{\vbseven}{1}
  }{
    \ifnumequal{\value{rolldice}}{2}{
      % variables 
      \renewcommand{\vbone}{CASABLANCA}
      \renewcommand{\vbtwo}{CARACAS}
      \renewcommand{\vbthree}{10}
      \renewcommand{\vbfour}{7}
      \renewcommand{\vbfive}{CA}
      \renewcommand{\vbsix}{2}
      \renewcommand{\vbseven}{2}
    }{
      % variables 
      \renewcommand{\vbone}{MUMBAI}
      \renewcommand{\vbtwo}{SHANGHAI}
      \renewcommand{\vbthree}{6}
      \renewcommand{\vbfour}{8}
      \renewcommand{\vbfive}{AI}
      \renewcommand{\vbsix}{1}
      \renewcommand{\vbseven}{1}
    }
  }
}

\SUBTRACT\vbthree{1}\p
\SUBTRACT\vbfour{1}\q
\FRACADD\vbsix\p\vbseven\q\m\n
\FRACDIV\vbsix\p\m\n\r\s

\question[4] A letter is known to have come either from $\vbone$ or from $\vbtwo$. 
But on the envelope, only two consecutive letters - $\vbfive$ - are visible. 
What is the probability that the letter came from $\vbone$? 


\watchout[-40pt]

\ifprintanswers
  \marginnote[10cm]{ Surprised? Intuitively, one might think that the answer 
  is $\frac{1}{2}$. But intuition is an unreliable guide when calculating 
  probabilities. Place your trust only in Maths}
\fi 

\begin{solution}[\halfpage]
Let $C_1$ be the event that the letter comes from $\vbone$ and $C_2$ 
the event that it comes from $\vbtwo$. Also, let $V$ be the event that 
the \textit{visible} pair of characters is $\vbfive$

Now, here is the catch. You have to count the \textbf{number of pairs} of characters 
that result if one goes from left-to-right along a city name. And here is the second catch. 
If the city's name is $N$ characters long, then there are $N-1$ such pairs - the last such 
pair being made up of the second last \& last characters

Hence, we can very quickly write 
\begin{align}
  P(V\vert\,C_1) &= \WRITEFRAC\vbsix\p \\ 
  P(V\vert\,C_2) &= \WRITEFRAC\vbseven\q \\ 
  P(C_1) = P(C_2) &= \dfrac{1}{2}
\end{align}
  The required probability, $P(C_1\vert\,V)$ is therefore
  \begin{align}
    P(C_1\vert\,V) &= \dfrac{P(V\vert\,C_1)\cdot P(C_1)}{P(V\vert\,C_1)\cdot P(C_1) + P(V\vert\,C_2)\cdot P(C_2)} \\
    &= \dfrac{\WRITEFRAC\vbsix\p\times\frac{1}{2}}
        {\frac{1}{2}\cdot\left( \WRITEFRAC\vbsix\p + \WRITEFRAC\vbseven\q \right)} = \WRITEFRAC\r\s
  \end{align}
  
\end{solution}

