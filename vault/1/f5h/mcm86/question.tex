

\ifnumodd{\value{rolldice}}{
  % variables 
  \renewcommand{\vbone}{8}
  \renewcommand{\vbthree}{1}
  \renewcommand{\vbfour}{ }
  \renewcommand{\vbfive}{1}
  \renewcommand{\vbsix}{\sqrt{2}}
}{
    \renewcommand{\vbone}{12}
    \renewcommand{\vbthree}{\dfrac{1}{\sqrt{3}}}
    \renewcommand{\vbfour}{\sqrt{3}}
    \renewcommand{\vbfive}{\sqrt{3}}
    \renewcommand{\vbsix}{2}
}

\DIVIDE\vbone{2}\vbtwo

\question[3] Without using a calculator, find the value of $\tan\frac{\pi}{\vbone}$

\watchout

\ifprintanswers
\fi 

\begin{solution}[\halfpage]
  We know the value of $\tan\frac\pi\vbtwo$. And furthermore, $\frac\pi\vbtwo = 2\times\frac\pi\vbone$
	\begin{align}
		\therefore\tan\frac{\pi}{\vbtwo} = \tan \left(2\cdot\frac{\pi}{\vbone}\right) &= \frac{2\tan\frac{\pi}{\vbone}}{1-\tan^2\frac{\pi}{\vbone}} = \vbthree \\
  \end{align}

  Now, if we let $\tan\frac\pi\vbone = x$, then 
  \begin{align}
    \dfrac{2x}{1-x^2} &= \vbthree \Rightarrow
		x^2 + 2\times\vbfour x - 1 = 0 \\
    \text{ or } x &= -\vbfive \pm \vbsix 
	\end{align}
	As $\tan\theta > 0 \,\forall \,\theta \in (0,\frac{\pi}{2})$, $\tan\frac\pi\vbone$ can only be equal to
	$\vbsix - \vbfive$
\end{solution}
