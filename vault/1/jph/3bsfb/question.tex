


\ifnumequal{\value{rolldice}}{0}{
  \renewcommand{\vbone}{1}
  \renewcommand{\vbtwo}{2}
  \renewcommand{\vbthree}{3}
  \renewcommand{\vbsix}{\dfrac{1}{12}}
}{
  \ifnumequal{\value{rolldice}}{1}{
    \renewcommand{\vbone}{3}
    \renewcommand{\vbtwo}{2}
    \renewcommand{\vbthree}{4}
    \renewcommand{\vbsix}{\dfrac{1}{32}}
  }{
    \ifnumequal{\value{rolldice}}{2}{
      \renewcommand{\vbone}{5}
      \renewcommand{\vbtwo}{3}
      \renewcommand{\vbthree}{3}
      \renewcommand{\vbsix}{\dfrac{1}{27}}
    }{
      \renewcommand{\vbone}{7}
      \renewcommand{\vbtwo}{2}
      \renewcommand{\vbthree}{5}
      \renewcommand{\vbsix}{\dfrac{1}{80}}
    }
  }
}

\POWER\vbtwo\vbthree\a
\ADD\a\vbone\vbfour
\SUBTRACT\vbfour\vbone\vbfive
\FRACMINUS{1}{\vbthree}{1}{1}\b\c

\question[4] Evaluate the following limit \[ \lim_{x\to\vbfour}\dfrac{\sqrt[\vbthree]{x-\vbone}-\vbtwo}{x-\vbfour}\]
\texttt{Hint:}$\vbtwo^? + \vbone = \vbfour$

\watchout[-40pt]

\begin{solution}[\halfpage]
  Half the battle is won if you figured out that $\vbtwo = \sqrt[\vbthree]{\vbfour - \vbone}$. 
  Which means 
  \begin{align}
    \lim_{x\to\vbfour}\dfrac{\sqrt[\vbthree]{x-\vbone}-\vbtwo}{x-\vbfour} &=
    \lim_{x\to\vbfour}\dfrac{\sqrt[\vbthree]{x-\vbone}-\sqrt[\vbthree]{\vbfour-\vbone}}
    {(x-\vbone)-(\vbfour-\vbone)}
  \end{align}
  Now, if we define $\bar{x}=x-1$ and $A=\vbfour-\vbone = \vbfive$, then the above limit 
  becomes 
  \begin{align}
    \lim_{\bar{x}\to\vbfive}\dfrac{\sqrt[\vbthree]{\bar{x}}-\sqrt[\vbthree]{A}}{\bar{x}-A}
    &= \dfrac{1}{\vbthree}\cdot A^{\frac{1}{\vbthree}-1} \\
    &= \dfrac{1}{\vbthree}\cdot\vbfive^{\WRITEFRAC\b\c} = \vbsix
  \end{align}
\end{solution}

