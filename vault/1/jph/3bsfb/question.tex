


\ifnumequal{\value{rolldice}}{0}{
  \renewcommand{\va}{1}
  \renewcommand{\vb}{2}
  \renewcommand{\vc}{3}
  \renewcommand{\vf}{\dfrac{1}{12}}
}{
  \ifnumequal{\value{rolldice}}{1}{
    \renewcommand{\va}{3}
    \renewcommand{\vb}{2}
    \renewcommand{\vc}{4}
    \renewcommand{\vf}{\dfrac{1}{32}}
  }{
    \ifnumequal{\value{rolldice}}{2}{
      \renewcommand{\va}{5}
      \renewcommand{\vb}{3}
      \renewcommand{\vc}{3}
      \renewcommand{\vf}{\dfrac{1}{27}}
    }{
      \renewcommand{\va}{7}
      \renewcommand{\vb}{2}
      \renewcommand{\vc}{5}
      \renewcommand{\vf}{\dfrac{1}{80}}
    }
  }
}

\POWER\vb\vc\a
\ADD\a\va\vd
\SUBTRACT\vd\va\ve
\FRACMINUS{1}{\vc}{1}{1}\b\c

\question[4] Evaluate the following limit \[ \lim_{x\to\vd}\dfrac{\sqrt[\vc]{x-\va}-\vb}{x-\vd}\]

\watchout[-40pt]

\begin{solution}[\halfpage]
  Half the battle is won if you figured out that \[ \vb = \sqrt[\vc]\ve = \sqrt[\vc]{\vd - \va} \] 
  Which means 
  \begin{align}
    \lim_{x\to\vd}\dfrac{\sqrt[\vc]{x-\va}-\vb}{x-\vd} &=
    \lim_{x\to\vd}\dfrac{\sqrt[\vc]{x-\va}-\sqrt[\vc]{\vd-\va}}
    {(x-\va)-(\vd-\va)}
  \end{align}
  If we now define 
    \[ z=x-\va\implies z\to\ve\text{ as }x\to\vd\text{ and } A=\vd-\va = \ve \] 
  then the above limits becomes  
  \begin{align}
    \lim_{z\to\ve}\dfrac{\sqrt[\vc]{z}-\sqrt[\vc]{A}}{z-A}
    &= \dfrac{1}{\vc}\cdot A^{\frac{1}{\vc}-1} \\
    &= \dfrac{1}{\vc}\cdot\ve^{\WRITEFRAC[false]\b\c} = \vf
  \end{align}
\end{solution}

\ifprintanswers\begin{codex}$\vf$\end{codex}\fi
