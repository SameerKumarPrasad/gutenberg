\ifnumequal{\value{rolldice}}{0}{
  \renewcommand{\va}{1}
}{
  \ifnumequal{\value{rolldice}}{1}{
    \renewcommand{\va}{2}
  }{
    \ifnumequal{\value{rolldice}}{2}{
      \renewcommand{\va}{3}
    }{
      \renewcommand{\va}{4}
    }
  }
}

\SQUARE\va\vb
\MULTIPLY\va{2}\vc
\MULTIPLY\vb{4}\vd
\MULTIPLY\vc{2}\ve

\question[4] The figure below shows a \textbf{fixed circle} $C_1$ with equation $(x-\va)^2 + y^2 = \vb$ and a
\textbf{shrinking circle} $C_2$ with radius $r$ centered at the origin. $P$ is the point $(0,r)$, $Q$ is 
the upper point of intersection of the two circles and $R$ is where the line $PQ$ intersects the $x-$axis. What 
happens to $R$ as $C_2$ shrinks, that is, as $r\to 0^+$?

\watchout[-45pt]

\figinit{pt}
  \def\radA{33} %C_1
  \def\radB{20} %C_2
  \figpt 1: (0,0) 
  \figpt 2: (\radB,0)
  \figpt 10:$P$(0,\radA)
  \figptsintercirc 50 [1,\radA;2,\radB]
  \figvectP 99 [1,2]
  \figvectP 100 [10,51]
  \figptinterlines 60:$R$ [1,99;10,100]
  \figptcirc 70 :$C_1$: 2;\radB (225)
  \figptcirc 71 :$C_2$: 1;\radA (225)
\figdrawbegin{}
  \drawAxes{1}{-55}{85}{-55}{55}
  \figdrawcirc 1(\radA)
  \figdrawcirc 2(\radB)
  \figdrawline [10,60]
\figdrawend
\figvisu{\figBoxA}{}{%
  \figwritene 70:(3)
  \figwritene 71:(3)
  \figset write(mark=$\bullet$)
  \figwritene 51:$Q$(3)
  \figwritenw 10:(3)
  \figwrites 60:(3)
}

\vspace{1cm}
\centerline{\box\figBoxA}

\begin{solution}[\fullpage]
  The equation of circle $C_2$ is simply $x^2 + y^2 = r^2$.
  And therefore, $Q$ would be a point where 
  \begin{align}
    y_Q^2 &= \vb - (x_Q-\va)^2 = r^2 - x_Q^2 \\
    \implies \vb - (x_Q^2 - \vc x + \vb) &= r^2 - x_Q^2 \text{ or } x_Q = \dfrac{r^2} {\vc} \\
    \text{ and }y_Q = \sqrt{r^2 - x_Q^2} &= \sqrt{r^2 - \left( \dfrac{r^2}{\vc}\right)^2}
    = \dfrac{r}{\vc}\cdot\sqrt{\vd - r^2}
  \end{align}
  
  Next, we find the \textbf{equation of line PQ}
  \begin{align}
    \dfrac{y-r}{x-0} &= \dfrac{y_Q-y_P}{x_Q-x_P} = 
    \dfrac{\frac{r}{\vc}\cdot\sqrt{\vd-r^2}-r}{\frac{r^2}{\vc} - 0}
  \end{align}
  And as $R$ lies on $PQ$ and given that $R=(x_R,0)$, the \textbf{coordinates of R} would be 
  \begin{align}
    \dfrac{0-r}{x_R-0} &= \dfrac{\frac{r}{\vc}\cdot\sqrt{\vd-r^2}-r}{\frac{r^2}{\vc}} \\
    \implies x_R &= \dfrac{r^2}{\vc - \sqrt{\vd - r^2}}
  \end{align}
  Notice that point $R$ exists only if radius of $C_2\leq 2\times$ radius of $C_1$.
  
  That aside, what we need to find is \[ \lim_{r\to 0^+} x_R\] - which, in its current form, 
  is of the $\frac{0}{0}$ type
  \begin{align}
    \lim_{r\to 0^+}x_R &= \lim_{r\to 0^+}\dfrac{r^2}{\vc - \sqrt{\vd - r^2}} \\
    &= \lim_{r\to 0+}\dfrac{r^2}{\vc - \sqrt{\vd - r^2}}\times
    \dfrac{\vc + \sqrt{\vd - r^2}}{\vc + \sqrt{\vd - r^2}} \\
    &= \lim_{r\to 0+}(\vc + \sqrt{\vd - r^2}) = \ve
  \end{align}
  Which means, that as $C_2$ shrinks to zero, $R\to (\ve, 0)$
\end{solution}
\ifprintanswers\begin{codex}$R\to(\ve,0)$\end{codex}\fi
