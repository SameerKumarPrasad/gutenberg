\ifnumequal{\value{rolldice}}{0}{
  % variables 
  \renewcommand{\vbone}{1}
}{
  \ifnumequal{\value{rolldice}}{1}{
    % variables 
    \renewcommand{\vbone}{2}
  }{
    \ifnumequal{\value{rolldice}}{2}{
      % variables 
      \renewcommand{\vbone}{3}
    }{
      % variables 
      \renewcommand{\vbone}{4}
    }
  }
}

\SQUARE\vbone\vbtwo
\MULTIPLY\vbone{2}\vbthree
\MULTIPLY\vbtwo{4}\vbfour
\MULTIPLY\vbthree{2}\vbfive

\question The figure alongside shows a \textbf{fixed circle} $C_1$ with equation $(x-\vbone)^2 + y^2 = \vbtwo$ and a
\textbf{shrinking circle} $C_2$ with radius $r$ centered at the origin. $P$ is the point $(0,r)$, $Q$ is 
the upper point of intersection of the two circles and $R$ is where the line $PQ$ intersects the $x-$axis. What 
happens to $R$ as $C_2$ shrinks, that is, as $r\to 0^+$?
\watchout[-45pt]

  % stuff to be shown only in the answer key - like explanatory margin figures
  \begin{marginfigure}[-20pt]
    \figinit{pt}
      \def\radA{33} %C_1
      \def\radB{20} %C_2
      \figpt 1: (0,0) 
      \figpt 2: (\radB,0)
      \figpt 10:$P$(0,\radA)
      \figptsintercirc 50 [1,\radA;2,\radB]
      \figvectP 99 [1,2]
      \figvectP 100 [10,51]
      \figptinterlines 60:$R$ [1,99;10,100]
      \figptcirc 70 :$C_1$: 2;\radB (225)
      \figptcirc 71 :$C_2$: 1;\radA (225)
    \figdrawbegin{}
      \drawAxes{1}{-55}{85}{-55}{55}
      \figdrawcirc 1(\radA)
      \figdrawcirc 2(\radB)
      \figdrawline [10,60]
    \figdrawend
    \figvisu{\figBoxA}{}{%
      \figwritene 70:(3)
      \figwritene 71:(3)
      \figset write(mark=$\bullet$)
      \figwritene 51:$Q$(3)
      \figwritenw 10:(3)
      \figwrites 60:(3)
    }
    \centerline{\box\figBoxA}
  \end{marginfigure}

\begin{solution}
  The equation of circle $C_2$ is simply $x^2 + y^2 = r^2$.
  And therefore, $Q$ would be a point where 
  \begin{align}
    y_Q^2 &= \vbtwo - (x_Q-\vbone)^2 = r^2 - x_Q^2 \\
    \implies \vbtwo - (x_Q^2 - \vbthree x + \vbtwo) &= r^2 - x_Q^2 \text{ or } x_Q = \dfrac{r^2} {\vbthree} \\
    \text{ and }y_Q = \sqrt{r^2 - x_Q^2} &= \sqrt{r^2 - \left( \dfrac{r^2}{\vbthree}\right)^2}
    = \dfrac{r}{\vbthree}\cdot\sqrt{\vbfour - r^2}
  \end{align}
  
  Next, we find the \textbf{equation of line PQ}
  \begin{align}
    \dfrac{y-r}{x-0} &= \dfrac{y_Q-y_P}{x_Q-x_P} = 
    \dfrac{\frac{r}{\vbthree}\cdot\sqrt{\vbfour-r^2}-r}{\frac{r^2}{\vbthree} - 0}
  \end{align}
  And as $R$ lies on $PQ$ and given that $R=(x_R,0)$, the \textbf{coordinates of R} would be 
  \begin{align}
    \dfrac{0-r}{x_R-0} &= \dfrac{\frac{r}{\vbthree}\cdot\sqrt{\vbfour-r^2}-r}{\frac{r^2}{\vbthree}} \\
    \implies x_R &= \dfrac{r^2}{\vbthree - \sqrt{\vbfour - r^2}}
  \end{align}
  Notice that point $R$ exists only if radius of $C_2\leq 2\times$ radius of $C_1$.
  
  That aside, what we need to find is \[ \lim_{r\to 0^+} x_R\] - which, in its current form, 
  is of the $\frac{0}{0}$ type
  \begin{align}
    \lim_{r\to 0^+}x_R &= \lim_{r\to 0^+}\dfrac{r^2}{\vbthree - \sqrt{\vbfour - r^2}} \\
    &= \lim_{r\to 0+}\dfrac{r^2}{\vbthree - \sqrt{\vbfour - r^2}}\times
    \dfrac{\vbthree + \sqrt{\vbfour - r^2}}{\vbthree + \sqrt{\vbfour - r^2}} \\
 i   &= \lim_{r\to 0+}(\vbthree + \sqrt{\vbfour - r^2}) = \vbfive
  \end{align}
  Which means, that as $C_2$ shrinks to zero, $R\to (\vbfive, 0)$
\end{solution}
