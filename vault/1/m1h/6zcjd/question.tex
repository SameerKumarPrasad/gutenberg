% This is an empty shell file placed for you by the 'examiner' script.
% You can now fill in the TeX for your question here.

% Now, down to brasstacks. ** Writing good solutions is an Art **. 
% Eventually, you will find your own style. But here are some thoughts 
% to get you started: 
%
%   1. Write the solution as if you are writing it for your favorite
%      14-17 year old to help him/her understand. Could be your nephew, 
%      your niece, a cousin perhaps or probably even you when you 
%      were that age. Just write for them.
%
%   2. Use margin-notes to "talk" to students about the critical insights
%      in the question. The tone can be - in fact, should be - informal
%
%   3. Don't shy away from creating margin-figures you think will help
%      students understand. Yes, it is a little more work per question. 
%      But the question & solution will be written only once. Make that
%      attempt at writing a solution count.
%
%   4. At the same time, do not be too verbose. A long solution can
%      - at first sight - make the student think, "God, that is a lot to know".
%      Our aim is not to scare students. Rather, our aim should be to 
%      create many "Aha!" moments everyday in classrooms around the world
% 
%   5. Ensure that there are *no spelling mistakes anywhere*. We are an 
%      education company. Bad spellings suggest that we ourselves 
%      don't have any education. Also, use American spellings by default
% 
%   6. If a question has multiple parts, then first delete lines 40-41
%   7. If a question does not have parts, then first delete lines 43-69

% \setcounter{rolldice}{2}

\ifnumequal{\value{rolldice}}{0}{
  \renewcommand{\vbone}{3}
  \renewcommand{\vbtwo}{6}
  \renewcommand{\vbthree}{9}
  \renewcommand{\vbfour}{18}
  \renewcommand{\vbfive}{54}
  \renewcommand{\vbsix}{18}
}{
  \ifnumodd{\value{rolldice}}{
    \renewcommand{\vbone}{2}
    \renewcommand{\vbtwo}{4}
    \renewcommand{\vbthree}{6}
    \renewcommand{\vbfour}{12}
    \renewcommand{\vbfive}{24}
    \renewcommand{\vbsix}{8}
  }{
    \renewcommand{\vbone}{1}
    \renewcommand{\vbtwo}{2}
    \renewcommand{\vbthree}{3}
    \renewcommand{\vbfour}{6}
    \renewcommand{\vbfive}{6}
    \renewcommand{\vbsix}{2}
  }
}

\question[3] Using second derivative, find the points of maxima and minima for $y=x^2(a-\vbone x)^2$

\insertQR{QRC}
\watchout

\ifprintanswers
  % stuff to be shown only in the answer key - like explanatory margin figures
\fi 

\begin{solution}[\fullpage]
  \begin{align}
     y &= x^2\cdot(a-\vbone x)^2 \\
     \Rightarrow \ln y &= 2\cdot \left[\ln x + \ln (a-\vbone x)\right] \\
     \Rightarrow \dfrac{1}{y}\dfrac{\ud y}{\ud x} &= 2\left[ \dfrac{1}{x} - \dfrac{\vbone}{a-\vbone x}\right] \\
     \Rightarrow \dfrac{\ud y}{\ud x} &= 2\left[ \dfrac{1}{x} - \dfrac{\vbone}{a-\vbone x}\right]
                                         \cdot x^2\cdot (a-\vbone x)^2 \\
                                      &= 2\cdot(a-\vbtwo x)\cdot x \cdot (a-\vbone x)
  \end{align}
  As a first step, we can say that the extrema are at $x=\dfrac{a}{\vbtwo}$, $x=0$ and $x=\dfrac{a}{\vbone}$.
  To classify them as maxima or minima, we need to calculate the second derivative
  
  \begin{align}
     \dfrac{\ud y}{\ud x} &= 2(a-\vbtwo x)x(a-\vbone x) \\
            &= 2\cdot(a^2x - \vbthree ax^2 + \vbsix x^3) \\
     \Rightarrow \dfrac{\ud^2 y}{\ud x^2} &= 2\cdot(a^2 - \vbfour ax + \vbfive x^2) \\
     &= \left\lbrace
        \begin{array}{l l}
          2a^2 & \text{ when } x = 0 \Rightarrow \text{ minima } \\
          \frac{-a^2}{2} & \text{ when } x = \frac{a}{\vbtwo} \Rightarrow \text{ maxima } \\
          2a^2 & \text{ when } x = \dfrac{a}{\vbone} \Rightarrow \text{ minima }
        \end{array}\right.
  \end{align}
  
\end{solution}
