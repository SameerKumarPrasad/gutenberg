

\ifnumequal{\value{rolldice}}{0}{
  \renewcommand{\vbone}{18}
}{
  \ifnumequal{\value{rolldice}}{1}{
    \renewcommand{\vbone}{20}
  }{
    \ifnumequal{\value{rolldice}}{2}{
      \renewcommand{\vbone}{24}
    }{
      \renewcommand{\vbone}{30}
    }
  }
}

\question[3] A right conical funnel with a slant height of $\vbone$ cm must be made. What should 
be its height for it to have the greatest possible volume?

\watchout

\ifprintanswers
\fi 

\begin{solution}[\halfpage]
   If $L$ be the slant height of the funnel and $V$ its volume, then 
   \begin{align}
       L^2 &= h^2 + R^2 \Rightarrow R^2 = L^2 - h^2 \\ 
       V &= \dfrac{\pi}{3}\cdot R^2\cdot h = \dfrac{\pi}{3}\cdot(L^2 - h^2)\cdot h
   \end{align}
   where $R$ is the radius of the cone's base and $h$ the cone's height
   
   And so, 
   \begin{align}
      \dfrac{\ud V}{\ud h} &= \dfrac{\pi}{3}\cdot\dfrac{\ud}{\ud h}(L^2-h^2)\cdot h \\
             &= \dfrac{\pi}{3}(L^2 - 3h^2) \\
      \Rightarrow\text{ A maxima or a minima when } h &= \sqrt{\dfrac{\vbone^2}{3}} = \dfrac{\vbone}{\sqrt{3}} \\
      \text{Also }\dfrac{\ud^2 V}{\ud h^2} &= -2\pi\cdot h < 0 \Rightarrow \text{ maxima }
   \end{align}
   
   The funnel would therefore have the greatest volume if its height is $\frac{\vbone}{\sqrt{3}}$ cm
   
\end{solution}
