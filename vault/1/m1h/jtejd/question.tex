% This is an empty shell file placed for you by the 'examiner' script.
% You can now fill in the TeX for your question here.

% Now, down to brasstacks. ** Writing good solutions is an Art **. 
% Eventually, you will find your own style. But here are some thoughts 
% to get you started: 
%
%   1. Write the solution as if you are writing it for your favorite
%      14-17 year old to help him/her understand. Could be your nephew, 
%      your niece, a cousin perhaps or probably even you when you 
%      were that age. Just write for them.
%
%   2. Use margin-notes to "talk" to students about the critical insights
%      in the question. The tone can be - in fact, should be - informal
%
%   3. Don't shy away from creating margin-figures you think will help
%      students understand. Yes, it is a little more work per question. 
%      But the question & solution will be written only once. Make that
%      attempt at writing a solution count.
%
%   4. At the same time, do not be too verbose. A long solution can
%      - at first sight - make the student think, "God, that is a lot to know".
%      Our aim is not to scare students. Rather, our aim should be to 
%      create many "Aha!" moments everyday in classrooms around the world
% 
%   5. Ensure that there are *no spelling mistakes anywhere*. We are an 
%      education company. Bad spellings suggest that we ourselves 
%      don't have any education. Also, use American spellings by default
% 
%   6. If a question has multiple parts, then first delete lines 40-41
%   7. If a question does not have parts, then first delete lines 43-69

\ifnumequal{\value{rolldice}}{0}{
  \renewcommand{\vbone}{18}
}{
  \ifnumequal{\value{rolldice}}{1}{
    \renewcommand{\vbone}{20}
  }{
    \ifnumequal{\value{rolldice}}{2}{
      \renewcommand{\vbone}{24}
    }{
      \renewcommand{\vbone}{30}
    }
  }
}

\question[3] A right conical funnel with a slant height of $\vbone$ cm must be made. What should 
be its height for it to have the greatest possible volume?

\insertQR{QRC}
\watchout

\ifprintanswers
%  \begin{marginfigure}
%    \figinit{pt}
%      \figpt 99: (60,0)
%      \figpt 100: (60,60)
%      \figpt 101: (20,60)
%      \figpt 102: (100,60)
%      \figpt 103: (60,80)
%      \figpt 104: (80,70)
%      \figpt 105: (65,40)
%      \figpt 106: (75,40)
%    \figdrawbegin{}
%      \figdrawarcell 100 ; 40,5 (0,360,0)
%      \figdrawline [99,101,102,99]
%      \figdrawline [99,103]
%      \figset arrowhead(fill=true,length=4)
%      \figdrawarrow [99,103]
%      \figwriten 104:$R$(2)
%      \figwritee 105:$h$(2)
%      \figwritee 106:$L$(2)
%    \figdrawend
%    \figvisu{\figBoxA}{}{
%    }
%    \centerline{\box\figBoxA}
%  \end{marginfigure}
\fi 

\begin{solution}[\halfpage]
   If $L$ be the slant height of the funnel and $V$ its volume, then 
   \begin{align}
       L^2 &= h^2 + R^2 \Rightarrow R^2 = L^2 - h^2 \\ 
       V &= \dfrac{\pi}{3}\cdot R^2\cdot h = \dfrac{\pi}{3}\cdot(L^2 - h^2)\cdot h
   \end{align}
   where $R$ is the radius of the cone's base and $h$ the cone's height
   
   And so, 
   \begin{align}
      \dfrac{\ud V}{\ud h} &= \dfrac{\pi}{3}\cdot\dfrac{\ud}{\ud h}(L^2-h^2)\cdot h \\
             &= \dfrac{\pi}{3}(L^2 - 3h^2) \\
      \Rightarrow\text{ A maxima or a minima when } h &= \sqrt{\dfrac{\vbone^2}{3}} = \dfrac{\vbone}{\sqrt{3}} \\
      \text{Also }\dfrac{\ud^2 V}{\ud h^2} &= -2\pi\cdot h < 0 \Rightarrow \text{ maxima }
   \end{align}
   
   The funnel would therefore have the greatest volume if its height is $\frac{\vbone}{\sqrt{3}}$ cm
   
\end{solution}
