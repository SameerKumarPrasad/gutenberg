% This is an empty shell file placed for you by the 'examiner' script.
% You can now fill in the TeX for your question here.

% Now, down to brasstacks. ** Writing good solutions is an Art **. 
% Eventually, you will find your own style. But here are some thoughts 
% to get you started: 
%
%   1. Write the solution as if you are writing it for your favorite
%      14-17 year old to help him/her understand. Could be your nephew, 
%      your niece, a cousin perhaps or probably even you when you 
%      were that age. Just write for them.
%
%   2. Use margin-notes to "talk" to students about the critical insights
%      in the question. The tone can be - in fact, should be - informal
%
%   3. Don't shy away from creating margin-figures you think will help
%      students understand. Yes, it is a little more work per question. 
%      But the question & solution will be written only once. Make that
%      attempt at writing a solution count.
%
%   4. At the same time, do not be too verbose. A long solution can
%      - at first sight - make the student think, "God, that is a lot to know".
%      Our aim is not to scare students. Rather, our aim should be to 
%      create many "Aha!" moments everyday in classrooms around the world
% 
%   5. Ensure that there are *no spelling mistakes anywhere*. We are an 
%      education company. Bad spellings suggest that we ourselves 
%      don't have any education. Also, use American spellings by default
% 
%   6. If a question has multiple parts, then first delete lines 40-41
%   7. If a question does not have parts, then first delete lines 43-69

\ifnumequal{\value{rolldice}}{0}{
  \renewcommand{\vbone}{8}
  \renewcommand{\vbtwo}{4}
  \renewcommand{\vbthree}{16}
  \renewcommand{\vbfour}{64}
  \renewcommand{\vbfive}{48}
}{
  \ifnumequal{\value{rolldice}}{1}{
    \renewcommand{\vbone}{10}
    \renewcommand{\vbtwo}{5}
    \renewcommand{\vbthree}{20}
    \renewcommand{\vbfour}{100}
    \renewcommand{\vbfive}{60}
  }{
    \ifnumequal{\value{rolldice}}{2}{
      \renewcommand{\vbone}{6}
      \renewcommand{\vbtwo}{3}
      \renewcommand{\vbthree}{12}
      \renewcommand{\vbfour}{36}
      \renewcommand{\vbfive}{36}
    }{
      \renewcommand{\vbone}{12}
      \renewcommand{\vbtwo}{6}
      \renewcommand{\vbthree}{24}
      \renewcommand{\vbfour}{144}
      \renewcommand{\vbfive}{72}
    }
  }
}

\question[2] Break up the number $\vbone$ into two summands such that the sum of their cubes
is the least possible

\watchout 
\insertQR{QRC}

\ifprintanswers
  % stuff to be shown only in the answer key - like explanatory margin figures
\fi

\begin{solution}[\halfpage]
   Basically, we have to minimize $y = x^3 + (\vbone-x)^3$
   \begin{align}
      y &= x^3 + (\vbone -x)^3, \text{ then } \\
      \dydx &= 3x^2 - 3(\vbone - x)^2 \text{ and } \\
      \dnydxn{2} &= 6x + 6(\vbone-x)
   \end{align}
   
   Now, 
   \begin{align}
      \dydx &= 0 \text { when } 3x^2 - 3(\vbone - x)^2 = 0 \\
      \Rightarrow x^2 &= \vbfour + x^2 - \vbthree x \text{ or } x = \vbtwo \\
      \text{Also, }\left[\dnydxn{2}\right]_{x=\vbtwo} &= \vbfive > 0 \Rightarrow \text{ minima }
   \end{align}
   
   Hence, the required split is $\vbtwo$ and $\vbtwo$
\end{solution}
