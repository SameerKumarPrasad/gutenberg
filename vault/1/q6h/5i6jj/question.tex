

 %\noprintanswers

\ifnumequal{\value{rolldice}}{0}{
  % variables 
  \renewcommand{\vbone}{4} %a1
  \renewcommand{\vbtwo}{7} %b1
  \renewcommand{\vbthree}{1} %c1
}{
  \ifnumequal{\value{rolldice}}{1}{
    % variables 
    \renewcommand{\vbone}{3}
    \renewcommand{\vbtwo}{5}
    \renewcommand{\vbthree}{6}
  }{
    \ifnumequal{\value{rolldice}}{2}{
      % variables 
      \renewcommand{\vbone}{1}
      \renewcommand{\vbtwo}{3}
      \renewcommand{\vbthree}{1}
    }{
      % variables 
      \renewcommand{\vbone}{5}
      \renewcommand{\vbtwo}{1}
      \renewcommand{\vbthree}{3}
    }
  }
}


\FRACMINUS\vbone{3}\vbtwo{3}\pax\qax
\FRACL{-1}{\vbone}\pax\qax\pay\qay

\FRACMULT\vbone{1}{7}{9}\tp\tq
\FRACMINUS\tp\tq\vbthree{9}\pbx\qbx
\FRACL{-1}\vbone\pbx\qbx\pby\qby

\FRACMULT\vbtwo{1}{7}{12}\tp\tq
\FRACMINUS\vbthree{12}\tp\tq\pcx\qcx
\FRACL{2}{\vbtwo}\pcx\qcx\pcy\qcy


\FRACADD\pax\qax\pbx\qbx\tp\tq
\FRACDIV\tp\tq{2}{1}\pxx\qxx % X_x
\FRACADD\pay\qay\pby\qby\tp\tq
\FRACDIV\tp\tq{2}{1}\pxy\qxy % X_y

\FRACADD\pax\qax\pcx\qcx\tp\tq
\FRACDIV\tp\tq{2}{1}\pyx\qyx % Y_x
\FRACADD\pay\qay\pcy\qcy\tp\tq
\FRACDIV\tp\tq{2}{1}\pyy\qyy % Y_y

\FRACADD\pbx\qbx\pcx\qcx\tp\tq
\FRACDIV\tp\tq{2}{1}\pzx\qzx % Z_x
\FRACADD\pby\qby\pcy\qcy\tp\tq
\FRACDIV\tp\tq{2}{1}\pzy\qzy % Z_y

\FRACMINUS\pxy\qxy\pxx\qxx\mx\nx
\FRACMULT\pyx\qyx{-1}{2}\tp\tq
\FRACMINUS\pyy\qyy\tp\tq\my\ny
\FRACMULT\pzx\qzx{-7}{2}\tp\tq
\FRACMINUS\pzy\qzy\tp\tq\mz\nz

\FRACMINUS\my\ny\mx\nx\ansxp\ansxq
\FRACMULT\ansxp\ansxq{2}{3}\ansxp\ansxq
\FRACADD\ansxp\ansxq\mx\nx\ansyp\ansyq

\question Three lines - $L_1: x + y = \vbone$, $L_2: 2x - y + \vbtwo = 0$  
and $L_3: 2x - 7y + \vbthree = 0$ intersect at points $A,\, B$ and $C$ - \asif - to form 
a triangle. The mid-points of the sides - also shown - are $X,\,Y$ and $Z$


\watchout

  % stuff to be shown only in the answer key - like explanatory margin figures
  \begin{marginfigure}
    \figinit{pt}
      \figpt 10:$B$(-10, 0)
      \figpt 20:$A$(70, 20)
      \figpt 30:$C$(50, -20)
      \figpt 11: $X$(30,10)
      \figpt 12: $L_1$(30,10)
      \figpt 21: $Y$(60,0)
      \figpt 22: $L_2$(60,0)
      \figpt 31: $Z$(20,-10)
      \figpt 32: $L_3$(20,-10)
    \figdrawbegin{}
      \figdrawline [10, 20]
      \figdrawline [20, 30]
      \figdrawline [30, 10]
    \figdrawend
    \figvisu{\figBoxA}{}{%
      \figwritesw 10:(5pt)
      \figwritene 20:(5pt)
      \figwritese 30:(5pt)
      \figwritese 12:(3)
      \figwritew 22:(3)
      \figwritene 32:(3)
      \figset write(mark=$\bullet$)
      \figwritenw 11:(4)
      \figwritee 21:(2)
      \figwritesw 31:(4)
    }
    \centerline{\box\figBoxA}
  \end{marginfigure}

\begin{parts}
  \part[4] Find the coordinates of the vertices of the triangle 

\begin{solution}[\halfpage]
    This is simple. Take two lines at a time and solve first for $x$ and then for $y$. Which means, that if 
    \begin{align}
      L_1: y &= \vbone - x\;(m_1 = -1) \\
      L_2: y &= 2x + \vbtwo\;(m_2 = 2)\\
      L_3: y &= \frac{2}{7}x + \frac{\vbthree}{7}\;(m_3 = \frac{2}{7})
    \end{align}
    , then solving $L_1$ and $L_2$ gives 
    $A = \left( \WRITEFRAC\pax\qax,\,\WRITEFRAC\pay\qay \right)$,
     solving $L_1$ and $L_3$ gives  
    $B = \left( \WRITEFRAC\pbx\qbx,\,\WRITEFRAC\pby\qby \right)$ and 
     solving $L_2$ and $L_3$ gives 
    $C = \left( \WRITEFRAC\pcx\qcx,\,\WRITEFRAC\pcy\qcy \right)$,
  \end{solution}

  \part[2] Find the coordinates of the \textit{mid-points} of the sides of the triangle

\begin{solution}[\mcq]
  	A point $J$ in the middle of two other points - $M = (x_m, y_m)$ and $N = (x_n, y_n)$ has coordinates  
  	$\left( \frac{x_m + x_n}{2}, \frac{y_m + y_m}{2}\right)$

  	Given this, its easy to find coordinates of $X$, $Y$ and $Z$
  	\begin{align}
      X &= \left( \frac{x_A + x_b}{2},\,\frac{y_A + y_B}{2}\right) 
      = \left( \WRITEFRAC\pxx\qxx ,\,\WRITEFRAC\pxy\qxy \right) \\
      Y &= \left( \frac{x_A + x_C}{2},\,\frac{y_A + y_C}{2}\right) 
      = \left( \WRITEFRAC\pyx\qyx,\,\WRITEFRAC\pyy\qyy \right)  \\
      Z &= \left( \frac{x_B + x_C}{2},\,\frac{y_B + y_C}{2}\right)
      = \left( \WRITEFRAC\pzx\qzx,\,\WRITEFRAC\pzy\qzy \right)
  	\end{align}
  \end{solution}

  \part[3] Find the \textit{circumcenter} of the triangle

\begin{solution}[\halfpage]
  	The circumcenter of a triangle is the points where the \textit{perpendicular} bisectors 
  	of its sides intersect
  	
  	Now, we know the slopes of the sides from part (a). And we can therefore write equations for 
    the perpendiculars ( lets call them $P_1,\, P_2$ and $P_3$ )
    \begin{align}
    	P_1: \dfrac{y-\WRITEFRAC\pxy\qxy}{x-\WRITEFRAC\pxx\qxx} &= 1 
    	\Rightarrow y = x + \WRITEFRAC\mx\nx \\
    	P_2: \dfrac{y-\WRITEFRAC\pyy\qyy}{x-\WRITEFRAC\pyx\qyx} &= -\frac{1}{2} 
    	\Rightarrow y = -\frac{x}{2} + \WRITEFRAC\my\ny \\
    	P_3: \dfrac{y-\WRITEFRAC\pzy\qzy}{x-\WRITEFRAC\pzx\qzx} &= -\frac{7}{2} 
    	\Rightarrow y = -\frac{7}{2}x + \WRITEFRAC\mz\nz
    \end{align}
    Solving any two of the above - say $P_1$ and $P_2$ - will give the circumcenter 
    $O = \left( \WRITEFRAC\ansxp\ansxq,\,\WRITEFRAC\ansyp\ansyq \right)$
    
    To close the solution, you should confirm that $O$ also lies on the third line - 
    in this case $P_3$. If it does, then $O$ is your point. If not, then you have made a mistake somewhere!
  \end{solution}

\end{parts}

