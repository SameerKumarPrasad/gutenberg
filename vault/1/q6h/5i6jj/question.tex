

 %\noprintanswers

\ifnumequal{\value{rolldice}}{0}{
  % variables 
  \renewcommand{\va}{4} %a1
  \renewcommand{\vb}{7} %b1
  \renewcommand{\vc}{1} %c1
}{
  \ifnumequal{\value{rolldice}}{1}{
    % variables 
    \renewcommand{\va}{3}
    \renewcommand{\vb}{5}
    \renewcommand{\vc}{6}
  }{
    \ifnumequal{\value{rolldice}}{2}{
      % variables 
      \renewcommand{\va}{1}
      \renewcommand{\vb}{3}
      \renewcommand{\vc}{1}
    }{
      % variables 
      \renewcommand{\va}{5}
      \renewcommand{\vb}{1}
      \renewcommand{\vc}{3}
    }
  }
}


\FRACMINUS\va{3}\vb{3}\pax\qax
\FRACL{-1}{\va}\pax\qax\pay\qay

\FRACMULT\va{1}{7}{9}\tp\tq
\FRACMINUS\tp\tq\vc{9}\pbx\qbx
\FRACL{-1}\va\pbx\qbx\pby\qby

\FRACMULT\vb{1}{7}{12}\tp\tq
\FRACMINUS\vc{12}\tp\tq\pcx\qcx
\FRACL{2}{\vb}\pcx\qcx\pcy\qcy


\FRACADD\pax\qax\pbx\qbx\tp\tq
\FRACDIV\tp\tq{2}{1}\pxx\qxx % X_x
\FRACADD\pay\qay\pby\qby\tp\tq
\FRACDIV\tp\tq{2}{1}\pxy\qxy % X_y

\FRACADD\pax\qax\pcx\qcx\tp\tq
\FRACDIV\tp\tq{2}{1}\pyx\qyx % Y_x
\FRACADD\pay\qay\pcy\qcy\tp\tq
\FRACDIV\tp\tq{2}{1}\pyy\qyy % Y_y

\FRACADD\pbx\qbx\pcx\qcx\tp\tq
\FRACDIV\tp\tq{2}{1}\pzx\qzx % Z_x
\FRACADD\pby\qby\pcy\qcy\tp\tq
\FRACDIV\tp\tq{2}{1}\pzy\qzy % Z_y

\FRACMINUS\pxy\qxy\pxx\qxx\mx\nx
\FRACMULT\pyx\qyx{-1}{2}\tp\tq
\FRACMINUS\pyy\qyy\tp\tq\my\ny
\FRACMULT\pzx\qzx{-7}{2}\tp\tq
\FRACMINUS\pzy\qzy\tp\tq\mz\nz

\FRACMINUS\my\ny\mx\nx\ansxp\ansxq
\FRACMULT\ansxp\ansxq{2}{3}\ansxp\ansxq
\FRACADD\ansxp\ansxq\mx\nx\ansyp\ansyq

\question Three lines - $AB: x + y = \va$, $AC: 2x - y + \vb = 0$  
and $BC: 2x - 7y + \vc = 0$ intersect at points $A,\, B$ and $C$ (see figure) to form 
a triangle. The mid-points of the sides - also shown - are $X,\,Y$ and $Z$.

\watchout

\figinit{pt}
  \figpt 10:$B$(-10, 0)
  \figpt 20:$A$(70, 20)
  \figpt 30:$C$(50, -20)
  \figpt 11: $X$(30,10)
  \figpt 12: $L_1$(30,10)
  \figpt 21: $Y$(60,0)
  \figpt 22: $L_2$(60,0)
  \figpt 31: $Z$(20,-10)
  \figpt 32: $L_3$(20,-10)
\figdrawbegin{}
  \figdrawline [10, 20]
  \figdrawline [20, 30]
  \figdrawline [30, 10]
\figdrawend
\figvisu{\figBoxA}{}{%
  \figwritesw 10:(5pt)
  \figwritene 20:(5pt)
  \figwritese 30:(5pt)
  \normalsize
  \figset write(mark=$\bullet$)
  \figwritenw 11:(4)
  \figwritee 21:(2)
  \figwritesw 31:(4)
}

\ifprintanswers
  \begin{marginfigure}
    \centerline{\box\figBoxA}
  \end{marginfigure}
\else
  \vspace{1cm}
  \centerline{\box\figBoxA}
  \vspace{1cm}
\fi

\begin{parts}
  \part[4] Find the coordinates of the vertices of the triangle 

\begin{solution}[\halfpage]
    This is simple. Take two lines at a time and solve first for $x$ and then for $y$. Which means, that if 
    \begin{align}
      AB: y &= \va - x\;(m_1 = -1) \\
      AC: y &= 2x + \vb\;(m_2 = 2)\\
      BC: y &= \frac{2}{7}x + \frac{\vc}{7}\;(m_3 = \frac{2}{7})
    \end{align}
    , then solving $AB$ and $AC$ gives 
    $A = \left( \WRITEFRAC[false]\pax\qax,\,\WRITEFRAC[false]\pay\qay \right)$,
     solving $AB$ and $BC$ gives  
    $B = \left( \WRITEFRAC[false]\pbx\qbx,\,\WRITEFRAC[false]\pby\qby \right)$ and 
     solving $AC$ and $BC$ gives 
    $C = \left( \WRITEFRAC[false]\pcx\qcx,\,\WRITEFRAC[false]\pcy\qcy \right)$,
  \end{solution}

  \part[2] Find the coordinates of the \textit{mid-points} of the sides of the triangle

\begin{solution}[\mcq]
  	A point $J$ in the middle of two other points - $M = (x_m, y_m)$ and $N = (x_n, y_n)$ has coordinates  
  	$\left( \frac{x_m + x_n}{2}, \frac{y_m + y_m}{2}\right)$

  	Given this, its easy to find coordinates of $X$, $Y$ and $Z$
  	\begin{align}
      X &= \left( \frac{x_A + x_b}{2},\,\frac{y_A + y_B}{2}\right) 
      = \left( \WRITEFRAC[false]\pxx\qxx ,\,\WRITEFRAC[false]\pxy\qxy \right) \\
      Y &= \left( \frac{x_A + x_C}{2},\,\frac{y_A + y_C}{2}\right) 
      = \left( \WRITEFRAC[false]\pyx\qyx,\,\WRITEFRAC[false]\pyy\qyy \right)  \\
      Z &= \left( \frac{x_B + x_C}{2},\,\frac{y_B + y_C}{2}\right)
      = \left( \WRITEFRAC[false]\pzx\qzx,\,\WRITEFRAC[false]\pzy\qzy \right)
  	\end{align}
  \end{solution}

  \part[3] Find the \textit{circumcenter} of the triangle

\begin{solution}[\halfpage]
  	The circumcenter of a triangle is the points where the \textit{perpendicular} bisectors 
  	of its sides intersect
  	
  	Now, we know the slopes of the sides from part (a). And we can therefore write equations for 
    the perpendiculars ( lets call them $P_1,\, P_2$ and $P_3$ )
    \begin{align}
    	P_1: \dfrac{y-\WRITEFRAC[false]\pxy\qxy}{x-\WRITEFRAC[false]\pxx\qxx} &= 1 
    	\implies y = x + \WRITEFRAC[false]\mx\nx \\
    	P_2: \dfrac{y-\WRITEFRAC[false]\pyy\qyy}{x-\WRITEFRAC[false]\pyx\qyx} &= -\frac{1}{2} 
    	\implies y = -\frac{x}{2} + \WRITEFRAC[false]\my\ny \\
    	P_3: \dfrac{y-\WRITEFRAC[false]\pzy\qzy}{x-\WRITEFRAC[false]\pzx\qzx} &= -\frac{7}{2} 
    	\implies y = -\frac{7}{2}x + \WRITEFRAC[false]\mz\nz
    \end{align}
    Solving any two of the above - say $P_1$ and $P_2$ - will give the circumcenter 
    $O = \left( \WRITEFRAC[false]\ansxp\ansxq,\,\WRITEFRAC[false]\ansyp\ansyq \right)$
    
    To close the solution, you should confirm that $O$ also lies on the third line - 
    in this case $P_3$. If it does, then $O$ is your point. If not, then you have made a mistake somewhere!
  \end{solution}

\end{parts}

\ifprintanswers
  \begin{codex}
    \begin{tabular}{l}
      $(a)\,A = \left( \WRITEFRAC[false]\pax\qax,\,\WRITEFRAC[false]\pay\qay \right),\quad
           B = \left( \WRITEFRAC[false]\pbx\qbx,\,\WRITEFRAC[false]\pby\qby \right),\quad
           C = \left( \WRITEFRAC[false]\pcx\qcx,\,\WRITEFRAC[false]\pcy\qcy \right)$ \\
      $(b)\, X = \left( \WRITEFRAC[false]\pxx\qxx ,\,\WRITEFRAC[false]\pxy\qxy \right),\quad
      Y= \left( \WRITEFRAC[false]\pyx\qyx,\,\WRITEFRAC[false]\pyy\qyy \right),\quad
      Z= \left( \WRITEFRAC[false]\pzx\qzx,\,\WRITEFRAC[false]\pzy\qzy \right)$ \\
      $(c)\, \left( \WRITEFRAC[false]\ansxp\ansxq,\,\WRITEFRAC[false]\ansyp\ansyq \right)$
    \end{tabular}
  \end{codex}
\fi
