% This is an empty shell file placed for you by the 'examiner' script.
% You can now fill in the TeX for your question here.

% Now, down to brasstacks. ** Writing good solutions is an Art **. 
% Eventually, you will find your own style. But here are some thoughts 
% to get you started: 
%
%   1. Write the solution as if you are writing it for your favorite
%      14-17 year old to help him/her understand. Could be your nephew, 
%      your niece, a cousin perhaps or probably even you when you 
%      were that age. Just write for them.
%
%   2. Use margin-notes to "talk" to students about the critical insights
%      in the question. The tone can be - in fact, should be - informal
%
%   3. Don't shy away from creating margin-figures you think will help
%      students understand. Yes, it is a little more work per question. 
%      But the question & solution will be written only once. Make that
%      attempt at writing a solution count.
%
%   4. At the same time, do not be too verbose. A long solution can
%      - at first sight - make the student think, "God, that is a lot to know".
%      Our aim is not to scare students. Rather, our aim should be to 
%      create many "Aha!" moments everyday in classrooms around the world
% 
%   5. Ensure that there are *no spelling mistakes anywhere*. We are an 
%      education company. Bad spellings suggest that we ourselves 
%      don't have any education. Also, use American spellings by default
% 
%   6. If a question has multiple parts, then first delete lines 40-41
%   7. If a question does not have parts, then first delete lines 43-69
\question[1] $AB$ and $CD$ are two \textit{equal} chords of a circle whose center is $O$ - \asif. 
$OM\perp AB$ and $ON\perp CD$. Prove that $\angle OMN = \angle ONM$

\insertQR{QRC}

\ifprintanswers
  % stuff to be shown only in the answer key - like explanatory margin figures
\fi 
  \begin{marginfigure}
    \figinit{pt}
    	\figpt 100: $O$(50,50)
    	\figpt 200:(95,50) % reference pt that is rotated
    	\figptrot 101:$A$= 200 /100,80/
    	\figptrot 102:$C$= 200 /100,100/
    	\figptrot 103:$B$= 200 /100,190/
    	\figptrot 104:$D$= 200 /100,350/
    	\figptorthoprojline 300:$M$= 100 /101,103/
    	\figptorthoprojline 301:$N$= 100 /102,104/
    \figdrawbegin{}
    	\figdrawcirc 100(45)
      \figdrawline [101,103] 
      \figdrawline [102,104] 
      \figdrawaltitude 5 [100,101,103]
      \figdrawaltitude 5 [100,102,104]
      \figdrawline [300,301]
    \figdrawend
    \figvisu{\figBoxA}{}{%
    	\figset write(mark=$\bullet$)
    	\figwrites 100:(2)
    	\figwriten 101:(3)
    	\figwriten 102:(3)
    	\figwritew 103:(3)
    	\figwritee 104:(3)
    	\figwriten 300:(3)
    	\figwriten 301:(3)
    }
    \centerline{\box\figBoxA}
  \end{marginfigure}

\begin{solution}[\mcq]
  Chords of the same length are equidistant from the center. Which means, $OM = ON \Rightarrow \bigtriangleup OMN$ 
  is an isoceles triangle. And therefore $\angle OMN = \angle ONM$
\end{solution}

