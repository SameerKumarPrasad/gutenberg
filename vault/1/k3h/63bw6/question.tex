
\ifnumequal{\value{rolldice}}{0}{
  % variables 
  \renewcommand{\va}{8}
  \renewcommand{\vb}{12}
  \renewcommand{\vc}{4}
}{
  \ifnumequal{\value{rolldice}}{1}{
    % variables 
    \renewcommand{\va}{6}
    \renewcommand{\vb}{10}
    \renewcommand{\vc}{2}
  }{
    \ifnumequal{\value{rolldice}}{2}{
      % variables 
      \renewcommand{\va}{6}
      \renewcommand{\vb}{8}
      \renewcommand{\vc}{1}
    }{
      % variables 
      \renewcommand{\va}{4}
      \renewcommand{\vb}{12}
      \renewcommand{\vc}{4}
    }
  }
}

\DIVIDE\va{2}\vp
\DIVIDE\vb{2}\vq
\SQUARE\vp\vj
\SQUARE\vq\vk
\SUBTRACT\vk\vj\vl

\EXPR[2]\vz{(\vl - (\vc * \vc))/(2*\vc)}
\SQUARE\vz\vx
\ADD\vx\vk\vw
\SQUAREROOT\vw\vv
\ROUND[2]\vv\vu

\question[3] Chords $AB$ and $CD$ (see figure) are of lengths $\va$ cm and $\vb$ cm respectively. They are also 
parallel to each other. If the distance between them (the chords) is $\vc$ cm, then what is 
the radius of the circle? 

\watchout

\begin{calcaid}
  \begin{tabular}{c c c c}
    $\sqrt{36.25}=6.02$ & $\sqrt{34}=5.83$ & $\sqrt{40}=6.32$
  \end{tabular}
\end{calcaid}

  % stuff to be shown only in the answer key - like explanatory margin figures
\figinit{pt}
  \large
  \figpt 100: $O$(50,50)
  \figpt 200: (95,50) % ref. pt 
  \figptrot 101: $D$= 200 /100,20/
  \figptrot 102: $B$= 200 /100,50/
  \figptrot 103: $C$= 200 /100,160/
  \figptrot 104: $A$= 200 /100,130/
  \figptorthoprojline 300:$M$= 100 /102,104/
  \figptorthoprojline 301:$N$= 100 /101,103/
\figdrawbegin{}
  \figdrawcirc 100(45)
  \figdrawline [101,103]
  \figdrawline [102,104]
  \figdrawaltitude 5 [100,102,104]
  \figdrawaltitude 5 [100,101,103]
  \ifprintanswers
    \figset (dash=5)
    \figdrawline [100,101]
    \figdrawline [100,102]
  \fi
\figdrawend
\figvisu{\figBoxA}{}{%
  \figset write(mark=$\bullet$)
  \figwrites 100:(3)
  \figwritee 101:(3)
  \figwritene 102:(3)
  \figwritew 103:(4)
  \figwritenw 104:(4)
  \figwriten 300:(3)
  \figwritene 301:(4)
}

\vspace{1cm}
\centerline{\box\figBoxA}

\begin{solution}[\halfpage]
	Here is what we know,
	\begin{align}
		MN &= \vc\text{ cm} \\
		OB^2 &= OM^2 + MB^2 \\
		OD^2 &= ON^2 + ND^2 \\
		OB = OD &= \text{ Radius } \\
		MB &= \frac{1}{2}AB = \vp\text{ cm} \\
		ND &= \frac{1}{2}CD = \vq\text{ cm}
	\end{align}
	And therefore, 
	\begin{align}
		OM^2 + MB^2 = (ON + \vc)^2 + \vp^2 &= ON^2 + \vq^2 \\
		\implies (ON+3)^2 - ON^2 &= \vk - \vj = \vl \\
		\implies \underbrace{(2\cdot ON + \vc)\cdot \vc}_{a^2-b^2 = (a-b)(a+b)} &= \vl \implies ON = \vz\text{ cm} \\
		\therefore OD = \text{ Radius } &= \sqrt{ON^2 + ND^2} \\ 
		&= \sqrt{\vz^2 + \vq^2} = \vu\text{ cm} 
	\end{align}
\end{solution}

\ifprintanswers\begin{codex}$\vu\text{ cm}$\end{codex}\fi
