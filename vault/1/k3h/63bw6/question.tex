

\question[3] Chords $AB$ and $CD$ are of lengths 5cm and 11cm respectively. They are also 
parallel to each other. If the distance between them (the chords) is 3cm, then what is 
the radius of the circle? 


  % stuff to be shown only in the answer key - like explanatory margin figures
  \begin{marginfigure}
    \figinit{pt}
      \figpt 100: $O$(50,50)
      \figpt 200: (95,50) % ref. pt 
      \figptrot 101: $D$= 200 /100,20/
      \figptrot 102: $B$= 200 /100,50/
      \figptrot 103: $C$= 200 /100,160/
      \figptrot 104: $A$= 200 /100,130/
      \figptorthoprojline 300:$M$= 100 /102,104/
      \figptorthoprojline 301:$N$= 100 /101,103/
    \figdrawbegin{}
      \figdrawcirc 100(45)
      \figdrawline [101,103]
      \figdrawline [102,104]
      \figdrawaltitude 5 [100,102,104]
      \figdrawaltitude 5 [100,101,103]
      \ifprintanswers
        \figset (dash=5)
        \figdrawline [100,101]
        \figdrawline [100,102]
      \fi
    \figdrawend
    \figvisu{\figBoxA}{}{%
      \figset write(mark=$\bullet$)
      \figwrites 100:(2)
      \figwritee 101:(3)
      \figwritee 102:(3)
      \figwritew 103:(3)
      \figwritew 104:(3)
      \figwritene 300:(3)
      \figwritene 301:(3)
    }
    \centerline{\box\figBoxA}
  \end{marginfigure}

\begin{solution}[\halfpage]
	Here is what we know,
	\begin{align}
		MN &= 3cm \\
		OB^2 &= OM^2 + MB^2 \\
		OD^2 &= ON^2 + ND^2 \\
		OB = OD &= \text{ Radius } \\
		MB &= \frac{1}{2}AB = 2.5cm \\
		ND &= \frac{1}{2}CD = 5.5cm
	\end{align}
	And therefore, 
	\begin{align}
		OM^2 + MB^2 = (ON + 3)^2 + 2.5^2 &= ON^2 + 5.5^2 \\
		\Rightarrow (ON+3)^2 - ON^2 &= 30.25 - 6.25 = 24 \\
		\Rightarrow \underbrace{(2\cdot ON + 3)\cdot 3}_{a^2-b^2 = (a-b)(a+b)} &= 24 \Rightarrow ON = 2.5cm \\
		\therefore OD = \text{ Radius } &= \sqrt{ON^2 + ND^2} \\ 
		&= \sqrt{2.5^2 + 5.5^2} = 6.04cm
	\end{align}
\end{solution}

