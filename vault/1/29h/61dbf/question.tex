% This is an empty shell file placed for you by the 'examiner' script.
% You can now fill in the TeX for your question here.

% Now, down to brasstacks. ** Writing good solutions is an Art **. 
% Eventually, you will find your own style. But here are some thoughts 
% to get you started: 
%
%   1. Write to be understood - but be crisp. Your own solution should not take 
%      more space than you will give to the student. Hence, if you take more than 
%      a half-page to write a solution, then give the student a full-page and so on...
%
%   2. Use margin-notes to "talk" to students about the critical insights
%      in the question. The tone can be - in fact, should be - informal
%
%   3. Don't shy away from creating margin-figures you think will help
%      students understand. Yes, it is a little more work per question. 
%      But the question & solution will be written only once. Make that
%      attempt at writing a solution count.
%      
%      3b. Use bc_to_fig.tex. Its an easier way to generate plots & graphs 
% 
%   4. Ensure that there are *no spelling mistakes anywhere*. We are an 
%      education company. Bad spellings suggest that we ourselves 
%      don't have any education. Also, use American spellings by default
% 
%   5. If a question has multiple parts, then first delete lines 40-41
%   6. If a question does not have parts, then first delete lines 43-69
%   
%   7. Create versions of the question when possible. Use commands defined in 
%      tufte-tweaks.sty to do so. Its easier than you think

%\noprintanswers
%\setcounter{rolldice}{0}

\ifnumequal{\value{rolldice}}{0}{
  % variables 
  \renewcommand{\vbone}{POSITRON } % word
  \renewcommand{\vbtwo}{S } % pivot
  \renewcommand{\vbthree}{5} % # of characters in target word 
  \renewcommand{\vbfour}{8} % length of original 
  \renewcommand{\vbfive}{7} % # distinct characters in original
  \renewcommand{\vbsix}{O } % repeating character
  \renewcommand{\vbseven}{720} % part a 
  \renewcommand{\vbeight}{1800} % part b
  \renewcommand{\vbnine}{7776} % part c
  \renewcommand{\vbten}{9031} % part d
}{
  \ifnumequal{\value{rolldice}}{1}{
    % variables 
    \renewcommand{\vbone}{NUCLEUS }
    \renewcommand{\vbtwo}{E }
    \renewcommand{\vbthree}{4}
    \renewcommand{\vbfour}{7}
    \renewcommand{\vbfive}{6}
    \renewcommand{\vbsix}{U }
    \renewcommand{\vbseven}{120}
    \renewcommand{\vbeight}{240}
    \renewcommand{\vbnine}{625}
    \renewcommand{\vbten}{671}
  }{
    \ifnumequal{\value{rolldice}}{2}{
      % variables 
      \renewcommand{\vbone}{QUASAR }
      \renewcommand{\vbtwo}{R }
      \renewcommand{\vbthree}{4}
      \renewcommand{\vbfour}{6}
      \renewcommand{\vbfive}{5}
      \renewcommand{\vbsix}{A }
      \renewcommand{\vbseven}{24}
      \renewcommand{\vbeight}{96}
      \renewcommand{\vbnine}{256}
      \renewcommand{\vbten}{369}
    }{
      % variables 
      \renewcommand{\vbone}{GALAXY }
      \renewcommand{\vbtwo}{X }
      \renewcommand{\vbthree}{4}
      \renewcommand{\vbfour}{6}
      \renewcommand{\vbfive}{5}
      \renewcommand{\vbsix}{A }
      \renewcommand{\vbseven}{24}
      \renewcommand{\vbeight}{96}
      \renewcommand{\vbnine}{256}
      \renewcommand{\vbten}{369}
    }
  }
}

\gcalcexpr[0]\tp{\vbfive - 1}
\gcalcexpr[0]\tq{\vbthree - 1}

\question How many $\vbthree$-letter words can be formed from \vbone if

\watchout

\ifprintanswers
\fi 

\begin{parts}
  \part[2] \vbtwo is \textit{never} included and none of the other characters can repeat? You might want to 
  think about the number of \textit{distinct} characters in \vbone

  \insertQR{QRC}
\begin{solution}[\mcq]
    The word has $\vbfour$ characters - with \vbsix occuring twice. And hence, there are $\vbfive$ \textit{distinct}
    characters in the word
    
    If, however, \vbtwo can never occur, then we have $\tp$ characters left to work with. And, as neither 
    can occur more than once, the required number of ways is 
    \begin{align}
    	N &= \encr\tp\vbthree\cdot\vbthree ! = \vbseven
    \end{align}
  \end{solution}

  \part[1] \vbtwo is \textit{always} included \textit{but} none of the included characters repeat
  \insertQR{QRC}
\begin{solution}[\mcq]
  	If one slot is booked for \vbtwo, then there are $\tq$ slots left to fill using the $\tp$ distinct characters
  	
  	The required number of ways then is
  	\begin{align}
  		N &= \vbthree \times \encr\tp\tq\cdot\tq ! = \vbeight
  	\end{align}
  \end{solution}

  \part[1] \vbtwo is \textit{never} included \textit{but} the other characters - if included - can repeat.
  \ifnumequal{\value{rolldice}}{0}{\texttt{Calculator: $6^4 = 1296$ and $7^4 = 2401$}}{}

  \insertQR{QRC}
\begin{solution}[\mcq]
  	This is easy. From the $\vbfive$ distinct characters, we always have to ignore one - \vbtwo. 
  	And given that any of the remaining $\tp$ can repeat, the required number of ways is 
  	\begin{align}
  		N &= \tp^\vbthree = \vbnine
  	\end{align}
  \end{solution}

  \part[3] \vbtwo is \textit{always} included \textit{and} the any of the included characters can repeat

  \insertQR{QRC}
\begin{solution}[\mcq]
  	Think of it like this. If, from all possible $\vbthree$-letter words made using $\vbfive$ distinct 
  	characters - with repetition - one were to remove all the words in part (c), then the remaining 
  	words would always have the character \vbtwo
  	\begin{align}
  		N &= \vbfive^\vbthree - \tp^\vbthree = \vbten
  	\end{align}
  \end{solution}


\end{parts}
