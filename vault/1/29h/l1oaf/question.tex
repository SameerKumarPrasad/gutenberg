% This is an empty shell file placed for you by the 'examiner' script.
% You can now fill in the TeX for your question here.

% Now, down to brasstacks. ** Writing good solutions is an Art **. 
% Eventually, you will find your own style. But here are some thoughts 
% to get you started: 
%
%   1. Write to be understood - but be crisp. Your own solution should not take 
%      more space than you will give to the student. Hence, if you take more than 
%      a half-page to write a solution, then give the student a full-page and so on...
%
%   2. Use margin-notes to "talk" to students about the critical insights
%      in the question. The tone can be - in fact, should be - informal
%
%   3. Don't shy away from creating margin-figures you think will help
%      students understand. Yes, it is a little more work per question. 
%      But the question & solution will be written only once. Make that
%      attempt at writing a solution count.
%      
%      3b. Use bc_to_fig.tex. Its an easier way to generate plots & graphs 
% 
%   4. Ensure that there are *no spelling mistakes anywhere*. We are an 
%      education company. Bad spellings suggest that we ourselves 
%      don't have any education. Also, use American spellings by default
% 
%   5. If a question has multiple parts, then first delete lines 40-41
%   6. If a question does not have parts, then first delete lines 43-69
%   
%   7. Create versions of the question when possible. Use commands defined in 
%      tufte-tweaks.sty to do so. Its easier than you think

% \noprintanswers
% \setcounter{rolldice}{3}

\ifnumequal{\value{rolldice}}{0}{
  % variables 
  \renewcommand{\vbone}{5} % # boys
  \renewcommand{\vbtwo}{3} % # girls
  \renewcommand{\vbthree}{boys } % remaining
  \renewcommand{\vbfour}{girls }
  \renewcommand{\vbfive}{1440}
  \renewcommand{\vbsix}{}
  \renewcommand{\vbseven}{}
  \renewcommand{\vbeight}{}
  \renewcommand{\vbnine}{}
  \renewcommand{\vbten}{}
}{
  \ifnumequal{\value{rolldice}}{1}{
    % variables 
    \renewcommand{\vbone}{3}
    \renewcommand{\vbtwo}{6}
    \renewcommand{\vbthree}{girls }
    \renewcommand{\vbfour}{boys }
    \renewcommand{\vbfive}{8640}
    \renewcommand{\vbsix}{}
    \renewcommand{\vbseven}{}
    \renewcommand{\vbeight}{}
    \renewcommand{\vbnine}{}
    \renewcommand{\vbten}{}
  }{
    \ifnumequal{\value{rolldice}}{2}{
      % variables 
      \renewcommand{\vbone}{6}
      \renewcommand{\vbtwo}{4}
      \renewcommand{\vbthree}{boys }
      \renewcommand{\vbfour}{girls }
      \renewcommand{\vbfive}{34,560}
      \renewcommand{\vbsix}{}
      \renewcommand{\vbseven}{}
      \renewcommand{\vbeight}{}
      \renewcommand{\vbnine}{}
      \renewcommand{\vbten}{}
    }{
      % variables 
      \renewcommand{\vbone}{3}
      \renewcommand{\vbtwo}{7}
      \renewcommand{\vbthree}{girls }
      \renewcommand{\vbfour}{boys }
      \renewcommand{\vbfive}{60,480}
      \renewcommand{\vbsix}{}
      \renewcommand{\vbseven}{}
      \renewcommand{\vbeight}{}
      \renewcommand{\vbnine}{}
      \renewcommand{\vbten}{}
    }
  }
}
\ifthenelse{\vbone > \vbtwo}{\gcalcexpr[0]\nremaining{\vbone - \vbtwo}}{\gcalcexpr[0]\nremaining{\vbtwo - \vbone}}
\ifthenelse{\vbone > \vbtwo}{\gcalcexpr[0]\nmin\vbtwo}{\gcalcexpr[0]\nmin\vbone}
\ifthenelse{\vbone > \vbtwo}{\gcalcexpr[0]\nmax\vbone}{\gcalcexpr[0]\nmax\vbtwo}

\question[2] $\vbone$ boys and $\vbtwo$ girls need to stand in a line so that, as far as possible, they alternate 
and when they can't, the remaining students line up randomly. In how many ways can they stand?

\insertQR{QRC}

\watchout

\ifprintanswers
\fi 

\begin{solution}[\mcq]
	As there are $\vbone$ boys and $\vbtwo$ girls, they can stand alternately only till such time that there 
	are \vbfour remaining to join the line. The remaining $\nremaining$ \vbthree must then stand in some 
	random order
	
	Also, the line can start with either a boy or a girl. Given this, the number of possible ways for them 
	to stand in a line is 
	
	\begin{align}
		N &= 2\times\underbrace{\left[ \encr\nmax\nmin\cdot\nmin !\cdot\nmin !\right]}_{\texttt{when alternating}}
		\times \overbrace{\nremaining !}^{\texttt{remaining}} = \vbfive
	\end{align}
\end{solution}
