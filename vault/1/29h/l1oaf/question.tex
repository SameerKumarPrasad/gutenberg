


\ifnumequal{\value{rolldice}}{0}{
  % variables 
  \renewcommand{\vbone}{5} % # boys
  \renewcommand{\vbtwo}{3} % # girls
  \renewcommand{\vbthree}{boys } % remaining
  \renewcommand{\vbfour}{girls }
  \renewcommand{\vbfive}{1440}
}{
  \ifnumequal{\value{rolldice}}{1}{
    % variables 
    \renewcommand{\vbone}{3}
    \renewcommand{\vbtwo}{6}
    \renewcommand{\vbthree}{girls }
    \renewcommand{\vbfour}{boys }
    \renewcommand{\vbfive}{8640}
  }{
    \ifnumequal{\value{rolldice}}{2}{
      % variables 
      \renewcommand{\vbone}{6}
      \renewcommand{\vbtwo}{4}
      \renewcommand{\vbthree}{boys }
      \renewcommand{\vbfour}{girls }
      \renewcommand{\vbfive}{34,560}
    }{
      % variables 
      \renewcommand{\vbone}{3}
      \renewcommand{\vbtwo}{7}
      \renewcommand{\vbthree}{girls }
      \renewcommand{\vbfour}{boys }
      \renewcommand{\vbfive}{60,480}
    }
  }
}
\ifthenelse{\vbone > \vbtwo}{\gcalcexpr[0]\nremaining{\vbone - \vbtwo}}{\gcalcexpr[0]\nremaining{\vbtwo - \vbone}}
\ifthenelse{\vbone > \vbtwo}{\gcalcexpr[0]\nmin\vbtwo}{\gcalcexpr[0]\nmin\vbone}
\ifthenelse{\vbone > \vbtwo}{\gcalcexpr[0]\nmax\vbone}{\gcalcexpr[0]\nmax\vbtwo}

\question[2] $\vbone$ boys and $\vbtwo$ girls need to stand in a line so that, as far as possible, they alternate 
and when they can't, the remaining students line up randomly. In how many ways can they stand?


\watchout

\ifprintanswers
\fi 

\begin{solution}[\mcq]
	As there are $\vbone$ boys and $\vbtwo$ girls, they can stand alternately only till such time that there 
	are \vbfour remaining to join the line. The remaining $\nremaining$ \vbthree must then stand in some 
	random order
	
	Also, the line can start with either a boy or a girl. Given this, the number of possible ways for them 
	to stand in a line is 
	
	\begin{align}
		N &= 2\times\underbrace{\left[ \encr\nmax\nmin\cdot\nmin !\cdot\nmin !\right]}_{\texttt{when alternating}}
		\times \overbrace{\nremaining !}^{\texttt{remaining}} = \vbfive
	\end{align}
\end{solution}
