% This is an empty shell file placed for you by the 'examiner' script.
% You can now fill in the TeX for your question here.

% Now, down to brasstacks. ** Writing good solutions is an Art **. 
% Eventually, you will find your own style. But here are some thoughts 
% to get you started: 
%
%   1. Write to be understood - but be crisp. Your own solution should not take 
%      more space than you will give to the student. Hence, if you take more than 
%      a half-page to write a solution, then give the student a full-page and so on...
%
%   2. Use margin-notes to "talk" to students about the critical insights
%      in the question. The tone can be - in fact, should be - informal
%
%   3. Don't shy away from creating margin-figures you think will help
%      students understand. Yes, it is a little more work per question. 
%      But the question & solution will be written only once. Make that
%      attempt at writing a solution count.
%      
%      3b. Use bc_to_fig.tex. Its an easier way to generate plots & graphs 
% 
%   4. Ensure that there are *no spelling mistakes anywhere*. We are an 
%      education company. Bad spellings suggest that we ourselves 
%      don't have any education. Also, use American spellings by default
% 
%   5. If a question has multiple parts, then first delete lines 40-41
%   6. If a question does not have parts, then first delete lines 43-69
%   
%   7. Create versions of the question when possible. Use commands defined in 
%      tufte-tweaks.sty to do so. Its easier than you think

% \noprintanswers
% \setcounter{rolldice}{3}

\ifnumequal{\value{rolldice}}{0}{
  % variables 
  \renewcommand{\vbone}{5} % country A > B 
  \renewcommand{\vbtwo}{3} % country B
  \renewcommand{\vbthree}{14,400}
}{
  \ifnumequal{\value{rolldice}}{1}{
    % variables 
    \renewcommand{\vbone}{7}
    \renewcommand{\vbtwo}{4}
    \renewcommand{\vbthree}{8,467,200}
  }{
    \ifnumequal{\value{rolldice}}{2}{
      % variables 
      \renewcommand{\vbone}{9}
      \renewcommand{\vbtwo}{4}
      \renewcommand{\vbthree}{\text{close to 2 billion ways}}
    }{
      % variables 
      \renewcommand{\vbone}{6}
      \renewcommand{\vbtwo}{3}
      \renewcommand{\vbthree}{151,200}
    }
  }
}

\gcalcexpr[0]\midslots{\vbtwo - 1}
\gcalcexpr[0]\totalslots{\vbtwo + 1}
\gcalcexpr[0]\remainingA{\vbone - \vbtwo + 1}
\gcalcexpr[0]\tmp{\vbone + 1}

\question[5] At an international conference, there are $\vbone$ representatives of country A and $\vbtwo$ representatives 
of country B. In how many ways can they be seated along a table so that no two representatives of country $B$ 
are sitting together? You can leave your answer at the final expression

\insertQR[10pt]{abc}

\watchout

\begin{solution}[\halfpage]
	\begin{fullwidth}
	First, let the $\vbtwo$ representatives of country B sit down. Between them, there are $\midslots$ empty 
	slots where if we were to place a representative of country A, then no two of B would be sitting 
	together. The number of ways to do this are 
	
	\begin{align}
		N_1 &= \encr\vbone\midslots\times\midslots ! \times\vbtwo !
	\end{align}
	We still have $\remainingA$ representatives of country A to seat. But they can sit anywhere - 
	in the $\midslots$ slots between or in extreme left and extreme right slots. In short, there are 
	$\totalslots = (\midslots + 2)$ slots for them to be in 
	
	This is just like the problem of distributing $N$ things amonst $M$ people. In this case, we 
	have to dsitribute $\remainingA$ amongst $\totalslots$. And the formula for that is
	\begin{align}
		N_2 &= \eDistNAmongstM\remainingA\totalslots\times\remainingA !
	\end{align}
	The required answer, therefore, is 
	\begin{align}
		N_{\texttt{total}} &= N_1 \times N_2 \\ 
		&= \fncr\vbone\midslots\times\midslots ! \times\vbtwo ! \times \eDistNAmongstM\remainingA\totalslots\times\remainingA ! \\
		&= \dfrac{\vbone ! \times \tmp !}{\remainingA !} = \vbthree
	\end{align}
	\end{fullwidth}
\end{solution}
