


\ifnumequal{\value{rolldice}}{0}{
  % variables 
  \renewcommand{\vbone}{even}
  \renewcommand{\vbtwo}{1,7,6,8,4,3}
  \renewcommand{\vbthree}{3} % # odd 
  \renewcommand{\vbfour}{3} % # even
  \renewcommand{\vbsix}{360}
}{
  \ifnumequal{\value{rolldice}}{1}{
    % variables 
    \renewcommand{\vbone}{odd}
    \renewcommand{\vbtwo}{5,8,3,2,4,1,9}
    \renewcommand{\vbthree}{4}
    \renewcommand{\vbfour}{3}
    \renewcommand{\vbsix}{2880}
  }{
    \ifnumequal{\value{rolldice}}{2}{
      % variables 
      \renewcommand{\vbone}{even}
      \renewcommand{\vbtwo}{7,8,3,4,1,6}
      \renewcommand{\vbthree}{3}
      \renewcommand{\vbfour}{3}
      \renewcommand{\vbsix}{360}
    }{
      % variables 
      \renewcommand{\vbone}{odd}
      \renewcommand{\vbtwo}{2,5,7,9,3,4}
      \renewcommand{\vbthree}{4}
      \renewcommand{\vbfour}{2}
      \renewcommand{\vbsix}{480}
    }
  }
}

\gcalcexpr[0]\tp{\vbthree + \vbfour}
\ifthenelse{\equal\vbone{odd}}{\renewcommand{\vbfive}{\vbthree}}{\renewcommand{\vbfive}{\vbfour}}

\question[1] How many \textit{\vbone} numbers can be formed using $\lbrace \vbtwo \rbrace$. All digits 
must be used and no digit repeats


\watchout

\ifprintanswers
\fi 

\begin{solution}[\mcq]
	There are a total of $\tp$ digits available - of which $\vbthree$ are odd and $\vbfour$ are even. 
	To form an \textit{\vbone} number, it must end with one of the $\vbfive$ \vbone digits. The other 
	digits can be in any order
	
	And therefore
	\begin{align}
		N &= (\tp - 1)!\times\vbfive = \vbsix
	\end{align}
\end{solution}
