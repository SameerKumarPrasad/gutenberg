


\ifnumequal{\value{rolldice}}{0}{
  % variables 
  \renewcommand{\vbone}{12}
  \renewcommand{\vbtwo}{5}
  \renewcommand{\vbthree}{210} % triangles
  \renewcommand{\vbfour}{546} % pentagons
  \renewcommand{\vbfive}{420} % quadrilaterals
  \renewcommand{\vbsix}{66} % lines
}{
  \ifnumequal{\value{rolldice}}{1}{
    % variables 
    \renewcommand{\vbone}{11}
    \renewcommand{\vbtwo}{6}
    \renewcommand{\vbthree}{145}
    \renewcommand{\vbfour}{180}
    \renewcommand{\vbfive}{215}
    \renewcommand{\vbsix}{55}
  }{
    \ifnumequal{\value{rolldice}}{2}{
      % variables 
      \renewcommand{\vbone}{9}
      \renewcommand{\vbtwo}{3}
      \renewcommand{\vbthree}{83}
      \renewcommand{\vbfour}{111}
      \renewcommand{\vbfive}{120}
      \renewcommand{\vbsix}{36}
    }{
      % variables 
      \renewcommand{\vbone}{10}
      \renewcommand{\vbtwo}{4}
      \renewcommand{\vbthree}{116}
      \renewcommand{\vbfour}{186}
	  \renewcommand{\vbfive}{185}
      \renewcommand{\vbsix}{45}
    }
  }
}

\gcalcexpr[0]\tmp{\vbone - \vbtwo}

\question There are $\vbone$ points in a plane of which $\vbtwo$ points are in a straight line 
and except these $\vbtwo$, no three of the others are in a straight line. A line-segment between two points
can pass through other points. Given this, find the number of

\watchout

\ifprintanswers
\fi 


\begin{parts}
  \part[1] \textbf{Triangles} that can be formed 

\begin{solution}[\mcq]
  	 We have two sets of points - set $A$ has the $\vbtwo$ collinear points and set $B$ has 
  	 the $\tmp$ non-collinear points
  	 
  	 A triangle can be formed either by picking 3 points from $B$ or 2 from $B$ and one from $A$ or 
  	 one from $B$ and two from $A$
  	 \begin{align}
  	 	N_{\delta} &= \encr\tmp{3} + \encr\tmp{2}\cdot\vbtwo + \tmp\cdot\encr\vbtwo{2} \\
  	 	&= \vbthree
  	 \end{align}
  \end{solution}

  \part[1] \textbf{Pentagons} that can be formed

\begin{solution}[\mcq]
  	\begin{align}
  		\ifthenelse{\tmp > 5}{ 
  			N &= \encr\tmp{5} + \encr\tmp{4}\cdot\vbtwo + \encr\tmp{3}\cdot\encr\vbtwo{2} \\
  			&= \vbfour
  		}{ 
  		   N &= \encr\tmp{4}\cdot\vbtwo + \encr\tmp{3}\cdot\encr\vbtwo{2} \\
  		   &= \vbfour
  		} 
  	\end{align}
  \end{solution}

  \part[1] \textbf{Quadrilaterals} that can be formed

\begin{solution}[\mcq]
  	\begin{align}
  		N &= \encr\tmp{4} + \encr\tmp{3}\cdot\vbtwo + \encr\tmp{2}\cdot\encr\vbtwo{2} \\
  		&= \vbfive
  	\end{align}
  \end{solution}

  \part[1] \textbf{Line segments} that can be formed

\begin{solution}[\mcq]
  	\begin{align}
  		N &= \encr\vbone{2} = \vbsix
  	\end{align}
  \end{solution}

\end{parts}
