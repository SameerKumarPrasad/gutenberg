% This is an empty shell file placed for you by the 'examiner' script.
% You can now fill in the TeX for your question here.

% Now, down to brasstacks. ** Writing good solutions is an Art **. 
% Eventually, you will find your own style. But here are some thoughts 
% to get you started: 
%
%   1. Write to be understood - but be crisp. Your own solution should not take 
%      more space than you will give to the student. Hence, if you take more than 
%      a half-page to write a solution, then give the student a full-page and so on...
%
%   2. Use margin-notes to "talk" to students about the critical insights
%      in the question. The tone can be - in fact, should be - informal
%
%   3. Don't shy away from creating margin-figures you think will help
%      students understand. Yes, it is a little more work per question. 
%      But the question & solution will be written only once. Make that
%      attempt at writing a solution count.
%      
%      3b. Use bc_to_fig.tex. Its an easier way to generate plots & graphs 
% 
%   4. Ensure that there are *no spelling mistakes anywhere*. We are an 
%      education company. Bad spellings suggest that we ourselves 
%      don't have any education. Also, use American spellings by default
% 
%   5. If a question has multiple parts, then first delete lines 40-41
%   6. If a question does not have parts, then first delete lines 43-69
%   
%   7. Create versions of the question when possible. Use commands defined in 
%      tufte-tweaks.sty to do so. Its easier than you think

%\noprintanswers
%\setcounter{rolldice}{3}

\ifnumequal{\value{rolldice}}{0}{
  % variables 
  \renewcommand{\vbone}{12}
  \renewcommand{\vbtwo}{5}
  \renewcommand{\vbthree}{210} % triangles
  \renewcommand{\vbfour}{546} % pentagons
  \renewcommand{\vbfive}{420} % quadrilaterals
  \renewcommand{\vbsix}{66} % lines
}{
  \ifnumequal{\value{rolldice}}{1}{
    % variables 
    \renewcommand{\vbone}{11}
    \renewcommand{\vbtwo}{6}
    \renewcommand{\vbthree}{145}
    \renewcommand{\vbfour}{180}
    \renewcommand{\vbfive}{215}
    \renewcommand{\vbsix}{55}
  }{
    \ifnumequal{\value{rolldice}}{2}{
      % variables 
      \renewcommand{\vbone}{9}
      \renewcommand{\vbtwo}{3}
      \renewcommand{\vbthree}{83}
      \renewcommand{\vbfour}{111}
      \renewcommand{\vbfive}{120}
      \renewcommand{\vbsix}{36}
    }{
      % variables 
      \renewcommand{\vbone}{10}
      \renewcommand{\vbtwo}{4}
      \renewcommand{\vbthree}{116}
      \renewcommand{\vbfour}{186}
	  \renewcommand{\vbfive}{185}
      \renewcommand{\vbsix}{45}
    }
  }
}

\gcalcexpr[0]\tmp{\vbone - \vbtwo}

\question There are $\vbone$ points in a plane of which $\vbtwo$ points are in a straight line and except these $\vbtwo$
, no three of the others are in a straight line. Given this, find the number of

\watchout

\ifprintanswers
\fi 


\begin{parts}
  \part[1] \textit{Triangles} that can be formed 

  \insertQR{QRC}
\begin{solution}[\mcq]
  	 We have two sets of points - set $A$ has the $\vbtwo$ collinear points and set $B$ has 
  	 the $\tmp$ non-collinear points
  	 
  	 A triangle can be formed either by picking 3 points from $B$ or 2 from $B$ and one from $A$ or 
  	 one from $B$ and two from $A$
  	 \begin{align}
  	 	N_{\delta} &= \encr\tmp{3} + \encr\tmp{2}\cdot\vbtwo + \tmp\cdot\encr\vbtwo{2} \\
  	 	&= \vbthree
  	 \end{align}
  \end{solution}

  \part[1] \textit{Pentagons} that can be formed

  \insertQR{QRC}
\begin{solution}[\mcq]
  	\begin{align}
  		\ifthenelse{\tmp > 5}{ 
  			N &= \encr\tmp{5} + \encr\tmp{4}\cdot\vbtwo + \encr\tmp{3}\cdot\encr\vbtwo{2} \\
  			&= \vbfour
  		}{ 
  		   N &= \encr\tmp{4}\cdot\vbtwo + \encr\tmp{3}\cdot\encr\vbtwo{2} \\
  		   &= \vbfour
  		} 
  	\end{align}
  \end{solution}

  \part[1] \textit{Quadrilaterals} that can be formed

  \insertQR{QRC}
\begin{solution}[\mcq]
  	\begin{align}
  		N &= \encr\tmp{4} + \encr\tmp{3}\cdot\vbtwo + \encr\tmp{2}\cdot\encr\vbtwo{2} \\
  		&= \vbfive
  	\end{align}
  \end{solution}

  \part[1] \text{Line segments} that can be formed

  \insertQR{QRC}
\begin{solution}[\mcq]
  	\begin{align}
  		N &= \encr\vbone{2} = \vbsix
  	\end{align}
  \end{solution}

\end{parts}
