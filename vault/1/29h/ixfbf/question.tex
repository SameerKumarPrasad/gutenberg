% This is an empty shell file placed for you by the 'examiner' script.
% You can now fill in the TeX for your question here.

% Now, down to brasstacks. ** Writing good solutions is an Art **. 
% Eventually, you will find your own style. But here are some thoughts 
% to get you started: 
%
%   1. Write to be understood - but be crisp. Your own solution should not take 
%      more space than you will give to the student. Hence, if you take more than 
%      a half-page to write a solution, then give the student a full-page and so on...
%
%   2. Use margin-notes to "talk" to students about the critical insights
%      in the question. The tone can be - in fact, should be - informal
%
%   3. Don't shy away from creating margin-figures you think will help
%      students understand. Yes, it is a little more work per question. 
%      But the question & solution will be written only once. Make that
%      attempt at writing a solution count.
%      
%      3b. Use bc_to_fig.tex. Its an easier way to generate plots & graphs 
% 
%   4. Ensure that there are *no spelling mistakes anywhere*. We are an 
%      education company. Bad spellings suggest that we ourselves 
%      don't have any education. Also, use American spellings by default
% 
%   5. If a question has multiple parts, then first delete lines 40-41
%   6. If a question does not have parts, then first delete lines 43-69
%   
%   7. Create versions of the question when possible. Use commands defined in 
%      tufte-tweaks.sty to do so. Its easier than you think

%\noprintanswers
%\setcounter{rolldice}{1}

\ifnumequal{\value{rolldice}}{0}{
  % variables 
  \renewcommand{\vbone}{10}
  \renewcommand{\vbtwo}{126.25}
}{
  \ifnumequal{\value{rolldice}}{1}{
    % variables 
    \renewcommand{\vbone}{15}
    \renewcommand{\vbtwo}{373.75}
  }{
    \ifnumequal{\value{rolldice}}{2}{
      % variables 
      \renewcommand{\vbone}{19}
      \renewcommand{\vbtwo}{717.25}
    }{
      % variables 
      \renewcommand{\vbone}{16}
      \renewcommand{\vbtwo}{446}
    }
  }
}

\question[4] Find the sum of the first $\vbone$ terms of the series $\dfrac{1^3}{1} + \dfrac{1^3+2^3}{1+3} + \dfrac{1^3+2^3+3^3}{1+3+5}\ldots$

\insertQR{QRC}

\watchout

\ifprintanswers
\fi 

\begin{solution}[\halfpage]
	The $n^{\text{th}}$ term of the series is of the form $a_n = \dfrac{\sum_{k=1}^{n}k^3}{\sum_{k=1}^{n}(2k-1)}
	= \dfrac{\eSumOfCubes{n}}{n^2} = \dfrac{1}{4}\cdot(n+1)^2$
	
	Hence, the sum of the first $M$ terms of the series ($M$ could be anything) would be given by 
	\begin{align}
		S_n &= \sum_{n=1}^{M}a_n = \dfrac{1}{4}\cdot\left[ \sum_{n=1}^{M}n^2 + 2\cdot\sum_{n=1}^{M}n + \sum_{n=1}^{M}1 \right] \\
		&= \dfrac{1}{4}\cdot\left[ \eSumOfSquares{M} + 2\eSumOfN{M} + M \right] \\ 
		&= \dfrac{M}{24}\cdot(2M^2 + 9M + 13)
	\end{align}
	For $M=\vbone$, $S_{\vbone} = \vbtwo$
\end{solution}
