% This is an empty shell file placed for you by the 'examiner' script.
% You can now fill in the TeX for your question here.

% Now, down to brasstacks. ** Writing good solutions is an Art **. 
% Eventually, you will find your own style. But here are some thoughts 
% to get you started: 
%
%   1. Write to be understood - but be crisp. Your own solution should not take 
%      more space than you will give to the student. Hence, if you take more than 
%      a half-page to write a solution, then give the student a full-page and so on...
%
%   2. Use margin-notes to "talk" to students about the critical insights
%      in the question. The tone can be - in fact, should be - informal
%
%   3. Don't shy away from creating margin-figures you think will help
%      students understand. Yes, it is a little more work per question. 
%      But the question & solution will be written only once. Make that
%      attempt at writing a solution count.
%      
%      3b. Use bc_to_fig.tex. Its an easier way to generate plots & graphs 
% 
%   4. Ensure that there are *no spelling mistakes anywhere*. We are an 
%      education company. Bad spellings suggest that we ourselves 
%      don't have any education. Also, use American spellings by default
% 
%   5. If a question has multiple parts, then first delete lines 40-41
%   6. If a question does not have parts, then first delete lines 43-69
%   
%   7. Create versions of the question when possible. Use commands defined in 
%      tufte-tweaks.sty to do so. Its easier than you think

% \noprintanswers
% \setcounter{rolldice}{3}

\ifnumequal{\value{rolldice}}{0}{
  % variables 
  \renewcommand{\vbone}{NEWSPAPER }
  \renewcommand{\vbtwo}{odd }
  \renewcommand{\vbthree}{9} % word-length
  \renewcommand{\vbfour}{3} % # of vowels
  \renewcommand{\vbfive}{10,800} % answer
  \renewcommand{\vbsix}{2!}
  \renewcommand{\vbseven}{2!}
  \renewcommand{\vbeight}{}
  \renewcommand{\vbnine}{}
  \renewcommand{\vbten}{}
}{
  \ifnumequal{\value{rolldice}}{1}{
    % variables 
    \renewcommand{\vbone}{AIRCRAFT }
    \renewcommand{\vbtwo}{even }
    \renewcommand{\vbthree}{8}
    \renewcommand{\vbfour}{3}
    \renewcommand{\vbfive}{720}
    \renewcommand{\vbsix}{2!}
    \renewcommand{\vbseven}{2!}
    \renewcommand{\vbeight}{}
    \renewcommand{\vbnine}{}
    \renewcommand{\vbten}{}
  }{
    \ifnumequal{\value{rolldice}}{2}{
      % variables 
      \renewcommand{\vbone}{TASMANIA }
      \renewcommand{\vbtwo}{odd }
      \renewcommand{\vbthree}{8}
      \renewcommand{\vbfour}{4}
      \renewcommand{\vbfive}{96}
      \renewcommand{\vbsix}{3!}
      \renewcommand{\vbseven}{1!}
      \renewcommand{\vbeight}{}
      \renewcommand{\vbnine}{}
      \renewcommand{\vbten}{}
    }{
      % variables 
      \renewcommand{\vbone}{LOPSIDED }
      \renewcommand{\vbtwo}{even }
      \renewcommand{\vbthree}{8}
      \renewcommand{\vbfour}{3}
      \renewcommand{\vbfive}{1440}
      \renewcommand{\vbsix}{1!}
      \renewcommand{\vbseven}{2!}
      \renewcommand{\vbeight}{}
      \renewcommand{\vbnine}{}
      \renewcommand{\vbten}{}
    }
  }
}

\gcalcexpr[0]\con{\vbthree - \vbfour}
\ifthenelse{\equal{\vbtwo}{odd }}{
	\gcalcexpr[0]\ntype{\vbthree / 2}
}{
  \ifthenelse{\isodd{\vbthree}}{
  	\gcalcexpr[0]\ntype{(\vbthree - 1) / 2}
  }{
  	\gcalcexpr[0]\ntype{\vbthree / 2}
  }
}

\question[2] How many words - with or without meaning - can be formed from the characters in the word 
\vbone such that the vowels are in the \vbtwo places

\insertQR{QRC}

\watchout

\ifprintanswers
\fi 

\begin{solution}[\mcq]
	The whole word is $\vbthree$ characters long and it has $\vbfour$ vowels. The remaining $\con$ characters 
	are obviously consonants
	
	Now, with $\vbthree$ characters, the word has $\ntype$ \vbtwo positions for the vowels
	Moreover, even though there are $\vbfour$ vowels, some of them are the same and therefore some permutations of them 
	would actually result in the same word. Same goes for consonants 
	
	We will leave it you to count the duplicate vowels and consonants - if any. But bearing the above in mind, 
	the required number of distinct words is
	
	\begin{align}
		N &= \underbrace{\encr\ntype\vbfour\cdot\dfrac{\vbfour !}{\vbsix}}_{\texttt{distinct permutations of vowels}}
		\times \overbrace{\dfrac{\con !}{\vbseven}}^{\texttt{\ldots and of consonants}} = \vbfive
	\end{align}
\end{solution}
