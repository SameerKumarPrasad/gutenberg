


\ifnumequal{\value{rolldice}}{0}{
  % variables 
  \renewcommand{\vbone}{NEWSPAPER }
  \renewcommand{\vbtwo}{odd }
  \renewcommand{\vbthree}{9} % word-length
  \renewcommand{\vbfour}{3} % # of vowels
  \renewcommand{\vbfive}{10,800} % answer
  \renewcommand{\vbsix}{2!}
  \renewcommand{\vbseven}{2!}
}{
  \ifnumequal{\value{rolldice}}{1}{
    % variables 
    \renewcommand{\vbone}{AIRCRAFT }
    \renewcommand{\vbtwo}{even }
    \renewcommand{\vbthree}{8}
    \renewcommand{\vbfour}{3}
    \renewcommand{\vbfive}{720}
    \renewcommand{\vbsix}{2!}
    \renewcommand{\vbseven}{2!}
  }{
    \ifnumequal{\value{rolldice}}{2}{
      % variables 
      \renewcommand{\vbone}{TASMANIA }
      \renewcommand{\vbtwo}{odd }
      \renewcommand{\vbthree}{8}
      \renewcommand{\vbfour}{4}
      \renewcommand{\vbfive}{96}
      \renewcommand{\vbsix}{3!}
      \renewcommand{\vbseven}{1!}
    }{
      % variables 
      \renewcommand{\vbone}{LOPSIDED }
      \renewcommand{\vbtwo}{even }
      \renewcommand{\vbthree}{8}
      \renewcommand{\vbfour}{3}
      \renewcommand{\vbfive}{1440}
      \renewcommand{\vbsix}{1!}
      \renewcommand{\vbseven}{2!}
    }
  }
}

\gcalcexpr[0]\con{\vbthree - \vbfour}
\ifthenelse{\equal{\vbtwo}{odd }}{
	\gcalcexpr[0]\ntype{\vbthree / 2}
}{
  \ifthenelse{\isodd{\vbthree}}{
  	\gcalcexpr[0]\ntype{(\vbthree - 1) / 2}
  }{
  	\gcalcexpr[0]\ntype{\vbthree / 2}
  }
}

\question[2] How many words - with or without meaning - can be formed from the characters in the word 
\vbone such that the vowels are in the \vbtwo places


\watchout

\ifprintanswers
\fi 

\begin{solution}[\mcq]
	The whole word is $\vbthree$ characters long and it has $\vbfour$ vowels. The remaining $\con$ characters 
	are obviously consonants
	
	Now, with $\vbthree$ characters, the word has $\ntype$ \vbtwo positions for the vowels
	Moreover, even though there are $\vbfour$ vowels, some of them are the same and therefore some permutations of them 
	would actually result in the same word. Same goes for consonants 
	
	We will leave it you to count the duplicate vowels and consonants - if any. But bearing the above in mind, 
	the required number of distinct words is
	
	\begin{align}
		N &= \underbrace{\encr\ntype\vbfour\cdot\dfrac{\vbfour !}{\vbsix}}_{\texttt{distinct permutations of vowels}}
		\times \overbrace{\dfrac{\con !}{\vbseven}}^{\texttt{\ldots and of consonants}} = \vbfive
	\end{align}
\end{solution}
