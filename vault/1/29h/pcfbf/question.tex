% This is an empty shell file placed for you by the 'examiner' script.
% You can now fill in the TeX for your question here.

% Now, down to brasstacks. ** Writing good solutions is an Art **. 
% Eventually, you will find your own style. But here are some thoughts 
% to get you started: 
%
%   1. Write to be understood - but be crisp. Your own solution should not take 
%      more space than you will give to the student. Hence, if you take more than 
%      a half-page to write a solution, then give the student a full-page and so on...
%
%   2. Use margin-notes to "talk" to students about the critical insights
%      in the question. The tone can be - in fact, should be - informal
%
%   3. Don't shy away from creating margin-figures you think will help
%      students understand. Yes, it is a little more work per question. 
%      But the question & solution will be written only once. Make that
%      attempt at writing a solution count.
%      
%      3b. Use bc_to_fig.tex. Its an easier way to generate plots & graphs 
% 
%   4. Ensure that there are *no spelling mistakes anywhere*. We are an 
%      education company. Bad spellings suggest that we ourselves 
%      don't have any education. Also, use American spellings by default
% 
%   5. If a question has multiple parts, then first delete lines 40-41
%   6. If a question does not have parts, then first delete lines 43-69
%   
%   7. Create versions of the question when possible. Use commands defined in 
%      tufte-tweaks.sty to do so. Its easier than you think

%\noprintanswers
%\setcounter{rolldice}{1}

\ifnumequal{\value{rolldice}}{0}{
  % variables 
  \renewcommand{\vbone}{16}
  \renewcommand{\vbtwo}{37}
}{
  \ifnumequal{\value{rolldice}}{1}{
    % variables 
    \renewcommand{\vbone}{12}
    \renewcommand{\vbtwo}{29}
  }{
    \ifnumequal{\value{rolldice}}{2}{
      % variables 
      \renewcommand{\vbone}{17}
      \renewcommand{\vbtwo}{44}
    }{
      % variables 
      \renewcommand{\vbone}{19}
      \renewcommand{\vbtwo}{25}
    }
  }
}

\gcalcexpr[0]\tp{\vbone - 1}
\gcalcexpr[0]\tq{\vbtwo - 1}
\gcalcexpr[0]\tr{\vbtwo - \vbone}

\question[3] The first, $\vbone^\text{th}$ and $\vbtwo^\text{th}$ terms of an arithmetic progression are also 
\textit{successive} terms of a geometric progression. Find the common ratio of the geometric progression

\insertQR[-10pt]{QRC}

\watchout

\ifprintanswers
\fi 

\begin{solution}[\halfpage]
	Here is what we know 
	\begin{align}
		a &= b \\
		a + \tp\cdot d &= b\cdot r \\
		a + \tq\cdot d &= b\cdot r^2 
	\end{align}
	
	From this, we can infer 
	\begin{align}
		(a + \tp\cdot d) - a &= b\cdot r - b = b\cdot(r-1) \\
		(a + \tq\cdot d) - a &= b\cdot r^2 - b = b\cdot(r^2-1) \\ 
		&= \underbrace{b\cdot(r-1)}_{\tp\cdot d}\cdot(r+1) \\
		\Rightarrow \tq\cdot d &= \tp\cdot d \cdot (r + 1) \\ 
		\Rightarrow r &= \dfrac{\tr}{\tp}
	\end{align}
\end{solution}
