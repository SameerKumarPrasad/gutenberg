


\ifnumequal{\value{rolldice}}{0}{
  % variables 
  \renewcommand{\vbone}{16}
  \renewcommand{\vbtwo}{37}
}{
  \ifnumequal{\value{rolldice}}{1}{
    % variables 
    \renewcommand{\vbone}{12}
    \renewcommand{\vbtwo}{29}
  }{
    \ifnumequal{\value{rolldice}}{2}{
      % variables 
      \renewcommand{\vbone}{17}
      \renewcommand{\vbtwo}{44}
    }{
      % variables 
      \renewcommand{\vbone}{19}
      \renewcommand{\vbtwo}{25}
    }
  }
}

\gcalcexpr[0]\tp{\vbone - 1}
\gcalcexpr[0]\tq{\vbtwo - 1}
\gcalcexpr[0]\tr{\vbtwo - \vbone}

\question[3] The first, $\vbone^\text{th}$ and $\vbtwo^\text{th}$ terms of an arithmetic progression are also 
\textit{successive} terms of a geometric progression. Find the common ratio of the geometric progression


\watchout

\ifprintanswers
\fi 

\begin{solution}[\halfpage]
	Here is what we know 
	\begin{align}
		a &= b \\
		a + \tp\cdot d &= b\cdot r \\
		a + \tq\cdot d &= b\cdot r^2 
	\end{align}
	
	From this, we can infer 
	\begin{align}
		(a + \tp\cdot d) - a &= b\cdot r - b = b\cdot(r-1) \\
		(a + \tq\cdot d) - a &= b\cdot r^2 - b = b\cdot(r^2-1) \\ 
		&= \underbrace{b\cdot(r-1)}_{\tp\cdot d}\cdot(r+1) \\
		\Rightarrow \tq\cdot d &= \tp\cdot d \cdot (r + 1) \\ 
		\Rightarrow r &= \dfrac{\tr}{\tp}
	\end{align}
\end{solution}
