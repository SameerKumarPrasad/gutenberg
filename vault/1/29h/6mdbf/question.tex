


\ifnumequal{\value{rolldice}}{0}{
  % variables 
  \renewcommand{\va}{6}
  \renewcommand{\vb}{30,240}
}{
  \ifnumequal{\value{rolldice}}{1}{
    % variables 
    \renewcommand{\va}{7}
    \renewcommand{\vb}{282,240}
  }{
    \ifnumequal{\value{rolldice}}{2}{
      % variables 
      \renewcommand{\va}{5}
      \renewcommand{\vb}{3600}
    }{
      % variables 
      \renewcommand{\va}{8}
      \renewcommand{\vb}{2,903,040}
    }
  }
}

\ADD\va{2}\vm
\SUBTRACT\vm{1}\vn

\question[3] At a buffet, guests get one type of soup and one type of dessert along with $\va$ other 
main-course dishes. In how many ways can the buffet table be laid if the soup and the 
dessert should \textit{never} be placed together? 

\watchout

\begin{solution}[\mcq]
  In all, $\vm$ distinct items ($\va$ main + soup + dessert)  must be laid 
  out on the buffet table.

  Any layout in which the soup and the dessert are together is invalid. 
  All other layouts are valid. Which means,
  \[ N_{valid} = N_{all} - N_{invalid} \]

  To count the \textbf{invalid} layouts, let us bundle the soup and the dessert together and see them as one unit. 
  With this new unit, there would now be $M = (\vm - 2) + 1 = \vn$ distinct items to lay on the table. 

  And hence, 
  \begin{align}
    N_{all} &= \vm ! \\
    N_{invalid} &= \vn ! \times 2! \\
    \implies N_{valid} &= \vm !- \vn !\times 2! = \vn !\cdot (\vm - 2) \nonumber\\
                       &= \vb
  \end{align}
\end{solution}

\ifprintanswers
  \begin{codex}
    $N = \vn !\cdot \va = \vb$
  \end{codex}
\fi 
