
\ifnumequal{\value{rolldice}}{0}{
  % variables 
  \renewcommand{\vbone}{1}
}{
  \ifnumequal{\value{rolldice}}{1}{
    % variables 
    \renewcommand{\vbone}{2}
  }{
    \ifnumequal{\value{rolldice}}{2}{
      % variables 
      \renewcommand{\vbone}{3}
    }{
      % variables 
      \renewcommand{\vbone}{4}
    }
  }
}

\SQUARE\vbone\vbtwo
\POWER\vbone{3}\vbthree
\MULTIPLY\vbtwo{3}\vbfour

\question Does the following limit exist? If yes, then what? If not, then why not?
\[ \lim_{x\to\vbone}\left[ \dfrac{x^3-\vbthree}{(x-\vbone)^2}\right]\]

\watchout[-40pt]

\begin{solution}
  In of itself, the expression is of the form $\frac{0}{0}$ for $x=\vbone$ - which cannot be determined. So, lets try to get it to be something that can be determined. Might not be possible - but lets try anyway
  
  \begin{align}
    \lim_{x\to\vbone}\left[ \dfrac{x^3-\vbthree}{(x-\vbone)^2}\right] &= 
    \lim_{x\to\vbone}\left[ \dfrac{(x-\vbone)\cdot (x^2 + \vbone x + \vbtwo)}{(x-\vbone)^2}\right]\\
    &= \lim_{x\to\vbone}\left( \dfrac{x^2+\vbone x + \vbtwo}{x-\vbone}\right)
  \end{align}
  
  Now, at $x=\vbone$, the \textbf{numerator} = \vbfour. However, note that 
  \begin{align}
    \lim_{x\to\vbone^+}\left( \dfrac{x^2+\vbone x + \vbtwo}{x-\vbone}\right) &= \infty \\
    \lim_{x\to\vbone^-}\left( \dfrac{x^2+\vbone x + \vbtwo}{x-\vbone}\right) &= -\infty
  \end{align}
  
  And as \[ \lim_{x\to\vbone^+} f(x) \neq \lim_{x\to\vbone^-} f(x)\], the \textbf{limit does not exist}
\end{solution}
