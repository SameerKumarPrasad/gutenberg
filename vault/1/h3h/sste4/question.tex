

\ifnumequal{\value{rolldice}}{0}{
  \renewcommand{\vbone}{2}
  \renewcommand{\vbtwo}{\dfrac{2}{3}}
  \renewcommand{\vbthree}{\dfrac{8\pi}{3}}
}{
  \ifnumodd{\value{rolldice}}{
    \renewcommand{\vbone}{3}
    \renewcommand{\vbtwo}{\dfrac{3}{3}}
    \renewcommand{\vbthree}{\dfrac{27\pi}{2}}
  }{
    \renewcommand{\vbone}{4}
    \renewcommand{\vbtwo}{\dfrac{4}{3}}
    \renewcommand{\vbthree}{\dfrac{128\pi}{3}}
  }
}

\question[4] Compute the volume of the solid obtained by revolving - about the $y-axis$ -
the figure bounded by the parabola $y=\vbone x-x^2$ and the $x-axis$

\watchout

\ifprintanswers
  % stuff to be shown only in the answer key - like explanatory margin figures
  \begin{marginfigure}
\figinit{pt}
\def\Xmin{-8.00000}
\def\Ymin{-25.22999}
\def\Xmax{72.00000}
\def\Ymax{44.77000}
\def\Xori{8.00000}
\def\Yori{25.22999}
\figpt0:(\Xori,\Yori)
\figpt 100:$\vbone,0$(71,24)
\figpt 101:$0,0$(9,24)
\figpt 102:$x_0$(47,26)
\figpt 103:(47,68)
\figdrawbegin{}
\def\Xmaxx{\Xmax} % To customize the position
\def\Ymaxx{\Ymax} % of the arrow-heads of the axes.
\figset arrowhead(length=4, fillmode=yes) % styling the arrowheads
\figdrawaxes 0(\Xmin, \Xmaxx, \Ymin, \Ymaxx)
\figdrawline [102,103]
\figdrawlineC(
0 0,
2.75862 9.33333,
5.51724 17.99999,
8.27586 25.99999,
11.03448 33.33333,
13.79310 39.99999,
16.55172 45.99999,
19.31034 51.33333,
22.06896 55.99999,
24.82758 59.99999,
27.58620 63.33333,
30.34482 65.99999,
33.10344 67.99999,
35.86206 69.33333,
38.62068 69.99999,
41.37931 69.99999,
44.13793 69.33333,
46.89655 68.00000,
49.65517 66.00000,
52.41379 63.33333,
55.17241 60.00000,
57.93103 56.00000,
60.68965 51.33333,
63.44827 46.00000,
66.20689 40.00000,
68.96551 33.33333,
71.72413 26.00000,
74.48275 18.00000,
77.24137 9.33333,
79.99999 0
)
\figdrawend
\figvisu{\figBoxA}{}{%
\figptsaxes 1:0(\Xmin, \Xmaxx, \Ymin, \Ymaxx)
\figwritee 1:(5pt)     \figwriten 2:(5pt)
\figptsaxes 1:0(\Xmin, \Xmax, \Ymin, \Ymax)
\figwritesw 100:(2)
\figwritese 101:(2)
\figwrites 102:(2)
}
\centerline{\box\figBoxA}

  \end{marginfigure}
\fi 

\begin{solution}[\halfpage]
   Imagine a thin strip at $x = x_0$ that is rotated around the y-axis. The volume 
   added to the total by this thin strip would be 
   \begin{align}
     \ud V &= \underbrace{2\pi\cdot x_0}_{\text{circumference}}\times y_0\ud x
   \end{align}
   
   This suggests that the total volume of the generated solid would be
   \begin{align}
   	 V &= \int_0^{\vbone} (2\pi\cdot x)\times(\vbone x-x^2)\ud x \\
   	   &= 2\pi\int_0^{\vbone} (\vbone x^2-x^3)\ud x \\
   	   &= 2\pi\times\left[ \left( \vbtwo x^3\right)_0^{\vbone} - \left( \dfrac{x^4}{4}\right)_0^{\vbone}\right] \\
   	   &= \vbthree
   \end{align}
\end{solution}
