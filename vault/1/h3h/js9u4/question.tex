% This is an empty shell file placed for you by the 'examiner' script.
% You can now fill in the TeX for your question here.

% Now, down to brasstacks. ** Writing good solutions is an Art **. 
% Eventually, you will find your own style. But here are some thoughts 
% to get you started: 
%
%   1. Write the solution as if you are writing it for your favorite
%      14-17 year old to help him/her understand. Could be your nephew, 
%      your niece, a cousin perhaps or probably even you when you 
%      were that age. Just write for them.
%
%   2. Use margin-notes to "talk" to students about the critical insights
%      in the question. The tone can be - in fact, should be - informal
%
%   3. Don't shy away from creating margin-figures you think will help
%      students understand. Yes, it is a little more work per question. 
%      But the question & solution will be written only once. Make that
%      attempt at writing a solution count.
%
%   4. At the same time, do not be too verbose. A long solution can
%      - at first sight - make the student think, "God, that is a lot to know".
%      Our aim is not to scare students. Rather, our aim should be to 
%      create many "Aha!" moments everyday in classrooms around the world
% 
%   5. Ensure that there are *no spelling mistakes anywhere*. We are an 
%      education company. Bad spellings suggest that we ourselves 
%      don't have any education. Also, use American spellings by default
% 
%   6. If a question has multiple parts, then first delete lines 40-41
%   7. If a question does not have parts, then first delete lines 43-69

\question[2] Two circles intersect at points $A$ and $B$ - \asif. If $AP$ and $AQ$ be their 
respective diameters, then prove that $PBQ$ is a line

\insertQR{QRC}

  % stuff to be shown only in the answer key - like explanatory margin figures
  \begin{marginfigure}
    \figinit{pt}
    	\figpt 100: $C_1$(40,40)
    	\figpt 101: $C_2$(85,40)
      \figpt 102: (70,40) % right extreme of C_1
      \figpt 103: (40,70) % top extreme of C_1
      \figpt 104: (50,40) % left C_2
      \figpt 105: (85,75)
      \figptsintercirc 200 [100,30;101,35] % 201 is the other point
      \figptsinterlinellP 300 [100,102,103 ; 201,100] % 301 is the other point
      \figptsinterlinellP 310 [101,104,105 ; 201,101] % 311 is the other point
    \figdrawbegin{}
      \figdrawcirc 100(30)
      \figdrawcirc 101(35)
      \figdrawline [201,301]
      \figdrawline [201,311]
      \figdrawline [301,311]
      \ifprintanswers
        \figset (dash=5)
        \figdrawline [201,200]
      \fi
    \figdrawend
    \figvisu{\figBoxA}{}{%
      \figset write(mark=$\bullet$)
      \figwritew 100:(2)
      \figwritee 101:(1)
      \figwrites 200:$B$(3) 
      \figwriten 201:$A$(3) 
      \figwrites 301:$P$(3)
      \figwrites 311:$Q$(3)
    }
    \centerline{\box\figBoxA}
  \end{marginfigure} 

\begin{solution}[\halfpage]
	If $AP$ and $AQ$ are diameters, then $\angle APB$ and $\angle APQ$ are angles
	within a semi-circle - which we know is always equal to $\ang{90}$
	
	And therefore, 
	\begin{align}
		\angle PBA + \angle QBA &= \ang{90} + \ang{90} \\
		                        &= \ang{180} \\
		     \Rightarrow \text{ points P,B and Q are collinear }
	\end{align}
\end{solution}

