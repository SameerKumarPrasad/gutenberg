
\ifnumequal{\value{rolldice}}{0}{
  \renewcommand\va{80}
  \renewcommand\vb{120}
}{
  \ifnumequal{\value{rolldice}}{1}{
    \renewcommand\va{64}
    \renewcommand\vb{84}
  }{
    \ifnumequal{\value{rolldice}}{2}{
      \renewcommand\va{78}
      \renewcommand\vb{54}
    }{
      \renewcommand\va{94}
      \renewcommand\vb{78}
    }
  }
}

\ADD\va\vb\vc 
\SUBTRACT{360}\vc\vd
\DIVIDE\vd{2}\ve
\EXPR[0]\vx{(180-\va)/2}
\EXPR[0]\vy{(180-\vb)/2}
\ADD\vx\vy\vz

\question[3]  In the adjoining figure, $O$ is the center of the circle. Find $\angle BAC$
given that $\angle BOA = \ang\va$ and $\angle AOC = \ang{120}$

\watchout

\figinit{pt}
  \figpt 1: (60,0)
  \figptcirc 2:$A$: 1;50(100)
  \figptcirc 3:$B$: 1;50(190)
  \figptcirc 4:$C$: 1;50(340)
\figdrawbegin{}
\figdrawcirc 1(50)
  \figdrawline [1,2]
  \figdrawline [1,3]
  \figdrawline [1,4]
  \figdrawarccircP 1;8 [2,3]
  \figdrawarccircP 1;10 [4,2]
  \ifprintanswers
    \figset (dash=4)
    \figdrawline [3,2,4]
  \fi
\figdrawend
\figvisu{\figBoxA}{}{
  \figsetmark{$\bullet$}
  \figwrites 1:$O$(5)
  \figwriten 2:(5)
  \figwritew 3:(5)
  \figwritee 4:(5)
}

\ifprintanswers
  \begin{marginfigure}
    \centerline{\box\figBoxA}
  \end{marginfigure}
\else
  \vspace{1cm}
  \centerline{\box\figBoxA}
\fi

\begin{solution}[\halfpage]
  \textbf{Method \#1}

  Note that 
  \[ \angle BAC = \dfrac{1}{2}\angle BOC \]
  Moreover, 
  \[ \angle BOC = \ang{360} - (\angle BOA + \angle AOC) \implies \angle BOC = \ang{360}-\ang\vc = \ang\vd \]
  And therefore, 
  \[ \angle BAC = \dfrac{1}{2}\ang\vd = \ang\ve \]
  \textbf{Method \#2}

	Alternatively, if we join $A$ with $B$ and $A$ with $C$ as shown, then we get two isoceles 
	triangles, $\triangle AOB$ and $\triangle AOC$
	
	In $\triangle AOB$,
	\begin{align}
		\angle OBA = \angle OAB &= \dfrac{1}{2}\cdot(\ang{180}-\angle AOB) = \ang\vx
	\end{align}
	Similarly, in $\triangle AOC$
	\begin{align}
		\angle OAC = \angle OCA &= \dfrac{1}{2}\cdot(\ang{180}-\angle AOC) = \ang\vy
	\end{align}
	And therefore,
	\begin{align}
		\angle BAC &= \angle OAB + \angle OAC = \ang\vx + \ang\vy = \ang\vz
	\end{align}
\end{solution}

\ifprintanswers\begin{codex}$\ang\vz$\end{codex}\fi
