

\question[3]  In the adjoining figure, $O$ is the center of the circle. Find $\angle BAC$
given that $\angle BOA = \ang{80}$ and $\angle AOC = \ang{120}$


\ifprintanswers
  % stuff to be shown only in the answer key - like explanatory margin figures
\fi 
\begin{marginfigure}
	\figinit{pt}
    \figpt 1: (60,0)
    \figptcirc 2:$A$: 1;50(100)
    \figptcirc 3:$B$: 1;50(190)
    \figptcirc 4:$C$: 1;50(340)
	\figdrawbegin{}
	\figdrawcirc 1(50)
    \figdrawline [1,2]
    \figdrawline [1,3]
    \figdrawline [1,4]
    \figdrawarccircP 1;8 [2,3]
    \figdrawarccircP 1;10 [4,2]
    \ifprintanswers
      \figset (dash=4)
      \figdrawline [3,2,4]
    \fi
	\figdrawend
  \figvisu{\figBoxA}{}{
    \figsetmark{$\bullet$}
    \figwrites 1:$O$(5)
    \figwriten 2:(5)
    \figwritew 3:(5)
    \figwritee 4:(5)
  }
  \centerline{\box\figBoxA}
\end{marginfigure}
\begin{solution}[\halfpage]
  For the short solution, one should note that $\angle BAC = \dfrac{1}{2}\angle BOC$,
  where $\angle BOC = \ang{360} - \angle BOA - \angle AOC = \ang{160} \Rightarrow \angle BAC = \ang{80}$

	Alternatively, if we join $A$ with $B$ and $A$ with $C$ as shown, then we get two isoceles 
	triangles, $\triangle AOB$ and $\triangle AOC$
	
	In $\triangle AOB$,
	\begin{align}
		\angle OBA = \angle OAB &= \dfrac{1}{2}\cdot(\ang{180}-\angle AOB) \\
		                        &= \ang{50}
	\end{align}
	
	Similarly, in $\triangle AOC$
	\begin{align}
		\angle OAC = \angle OCA &= \dfrac{1}{2}\cdot(\ang{180}-\angle AOC) \\
		                        &= \ang{30}
	\end{align}
	
	And therefore,
	\begin{align}
		\angle BAC &= \angle OAB + \angle OAC \\
		           &= \ang{50} + \ang{30} = \ang{80}
	\end{align}

\end{solution}
