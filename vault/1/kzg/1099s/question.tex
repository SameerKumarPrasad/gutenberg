
\ifnumequal{\value{rolldice}}{0}{
  \renewcommand\va{100}
}{
  \ifnumequal{\value{rolldice}}{1}{
    \renewcommand\va{64}
  }{
    \ifnumequal{\value{rolldice}}{2}{
      \renewcommand\va{78}
    }{
      \renewcommand\va{94}
    }
  }
}

\DIVIDE\va{2}\vb
\SUBTRACT{180}\vb\vc

\question[2]  If $O$ be the center of a circle and $\angle BOC = \ang\va$, then 
find $\angle BDC$.

\watchout

\figinit{pt}
  \figpt 1: (60,0)
  \figptcirc 2:$A$: 1;50(70)
  \figptcirc 3:$B$: 1;50(200)
  \figptcirc 4:$C$: 1;50(310)
  \figptcirc 5:$D$: 1;50(250)
\figdrawbegin{}
\figdrawcirc 1(50)
  \figdrawline [3,1,4,5,3]
  \ifprintanswers
    \figdrawline [3,2,4]
  \fi
\figdrawend
\figvisu{\figBoxA}{}{
  \Large
  \figsetmark{$\bullet$}
  \figwritene 1:$O$(3)
  \ifprintanswers
    \figwriten 2:(5)
  \fi
  \figwritew 3:(5)
  \figwritee 4:(5)
  \figwrites 5:(5)
}

\vspace{1cm}
\centerline{\box\figBoxA}

\begin{solution}[\halfpage]
	If we pick a point $A$ on the circle's circumfrence, then $ACDB$ is a cyclic 
	quadrilateral in which,
	\begin{align}
		\angle BAC &= \dfrac{1}{2}\cdot\angle BOC = \ang\vb
	\end{align}
	Moreover, because it is a cyclic quadrilateral, 
	\begin{align}
		\angle BAC + \angle BDC &= \ang{180} \\
		\implies \angle BDC &= \ang{180} - \angle BAC = \ang\vc
	\end{align}
\end{solution}
\ifprintanswers\begin{codex}$\ang\vc$\end{codex}\fi
