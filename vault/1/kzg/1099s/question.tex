

\question[2]  In the adjoining figure, $O$ is the center of the circle and $\angle BOC = \ang{100}$.
What is $\angle BDC$ equal to ?


\ifprintanswers
  % stuff to be shown only in the answer key - like explanatory margin figures
\fi 
\begin{marginfigure}
	\figinit{pt}
    \figpt 1: (60,0)
    \figptcirc 2:$A$: 1;50(70)
    \figptcirc 3:$B$: 1;50(200)
    \figptcirc 4:$C$: 1;50(310)
    \figptcirc 5:$D$: 1;50(250)
	\figdrawbegin{}
	\figdrawcirc 1(50)
    \figdrawline [3,1,4,5,3]
    \ifprintanswers
      \figdrawline [3,2,4]
    \fi
	\figdrawend
  \figvisu{\figBoxA}{}{
    \figsetmark{$\bullet$}
    \figwrites 1:$O$(5)
    \ifprintanswers
      \figwriten 2:(5)
    \fi
    \figwritew 3:(5)
    \figwritee 4:(5)
    \figwrites 5:(5)
  }
  \centerline{\box\figBoxA}
\end{marginfigure}
\begin{solution}[\halfpage]
	If we pick a point $A$ on the circle's circumfrence, then $ACDB$ is a cyclic 
	quadrilateral in which,
	\begin{align}
		\angle BAC &= \dfrac{1}{2}\cdot\angle BOC = \ang{50}
	\end{align}
	Moreover, because it is a cyclic quadrilateral, 
	\begin{align}
		\angle BAC + \angle BDC &= \ang{180} \\
		\Rightarrow \angle BDC &= \ang{180} - \angle BAC = \ang{130}
	\end{align}
\end{solution}
