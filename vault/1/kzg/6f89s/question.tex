

\ifnumequal{\value{rolldice}}{0}{
  % variables 
  \renewcommand{\va}{84}
  \renewcommand{\vb}{29}
}{
  \ifnumequal{\value{rolldice}}{1}{
    % variables 
    \renewcommand{\va}{68}
    \renewcommand{\vb}{17}
  }{
    \ifnumequal{\value{rolldice}}{2}{
      % variables 
      \renewcommand{\va}{78}
      \renewcommand{\vb}{23}
    }{
      % variables 
      \renewcommand{\va}{96}
      \renewcommand{\vb}{39}
    }
  }
}

\DIVIDE\va{2}\vc
\SUBTRACT\vc\vb\vd

\question[2]  In the adjoining figure, what is the value of $\angle ABO$ if
$\angle BOC = \ang\va$ and $\angle ACO = \ang\vb$?

\watchout

\figinit{pt}
  \figpt 1: (60,0)
  \figptcirc 2:$A$: 1;50(70)
  \figptcirc 3:$B$: 1;50(200)
  \figptcirc 4:$C$: 1;50(310)
\figdrawbegin{}
  \figdrawcirc 1(50)
  \figdrawline [2,3,1,4,2]
  \ifprintanswers
    \figdrawline [1,2]
  \fi
\figdrawend
\figvisu{\figBoxA}{}{
  \figsetmark{$\bullet$}
  \figwrites 1:$O$(5)
  \figwriten 2:(5)
  \figwritew 3:(5)
  \figwritee 4:(5)
}

\ifprintanswers
  \begin{marginfigure}
    \centerline{\box\figBoxA}
  \end{marginfigure}
\else
  \vspace{1cm}
  \centerline{\box\figBoxA}
\fi 

\begin{solution}[\halfpage]
	In the figure, 
	\begin{align}
 		\angle BAC &= \dfrac{1}{2}\angle BOC = \dfrac{1}{2}\cdot\ang\va = \ang\vc
	\end{align}
	
	Also, 
	\begin{align}
		\angle BAC &= \angle BAO + \angle OAC \\
		           &= \angle BAO + \ang\vb \\
		\implies \angle BAO &= \angle OBA = \ang\vc - \ang\vb = \ang\vd
	\end{align}
\end{solution}
\ifprintanswers\begin{codex}$\ang\vd$\end{codex}\fi
