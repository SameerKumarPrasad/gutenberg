

\question[3]   A bank is offering two investment schemes. The first scheme offers
7\% compound interest in the first year and - in each subsequent year - 25\% less 
than what was offered in the previous year. The second scheme offers a constant 5.75\%
throughout. If you are investing for a 2-year period, which scheme would you choose?


\ifprintanswers
  % stuff to be shown only in the answer key - like explanatory margin figures
\fi 

\begin{solution}[\fullpage]
	The amount in-hand at the end of 2 years with the first scheme is
	\begin{align}
		P_1 &= P_0\cdot\underbrace{\left( 1 + \dfrac{7}{100}\right)}_{\texttt{first year}}
		\cdot\underbrace{\left( 1 + \dfrac{5.25}{100}\right)}_{\texttt{second year}} \\
		&= 1.126\cdot P_0
	\end{align}
	
	Similarly, with the second scheme, you would get
	\begin{align}
		P_2 &= P_0\cdot\left( 1 + \dfrac{5.75}{100}\right)^{2} \\
		    &= 1.118\cdot P_0
	\end{align}
	
	Clearly, you would be better off going with the first scheme
\end{solution}
