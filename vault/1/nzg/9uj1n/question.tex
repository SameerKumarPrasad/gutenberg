

\question[6]  Shown alongside is a circle centered at $O$ having radius = $4\sqrt2$.
The two chords - $AB$ and $CD$ - intersect at $X$. If $X$ divides $AB$ in the ratio $2:1$
and $CD$ in the ratio $3:1$, then what is $\angle OCD$ given that length of $AB = 3\sqrt3$ units?


\begin{marginfigure}
	\figinit{pt}
    \figpt 0: (60,0)
    \figptcirc 1:$A$: 0;50(40)
    \figptcirc 2:$B$: 0;50(140)
    \figptcirc 3:$C$: 0;50(200)
    \figptcirc 4:$D$: 0;50(100)
    \figvectP 10 [1,2]
    \figvectP 20 [3,4]
    \figptinterlines 5:$X$[1,10;3,20]
    \figptorthoprojline 6:$M$= 0/3,4/
	\figdrawbegin{}
		\figdrawcirc 0(50)
    \figdrawline [1,2]
    \figdrawline [3,4]
    \figdrawarccircP 3;10 [0,4]
    \figset (dash=7)
    \figdrawline [0,3]
    \ifprintanswers
      \figdrawline [0,6]
    \fi
	\figdrawend
  \figvisu{\figBoxA}{}{
    \figsetmark{$\bullet$}
    \figwrites 0:$O$(5)
    \figwriten 1:(5)
    \figwritew 2:(5)
    \figwrites 3:(5)
    \figwriten 4:(5)
    \ifprintanswers
      \figwritese 5:(5)
      \figwritew 6:(5)
    \fi
  }
  \centerline{\box\figBoxA}
\end{marginfigure}

\ifprintanswers
  % stuff to be shown only in the answer key - like explanatory margin figures
  \marginnote[0.5cm]{Whether $\frac{AX}{BX} = \frac{2}{1}$ or $\frac{BX}{AX} = \frac{2}{1}$ is 
  unimportant. We are only interested in the total length of the chord}
  \marginnote[0.5cm]{Same goes for chord $CD$ }
\fi 

\begin{solution}[\fullpage]
	Here is what we know about intersecting chords in a circle,
	\begin{align}
		AX\cdot BX &= CX\cdot DX
	\end{align}
	Assuming that $\dfrac{AX}{BX} = \dfrac{2}{1}$ and $\dfrac{CX}{DX} = \dfrac{3}{1}$, we get
	\begin{align}
		\dfrac{2}{3}L_{AB}\cdot\dfrac{1}{3}L_{AB} &= \dfrac{3}{4}L_{CD}\cdot\dfrac{1}{4}L_{CD} \\
		\Rightarrow \dfrac{2}{9}\cdot(3\sqrt{3})^2 &= \dfrac{3}{16}\cdot L_{CD}^2 \\
		\Rightarrow L_{CD} &= 4\sqrt{2}
	\end{align}
	
	Now, if we drop a perpendicular from $O$ on $CD$ at $M$, then we know that 
	$CM = MD = \dfrac{4\sqrt2}{2} = 2\sqrt{2}$
	
	And therefore, in $\triangle OCM$, 
	\begin{align}
		\cos\angle OCM &= \dfrac{CM}{OC} \\
		               &= \dfrac{2\sqrt{2}}{4\sqrt{2}} = \dfrac{1}{2} \\
		\Rightarrow \angle OCM &= \cos^{-1}\dfrac{1}{2} = \ang{60}
	\end{align}
\end{solution}
