


\ifnumequal{\value{rolldice}}{0}{
  % variables 
  \renewcommand{\vbone}{1}
  \renewcommand{\vbtwo}{3}
  \renewcommand{\vbthree}{3}
  \renewcommand{\vbfour}{5}
  \renewcommand{\vbfive}{1} % original slope
  \renewcommand{\vbsix}{45} % original angle 
  \renewcommand{\vbseven}{\sqrt{3}} % new slope 
  \renewcommand{\vbeight}{1.73} 
  \renewcommand{\vbnine}{1} % value of vbone as a real number
}{
  \ifnumequal{\value{rolldice}}{1}{
    % variables 
    \renewcommand{\vbone}{\sqrt{3}}
    \renewcommand{\vbtwo}{3}
    \renewcommand{\vbthree}{2\sqrt{3}}
    \renewcommand{\vbfour}{4}
    \renewcommand{\vbfive}{\dfrac{1}{\sqrt{3}}}
    \renewcommand{\vbsix}{30}
    \renewcommand{\vbseven}{1}
    \renewcommand{\vbeight}{1}
    \renewcommand{\vbnine}{1.73}
  }{
    \ifnumequal{\value{rolldice}}{2}{
      % variables 
      \renewcommand{\vbone}{5}
      \renewcommand{\vbtwo}{7}
      \renewcommand{\vbthree}{2}
      \renewcommand{\vbfour}{4}
      \renewcommand{\vbfive}{1}
      \renewcommand{\vbsix}{45}
      \renewcommand{\vbseven}{\sqrt{3}}
      \renewcommand{\vbeight}{1.73}
      \renewcommand{\vbnine}{5}
    }{
      % variables 
      \renewcommand{\vbone}{\sqrt{3}}
      \renewcommand{\vbtwo}{5}
      \renewcommand{\vbthree}{2\sqrt{3}}
      \renewcommand{\vbfour}{6}
      \renewcommand{\vbfive}{\dfrac{1}{\sqrt{3}}}
      \renewcommand{\vbsix}{30}
      \renewcommand{\vbseven}{1}
      \renewcommand{\vbeight}{1}
      \renewcommand{\vbnine}{1.73}
     }
  }
}

\gcalcexpr[2]{\vbeight}{\vbtwo - (\vbeight * \vbnine)}

\question[3] If the line joining the points $A = (\vbone, \vbtwo)$ and $B = (\vbthree, \vbfour)$ 
\ is rotated \textit{counter-clockwise} about $A$ by $\ang{15}$, then what is the equation 
of the resulting new line ?


\watchout

\ifprintanswers
  % stuff to be shown only in the answer key - like explanatory margin figures
  \begin{marginfigure}
    \figinit{pt}
      \figpt 100:(0,0)
      \figpt 101:(0,0)
    \figdrawbegin{}
      \figdrawline [100,101]
    \figdrawend
    \figvisu{\figBoxA}{}{%
    }
    \centerline{\box\figBoxA}
  \end{marginfigure}
\fi 

\begin{solution}[\halfpage]
    First, the slope of $AB$ is given by 
    \begin{align}
        m &= \tan\theta = \dfrac{\vbfour - \vbtwo}{\vbthree - \vbone} = \vbfive \\
        \Rightarrow \theta &= \ang{\vbsix}
    \end{align}
    The slope of the \textit{new} line would therefore be $ = \tan(\theta + \ang{15}) = \vbseven$. Its equation
    , therefore, would be 
    \begin{align}
      \fEqnSlope &= \vbseven \text{ where } (x_1, y_1) = A = (\vbone, \vbtwo) \\
      \Rightarrow y &= \vbseven\cdot x + (\vbtwo - \vbone\cdot\vbseven) \\
      &= \vbseven\cdot x + \vbeight
    \end{align}
\end{solution}
