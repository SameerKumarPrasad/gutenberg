
% \noprintanswers
%\setcounter{rolldice}{2}
%\printrubric

\ifnumequal{\value{rolldice}}{0}{
  % variables 
  \renewcommand{\vbone}{2}
  \renewcommand{\vbtwo}{3}
  \renewcommand{\vbthree}{3}
  \renewcommand{\vbfour}{-1}
  \renewcommand{\vbfive}{5}
  \renewcommand{\vbsix}{2}
}{
  \ifnumequal{\value{rolldice}}{1}{
    % variables 
    \renewcommand{\vbone}{7}
    \renewcommand{\vbtwo}{4}
    \renewcommand{\vbthree}{3}
    \renewcommand{\vbfour}{1}
    \renewcommand{\vbfive}{5}
    \renewcommand{\vbsix}{3}
  }{
    \ifnumequal{\value{rolldice}}{2}{
      % variables 
      \renewcommand{\vbone}{1}
      \renewcommand{\vbtwo}{4}
      \renewcommand{\vbthree}{-11}
      \renewcommand{\vbfour}{2}
      \renewcommand{\vbfive}{3}
      \renewcommand{\vbsix}{7}
    }{
      % variables 
      \renewcommand{\vbone}{4}
      \renewcommand{\vbtwo}{2}
      \renewcommand{\vbthree}{2}
      \renewcommand{\vbfour}{6}
      \renewcommand{\vbfive}{4}
      \renewcommand{\vbsix}{5}
    }
  }
}

\SUBTRACT\vbfour\vbtwo\yd
\SUBTRACT\vbthree\vbone\xd
\FRACDIV{-1}{1}\yd\xd\p\q
\EXPR[0]\r{(\q*\vbsix-\p*\vbfive)}

\question[2] Find the equation of the line that passes through $(\vbfive,\vbsix)$ and 
is perpendicular to the line joining the points $(\vbone,\vbtwo)$ and $(\vbthree, \vbfour)$

\insertQR{QRC}

\watchout

\ifprintanswers
\fi 

\begin{solution}[\halfpage]
	For the line passing through $(\vbone, \vbtwo)$ and $(\vbthree, \vbfour)$, the slope is given by 
	\begin{align}
		& \fSlope{1}{2} = \dfrac{\vbfour - \vbtwo}{\vbthree - \vbone} = \WRITEFRAC\yd\xd \\
		& \Rightarrow m_2 = \text{ slope of the perpendicular line } = \dfrac{-1}{m_1} = \WRITEFRAC\p\q
	\end{align}
	
	And hence, the equation of the line passing through $(\vbfive,\vbsix)$ and with slope $= \WRITEFRAC\p\q$, is
	\begin{align}
		\dfrac{y-\vbsix}{x-\vbfive} &= \WRITEFRAC\p\q \\
		\text{ or } \q y &= \p x + \r 
	\end{align}
\end{solution}

\ifprintrubric
  \begin{table}
  	\begin{tabular}{ p{5cm}p{5cm} }
  		\toprule % in brief (4-6 words), what should a grader be looking for for insights & formulations
  		  \sc{\textcolor{blue}{Insight}} & \sc{\textcolor{blue}{Formulation}} \\ 
  		\midrule % ***** Insights & formulations ******
        Inferred the slope of the $\perp$ line & \\
  		\toprule % final numerical answers for the various versions
        \sc{\textcolor{blue}{If question has $\ldots$}} & \sc{\textcolor{blue}{Final answer}} \\
  		\midrule % ***** Numerical answers (below) **********
        Passes through $(5,2)$ & $4y = x + 3$ \\
        Passes through $(5,3)$ & $3y = -4x + 29$ \\
        Passes through $(3,7)$ & $y = -6x + 25$ \\
        Passes through $(4,5)$ & $2y = x + 6$ \\
  		\bottomrule
  	\end{tabular}
  \end{table}
\fi
