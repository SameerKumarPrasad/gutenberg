


\ifnumequal{\value{rolldice}}{0}{
  % variables 
  \renewcommand{\vbone}{2}
  \renewcommand{\vbtwo}{1}
  \renewcommand{\vbthree}{5}
  \renewcommand{\vbfour}{7}
  \renewcommand{\vbseven}{-\sqrt{3}}
  \renewcommand{\vbeight}{120}
}{
  \ifnumequal{\value{rolldice}}{1}{
    % variables 
    \renewcommand{\vbone}{2\sqrt{11}}
    \renewcommand{\vbtwo}{5}
    \renewcommand{\vbthree}{5}
    \renewcommand{\vbfour}{11}
    \renewcommand{\vbseven}{\dfrac{1}{\sqrt{3}}}
    \renewcommand{\vbeight}{30}
  }{
    \ifnumequal{\value{rolldice}}{2}{
      % variables 
      \renewcommand{\vbone}{\sqrt{23}}
      \renewcommand{\vbtwo}{4}
      \renewcommand{\vbthree}{2}
      \renewcommand{\vbfour}{3}
      \renewcommand{\vbseven}{\dfrac{1}{\sqrt{3}}}
      \renewcommand{\vbeight}{30}
    }{
      % variables 
      \renewcommand{\vbone}{\sqrt{7}}
      \renewcommand{\vbtwo}{2}
      \renewcommand{\vbthree}{2}
      \renewcommand{\vbfour}{4}
      \renewcommand{\vbseven}{\sqrt{3}}
      \renewcommand{\vbeight}{60}
      }
  }
}

\renewcommand{\vbfive}{(\vbone - \vbtwo\cdot\sqrt{3})}
\renewcommand{\vbsix}{(\vbone + \vbtwo\cdot\sqrt{3})}
\gcalcexpr[0]{\vbnine}{180-\vbeight}

\question[3] Find the angle at which the two lines $y=\vbfive\cdot(x + \vbthree)$ and 
$y=\vbsix\cdot(x - \vbfour)$ intersect


\watchout

\ifprintanswers

\fi 

\begin{solution}[\halfpage]
	The equations of the two lines can be re-written as 
	\begin{align}
		y &= \overbrace{\vbfive\cdot x}^{m_1} + \overbrace{\vbfive\cdot \vbthree}^{\text{not important}} \\
		y &= \underbrace{\vbsix\cdot x}_{m_2} - \vbsix\cdot \vbfour \\
		\Rightarrow & \fAngleOfIntersection{1}{2} = \dfrac{\vbfive - \vbsix}{1 + \vbfive\cdot\vbsix} \\
		&= \vbseven \\ 
		\Rightarrow \theta_1 &= \ang{\vbeight} \text{ or } \ang{\vbnine}
	\end{align}
\end{solution}

