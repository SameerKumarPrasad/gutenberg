


\ifnumequal{\value{rolldice}}{0}{
  % variables 
  \renewcommand{\va}{1}
  \renewcommand{\vb}{2}
  \renewcommand{\vc}{3}
  \renewcommand{\vd}{4}
  \renewcommand\ve{7}
  \renewcommand\vf{3}
  \renewcommand\vz{10.30}
}{
  \ifnumequal{\value{rolldice}}{1}{
    % variables 
    \renewcommand{\va}{4}
    \renewcommand{\vb}{5}
    \renewcommand{\vc}{2}
    \renewcommand{\vd}{7}
    \renewcommand\ve{3}
    \renewcommand\vf{5}
    \renewcommand\vz{22.71}
  }{
    \ifnumequal{\value{rolldice}}{2}{
      % variables 
      \renewcommand{\va}{9}
      \renewcommand{\vb}{5}
      \renewcommand{\vc}{7}
      \renewcommand{\vd}{3}
      \renewcommand\ve{7}
      \renewcommand\vf{4}
      \renewcommand\vz{5.86}
    }{
      % variables 
      \renewcommand{\va}{11}
      \renewcommand{\vb}{7}
      \renewcommand{\vc}{5}
      \renewcommand{\vd}{13}
      \renewcommand\ve{4}
      \renewcommand\vf{6}
      \renewcommand\vz{36.49}
    }
  }
}

\MULTIPLY\ve\va\vj
\MULTIPLY\vf\vc\vk
\FRACMINUS\va\vb\vc\vd\vp\vq
\FRACMULT\va\vb\vc\vd\vr\vs
\FRACADD{1}{1}\vr\vs\vt\vu
\FRACDIV\vp\vq\vt\vu\vv\vw
\ABSVALUE\vv\vx

\question[3] Find the angle at which the following two lines intersect.  
  \[ L_1: \vb y = \va\cdot(x-\ve) \]
  \[ L_2: \vd y = \vc\cdot(x-\vf) \]

\marginnote[-60pt]{ 
  \begin{tabular}{c c}
    $\tan\ang{10.30}=2/11$ & $\tan\ang{36.49}=54/73$ \\
    $\tan\ang{22.71}=18/43$ & $\tan\ang{5.86}=4/39$ \\
  \end{tabular} 
} 
\watchout

\begin{solution}[\halfpage]
	The equations of the two lines can be re-written as 
  \begin{align}
    y &= \frac\va\vb\cdot(x-\ve) = \frac\va\vb x - \frac\vj\vb \\
    y &= \frac\vc\vd\cdot(x-\vf) = \frac\vc\vd x - \frac\vk\vd
  \end{align}
  which gives us the slopes of the two lines as 
  \[ m_1 = \tan\theta_1 = \frac\va\vb\text{ and } m_2 = \tan\theta_2 = \frac\vc\vd \]
  where $\theta_1$ and $\theta_2$ are the angles the two lines make with the $x-$axis.

  The \textbf{acute angle of intersection} - $\beta$ - is given by 
  \begin{align}
    \tan\beta &= \vert \tan (\theta_1 - \theta_2 ) \vert = \vert \dfrac{m_1 - m_2}{1+m_1\cdot m_2} \vert \\
              &= \vert \dfrac{\frac\va\vb - \frac\vc\vd}{1+\frac\va\vb\cdot\frac\vc\vd} \vert = \frac\vx\vw \\
              \implies\beta &=\tan^{-1}\frac\vx\vw = \ang\vz
  \end{align}
\end{solution}

\ifprintanswers
  \begin{codex}
    $\ang\vz$
  \end{codex}
\fi

