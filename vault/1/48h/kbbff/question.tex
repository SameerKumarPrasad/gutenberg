

\ifnumequal{\value{rolldice}}{0}{
  % variables 
  \renewcommand{\vbone}{5} % a1
  \renewcommand{\vbtwo}{6} % b1
  \renewcommand{\vbthree}{1} % c1
  \renewcommand{\vbfour}{3} %a2
  \renewcommand{\vbfive}{5} % c2
  \renewcommand{\vbsix}{3} % k = b1/b2
}{
  \ifnumequal{\value{rolldice}}{1}{
    % variables 
    \renewcommand{\vbone}{3}
    \renewcommand{\vbtwo}{4}
    \renewcommand{\vbthree}{7}
    \renewcommand{\vbfour}{4}
    \renewcommand{\vbfive}{5}
    \renewcommand{\vbsix}{2}
  }{
    \ifnumequal{\value{rolldice}}{2}{
      % variables 
      \renewcommand{\vbone}{4}
      \renewcommand{\vbtwo}{7}
      \renewcommand{\vbthree}{11}
      \renewcommand{\vbfour}{2}
      \renewcommand{\vbfive}{13}
      \renewcommand{\vbsix}{1}
    }{
      % variables 
      \renewcommand{\vbone}{5}
      \renewcommand{\vbtwo}{8}
      \renewcommand{\vbthree}{7}
      \renewcommand{\vbfour}{3}
      \renewcommand{\vbfive}{9}
      \renewcommand{\vbsix}{2}
    }
  }
}

\EXPR[0]{\vbseven}{\vbtwo / \vbsix} % b2
\MULTIPLY\vbthree\vbfour\p
\MULTIPLY\vbfour\vbtwo\q
\MULTIPLY\vbseven\vbone\j
\MULTIPLY\vbfive\vbone\k
\MULTIPLY\vbseven\vbthree\m
\MULTIPLY\vbfive\vbtwo\n

\SUBTRACT\j\q\a
\SUBTRACT\p\k\b
\SUBTRACT\m\n\d
\FRACTIONSIMPLIFY\b\a\bb\aa
\FRACTIONSIMPLIFY\d\a\cc\dd
\FRACMULT\cc\dd{-1}{1}\ee\ff

\question[4] Find the equation of the straight line that passes through the point of intersection of 
$\vbone x + \vbtwo y + \vbthree = 0$ and $\vbfour x + \vbseven y + \vbfive = 0$ \textit{and} is 
perpendicular to the line $3x-5y+11 = 0$

\watchout

\ifprintanswers
\fi 

\begin{solution}[\halfpage]
	First, we must find that point of intersection. The two lines will intersect at an $x$ where
	\begin{align}
	    \vbone x + \vbtwo y + \vbthree &= 0 \text{ and } \\
	    \vbfour x + \vbseven y + \vbfive &= 0 \\ 
      \Rightarrow (x,y) &= (\WRITEFRAC\ee\ff,\WRITEFRAC\bb\aa) 
	\end{align}
	
	Now, the slope of the line \textit{ perpendicular } to 
	$3x - 5y + 11 = 0 \Rightarrow y = \frac{3}{5}x + \frac{11}{5}$ is $-\frac{5}{3}$. And so, its equation
	would be 

  \FRACMULT\bb\aa{-3}{1}\jj\kk
  \FRACMULT\ee\ff{-5}{1}\mm\nn
  \FRACADD\jj\kk\mm\nn\y\z

	\begin{align}
		\dfrac{y - \WRITEFRAC\bb\aa}{x - \WRITEFRAC\ee\ff} &= -\dfrac{5}{3} \\
		\Rightarrow 3y - 3\times\WRITEFRAC\bb\aa &= -5x + 5\times\WRITEFRAC\ee\ff \\
		\text{ or } 3y + 5x + \WRITEFRAC\y\z &= 0
	\end{align}
	
\end{solution}

\ifprintrubric
  \begin{table}
  	\begin{tabular}{ p{5cm}p{5cm} }
  		\toprule % in brief (4-6 words), what should a grader be looking for for insights & formulations
  		  \sc{\textcolor{blue}{Insight}} & \sc{\textcolor{blue}{Formulation}} \\ 
  		\midrule % ***** Insights & formulations ******
        $m_\perp = -\dfrac{5}{3}$ & \\
  		\toprule % final numerical answers for the various versions
        \sc{\textcolor{blue}{If question has $\ldots$}} & \sc{\textcolor{blue}{Final answer}} \\
  		\midrule % ***** Numerical answers (below) **********
        $5x+6y+1 = 0$ & $3y+5x+\frac{37}{4} = 0$ \\
        $3x+4y+7 = 0$ & $3y+5x+\frac{69}{10} = 0$ \\
        $4x+7y+11 = 0$ & $3y+5x+\frac{10}{7}=0$ \\
        $5x+8y+7 = 0$ & $3y+5x+37=0$ \\
  		\bottomrule
  	\end{tabular}
  \end{table}
\fi
