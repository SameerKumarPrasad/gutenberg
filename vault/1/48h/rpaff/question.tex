


\ifnumequal{\value{rolldice}}{0}{
  % variables 
  \renewcommand{\vbone}{1} % x 
  \renewcommand{\vbtwo}{3} % y = 3x
  \renewcommand{\vbthree}{3} % p
  \renewcommand{\vbfour}{4} % q 
  \renewcommand{\vbfive}{5} % d 
  \renewcommand{\vbeight}{1}
}{
  \ifnumequal{\value{rolldice}}{1}{
    % variables 
    \renewcommand{\vbone}{2}
    \renewcommand{\vbtwo}{6}
    \renewcommand{\vbthree}{3}
    \renewcommand{\vbfour}{4}
    \renewcommand{\vbfive}{15}
    \renewcommand{\vbeight}{2}
    }{
    \ifnumequal{\value{rolldice}}{2}{
      % variables 
      \renewcommand{\vbone}{2}
      \renewcommand{\vbtwo}{6}
      \renewcommand{\vbthree}{4}
      \renewcommand{\vbfour}{3}
      \renewcommand{\vbfive}{10}
      \renewcommand{\vbeight}{2}
    }{
      % variables 
      \renewcommand{\vbone}{1}
      \renewcommand{\vbtwo}{3}
      \renewcommand{\vbthree}{12}
      \renewcommand{\vbfour}{5}
      \renewcommand{\vbfive}{13}
      \renewcommand{\vbeight}{1}
    }
  }
}

\FRACMULT\vbthree\vbfour\vbone{1}\p\q
\FRACMINUS\vbtwo{1}\p\q\n\m
\SQUARE\vbfour\a
\SQUARE\vbfive\b
\SQUARE\vbthree\c
\MULTIPLY\a\b\d
\ADD\a\c\e
\DIVIDE\d\e\f
\SQUAREROOT\f\g
\SUBTRACT\vbone\g\h
\ADD\vbone\g\j

\FRACMULT\vbthree\vbfour\h{1}\k\l
\FRACMULT\vbthree\vbfour\j{1}\y\z
\FRACADD\k\l\n\m\aa\ab
\FRACADD\y\z\n\m\ba\bb

\renewcommand{\vbsix}{\dfrac{\vbthree}{\vbfour}}

\question[4] If a line that passes through a point $A = (\vbone, \vbtwo)$ has slope equal to $\dfrac\vbthree\vbfour$, then 
find the points on it that are $\vbfive$ units away from $A$

\watchout

\ifprintanswers
\fi 

\begin{solution}[\halfpage]
	The equation of the given line would be 
	\begin{align}
		\dfrac{y-\vbtwo}{x-\vbone} &= \vbsix \Rightarrow y = \dfrac\vbthree\vbfour\cdot x + \WRITEFRAC\n\m
	\end{align}
	Moreover, any point that is $\vbfive$ units away from $A$ would satisy 
	the following condition 
	\begin{align}
		(x - \vbone)^2 + (y-\vbtwo)^2 &= (\vbfive)^2 
	\end{align}
	And if this point also happens to be on the line, then 
	\begin{align}
		(x-\vbone)^2 + \left[ \left( \vbsix\cdot x + \dfrac\n\m\right) - \vbtwo \right]^2 &= \vbfive^2 \\
		\Rightarrow (x - \vbone)^2\cdot\left[ 1 + \left( \vbsix \right)^2\right] &= \vbfive^2 \\
    \Rightarrow (x - \vbone)^2 &= \f  \\ 
    \Rightarrow x &=  \h,\,\j
	\end{align}
	
    
  Plug these values of $x$ into the equation for the line - $(1)$ - and you will 
  get the required coordinates -  $(\h,\WRITEFRAC\aa\ab)$ and $(\j,\WRITEFRAC\ba\bb)$
\end{solution}

\ifprintrubric
  \begin{table}
  	\begin{tabular}{ p{5cm}p{5cm} }
  		\toprule % in brief (4-6 words), what should a grader be looking for for insights & formulations
  		  \sc{\textcolor{blue}{Insight}} & \sc{\textcolor{blue}{Formulation}} \\ 
  		\midrule % ***** Insights & formulations ******
        There are two(2) points - one each on either side of the line & 
        Inferred equation of line passing through given point \\
        & Used formula for distance between points \\
  		\toprule % final numerical answers for the various versions
        \sc{\textcolor{blue}{If question has $\ldots$}} & \sc{\textcolor{blue}{Final answer}} \\
  		\midrule % ***** Numerical answers (below) **********
        $A = (1,3)$, 5 units & $(-3,0)$ and $(5,6)$ \\
        $A = (2,6)$, 10 units & $(-4,-2)$ and $(8,14)$ \\
        $A = (2,6)$, 15 units  & $(-10,-3)$ and $(14,15)$ \\
        $A = (1,3)$, 13 units & $(-4,-9)$ and $(6,15)$ \\
  		\bottomrule
  	\end{tabular}
  \end{table}
\fi
