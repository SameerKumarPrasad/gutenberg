


\ifnumequal{\value{rolldice}}{0}{
  % variables 
  \renewcommand{\vbone}{2}
  \renewcommand{\vbtwo}{3}
}{
  \ifnumequal{\value{rolldice}}{1}{
    % variables 
    \renewcommand{\vbone}{3}
    \renewcommand{\vbtwo}{7}
  }{
    \ifnumequal{\value{rolldice}}{2}{
      % variables 
      \renewcommand{\vbone}{6}
      \renewcommand{\vbtwo}{9}
    }{
      % variables 
      \renewcommand{\vbone}{13}
      \renewcommand{\vbtwo}{15}
    }
  }
}

\gcalcexpr[0]{\vbthree}{\vbtwo - \vbone}
\gcalcexpr[0]{\vbfour}{\vbtwo + \vbone}

\question[2] Find the equation of the line - or lines - that are equally inclined to both the $x$ and $y$ axes 
and pass through the point $(\vbone, \vbtwo)$


\watchout

\ifprintanswers
  % stuff to be shown only in the answer key - like explanatory margin figures
  \begin{marginfigure}
  % 1. Definition of characteristic points
\figinit{pt}
\def\Xmin{-39.99999}
\def\Ymin{-22.22222}
\def\Xmax{39.99999}
\def\Ymax{57.77777}
\def\Xori{39.99999}
\def\Yori{22.22222}
\figpt 100:$\theta$(76,22)
\figpt 101:$\theta$(42,57)
\figpt 102:$\psi$(27,27)
\figpt 103:$\psi$(40,40)
\figpt0:(\Xori,\Yori)
\figdrawbegin{}
\def\Xmaxx{\Xmax} % To customize the position
\def\Ymaxx{\Ymax} % of the arrow-heads of the axes.
\drawAxes{0}\Xmin\Xmaxx\Ymin\Ymaxx
\figdrawlineC(
0 97.77777, % y = 8.50
2.75862 95.01915, % y = 8.18
5.51724 92.26053, % y = 7.87
8.27586 89.50191, % y = 7.56
11.03448 86.74329, % y = 7.25
13.79310 83.98467, % y = 6.94
16.55172 81.22605, % y = 6.63
19.31034 78.46743, % y = 6.32
22.06896 75.70881, % y = 6.01
24.82758 72.95019, % y = 5.70
27.58620 70.19157, % y = 5.39
30.34482 67.43295, % y = 5.08
33.10344 64.67432, % y = 4.77
35.86206 61.91570, % y = 4.46
38.62068 59.15708, % y = 4.15
41.37931 56.39846, % y = 3.84
44.13793 53.63984, % y = 3.53
46.89655 50.88122, % y = 3.22
49.65517 48.12260, % y = 2.91
52.41379 45.36398, % y = 2.60
55.17241 42.60536, % y = 2.29
57.93103 39.84674, % y = 1.98
60.68965 37.08812, % y = 1.67
63.44827 34.32950, % y = 1.36
66.20689 31.57088, % y = 1.05
68.96551 28.81226, % y = .74
71.72413 26.05363, % y = .43
74.48275 23.29501, % y = .12
77.24137 20.53639, % y = -.18
79.99999 17.77777
)
\figdrawlineC(
0 0, % y = -2.50
2.75862 2.75862, % y = -2.18
5.51724 5.51724, % y = -1.87
8.27586 8.27586, % y = -1.56
11.03448 11.03448, % y = -1.25
13.79310 13.79310, % y = -.94
16.55172 16.55172, % y = -.63
19.31034 19.31034, % y = -.32
22.06896 22.06896, % y = -.01
24.82758 24.82758, % y = .29
27.58620 27.58620, % y = .60
30.34482 30.34482, % y = .91
33.10344 33.10344, % y = 1.22
35.86206 35.86206, % y = 1.53
38.62068 38.62068, % y = 1.84
41.37931 41.37931, % y = 2.15
44.13793 44.13793, % y = 2.46
46.89655 46.89655, % y = 2.77
49.65517 49.65517, % y = 3.08
52.41379 52.41379, % y = 3.39
55.17241 55.17241, % y = 3.70
57.93103 57.93103, % y = 4.01
60.68965 60.68965, % y = 4.32
63.44827 63.44827, % y = 4.63
66.20689 66.20689, % y = 4.94
68.96551 68.96551, % y = 5.25
71.72413 71.72413, % y = 5.56
74.48275 74.48275, % y = 5.87
77.24137 77.24137, % y = 6.18
79.99999 79.99999
)
\figdrawend
\figvisu{\figBoxA}{}{%
\figptsaxes 1:0(\Xmin, \Xmaxx, \Ymin, \Ymaxx)
\figwritee 1:(5pt)     \figwriten 2:(5pt)
\figptsaxes 1:0(\Xmin, \Xmax, \Ymin, \Ymax)
\figwriten 100:(4)
\figwritene 101:(2)
\figwritee 102:(2)
\figwrites 103:(2)
}
\centerline{\box\figBoxA}

  \end{marginfigure}
\fi 

\begin{solution}[\halfpage]
	The situation is \asif. There are two lines that form equal angles with both the $x$ and $y$ axes. One angle ,
    however, is acute and the other obtuse
    
    Now, its easy to prove that $\psi = \frac{\pi}{4}$ and $\theta = \frac{3\pi}{4}$. Hence, the slopes of the 
    required lines are $+1$ and $-1$
    \begin{align}
    	\Rightarrow \dfrac{y-\vbtwo}{x-\vbone} &= 1 \Rightarrow y = x + \vbthree\\
    	\text{ and } \dfrac{y-\vbtwo}{x-\vbone} &= -1 \Rightarrow y = -x + \vbfour
    \end{align}
\end{solution}
