

 %\noprintanswers
 %\setcounter{rolldice}{0}
 %\printrubric

\ifnumequal{\value{rolldice}}{0}{
  % variables 
  \renewcommand{\vbone}{2}   %y = 2
  \renewcommand{\vbtwo}{1}    %coeff x
  \renewcommand{\vbthree}{-4}%coeff y
  \renewcommand{\vbfour}{-4}  %coeff const
  \renewcommand{\vbfive}{4}  %4sqrt(2) dist
}{
  \ifnumequal{\value{rolldice}}{1}{
    % variables 
    \renewcommand{\vbone}{1} 
    \renewcommand{\vbtwo}{2} 
    \renewcommand{\vbthree}{-3}
    \renewcommand{\vbfour}{3} 
    \renewcommand{\vbfive}{3} 
  }{
    \ifnumequal{\value{rolldice}}{2}{
      % variables 
      \renewcommand{\vbone}{4}
      \renewcommand{\vbtwo}{3}
      \renewcommand{\vbthree}{-2}
      \renewcommand{\vbfour}{-2}
      \renewcommand{\vbfive}{4}
    }{
      % variables 
      \renewcommand{\vbone}{3}
      \renewcommand{\vbtwo}{2}
      \renewcommand{\vbthree}{-1}
      \renewcommand{\vbfour}{-1}
      \renewcommand{\vbfive}{3}
     }
  }
}

\MULTIPLY\vbthree{-1}\a
\SOLVELINEARSYSTEM(0, 1; \vbtwo, \vbthree)(\vbone,\vbfour)(\p,\q)

\ADD\p{1}\j
\SQUARE\j\jj

\ADD\q{1}\k
\SQUARE\k\kk

\SQUARE\vbfive\c
\MULTIPLY\c{2}\d

\SUBTRACT\d\jj\aa
\MULTIPLY\j\k\x
\MULTIPLY\x{2}\bb
\SUBTRACT\d\kk\cc

\gcalcexpr[0]\disc{(\bb^2 - 4*\aa*\cc)^{0.5}}
\MULTIPLY\bb{-1}\mbb
\MULTIPLY\aa{2}\dnm

\ADD\mbb\disc\f
\SUBTRACT\mbb\disc\g


\question[4] Find the equation of a straight lines that pass
 through the intersection of lines $L_1: y=\vbone$ and 
 $L_2: \vbtwo x \vbthree y = \vbfour$ and are at a distance of $\vbfive\sqrt{2}$
 units from $(-1, -1)$.

\insertQR{QRC}

\watchout

\ifprintanswers
  % stuff to be shown only in the answer key - like explanatory margin figures
  \begin{marginfigure}[+40 pt]
    \figinit{pt}
      \figpt 100:(0,20)
      \figpt 101:$R_1$(90,50)
      \figpt 102:(0,40)
      \figpt 103:$R_2$(90,15)
      \figvectP 201[100,101]
      \figvectP 203[102,103]
      \figptinterlines 151:$A$[101,201;103,203] % pt of intersection
      \figpt 300:$B$ (90,33)
      \figptorthoprojline 251:$X$=300/100,101/
      \figptorthoprojline 253:$Y$=300/102,103/
    \figdrawbegin{}
      \figdrawline [100,101]
      \figdrawline [102,103]
      %\figdrawline [300, 251]
      %\figdrawline [300, 253]
      \figdrawaltitude 5 [300,101,100]
      \figdrawaltitude 5 [300, 103, 102]
      %\figdrawaltitude 5[100,102,101]
    \figdrawend
    \figvisu{\figBoxA}{}{%
      \figwritene 101:(2 pt)
      \figsetmark{$\bullet$}
      \figwriten 151:(4 pt)
      \figwritee 300:(2 pt)
      \figsetmark{}
      \figwritese 103:(2 pt)
      \figwritenw 251:(2 pt)
      \figwritesw 253:(2 pt)
    }
    \centerline{\box\figBoxA}
  \end{marginfigure}
\fi 

\begin{solution}[\fullpage]
  Let $R_1$ and $R_2$ be the two required lines. They pass through $A$ - which is the 
  point of intersection of $L_1$ and $L_2$ (not shown). The distance of $R_1$ and $R_2$ from 
  $B = (-1,-1)$ is $\vbfive\sqrt{2}$ units
  
  Now, lines $L_1$ and $L_2$ intersect when 
  \begin{align}
    y &= \vbone \text{ and, } \\
    \a y &= \vbtwo x - \vbfour
  \end{align}
  Solving the above equations gives us $A = (\p,\q)$

  The equation of any line passing through $A$ - like $R_1$ and $R_2$ - is given by 
  \begin{align}
    \dfrac{y-\q}{x-\p} &= m \\ 
    \implies \underbrace{mx - y + (\q - \p m) = 0}_{Ax + By + C = 0}    
  \end{align}
  
  Moreover, the distance of any such line from $B = (-1,-1)$ will be given by 
  \begin{align}
    D &= \dfrac{\vert Ax_B + By_B + C\vert}{\sqrt{A^2+B^2}} \\
      &= \dfrac{\vert -\j m + \k \vert }{\sqrt{m^2 + 1}} \\
      \implies D^2 &= \dfrac{(\j m - \k)^2}{m^2 + 1} = \d \\
      \implies \aa m^2 + \bb m + \cc &= 0 \text{ or } m = \WRITEFRAC\f\dnm, \WRITEFRAC\g\dnm 
  \end{align}

  So, we know that $R_1$ and $R_2$ pass through $A = (\p,\q)$ and that they have slopes 
  given by $(8)$ 

  It is now a simple case of plugging the values of $m$ we found in $(8)$ into $(3)$ 
  to get \textbf{equations} for the two lines. We leave this as an exercise
\end{solution}

