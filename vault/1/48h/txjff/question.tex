


\ifnumequal{\value{rolldice}}{0}{
  % variables 
  \renewcommand{\va}{1} % a 
  \renewcommand{\vb}{2} % b
  \renewcommand{\vc}{5} % m
  \renewcommand{\vd}{3} % n
}{
  \ifnumequal{\value{rolldice}}{1}{
    % variables 
    \renewcommand{\va}{2}
    \renewcommand{\vb}{3}
    \renewcommand{\vc}{7}
    \renewcommand{\vd}{11}
  }{
    \ifnumequal{\value{rolldice}}{2}{
      % variables 
      \renewcommand{\va}{9}
      \renewcommand{\vb}{1}
      \renewcommand{\vc}{11}
      \renewcommand{\vd}{16}
    }{
      % variables 
      \renewcommand{\va}{5}
      \renewcommand{\vb}{8}
      \renewcommand{\vc}{9}
      \renewcommand{\vd}{17}
    }
  }
}

\SUBTRACT{0}\vb\ve
\SUBTRACT{0}\vd\vf
\MULTIPLY\ve{-\vc}\vg
\MULTIPLY\vf{-\va}\vh
\SUBTRACT\ve\vf\vi
\SUBTRACT\vh\vg\vj


\question[3] A ray of light coming from a point $A = (\va, \vb)$ is reflected at a point $B$ \textit{on the x-axis} 
before passing through a point $C = (\vc, \vd)$ (see figure). Find the coordinates of point $B$. The angle at which light
hits $B$ - and is reflected back - is $\theta$

\watchout

\figinit{pt}
\def\Xmin{0}
\def\Ymin{1.57894}
\def\Xmax{60.00000}
\def\Ymax{61.57894}
\def\Xori{0}
\def\Yori{-1.57894}
\figpt0:(\Xori,\Yori)

\figpt 100: $A$(11,37) % A
\figpt 101: $C$(48,35) % C
\figpt 102: $B$(30,0) % B
\figpt 103: $\theta$(24,4)
\figpt 104: $\theta$(36,4)
\figpt 105: $M$(53,-1)

\figdrawbegin{}
\def\Xmaxx{\Xmax} % To customize the position
\def\Ymaxx{\Ymax} % of the arrow-heads of the axes.
\figset arrowhead(length=4, fillmode=yes) % styling the arrowheads
\figdrawaxes 0(\Xmin, \Xmaxx, \Ymin, \Ymaxx)
\figdrawlineC(
0 59.99999,
1.53846 56.84210,
3.07692 53.68421,
4.61538 50.52631,
6.15384 47.36842,
7.69230 44.21052,
9.23076 41.05263,
10.76923 37.89473,
12.30769 34.73684,
13.84615 31.57894,
15.38461 28.42105,
16.92307 25.26315,
18.46153 22.10526,
19.99999 18.94736,
21.53846 15.78947,
23.07692 12.63157,
24.61538 9.47368,
26.15384 6.31578,
27.69230 3.15789,
29.23076 0,
30.76923 0,
32.30769 3.15789,
33.84615 6.31578,
35.38461 9.47368,
36.92307 12.63157,
38.46153 15.78947,
39.99999 18.94736,
41.53846 22.10526,
43.07692 25.26315,
44.61538 28.42105,
46.15384 31.57894,
47.69230 34.73684,
49.23076 37.89473,
50.76923 41.05263,
52.30769 44.21052,
53.84615 47.36842,
55.38461 50.52631,
56.92307 53.68421,
58.46153 56.84210,
59.99999 59.99999
)
\figdrawend
\figvisu{\figBoxA}{}{%
\figptsaxes 1:0(\Xmin, \Xmaxx, \Ymin, \Ymaxx)
\figwritee 1:(5pt)     \figwriten 2:(5pt)
\figptsaxes 1:0(\Xmin, \Xmax, \Ymin, \Ymax)
\figwritew 103:(2)
\figwritee 104:(2)
\figset write(mark=$\bullet$)
\figwritee 100:(2)
\figwritee 101:(2)
\figwrites 102:(4)
\figwrites 105:(4)
}

\vspace{1cm}
\centerline{\box\figBoxA}

\begin{solution}[\halfpage]
	The slope of the \textbf{incoming} beam of light $AB$ is 
  \[\tan\angle{MBA} = \tan (\pi-\theta) = -\tan\theta\]
	
  The slope of the \textbf{reflected} beam $BC$ is 
  \[ \tan\angle{MBC} = \tan\theta = -\tan\angle{MBA} \]
	
  The light hits a point $B = (x,0)$ on the $x-$axis. And therefore, 
	
	\begin{align}
		\underbrace{\dfrac{0-\vb}{x-\va}}_{\tan\angle{MBA}} &= 
		-\underbrace{\dfrac{0-\vd}{x-\vc}}_{\tan\angle{MBC}} \\
		\implies \ve\cdot(x-\vc) &= \vf\cdot(x -\va) \\
		\implies x &= \WRITEFRAC\vj\vi 
	\end{align}
	Hence, the light hits the flat surface at $B = \left(\WRITEFRAC\vj\vi, 0\right)$.
\end{solution}

\ifprintanswers
  \begin{codex}
    $\left(\WRITEFRAC\vj\vi,0 \right)$
  \end{codex}
\fi
