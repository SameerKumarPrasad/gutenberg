


\ifnumequal{\value{rolldice}}{0}{
  % variables 
  \renewcommand{\va}{5}
  \renewcommand{\vb}{12}
  \renewcommand{\vc}{26}
  \renewcommand{\vd}{4}
}{
  \ifnumequal{\value{rolldice}}{1}{
    % variables 
    \renewcommand{\va}{6}
    \renewcommand{\vb}{8}
    \renewcommand{\vc}{11}
    \renewcommand{\vd}{6}
  }{
    \ifnumequal{\value{rolldice}}{2}{
      % variables 
      \renewcommand{\va}{3}
      \renewcommand{\vb}{4}
      \renewcommand{\vc}{12}
      \renewcommand{\vd}{6}
    }{
      % variables 
      \renewcommand{\va}{12}
      \renewcommand{\vb}{5}
      \renewcommand{\vc}{14}
      \renewcommand{\vd}{7}
     }
  }
}

\gcalcHypotenuse[0]{\ve}{\va}{\vb}
\MULTIPLY\vd\ve\vf
\SUBTRACT\vc\vf\vg
\ADD\vc\vf\vh

\question[3] Find the equation of the straight line(s) parallel to $\va x - \vb y + \vc = 0$ 
and at a distance of $\vd$ units from it.

\watchout

\begin{solution}[\halfpage]
	The distance between two parallel lines - $Ax + By + C_1 = 0$ and $Ax+By+C_2 = 0$ is given by 
	\[ d = \dfrac{\vert C_1 - C_2 \vert}{\sqrt{A^2+ B^2}}\]
  
  Note that there would be \textit{two} such lines, 
	one on either side of the original line. So,
	\begin{align}
		\vd &= \dfrac{\vert \vc - C_2 \vert}{\sqrt{\va^2 + \vb^2}} \\
		\implies \vc - C_2 &= \vf, \text{ for when } C_2 < \vc \\
		\text{ and } C_2 - \vc &= \vf, \text{ for when } C_2 > \vc
	\end{align}
	Solving (2) and (3) above, we get 
	\begin{align}
		C_2 &= \vg \text{ and } C_2 = \vh
	\end{align}
	
	Hence, the two lines are 
  \[\va x - \vb y + \vg = 0\] 
  \[\va x - \vb y + \vh = 0\]
\end{solution}

\ifprintanswers
  \begin{codex}
    $\va x -\vb y + \vg = 0\text{ and } \va x - \vb y + \vh = 0$
  \end{codex}
\fi
