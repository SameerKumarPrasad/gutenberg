% This is an empty shell file placed for you by the 'examiner' script.
% You can now fill in the TeX for your question here.

% Now, down to brasstacks. ** Writing good solutions is an Art **. 
% Eventually, you will find your own style. But here are some thoughts 
% to get you started: 
%
%   1. Write to be understood - but be crisp. Your own solution should not take 
%      more space than you will give to the student. Hence, if you take more than 
%      a half-page to write a solution, then give the student a full-page and so on...
%
%   2. Use margin-notes to "talk" to students about the critical insights
%      in the question. The tone can be - in fact, should be - informal
%
%   3. Don't shy away from creating margin-figures you think will help
%      students understand. Yes, it is a little more work per question. 
%      But the question & solution will be written only once. Make that
%      attempt at writing a solution count.
%      
%      3b. Use bc_to_fig.tex. Its an easier way to generate plots & graphs 
% 
%   4. Ensure that there are *no spelling mistakes anywhere*. We are an 
%      education company. Bad spellings suggest that we ourselves 
%      don't have any education. Also, use American spellings by default
% 
%   5. If a question has multiple parts, then first delete lines 40-41
%   6. If a question does not have parts, then first delete lines 43-69
%   
%   7. Create versions of the question when possible. Use commands defined in 
%      tufte-tweaks.sty to do so. Its easier than you think

%\noprintanswers
%\setcounter{rolldice}{3}
%\printrubric

\ifnumequal{\value{rolldice}}{0}{
  % variables 
  \renewcommand{\vbone}{5}
  \renewcommand{\vbtwo}{4}
  \renewcommand{\vbthree}{2}
  \renewcommand{\vbfour}{1}
}{
  \ifnumequal{\value{rolldice}}{1}{
    % variables 
    \renewcommand{\vbone}{2}
    \renewcommand{\vbtwo}{7}
    \renewcommand{\vbthree}{3}
    \renewcommand{\vbfour}{7}
  }{
    \ifnumequal{\value{rolldice}}{2}{
      % variables 
      \renewcommand{\vbone}{7}
      \renewcommand{\vbtwo}{10}
      \renewcommand{\vbthree}{4}
      \renewcommand{\vbfour}{5}
    }{
      % variables 
      \renewcommand{\vbone}{5}
      \renewcommand{\vbtwo}{9}
      \renewcommand{\vbthree}{7}
      \renewcommand{\vbfour}{9}
    }
  }
}

\gcalcexpr[0]{\vbfive}{\vbtwo * (\vbthree + \vbfour)}
\gcalcexpr[0]{\vbsix}{-\vbone*(\vbthree + \vbfour)} % numerator of m
\gcalcexpr[0]{\vbseven}{\vbthree + \vbfour}
\gcalcexpr[0]{\vbeight}{\vbfive / \vbfour } % vbfive = n  guaranteed by us to be an integer

\question[5] The intercept of a line - \asif - is divided by the point $(-\vbone, \vbtwo)$ so that 
$\frac{AN}{AM} = \frac{\vbthree}{\vbfour}$. Find the equation of the line 

\insertQR{QRC}

\watchout

  \begin{marginfigure}
% 1. Definition of characteristic points
\figinit{pt}
\def\Xmin{-72.72727}
\def\Ymin{-18.18181}
\def\Xmax{7.27272}
\def\Ymax{61.81818}
\def\Xori{72.72727}
\def\Yori{18.18181}
\figpt0:(\Xori,\Yori)
\figpt 100:$M$(18,18)
\figpt 101:$N$(72,72)
\figpt 102:$A$(47,47)
\figpt 103:$J$(\Xori, 47)
\figpt 104:$K$(47,\Yori)
% 2. Creation of the graphical file
\figdrawbegin{}
\def\Xmaxx{\Xmax} % To customize the position
\def\Ymaxx{\Ymax} % of the arrow-heads of the axes.
\figset arrowhead(length=4, fillmode=yes) % styling the arrowheads
\figdrawaxes 0(\Xmin, \Xmaxx, \Ymin, \Ymaxx)
\figdrawlineC(
0 0, % y = -4.00
2.75862 2.75862, % y = -3.39
5.51724 5.51724, % y = -2.78
8.27586 8.27586, % y = -2.17
11.03448 11.03448, % y = -1.57
13.79310 13.79310, % y = -.96
16.55172 16.55172, % y = -.35
19.31034 19.31034, % y = .24
22.06896 22.06896, % y = .85
24.82758 24.82758, % y = 1.46
27.58620 27.58620, % y = 2.06
30.34482 30.34482, % y = 2.67
33.10344 33.10344, % y = 3.28
35.86206 35.86206, % y = 3.88
38.62068 38.62068, % y = 4.49
41.37931 41.37931, % y = 5.10
44.13793 44.13793, % y = 5.71
46.89655 46.89655, % y = 6.31
49.65517 49.65517, % y = 6.92
52.41379 52.41379, % y = 7.53
55.17241 55.17241, % y = 8.13
57.93103 57.93103, % y = 8.74
60.68965 60.68965, % y = 9.35
63.44827 63.44827, % y = 9.95
66.20689 66.20689, % y = 10.56
68.96551 68.96551, % y = 11.17
71.72413 71.72413, % y = 11.77
74.48275 74.48275, % y = 12.38
77.24137 77.24137, % y = 12.99
79.99999 79.99999
)
\ifprintanswers
  \figset (dash=4)
  \figdrawline [102,103]
  \figdrawline [102,104]
\fi
\figdrawend
% 3. Writing text on the figure
\figvisu{\figBoxA}{}{%
\figptsaxes 1:0(\Xmin, \Xmaxx, \Ymin, \Ymaxx)
% Points 1 and 2 are the end points of the arrows
\figwritee 1:(5pt)     \figwriten 2:(5pt)
\figptsaxes 1:0(\Xmin, \Xmax, \Ymin, \Ymax)
\figset write(mark=$\bullet$)
\figwritese 100:(3)
\figwritese 101:(3)
\figwritese 0: $O$(3)
\figwritese 102:(3)
\ifprintanswers
  \figwritee 103:(2)
  \figwrites 104:(2)
\fi
}
\centerline{\box\figBoxA}

  \end{marginfigure}


\begin{solution}[\halfpage]
	First, drop perpendiculars from $A$ on the axes to get points $J(0,\vbtwo)$ and $K(-\vbone,0)$. 
  The other points in the figure are $M = (m,0)$ and $N = (0,n)$
	
	Now, $\bigtriangleup NAJ$ and $\bigtriangleup NMO$ are similar. Which means, 
	\begin{align}
		\dfrac{NJ}{NO} &= \dfrac{NA}{NM} \Rightarrow \dfrac{n-\vbtwo}{n} = \dfrac{\vbthree}{\vbseven} \\
		\Rightarrow n &= \dfrac{\vbfive}{\vbfour} = \vbeight \text{ or, in other words } N = (0, \vbeight)
	\end{align}
	
	\gcalcexpr[0]{\vbnine}{\vbeight - \vbtwo}
	\gcalcexpr[0]{\vbten}{\vbeight * \vbone }
	
	We now have two points on a line. And that is enough to calculate its equation
	\begin{align}
		\dfrac{y-\vbeight}{x} &= \dfrac{\vbeight - \vbtwo}{0 + \vbone} \\
		\Rightarrow \vbone\cdot y - \vbten &= \vbnine\cdot x
	\end{align}
	
\end{solution}

\ifprintrubric
  \begin{table}
  	\begin{tabular}{ p{5cm}p{5cm} }
  		\toprule % in brief (4-6 words), what should a grader be looking for for insights & formulations
  		  \sc{\textcolor{blue}{Insight}} & \sc{\textcolor{blue}{Formulation}} \\ 
  		\midrule % ***** Insights & formulations ******
        Drop perpendiculars to get similar triangles & \\
        Get a second point using ratio of sides of similar triangles & Expressed the ratio \\
  		\toprule % final numerical answers for the various versions
        \sc{\textcolor{blue}{If question has $\ldots$}} & \sc{\textcolor{blue}{Final answer}} \\
  		\midrule % ***** Numerical answers (below) **********
        $A = (-5,4)$ & $5y - 60 = 8x$ \\
        $A = (-2,7)$ & $2y - 20 = 3x$ \\
        $A = (-7,10)$ & $7y - 126 = 8x$ \\
        $A = (-5,9)$ & $5y-80=7x$ \\
  		\bottomrule
  	\end{tabular}
  \end{table}
\fi
