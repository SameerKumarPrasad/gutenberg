


\ifnumequal{\value{rolldice}}{0}{
  % variables 
  \renewcommand{\va}{5}
  \renewcommand{\vb}{4}
  \renewcommand{\vc}{2}
  \renewcommand{\vd}{1}
}{
  \ifnumequal{\value{rolldice}}{1}{
    % variables 
    \renewcommand{\va}{2}
    \renewcommand{\vb}{7}
    \renewcommand{\vc}{3}
    \renewcommand{\vd}{7}
  }{
    \ifnumequal{\value{rolldice}}{2}{
      % variables 
      \renewcommand{\va}{7}
      \renewcommand{\vb}{10}
      \renewcommand{\vc}{4}
      \renewcommand{\vd}{5}
    }{
      % variables 
      \renewcommand{\va}{5}
      \renewcommand{\vb}{9}
      \renewcommand{\vc}{7}
      \renewcommand{\vd}{9}
    }
  }
}

\EXPR[0]{\ve}{\vb * (\vc + \vd)}
\EXPR[0]{\vf}{-\va*(\vc + \vd)} % numerator of m
\EXPR[0]{\vg}{\vc + \vd}
\EXPR[0]{\vh}{\ve / \vd } % ve = n  guaranteed by us to be an integer
\SUBTRACT\vh\vb\vi
\MULTIPLY\vh\va\vj

\question[5] The intercept of a line is divided by a point $A= (-\va, \vb)$ so that 
$\frac{AN}{AM} = \frac{\vc}{\vd}$ (see figure). What is the equation of the line? 

\watchout

\figinit{pt}
\def\Xmin{-72.72727}
\def\Ymin{-18.18181}
\def\Xmax{7.27272}
\def\Ymax{61.81818}
\def\Xori{72.72727}
\def\Yori{18.18181}
\figpt0:(\Xori,\Yori)
\figpt 100:$M$(18,18)
\figpt 101:$N$(72,72)
\figpt 102:$A$(47,47)
\figpt 103:$J$(\Xori, 47)
\figpt 104:$K$(47,\Yori)
\figdrawbegin{}
\def\Xmaxx{\Xmax} % To customize the position
\def\Ymaxx{\Ymax} % of the arrow-heads of the axes.
\figset arrowhead(length=4, fillmode=yes) % styling the arrowheads
\figdrawaxes 0(\Xmin, \Xmaxx, \Ymin, \Ymaxx)
\figdrawlineC(
0 0, % y = -4.00
2.75862 2.75862, % y = -3.39
5.51724 5.51724, % y = -2.78
8.27586 8.27586, % y = -2.17
11.03448 11.03448, % y = -1.57
13.79310 13.79310, % y = -.96
16.55172 16.55172, % y = -.35
19.31034 19.31034, % y = .24
22.06896 22.06896, % y = .85
24.82758 24.82758, % y = 1.46
27.58620 27.58620, % y = 2.06
30.34482 30.34482, % y = 2.67
33.10344 33.10344, % y = 3.28
35.86206 35.86206, % y = 3.88
38.62068 38.62068, % y = 4.49
41.37931 41.37931, % y = 5.10
44.13793 44.13793, % y = 5.71
46.89655 46.89655, % y = 6.31
49.65517 49.65517, % y = 6.92
52.41379 52.41379, % y = 7.53
55.17241 55.17241, % y = 8.13
57.93103 57.93103, % y = 8.74
60.68965 60.68965, % y = 9.35
63.44827 63.44827, % y = 9.95
66.20689 66.20689, % y = 10.56
68.96551 68.96551, % y = 11.17
71.72413 71.72413, % y = 11.77
74.48275 74.48275, % y = 12.38
77.24137 77.24137, % y = 12.99
79.99999 79.99999
)
\ifprintanswers
  \figset (dash=4)
  \figdrawline [102,103]
  \figdrawline [102,104]
\fi
\figdrawend
\figvisu{\figBoxA}{}{%
\large
\figptsaxes 1:0(\Xmin, \Xmaxx, \Ymin, \Ymaxx)
\figwritee 1:(5pt)     \figwriten 2:(5pt)
\figptsaxes 1:0(\Xmin, \Xmax, \Ymin, \Ymax)
\figset write(mark=$\bullet$)
\figwritese 100:(3)
\figwritese 101:(3)
\figwritesw 0: $O$(3)
\figwritese 102:(3)
\ifprintanswers
  \figwritee 103:(3)
  \figwrites 104:(3)
\fi
}

\vspace{0.7cm}
\centerline{\box\figBoxA}

\begin{solution}[\halfpage]
	First, drop perpendiculars from $A$ on the axes to get points $J(0,\vb)$ and $K(-\va,0)$. 
  The other points in the figure are $M = (m,0)$ and $N = (0,n)$
	
	Now, $\bigtriangleup NAJ$ and $\bigtriangleup NMO$ are similar. Which means, 
	\begin{align}
		\dfrac{NJ}{NO} &= \dfrac{NA}{NM} \implies \dfrac{n-\vb}{n} = \dfrac{\vc}{\vg} \\
		\implies n &= \dfrac{\ve}{\vd} = \vh \text{ or, in other words } N = (0, \vh)
	\end{align}
	
	We now have two points on a line. And that is enough to calculate its equation
	\begin{align}
		\dfrac{y-\vh}{x} &= \dfrac{\vh - \vb}{0 + \va} \\
		\implies \va y - \vj &= \vi x
	\end{align}
	
\end{solution}

\ifprintanswers
  \begin{codex}
    $\vi x-\va y + \vj = 0$
  \end{codex}
\fi
