
\ifnumequal{\value{rolldice}}{0}{
  \renewcommand{\vbone}{-2}
  \renewcommand{\vbtwo}{-5}
}{
  \ifnumequal{\value{rolldice}}{1}{
    \renewcommand{\vbone}{-3}
    \renewcommand{\vbtwo}{-7}
  }{
    \ifnumequal{\value{rolldice}}{2}{
      \renewcommand{\vbone}{-4}
      \renewcommand{\vbtwo}{-9}
    }{
      \renewcommand{\vbone}{-5}
      \renewcommand{\vbtwo}{-6}
    }
  }
}

\ADD\vbone\vbtwo\a
\MULTIPLY\vbone\vbtwo\b
\MULTIPLY\a{-1}\c

\MULTIPLY\vbone{-1}\p
\MULTIPLY\vbtwo{-1}\q

\question[3] What would the \textbf{equivalent roster form} of the following set be? The roster form is one 
in which all elements are listed individually  - for example - $C = \lbrace a,b,c\ldots z\rbrace$
\[ C = \lbrace x\,:\, x^2 + \c x + \b = 0, x\in\mathbb{N}\rbrace  \]

\watchout[-50pt]

\begin{solution}[\mcq]
  First, lets find the $x$ that satisfy the quadratic equation
  \begin{align}
    x^2 + \c x + \b &= 0 \implies (x + \p)\cdot (x + \q) = 0 \\
    \implies x &= \vbone \text{ or } x = \vbtwo
  \end{align}

  Hence, $C$ \textbf{could be} $= \lbrace \vbone, \vbtwo \rbrace$. But note that $x\in\mathbb{N}$.
  And as neither $\vbone\notin\mathbb{N}$ nor $\vbtwo\notin\mathbb{N}$, hence $C=\phi$ (empty set)
\end{solution}

