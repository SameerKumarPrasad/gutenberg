
\ifnumequal{\value{rolldice}}{0}{
  \renewcommand{\vbone}{7}
  \renewcommand{\vbtwo}{1}
  \renewcommand{\vbthree}{5}
  \renewcommand{\vbfour}{3}
  \renewcommand{\vbfive}{16}
  \renewcommand{\vbsix}{0.845}
  \renewcommand{\vbseven}{0.699}
}{
  \ifnumequal{\value{rolldice}}{1}{
    \renewcommand{\vbone}{3}
    \renewcommand{\vbtwo}{3}
    \renewcommand{\vbthree}{7}
    \renewcommand{\vbfour}{2}
    \renewcommand{\vbfive}{18}
    \renewcommand{\vbsix}{0.477}
    \renewcommand{\vbseven}{0.845}
  }{
    \ifnumequal{\value{rolldice}}{2}{
      \renewcommand{\vbone}{2}
      \renewcommand{\vbtwo}{7}
      \renewcommand{\vbthree}{9}
      \renewcommand{\vbfour}{2}
      \renewcommand{\vbfive}{17}
      \renewcommand{\vbsix}{0.301}
      \renewcommand{\vbseven}{0.954}
    }{
      \renewcommand{\vbone}{5}
      \renewcommand{\vbtwo}{4}
      \renewcommand{\vbthree}{3}
      \renewcommand{\vbfour}{2}
      \renewcommand{\vbfive}{21}
      \renewcommand{\vbsix}{0.699}
      \renewcommand{\vbseven}{0.477}
    }
  }
}

\POWER\vbone\vbtwo\m
\POWER\vbthree\vbfour\n
\MULTIPLY\m\n\p
\EXPR[2]\q{(\vbfive * (\vbtwo * \vbsix + \vbfour * \vbseven ) )}
\FLOOR\q\a
\ADD\a{1}\b

\question Given a number $N = \p^{\vbfive}$

\watchout
\marginnote{ 

  $\log_{10}\vbone = \vbsix$

  $\log_{10}\vbthree = \vbseven$
}

\begin{parts}
  \part[2] Find $\log_{10} N$

\begin{solution}[\mcq]
    \begin{align}
      \log_{10} N &= \log_{10} \p^{\vbfive} = \vbfive\cdot\log_{10}\p \\
      &= \vbfive\cdot\log_{10}(\vbone^\vbtwo\times\vbthree^\vbfour) \\
      &= \vbfive\cdot (\vbtwo\cdot\log_{10}\vbone + \vbfour\cdot\log_{10}\vbthree) = \q
    \end{align}
  \end{solution}

  \part[1] How many digits would there be in an \textbf{integer} M that lies between $10^{k}$ and $10^{k+1},\,k\in\mathbb{N}$? Your answer should be in terms of $k$

\begin{solution}[\mcq]
    There are \textbf{k+1} digits in an \textbf{integer} that lies between $10^k$ and $10^{k+1}$.
    
    Think about integers that lie between $10^0 = 1$ and $10^1 = 10$. Or, for that matter
    between $10^2 = 100$ and $10^3 = 1000$.  
  \end{solution}

  \part[2] Using your answers for part (a) and (b), how many digits do you think there are in  $\p^{\vbfive}$? 

\begin{solution}[\mcq]
    In part (a), we found $\log_{10}\p^{\vbfive} = \q\implies 10^{\q} = \p^{\vbfive}$.
    
    It also means that $\p^{\vbfive} \in [10^{\a}, 10^{\b}]$. And from 
    part (b), we know that $N$ would therefore have $\b$ digits
  \end{solution}

\end{parts}

