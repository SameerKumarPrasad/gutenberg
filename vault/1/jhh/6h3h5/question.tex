


\ifnumequal{\value{rolldice}}{0}{
  % variables 
  \renewcommand{\vbone}{2}
  \renewcommand{\vbtwo}{3}
}{
  \ifnumequal{\value{rolldice}}{1}{
    % variables 
    \renewcommand{\vbone}{3}
    \renewcommand{\vbtwo}{2}
  }{
    \ifnumequal{\value{rolldice}}{2}{
      % variables 
      \renewcommand{\vbone}{2}
      \renewcommand{\vbtwo}{4}
    }{
      % variables 
      \renewcommand{\vbone}{4}
      \renewcommand{\vbtwo}{2}
    }
  }
}

\POWER\vbone\vbtwo\a
\SUBTRACT\a{1}\b % 
\POWER\b{2}\c

\question Prove that $(\vbone^\vbtwo)^{n} - \b n - 1$ is a multiple of 
$\c\quad\forall n\in\mathbb{N}, n\geq 2$

\insertQR{}

\watchout

\ifprintanswers
\fi 

\begin{solution}
  \begin{align}
   (\vbone^\vbtwo)^{n} - \b n - 1 &= \underbrace{(1 + \b)^{n}}_{(1+x)^n} - \b n - 1 \\
   &= \overbrace{\left[\sum_{k=0}^{n}\binom{n}{k} \b^{k}\right]}- \b n - 1 \\
   &= \left[\underbrace{1 + \b n}_{\texttt{First two terms}} 
   + \sum_{k=2}^{n} \binom{n}{k}\b^{k}\right] - \b n - 1 \\
   &= \b^{2}\cdot\sum_{k=2}^{n} \binom{n}{k} \b^{k-2} = \c\cdot\sum_{k=2}^{n} \binom{n}{k} \b^{k-2}
  \end{align}
  
  In short, we have shown that $(\vbone^\vbtwo)^{n} - \b n - 1 =  \c\times\texttt{something} \Rightarrow$ multiple of $\c$
\end{solution}

\ifprintrubric
  \begin{table}
  	\begin{tabular}{ p{5cm}p{5cm} }
  		\toprule % in brief (4-6 words), what should a grader be looking for for insights & formulations
  		  \sc{\textcolor{blue}{Insight}} & \sc{\textcolor{blue}{Formulation}} \\ 
  		\midrule % ***** Insights & formulations ******
  		\toprule % final numerical answers for the various versions
        \sc{\textcolor{blue}{If question has $\ldots$}} & \sc{\textcolor{blue}{Final answer}} \\
  		\midrule % ***** Numerical answers (below) **********
  		\bottomrule
  	\end{tabular}
  \end{table}
\fi
