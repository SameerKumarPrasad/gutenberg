


\ifnumequal{\value{rolldice}}{0}{
  % variables 
  \renewcommand{\vbone}{4} %a
  \renewcommand{\vbtwo}{5} % b
  \renewcommand{\vbthree}{5} %c
  \renewcommand{\vbfour}{2} %d
  \renewcommand{\vbfive}{9} % N
  \renewcommand{\vbsix}{4}
  \renewcommand{\vbseven}{10,500}
}{
  \ifnumequal{\value{rolldice}}{1}{
    % variables 
    \renewcommand{\vbone}{3}
    \renewcommand{\vbtwo}{4}
    \renewcommand{\vbthree}{2}
    \renewcommand{\vbfour}{3}
    \renewcommand{\vbfive}{10}
    \renewcommand{\vbsix}{5}
    \renewcommand{\vbseven}{\dfrac{35}{6}}
  }{
    \ifnumequal{\value{rolldice}}{2}{
      % variables 
      \renewcommand{\vbone}{2}
      \renewcommand{\vbtwo}{3}
      \renewcommand{\vbthree}{3}
      \renewcommand{\vbfour}{4}
      \renewcommand{\vbfive}{9}
      \renewcommand{\vbsix}{4}
      \renewcommand{\vbseven}{\dfrac{567}{128}}
    }{
      % variables 
      \renewcommand{\vbone}{4}
      \renewcommand{\vbtwo}{3}
      \renewcommand{\vbthree}{3}
      \renewcommand{\vbfour}{2}
      \renewcommand{\vbfive}{8}
      \renewcommand{\vbsix}{3}
      \renewcommand{\vbseven}{567}
    }
  }
}

\ADD\vbfive{1}\t % total # of terms
\SUBTRACT\t\vbsix\k % desired term
\SUBTRACT\vbfive\k\a
\SUBTRACT\a\k\b

\question If $\left( \dfrac{\vbone x}{\vbtwo} - \dfrac{\vbthree}{\vbfour x}\right)^{\vbfive}$ is expanded 
so that the power of $x$ is highest in the first term and lowest in the last, then what 
would the $\vbsix$th term from the end be?

\insertQR{}

\watchout

\ifprintanswers
\fi 

\begin{solution}
  For the first term to have the highest power of $x$ and the last to have the lowest, 
  we would expand the expression as follows
  \begin{align}
    \left( \dfrac{\vbone x}{\vbtwo} - \dfrac{\vbthree}{\vbfour x}\right)^{\vbfive}
    &= \sum_{k=0}^{n}\binom{n}{k}\left(\dfrac{\vbone x}{\vbtwo}\right)^{n-k}\cdot
    \left( -\dfrac{\vbthree}{\vbfour x}\right)^{k}
  \end{align}
  
  Moreover, there would be $\t$ terms in the expansion. And if the terms go from 
  $T_0\rightarrow T_{\vbfive}$, then the $\vbsix$th term from the end would 
  be $T_{\k}$
  
  \begin{align}
    T_{\k} &= \binom\vbfive\k
    \left(\dfrac{\vbone x}{\vbtwo}\right)^{\vbfive - \k}\times
    \left(\dfrac{\vbthree}{\vbfour x}\right)^{\k}\times (-1)^{\k} \\
    &= \vbseven\cdot x^{\b}
  \end{align}
\end{solution}

\ifprintrubric
  \begin{table}
  	\begin{tabular}{ p{5cm}p{5cm} }
  		\toprule % in brief (4-6 words), what should a grader be looking for for insights & formulations
  		  \sc{\textcolor{blue}{Insight}} & \sc{\textcolor{blue}{Formulation}} \\ 
  		\midrule % ***** Insights & formulations ******
  		\toprule % final numerical answers for the various versions
        \sc{\textcolor{blue}{If question has $\ldots$}} & \sc{\textcolor{blue}{Final answer}} \\
  		\midrule % ***** Numerical answers (below) **********
  		\bottomrule
  	\end{tabular}
  \end{table}
\fi
