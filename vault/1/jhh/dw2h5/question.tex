


\ifnumequal{\value{rolldice}}{0}{
  % variables 
  \renewcommand{\va}{4} %a
  \renewcommand{\vb}{5} % b
  \renewcommand{\vc}{5} %c
  \renewcommand{\vd}{2} %d
  \renewcommand{\ve}{9} % N
  \renewcommand{\vf}{4}
  \renewcommand{\vg}{10,500}
}{
  \ifnumequal{\value{rolldice}}{1}{
    % variables 
    \renewcommand{\va}{3}
    \renewcommand{\vb}{4}
    \renewcommand{\vc}{2}
    \renewcommand{\vd}{3}
    \renewcommand{\ve}{10}
    \renewcommand{\vf}{5}
    \renewcommand{\vg}{\dfrac{35}{6}}
  }{
    \ifnumequal{\value{rolldice}}{2}{
      % variables 
      \renewcommand{\va}{2}
      \renewcommand{\vb}{3}
      \renewcommand{\vc}{3}
      \renewcommand{\vd}{4}
      \renewcommand{\ve}{9}
      \renewcommand{\vf}{4}
      \renewcommand{\vg}{\dfrac{567}{128}}
    }{
      % variables 
      \renewcommand{\va}{4}
      \renewcommand{\vb}{3}
      \renewcommand{\vc}{3}
      \renewcommand{\vd}{2}
      \renewcommand{\ve}{8}
      \renewcommand{\vf}{3}
      \renewcommand{\vg}{567}
    }
  }
}

\ADD\ve{1}\t % total # of terms
\SUBTRACT\t\vf\k % desired term
\SUBTRACT\ve\k\a
\SUBTRACT\a\k\b

\question[3] What would the $\vf^{\text{th}}$ term \textbf{from the end} be if the following 
is expanded so that the highest powers of $x$ are written first and the lowest last? 
 \[ \left( \dfrac{\va x}{\vb} - \dfrac{\vc}{\vd x}\right)^{\ve} \]

\watchout

\begin{solution}[\halfpage]
  For the first term to have the highest power of $x$ and the last to have the lowest, 
  we would expand the expression as follows
  \begin{align}
    \left( \dfrac{\va x}{\vb} - \dfrac{\vc}{\vd x}\right)^{\ve}
    &= \sum_{k=0}^{n}\binom{n}{k}\left(\dfrac{\va x}{\vb}\right)^{n-k}\cdot
    \left( -\dfrac{\vc}{\vd x}\right)^{k}
  \end{align}
  
  Moreover, there would be $\t$ terms in the expansion. And if the terms go from 
  $T_0\rightarrow T_{\ve}$, then the $\vf$th term from the end would 
  be $T_{\k}$
  
  \begin{align}
    T_{\k} &= \binom\ve\k
    \left(\dfrac{\va x}{\vb}\right)^{\ve - \k}\times
    \left(\dfrac{\vc}{\vd x}\right)^{\k}\times (-1)^{\k} = \vg\cdot x^{\b}
  \end{align}
\end{solution}

\ifprintanswers\begin{codex}$\vg\cdot x^{\b}$\end{codex}\fi
