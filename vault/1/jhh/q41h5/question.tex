

\ifnumequal{\value{rolldice}}{0}{
  % variables 
  \renewcommand{\vbone}{\dfrac{1}{4}}
  \renewcommand{\vbtwo}{\dfrac{1}{4}}
  \renewcommand{\vbthree}{\dfrac{3}{4}}
  \renewcommand{\vbfour}{1}
  \renewcommand{\vbfive}{3}
  \renewcommand{\vbsix}{7}
  \renewcommand{\vbseven}{}
  \renewcommand{\vbeight}{}
  \renewcommand{\vbnine}{}
  \renewcommand{\vbten}{}
}{
  \ifnumequal{\value{rolldice}}{1}{
    % variables 
    \renewcommand{\vbone}{1}
    \renewcommand{\vbtwo}{5}
    \renewcommand{\vbthree}{10}
    \renewcommand{\vbfour}{5}
    \renewcommand{\vbfive}{2}
    \renewcommand{\vbsix}{11}
    \renewcommand{\vbseven}{}
    \renewcommand{\vbeight}{}
    \renewcommand{\vbnine}{}
    \renewcommand{\vbten}{}
  }{
    \ifnumequal{\value{rolldice}}{2}{
      % variables 
      \renewcommand{\vbone}{\dfrac{1}{2}}
      \renewcommand{\vbtwo}{1}
      \renewcommand{\vbthree}{3}
      \renewcommand{\vbfour}{2}
      \renewcommand{\vbfive}{3}
      \renewcommand{\vbsix}{8}
      \renewcommand{\vbseven}{}
      \renewcommand{\vbeight}{}
      \renewcommand{\vbnine}{}
      \renewcommand{\vbten}{}
    }{
      % variables 
      \renewcommand{\vbone}{\dfrac{-1}{3}}
      \renewcommand{\vbtwo}{-1}
      \renewcommand{\vbthree}{-5}
      \renewcommand{\vbfour}{3}
      \renewcommand{\vbfive}{5}
      \renewcommand{\vbsix}{12}
      \renewcommand{\vbseven}{}
      \renewcommand{\vbeight}{}
      \renewcommand{\vbnine}{}
      \renewcommand{\vbten}{}
    }
  }
}

\question[2] Let $\vec{a}=\vbfour\hat{i}-\hat{j}$, $\vec{b}=\vbfive\hat{j}-\hat{k}$ 
and $\vec{c}=\vbsix\hat{i}-\hat{k}$. Find a vector $\vec{d}$ which is perpendicular
to both $\vec{a}$ and $\vec{b}$ such that $\vec{c}\cdot\vec{d}=1$.

\insertQR{}

\watchout

\ifprintanswers
  % stuff to be shown only in the answer key - like explanatory margin figures
  \begin{marginfigure}
    \figinit{pt}
      \figpt 100:(0,0)
      \figpt 101:(0,0)
    \figdrawbegin{}
      \figdrawline [100,101]
    \figdrawend
    \figvisu{\figBoxA}{}{%
    }
    \centerline{\box\figBoxA}
  \end{marginfigure}
\fi 

\begin{solution}[\halfpage]
Let $\vec{d}= l\hat{i}+m\hat{j}+n\hat{k}$. \\
Since $\vec{d}$ is perpendicular to $\vec{a}$ and $\vec{b}$ 
and $\vec{c}\cdot\vec{d}=0$ therefore,
\begin{align}
  \vec{a}\cdot\vec{d}&=0 \Rightarrow \vbfour\cdot\text{l} + (-1)\cdot\text{m} = 0 \\
  \vec{b}\cdot\vec{d}&=0 \Rightarrow \vbfive\cdot\text{m} + (-1)\cdot\text{n} = 0 \\
  \vec{c}\cdot\vec{d}&=1 \Rightarrow \vbsix\cdot\text{l} + (-1)\cdot\text{n} = 1 
\end{align}
From equations (1), (2) and (3),
\begin{align}
  \vec{d}=\vbone\hat{i} + \vbtwo\hat{j} + \vbthree\hat{k} 
\end{align}

\end{solution}

\ifprintrubric
  \begin{table}
  	\begin{tabular}{ p{5cm}p{5cm} }
  		\toprule % in brief (4-6 words), what should a grader be looking for for insights & formulations
  		  \sc{\textcolor{blue}{Insight}} & \sc{\textcolor{blue}{Formulation}} \\ 
  		\midrule % ***** Insights & formulations ******
  		\toprule % final numerical answers for the various versions
        \sc{\textcolor{blue}{If question has $\ldots$}} & \sc{\textcolor{blue}{Final answer}} \\
  		\midrule % ***** Numerical answers (below) **********
  		\bottomrule
  	\end{tabular}
  \end{table}
\fi
