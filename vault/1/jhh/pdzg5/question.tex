
% \noprintanswers
%\setcounter{rolldice}{2}
% \printrubric

\ifnumequal{\value{rolldice}}{0}{
  % variables 
  \renewcommand{\vbone}{3}
  \renewcommand{\vbtwo}{2}
  \renewcommand{\vbthree}{9}
  \renewcommand{\vbfour}{1}
}{
  \ifnumequal{\value{rolldice}}{1}{
    % variables 
    \renewcommand{\vbone}{7}
    \renewcommand{\vbtwo}{4}
    \renewcommand{\vbthree}{7}
    \renewcommand{\vbfour}{3}
  }{
    \ifnumequal{\value{rolldice}}{2}{
      % variables 
      \renewcommand{\vbone}{4}
      \renewcommand{\vbtwo}{5}
      \renewcommand{\vbthree}{6}
      \renewcommand{\vbfour}{3}
    }{
      % variables 
      \renewcommand{\vbone}{5}
      \renewcommand{\vbtwo}{3}
      \renewcommand{\vbthree}{5}
      \renewcommand{\vbfour}{6}
    }
  }
}

\FRACMULT\vbfour\vbone\vbthree{1}\vbfive\a
\FRACMULT\vbfour\vbone\vbtwo{1}\p\q

\question[1] For what value of $p$ would the vectors $\vec{a} = \WRITEVEC{\vbone}{\vbtwo}{\vbthree}$ and 
$\vec{b} = \WRITEVECGENERAL{\vbfour}{p}{\vbfive}$ be parallel?

\insertQR[-15pt]{}

\watchout

\ifprintanswers
\fi 

\begin{solution}[\mcq]
	For two vectors $\vec{a} = \WRITEVECGENERAL{a_1}{a_2}{a_3}$ and $\vec{b} = \WRITEVECGENERAL{b_1}{b_2}{b_3}$ 
	to be parallel to each other, 
	\begin{align}
		\dfrac{a_1}{b_1} &= \dfrac{a_2}{b_2} = \dfrac{a_3}{b_3} \\
		\Rightarrow \dfrac{\vbone}{\vbfour} &= \dfrac{\vbtwo}{p} = \dfrac{\vbthree}{\vbfive} \\
		\Rightarrow p &= \dfrac{\p}{\q}
	\end{align}
\end{solution}

\ifprintrubric
  \begin{table}
  	\begin{tabular}{ p{5cm}p{5cm} }
  		\toprule % in brief (4-6 words), what should a grader be looking for for insights & formulations
  		  \sc{\textcolor{blue}{Insight}} & \sc{\textcolor{blue}{Formulation}} \\ 
  		\midrule % ***** Insights & formulations ******
  			Correctly stated relationship between coefficients of parallel vectors \\
  		\toprule % final numerical answers for the various versions
        \sc{\textcolor{blue}{If question has $\ldots$}} & \sc{\textcolor{blue}{Final answer}} \\
  		\midrule % ***** Numerical answers (below) **********
  			$\vec{a} = 3\hat{i} + 2\hat{j} + 9\hat{k}$ & $p = \frac{2}{3}$ \\
  			$\vec{a} = 7\hat{i} + 4\hat{j} + 7\hat{k}$ & $p = \frac{12}{7}$ \\
  			$\vec{a} = 4\hat{i} + 5\hat{j} + 6\hat{k}$ & $p = \frac{15}{4}$ \\
  			$\vec{a} = 5\hat{i} + 3\hat{j} + 5\hat{k}$ & $p = \frac{18}{5}$ \\
  		\bottomrule
  	\end{tabular}
  \end{table}
\fi
