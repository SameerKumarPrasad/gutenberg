\ifnumequal{\value{rolldice}}{0}{
  % variables 
  \renewcommand{\vbone}{2}
  \renewcommand{\vbtwo}{-1}
  \renewcommand{\vbthree}{+1}
  \renewcommand{\vbfour}{1}
  \renewcommand{\vbfive}{-3}
  \renewcommand{\vbsix}{-5}
  \renewcommand{\vbseven}{3}
  \renewcommand{\vbeight}{-4}
  \renewcommand{\vbnine}{-4}
}{
  \ifnumequal{\value{rolldice}}{1}{
    % variables 
    \renewcommand{\vbone}{8}
    \renewcommand{\vbtwo}{+10}
    \renewcommand{\vbthree}{+12}
    \renewcommand{\vbfour}{1}
    \renewcommand{\vbfive}{-3}
    \renewcommand{\vbsix}{-5}
    \renewcommand{\vbseven}{3}
    \renewcommand{\vbeight}{-4}
    \renewcommand{\vbnine}{-4}
  }{
    \ifnumequal{\value{rolldice}}{2}{
      % variables 
      \renewcommand{\vbone}{2}
      \renewcommand{\vbtwo}{-1}
      \renewcommand{\vbthree}{+1}
      \renewcommand{\vbfour}{11}
      \renewcommand{\vbfive}{+2}
      \renewcommand{\vbsix}{+1}
      \renewcommand{\vbseven}{3}
      \renewcommand{\vbeight}{-4}
      \renewcommand{\vbnine}{-4}
    }{
      % variables 
      \renewcommand{\vbone}{2}
      \renewcommand{\vbtwo}{-1}
      \renewcommand{\vbthree}{+1}
      \renewcommand{\vbfour}{1}
      \renewcommand{\vbfive}{-3}
      \renewcommand{\vbsix}{-5}
      \renewcommand{\vbseven}{3}
      \renewcommand{\vbeight}{-4}
      \renewcommand{\vbnine}{-4}
    }
  }
}

\question[3] Show that the position vectors $\vbone\hat{i} \vbtwo\hat{j} \vbthree\hat{k}$,
$\vbfour\hat{i} \vbfive\hat{j} \vbsix\hat{k}$ and 
$\vbseven\hat{i} \vbeight\hat{j} \vbnine\hat{k}$ form the vertices of a right angled 
triangle.

\insertQR{}

\watchout

\ifprintanswers
  % stuff to be shown only in the answer key - like explanatory margin figures
  \begin{marginfigure}
    \figinit{pt}
      \figpt 100:(0,0)
      \figpt 101:(0,0)
    \figdrawbegin{}
      \figdrawline [100,101]
    \figdrawend
    \figvisu{\figBoxA}{}{%
    }
    \centerline{\box\figBoxA}
  \end{marginfigure}
\fi 

\begin{solution}[\fullpage]
Let $A$, $B$ and $C$ be the position vectors. The vectors 
representing the sides of the triangle are $\vec{AB}$, $\vec{BC}$
and $\vec{CA}$.
\begin{align}
  \vec{AB} &= (\vbone - \vbfour)\hat{i} + (\vbtwo - (\vbfive))\hat{j} + (\vbthree - (\vbsix))\hat{k} \\
  \vec{BC} &= (\vbfour - \vbseven)\hat{i} + (\vbfive - (\vbeight))\hat{j} + (\vbsix - (\vbnine))\hat{k} \\
  \vec{CA} &= (\vbseven - \vbone)\hat{i} + (\vbeight - (\vbtwo))\hat{j} + (\vbnine - (\vbthree))\hat{k}
\end{align}
For the triangle to be right angled, one of the following conditions
needs to be true,
\begin{align}
  \vec{AB} \perp \vec{BC} \\
  \vec{BC} \perp \vec{CA} \\
  \vec{CA} \perp \vec{AB}
\end{align}
We can find this out by computing the dot product of each of the above two pairs of vectors above and verifying 
that \textbf{any one} of them is equal to zero. In this case,
\begin{align}
\ifnumequal{\value{rolldice}}{0}{
  \vec{BC}\cdot\vec{CA} = 0
}{
  \ifnumequal{\value{rolldice}}{1}{
    \vec{AB}\cdot\vec{BC} = 0
  }{
    \ifnumequal{\value{rolldice}}{2}{
      \vec{CA}\cdot \vec{AB} = 0
    }{
      \vec{AB}\cdot{BC} = 0
    }
  }
}    
\end{align}

\end{solution}

\ifprintrubric
  \begin{table}
  	\begin{tabular}{ p{5cm}p{5cm} }
  		\toprule % in brief (4-6 words), what should a grader be looking for for insights & formulations
  		  \sc{\textcolor{blue}{Insight}} & \sc{\textcolor{blue}{Formulation}} \\ 
  		\midrule % ***** Insights & formulations ******
  		\toprule % final numerical answers for the various versions
        \sc{\textcolor{blue}{If question has $\ldots$}} & \sc{\textcolor{blue}{Final answer}} \\
  		\midrule % ***** Numerical answers (below) **********
  		\bottomrule
  	\end{tabular}
  \end{table}
\fi
