
 %\setcounter{rolldice}{1}

\renewcommand{\vc}{(\lambda + 2)^2 + 40}
\ifnumequal{\value{rolldice}}{0}{
  % variables 
  \renewcommand{\va}{1}
  \renewcommand{\vb}{1}
}{
  \ifnumequal{\value{rolldice}}{1}{
    % variables 
    \renewcommand{\va}{2}
    \renewcommand{\vb}{3}
  }{
    \ifnumequal{\value{rolldice}}{2}{
      % variables 
      \renewcommand{\va}{5}
      \renewcommand{\vb}{4}
    }{
      % variables 
      \renewcommand{\va}{3}
      \renewcommand{\vb}{8}
    }
  }
}

\MULTIPLY\va{6}\a
\MULTIPLY\vb{2}\b
\SUBTRACT\a\b\c
\ADD\c{2}\d
\MULTIPLY\d{2}\e
\SQUARE\d\f
\SUBTRACT{44}\f\n
\SUBTRACT\e{4}\d

\question[2] The scalar product of the vector 
\[ \vec{V} = \hat{i} + \va\hat{j} + \vb\hat{k} \] 
and the unit vector along the sum of 
\[ \vec{A} = \WRITEVEC{2}{4}{-5}\text{ and }\vec{B} =\WRITEVECGENERAL{\lambda}{2}{3}\] 
is $1$. Find the value of $\lambda$.


\watchout

\ifprintanswers
\fi 

\begin{solution}[\mcq]
	\begin{align}
		\vec{A} + \vec{B} &= (\lambda + 2)\hat{i} + 6\hat{j} - 2\hat{k}
	\end{align}
	And therefore, the unit vector along $\vec{A} + \vec{B}$ would be 
	\begin{align}
		= \dfrac{\lambda + 2}{\sqrt{\vc}}\hat{i} + \dfrac{6}{\sqrt{\vc}}\hat{j} 
		- \dfrac{2}{\sqrt{\vc}} \hat{k}
	\end{align}
	Now, if the dot product of $\vec{V}$ and the unit vector is $1$, then 
	\begin{align}
		\dfrac{\lambda + 2}{\sqrt{\vc}}\times{1} &+ \dfrac{6}{\sqrt{\vc}}\times\va
		- \dfrac{2}{\sqrt{\vc}}\times\vb = 1 \\
		\implies (\lambda + 2) + \c &= \sqrt{\vc} \\
		\implies \lambda^{2} + \e\lambda + \f &= \lambda^{2} + 4\lambda + 44 \\
		\implies \lambda &= \WRITEFRAC\n\d
	\end{align}
\end{solution}

\ifprintanswers\begin{codex}$\WRITEFRAC\n\d$\end{codex}\fi
