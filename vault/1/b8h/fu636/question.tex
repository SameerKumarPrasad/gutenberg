


\ifnumequal{\value{rolldice}}{0}{
  % variables 
  \renewcommand{\vbone}{8}
  \renewcommand{\vbtwo}{4}
  \renewcommand{\vbfive}{240}
}{
  \ifnumequal{\value{rolldice}}{1}{
    % variables 
    \renewcommand{\vbone}{10}
    \renewcommand{\vbtwo}{5}
    \renewcommand{\vbfive}{4200}
  }{
    \ifnumequal{\value{rolldice}}{2}{
      % variables 
      \renewcommand{\vbone}{11}
      \renewcommand{\vbtwo}{6}
      \renewcommand{\vbfive}{40,320}
    }{
      % variables 
      \renewcommand{\vbone}{9}
      \renewcommand{\vbtwo}{3}
      \renewcommand{\vbfive}{90}
    }
  }
}
\gcalcexpr[0]{\vbthree}{\vbtwo - 1}
\gcalcexpr[0]{\vbfour}{\vbone - 3}

\question[2] There are $\vbone$ persons ( lets call them $P_1, P_2 \ldots P_{\vbone}$ ) of whom $\vbtwo$ must be arranged
in a line. However, $P_1$ must \textit{always} be present in the line whereas $P_4$ and $P_5$ must \textit{never} be. How many such arrangements are possible?


\watchout[-30pt]

\ifprintanswers
\fi 

\begin{solution}[\mcq]
	One out of $\vbtwo$ places in the line is always taken by $P_1$. Hence, its now a question of picking
	$\vbthree$ other people from amongst the $\vbfour$ remaining ( after excluding $P_4$ and $P_5$ ). Moreover,
	the order in which individuals stand in the line is important. Hence, 
	\begin{align}
		N_{\texttt{total}} &= \encr\vbfour\vbthree\cdot \vbtwo\,! \\
		&= \fncr\vbfour\vbthree\cdot \vbtwo\,! \\
		&= \vbfive
	\end{align}
\end{solution}
