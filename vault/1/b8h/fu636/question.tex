% This is an empty shell file placed for you by the 'examiner' script.
% You can now fill in the TeX for your question here.

% Now, down to brasstacks. ** Writing good solutions is an Art **. 
% Eventually, you will find your own style. But here are some thoughts 
% to get you started: 
%
%   1. Write to be understood - but be crisp. Your own solution should not take 
%      more space than you will give to the student. Hence, if you take more than 
%      a half-page to write a solution, then give the student a full-page and so on...
%
%   2. Use margin-notes to "talk" to students about the critical insights
%      in the question. The tone can be - in fact, should be - informal
%
%   3. Don't shy away from creating margin-figures you think will help
%      students understand. Yes, it is a little more work per question. 
%      But the question & solution will be written only once. Make that
%      attempt at writing a solution count.
%      
%      3b. Use bc_to_fig.tex. Its an easier way to generate plots & graphs 
% 
%   4. Ensure that there are *no spelling mistakes anywhere*. We are an 
%      education company. Bad spellings suggest that we ourselves 
%      don't have any education. Also, use American spellings by default
% 
%   5. If a question has multiple parts, then first delete lines 40-41
%   6. If a question does not have parts, then first delete lines 43-69
%   
%   7. Create versions of the question when possible. Use commands defined in 
%      tufte-tweaks.sty to do so. Its easier than you think

% \noprintanswers
% \setcounter{rolldice}{3}

\ifnumequal{\value{rolldice}}{0}{
  % variables 
  \renewcommand{\vbone}{8}
  \renewcommand{\vbtwo}{4}
  \renewcommand{\vbthree}{}
  \renewcommand{\vbfour}{}
  \renewcommand{\vbfive}{240}
  \renewcommand{\vbsix}{}
  \renewcommand{\vbseven}{}
  \renewcommand{\vbeight}{}
  \renewcommand{\vbnine}{}
  \renewcommand{\vbten}{}
}{
  \ifnumequal{\value{rolldice}}{1}{
    % variables 
    \renewcommand{\vbone}{10}
    \renewcommand{\vbtwo}{5}
    \renewcommand{\vbthree}{}
    \renewcommand{\vbfour}{}
    \renewcommand{\vbfive}{4200}
    \renewcommand{\vbsix}{}
    \renewcommand{\vbseven}{}
    \renewcommand{\vbeight}{}
    \renewcommand{\vbnine}{}
    \renewcommand{\vbten}{}
  }{
    \ifnumequal{\value{rolldice}}{2}{
      % variables 
      \renewcommand{\vbone}{11}
      \renewcommand{\vbtwo}{6}
      \renewcommand{\vbthree}{}
      \renewcommand{\vbfour}{}
      \renewcommand{\vbfive}{40,320}
      \renewcommand{\vbsix}{}
      \renewcommand{\vbseven}{}
      \renewcommand{\vbeight}{}
      \renewcommand{\vbnine}{}
      \renewcommand{\vbten}{}
    }{
      % variables 
      \renewcommand{\vbone}{9}
      \renewcommand{\vbtwo}{3}
      \renewcommand{\vbthree}{}
      \renewcommand{\vbfour}{}
      \renewcommand{\vbfive}{90}
      \renewcommand{\vbsix}{}
      \renewcommand{\vbseven}{}
      \renewcommand{\vbeight}{}
      \renewcommand{\vbnine}{}
      \renewcommand{\vbten}{}
    }
  }
}
\gcalcexpr[0]{\vbthree}{\vbtwo - 1}
\gcalcexpr[0]{\vbfour}{\vbone - 3}

\question[2] There are $\vbone$ persons ( lets call them $P_1, P_2 \ldots P_{\vbone}$ ) of whom $\vbtwo$ must be arranged
in a line. However, $P_1$ must \textit{always} be present in the line whereas $P_4$ and $P_5$ must \textit{never} be. How many such arrangements are possible?

\insertQR{QRC}

\watchout[-30pt]

\ifprintanswers
\fi 

\begin{solution}[\mcq]
	One out of $\vbtwo$ places in the line is always taken by $P_1$. Hence, its now a question of picking
	$\vbthree$ other people from amongst the $\vbfour$ remaining ( after excluding $P_4$ and $P_5$ ). Moreover,
	the order in which individuals stand in the line is important. Hence, 
	\begin{align}
		N_{\texttt{total}} &= \encr\vbfour\vbthree\cdot \vbtwo\,! \\
		&= \fncr\vbfour\vbthree\cdot \vbtwo\,! \\
		&= \vbfive
	\end{align}
\end{solution}
