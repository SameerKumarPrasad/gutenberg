


\ifnumequal{\value{rolldice}}{0}{
  % variables 
  \renewcommand{\vbone}{5} % required
  \renewcommand{\vbtwo}{9} % total 
  \renewcommand{\vbthree}{91}
}{
  \ifnumequal{\value{rolldice}}{1}{
    % variables 
    \renewcommand{\vbone}{6}
    \renewcommand{\vbtwo}{8}
    \renewcommand{\vbthree}{13}
  }{
    \ifnumequal{\value{rolldice}}{2}{
      % variables 
      \renewcommand{\vbone}{7}
      \renewcommand{\vbtwo}{11}
      \renewcommand{\vbthree}{204}
    }{
      % variables 
      \renewcommand{\vbone}{4}
      \renewcommand{\vbtwo}{7}
      \renewcommand{\vbthree}{25}
    }
  }
}

\gcalcexpr[0]{\tp}{\vbtwo - 2} 
\gcalcexpr[0]{\tq}{\vbone - 1} 

\question[2] A committee of $\vbone$ persons needs to be formed from amongst $\vbtwo$ people. However, 
Mr. X and Ms. Y do not like each other. And so, if one is picked, then the other cannot be. In how many 
ways then can the committee be formed?


\watchout[-30pt]

\ifprintanswers
\fi 

\begin{solution}[\mcq]
	If Mr. X is picked, then Ms. Y is out of consideration $\Rightarrow$ $\tq$ out of $\tp$ 
	will remain to be picked. Same goes for if Ms. Y is picked first

  And then there are committees in which neither Mr. X \textit{nor} Ms. Y are members $\Rightarrow$ 
  $\vbone$ out of $\tp$ need to be picked
	
	\begin{align}
		N_{\texttt{total}} &= \overbrace{2\times\encr\tp\tq}^{\texttt{either picked}} + 
    \overbrace{\encr\tp\vbone}^{\texttt{neither picked}} \\
		&= \vbthree
	\end{align}
\end{solution}
