

 %\noprintanswers
\setcounter{rolldice}{3}

\ifnumequal{\value{rolldice}}{0}{
  % variables 
  \renewcommand{\vbone}{9}
  \renewcommand{\vbtwo}{3}
  \renewcommand{\vbthree}{55}
  \renewcommand{\vbfour}{28}
  \renewcommand{\vbfive}{1680}
}{
  \ifnumequal{\value{rolldice}}{1}{
    % variables 
    \renewcommand{\vbone}{8}
    \renewcommand{\vbtwo}{2}
    \renewcommand{\vbthree}{9}
    \renewcommand{\vbfour}{7}
    \renewcommand{\vbfive}{70}
  }{
    \ifnumequal{\value{rolldice}}{2}{
      % variables 
      \renewcommand{\vbone}{12}
      \renewcommand{\vbtwo}{3}
      \renewcommand{\vbthree}{91}
      \renewcommand{\vbfour}{55}
      \renewcommand{\vbfive}{34,650}
    }{
      % variables 
      \renewcommand{\vbone}{10}
      \renewcommand{\vbtwo}{2}
      \renewcommand{\vbthree}{11}
      \renewcommand{\vbfour}{9}
      \renewcommand{\vbfive}{252}
    }
  }
}

\gcalcexpr[0]{\tmp}{\vbone - \vbtwo}
\gcalcexpr[0]{\per}{\vbone / \vbtwo}
\gcalcexpr[0]{\tp}{\vbone - \per}
\gcalcexpr[0]{\tq}{\tp - \per}

\question A man wants to invite $\vbone$ friends for a get-together at his place and he has 
$\vbtwo$ messengers he can use to deliver the invitations. The invitations are \textit{not}
personalized - that is, any one of the invited people can receive any of the invitation cards

\watchout

\ifprintanswers
 
\fi 

\begin{parts} 
  \part[2] In how many ways can the invitations be delivered if its ok to \textit{not} use one or more of the messengers? 

\begin{solution}[\mcq]
     As the invitations are \textit{not} personalized, the problem is the same as the one about the numbers of ways 
     in which $N$ items can be distributed amongst $M$ individuals 
     \begin{align}
     	N &= \eDistNAmongstM{N}{M} \\
        \text{ which in this case} &= \eDistNAmongstM{\vbone}{\vbtwo} = \vbthree
     \end{align}
  \end{solution}

  \part[2] In how many ways can the invitations be delivered if \textit{all} messengers must be used - that is, each messenger
  must carry \textit{atleast} one invitation?

\begin{solution}[\mcq]
    First, give one invitation to each of the messengers so that they are all used. Now, its a question of 
    distributing the remaining $\vbone - \vbtwo = \tmp$ messages amongst the $\vbtwo$ messengers
    \begin{align}
     	N &= \eDistNAmongstM{\tmp}{\vbtwo} = \vbfour
    \end{align}
  \end{solution}

  \part[2] Now assume that the invitations are personalized - that is, an invitation for $A$ cannot be given to $B$. 
  In how many ways can the invitations be delivered if each messenger delivers an \textit{equal} number of 
  invitations?

\begin{solution}[\mcq]
  	Each messenger has to deliver $\per$ invitations. Let the first messenger therefore pick $\per$ from $\vbone$, 
  	the next $\per$ from the remaining and so on \ldots 
  	\begin{align}
  		\ifnumequal\vbtwo{4}{ % # messengers = 4
  			N &= \encr\vbone\per \cdot \encr\tp\per \cdot \encr\tq\per \cdot \encr\per\per = \vbfive
  		}{
  			\ifnumequal\vbtwo{3} { % # messengers = 3
  				N &= \encr\vbone\per \cdot \encr\tp\per \cdot \encr\per\per = \vbfive
  			}{ % number of messengers = 2
  			  N &= \encr\vbone\per \cdot \encr\tp\per = \vbfive
  			}
  		}
  	\end{align}
  \end{solution}

\end{parts}
