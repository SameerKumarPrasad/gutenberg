
\ifnumequal{\value{rolldice}}{0}{
  % variables 
  \renewcommand{\vbone}{5}
  \renewcommand{\vbtwo}{3}
  \renewcommand{\vbthree}{7}
  \renewcommand{\vbfour}{1}
}{
  \ifnumequal{\value{rolldice}}{1}{
    % variables 
    \renewcommand{\vbone}{4}
    \renewcommand{\vbtwo}{1}
    \renewcommand{\vbthree}{2}
    \renewcommand{\vbfour}{6}
  }{
    \ifnumequal{\value{rolldice}}{2}{
      % variables 
      \renewcommand{\vbone}{6}
      \renewcommand{\vbtwo}{7}
      \renewcommand{\vbthree}{3}
      \renewcommand{\vbfour}{4}
    }{
      % variables 
      \renewcommand{\vbone}{9}
      \renewcommand{\vbtwo}{7}
      \renewcommand{\vbthree}{2}
      \renewcommand{\vbfour}{5}
    }
  }
}

\question[2] In a certain city, all telephone numbers have six digits - with the first two digits
always being either $\vbone\vbtwo$ or $\vbone\vbthree$ or $\vbone\vbfour$ or $\vbfour\vbthree$ or 
$\vbfour\vbone$. Given this, how many telephone numbers have all six digits distinct?


\watchout[-20pt]

\ifprintanswers
\fi 

\begin{solution}[\mcq]
	There are 5 combinations for the first two numbers - each using two distinct digits. Hence, 
	for the whole telephone number to have all digits different, the next \textit{four} digits 
	must be anything other than the first two
	\begin{align}
		\Rightarrow N_{\texttt{total}} &= 5 \times \underbrace{(8 \times 7 \times 6 \times 5 )}_{\texttt{\# distinct combinations}}
		 = 8,400
	\end{align}
\end{solution}
