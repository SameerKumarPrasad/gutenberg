% This is an empty shell file placed for you by the 'examiner' script.
% You can now fill in the TeX for your question here.

% Now, down to brasstacks. ** Writing good solutions is an Art **. 
% Eventually, you will find your own style. But here are some thoughts 
% to get you started: 
%
%   1. Write to be understood - but be crisp. Your own solution should not take 
%      more space than you will give to the student. Hence, if you take more than 
%      a half-page to write a solution, then give the student a full-page and so on...
%
%   2. Use margin-notes to "talk" to students about the critical insights
%      in the question. The tone can be - in fact, should be - informal
%
%   3. Don't shy away from creating margin-figures you think will help
%      students understand. Yes, it is a little more work per question. 
%      But the question & solution will be written only once. Make that
%      attempt at writing a solution count.
%      
%      3b. Use bc_to_fig.tex. Its an easier way to generate plots & graphs 
% 
%   4. Ensure that there are *no spelling mistakes anywhere*. We are an 
%      education company. Bad spellings suggest that we ourselves 
%      don't have any education. Also, use American spellings by default
% 
%   5. If a question has multiple parts, then first delete lines 40-41
%   6. If a question does not have parts, then first delete lines 43-69
%   
%   7. Create versions of the question when possible. Use commands defined in 
%      tufte-tweaks.sty to do so. Its easier than you think

% \noprintanswers
% \setcounter{rolldice}{1}

\ifnumequal{\value{rolldice}}{0}{
  % variables 
  \renewcommand{\vbone}{9}
}{
  \ifnumequal{\value{rolldice}}{1}{
    % variables 
    \renewcommand{\vbone}{8}
  }{
    \ifnumequal{\value{rolldice}}{2}{
      % variables 
      \renewcommand{\vbone}{10}
    }{
      % variables 
      \renewcommand{\vbone}{11}
    }
  }
}

\gcalcexpr[0]{\vbtwo}{((\vbone + 2) * (\vbone + 1)) / 2}
\gcalcexpr[0]{\vbthree}{\vbone - 2}
\gcalcexpr[0]{\vbfour}{3 * (\vbone - 1)}
\gcalcexpr[0]{\vbfive}{\vbfour + 3}

\question $\vbone$ oranges have to be distributed amongst 3 children. If a child gets an orange, then 
she gets a whole orange - not half or a third or a quarter


\watchout[-30pt]

\ifprintanswers
\fi 

\begin{parts}
  \part[2] In how many ways then can the $\vbone$ oranges be distributed?  

  \insertQR{QRC}
\begin{solution}[\mcq]
  	When its a question of distributing $N$ things amongst $M$ people, imagine the $N$ things lined
  	up with $M-1$ partitions between them. The first person gets whatever is on the left of the 
  	first partition, the second what is between the first and the second partition and so on. The number 
  	of distinct distributions then is the number of distinct permutations of $N+M-1$ items where $N$
  	are of one kind and $M-1$ of the other
  	\begin{align}
  		N_{\texttt{ways}} &= \dfrac{(\vbone + 3 - 1)\,!}{\vbone\,!\cdot (3-1)\,!} = \vbtwo
  	\end{align}
  \end{solution}

  \part[2] How many of these ways are grossly unfair - meaning, one or more children do not get any oranges?

  \insertQR{QRC}
\begin{solution}[\mcq]
  	If two of the children do not get anything, then the third child must get everything. As there are only
  	3 children, there can be only 3 ways in which this can happen
  	
  	However, if only one of the three is left out, then the $\vbone$ oranges are distributed amongst the other two. 
  	Moreover, each of the other two gets \textit{atleast} one orange. So, it comes down to distributing 
  	$\vbthree$ oranges amonst two children
  	
  	\begin{align}
  		N_{\texttt{unfair}} &= 3\times\dfrac{(\vbthree + 2 - 1)\,!}{\vbthree\,!\cdot(2-1)\,!} = \vbfour
  	\end{align}
  	Hence, there are $\vbfour + 3 = \vbfive$ grossly unfair ways of distributing the $\vbone$ oranges
  \end{solution}

\end{parts}
