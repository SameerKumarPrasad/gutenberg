
\ifnumequal{\value{rolldice}}{0}{
  % variables 
  \renewcommand{\va}{4}
  \renewcommand{\vb}{7}
  \renewcommand{\vc}{2}
  \renewcommand{\vd}{5}
  \renewcommand{\vg}{10}
}{
  \ifnumequal{\value{rolldice}}{1}{
    % variables 
    \renewcommand{\va}{5}
    \renewcommand{\vb}{8}
    \renewcommand{\vc}{3}
    \renewcommand{\vd}{5}
    \renewcommand{\vg}{20}
  }{
    \ifnumequal{\value{rolldice}}{2}{
      % variables 
      \renewcommand{\va}{5}
      \renewcommand{\vb}{9}
      \renewcommand{\vc}{3}
      \renewcommand{\vd}{7}
      \renewcommand{\vg}{35}
    }{
      % variables 
      \renewcommand{\va}{4}
      \renewcommand{\vb}{5}
      \renewcommand{\vc}{1}
      \renewcommand{\vd}{2}
      \renewcommand{\vg}{3}
    }
  }
}

\gcalcexpr[0]{\ve}{\vb - 2}
\gcalcexpr[0]{\vf}{\va - 2}

\question[1] In an examination, a student has to answer $\va$ out of $\vb$ questions. Questions $\vc$ 
and $\vd$, however, are compulsory. In how many ways then can the student make the choice of $\va$ questions 
to attempt?

\watchout[-20pt]

\begin{solution}[\mcq]
	With two out of $\vb$ questions fixed, the student has to choose another $\vf$ questions 
	from the remaining $\ve$. Hence
	\begin{align}
		N_{\texttt{total}} &= \encr\ve\vf = \fncr\ve\vf = \vg
	\end{align}
\end{solution}

\ifprintanswers
  \begin{codex}
    $\vg$
  \end{codex}
\fi 

