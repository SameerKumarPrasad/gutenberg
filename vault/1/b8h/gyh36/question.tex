


\ifnumequal{\value{rolldice}}{0}{
  % variables 
  \renewcommand{\vbone}{4}
  \renewcommand{\vbtwo}{7}
  \renewcommand{\vbthree}{2}
  \renewcommand{\vbfour}{5}
  \renewcommand{\vbseven}{10}
}{
  \ifnumequal{\value{rolldice}}{1}{
    % variables 
    \renewcommand{\vbone}{5}
    \renewcommand{\vbtwo}{8}
    \renewcommand{\vbthree}{3}
    \renewcommand{\vbfour}{5}
    \renewcommand{\vbseven}{20}
  }{
    \ifnumequal{\value{rolldice}}{2}{
      % variables 
      \renewcommand{\vbone}{5}
      \renewcommand{\vbtwo}{9}
      \renewcommand{\vbthree}{3}
      \renewcommand{\vbfour}{7}
      \renewcommand{\vbseven}{35}
    }{
      % variables 
      \renewcommand{\vbone}{4}
      \renewcommand{\vbtwo}{5}
      \renewcommand{\vbthree}{1}
      \renewcommand{\vbfour}{2}
      \renewcommand{\vbseven}{3}
    }
  }
}

\gcalcexpr[0]{\vbfive}{\vbtwo - 2}
\gcalcexpr[0]{\vbsix}{\vbone - 2}

\question[1] In an examination, a student has to answer $\vbone$ out of $\vbtwo$ questions. Questions $\vbthree$ 
and $\vbfour$, however, are compulsory. In how many ways then can the student make the choice of $\vbone$ questions 
to attempt?


\watchout[-20pt]

\ifprintanswers
\fi 

\begin{solution}[\mcq]
	With two out of $\vbtwo$ questions fixed, the student has to choose another $\vbsix$ questions 
	from the remaining $\vbfive$. Hence
	\begin{align}
		N_{\texttt{total}} &= \encr\vbfive\vbsix = \fncr\vbfive\vbsix = \vbseven
	\end{align}
\end{solution}
