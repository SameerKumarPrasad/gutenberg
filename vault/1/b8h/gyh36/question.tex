% This is an empty shell file placed for you by the 'examiner' script.
% You can now fill in the TeX for your question here.

% Now, down to brasstacks. ** Writing good solutions is an Art **. 
% Eventually, you will find your own style. But here are some thoughts 
% to get you started: 
%
%   1. Write to be understood - but be crisp. Your own solution should not take 
%      more space than you will give to the student. Hence, if you take more than 
%      a half-page to write a solution, then give the student a full-page and so on...
%
%   2. Use margin-notes to "talk" to students about the critical insights
%      in the question. The tone can be - in fact, should be - informal
%
%   3. Don't shy away from creating margin-figures you think will help
%      students understand. Yes, it is a little more work per question. 
%      But the question & solution will be written only once. Make that
%      attempt at writing a solution count.
%      
%      3b. Use bc_to_fig.tex. Its an easier way to generate plots & graphs 
% 
%   4. Ensure that there are *no spelling mistakes anywhere*. We are an 
%      education company. Bad spellings suggest that we ourselves 
%      don't have any education. Also, use American spellings by default
% 
%   5. If a question has multiple parts, then first delete lines 40-41
%   6. If a question does not have parts, then first delete lines 43-69
%   
%   7. Create versions of the question when possible. Use commands defined in 
%      tufte-tweaks.sty to do so. Its easier than you think

% \noprintanswers
% \setcounter{rolldice}{2}

\ifnumequal{\value{rolldice}}{0}{
  % variables 
  \renewcommand{\vbone}{4}
  \renewcommand{\vbtwo}{7}
  \renewcommand{\vbthree}{2}
  \renewcommand{\vbfour}{5}
  \renewcommand{\vbfive}{}
  \renewcommand{\vbsix}{}
  \renewcommand{\vbseven}{10}
  \renewcommand{\vbeight}{}
  \renewcommand{\vbnine}{}
  \renewcommand{\vbten}{}
}{
  \ifnumequal{\value{rolldice}}{1}{
    % variables 
    \renewcommand{\vbone}{5}
    \renewcommand{\vbtwo}{8}
    \renewcommand{\vbthree}{3}
    \renewcommand{\vbfour}{5}
    \renewcommand{\vbfive}{}
    \renewcommand{\vbsix}{}
    \renewcommand{\vbseven}{20}
    \renewcommand{\vbeight}{}
    \renewcommand{\vbnine}{}
    \renewcommand{\vbten}{}
  }{
    \ifnumequal{\value{rolldice}}{2}{
      % variables 
      \renewcommand{\vbone}{5}
      \renewcommand{\vbtwo}{9}
      \renewcommand{\vbthree}{3}
      \renewcommand{\vbfour}{7}
      \renewcommand{\vbfive}{}
      \renewcommand{\vbsix}{}
      \renewcommand{\vbseven}{35}
      \renewcommand{\vbeight}{}
      \renewcommand{\vbnine}{}
      \renewcommand{\vbten}{}
    }{
      % variables 
      \renewcommand{\vbone}{4}
      \renewcommand{\vbtwo}{5}
      \renewcommand{\vbthree}{1}
      \renewcommand{\vbfour}{2}
      \renewcommand{\vbfive}{}
      \renewcommand{\vbsix}{}
      \renewcommand{\vbseven}{3}
      \renewcommand{\vbeight}{}
      \renewcommand{\vbnine}{}
      \renewcommand{\vbten}{}
    }
  }
}

\gcalcexpr[0]{\vbfive}{\vbtwo - 2}
\gcalcexpr[0]{\vbsix}{\vbone - 2}

\question[1] In an examination, a student has to answer $\vbone$ out of $\vbtwo$ questions. Questions $\vbthree$ 
and $\vbfour$, however, are compulsory. In how many ways then can the student make the choice of $\vbone$ questions 
to attempt?

\insertQR[-10pt]{QRC}

\watchout[-20pt]

\ifprintanswers
\fi 

\begin{solution}[\mcq]
	With two out of $\vbtwo$ questions fixed, the student has to choose another $\vbsix$ questions 
	from the remaining $\vbfive$. Hence
	\begin{align}
		N_{\texttt{total}} &= \encr\vbfive\vbsix = \fncr\vbfive\vbsix = \vbseven
	\end{align}
\end{solution}
