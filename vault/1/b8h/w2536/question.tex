% This is an empty shell file placed for you by the 'examiner' script.
% You can now fill in the TeX for your question here.

% Now, down to brasstacks. ** Writing good solutions is an Art **. 
% Eventually, you will find your own style. But here are some thoughts 
% to get you started: 
%
%   1. Write to be understood - but be crisp. Your own solution should not take 
%      more space than you will give to the student. Hence, if you take more than 
%      a half-page to write a solution, then give the student a full-page and so on...
%
%   2. Use margin-notes to "talk" to students about the critical insights
%      in the question. The tone can be - in fact, should be - informal
%
%   3. Don't shy away from creating margin-figures you think will help
%      students understand. Yes, it is a little more work per question. 
%      But the question & solution will be written only once. Make that
%      attempt at writing a solution count.
%      
%      3b. Use bc_to_fig.tex. Its an easier way to generate plots & graphs 
% 
%   4. Ensure that there are *no spelling mistakes anywhere*. We are an 
%      education company. Bad spellings suggest that we ourselves 
%      don't have any education. Also, use American spellings by default
% 
%   5. If a question has multiple parts, then first delete lines 40-41
%   6. If a question does not have parts, then first delete lines 43-69
%   
%   7. Create versions of the question when possible. Use commands defined in 
%      tufte-tweaks.sty to do so. Its easier than you think

%\noprintanswers
%\setcounter{rolldice}{3}

\ifnumequal{\value{rolldice}}{0}{
  % variables 
  \renewcommand{\vbone}{7} % attempt
  \renewcommand{\vbtwo}{12} % total
  \renewcommand{\vbthree}{6} % per group
  \renewcommand{\vbfour}{5} % atmost 
  \renewcommand{\vbfive}{}
  \renewcommand{\vbsix}{}
  \renewcommand{\vbseven}{}
  \renewcommand{\vbeight}{780} % n-ways 
  \renewcommand{\vbnine}{}
  \renewcommand{\vbten}{}
}{
  \ifnumequal{\value{rolldice}}{1}{
    % variables 
    \renewcommand{\vbone}{5}
    \renewcommand{\vbtwo}{10}
    \renewcommand{\vbthree}{5}
    \renewcommand{\vbfour}{4}
    \renewcommand{\vbfive}{}
    \renewcommand{\vbsix}{}
    \renewcommand{\vbseven}{}
    \renewcommand{\vbeight}{250}
    \renewcommand{\vbnine}{}
    \renewcommand{\vbten}{}
  }{
    \ifnumequal{\value{rolldice}}{2}{
      % variables 
      \renewcommand{\vbone}{9}
      \renewcommand{\vbtwo}{14}
      \renewcommand{\vbthree}{7}
      \renewcommand{\vbfour}{6}
      \renewcommand{\vbfive}{}
      \renewcommand{\vbsix}{}
      \renewcommand{\vbseven}{}
      \renewcommand{\vbeight}{1,960}
      \renewcommand{\vbnine}{}
      \renewcommand{\vbten}{}
    }{
      % variables 
      \renewcommand{\vbone}{7}
      \renewcommand{\vbtwo}{14}
      \renewcommand{\vbthree}{7}
      \renewcommand{\vbfour}{5}
      \renewcommand{\vbfive}{}
      \renewcommand{\vbsix}{}
      \renewcommand{\vbseven}{}
      \renewcommand{\vbeight}{3,332}
      \renewcommand{\vbnine}{}
      \renewcommand{\vbten}{}
    }
  }
}

\gcalcexpr[0]{\vbfive}{\vbone - \vbfour} % min 
\gcalcexpr[0]{\vbsix}{\vbfour - 1 }
\gcalcexpr[0]{\vbseven}{\vbone - \vbsix}

\question[2] In an examination, a candidate is required to answer $\vbone$ out of $\vbtwo$ questions. The
questions are divided into two sections - each section having $\vbthree$ questions. However, the candidate can attempt
\textit{atmost} $\vbfour$ questions from each section. In how many different ways can the candidate attempt the $\vbone$
questions ?

\insertQR[10pt]{QRC}

\watchout[-40pt]

\ifprintanswers
	\begin{table}
		\begin{tabular}{cccccccc}
		   \toprule
		   	  $A_1$ & $B_1$ & $A_2$ & $B_2$ & $A_3$ & $B_3$ & $A_4$ & $B_4$ \\
		   \midrule
		      \vbfour & \vbfive & \vbsix & \vbseven & \vbseven & \vbsix & \vbfive & \vbfour \\
		   \bottomrule
		\end{tabular}
	\end{table}
\fi 

\begin{solution}[\halfpage]
	If the two sections be A and B, then there are \textit{four} possible ways in which 
	the candidate could pick $\vbone$ questions given the constraints - as shown in the table above
	
	The total number of ways, therefore, are
	\begin{align}
		N_{\texttt{total}} &= \encr\vbthree\vbfour\cdot\encr\vbthree\vbfive + 
		     \encr\vbthree\vbsix\cdot\encr\vbthree\vbseven + \encr\vbthree\vbseven\cdot\encr\vbthree\vbsix + 
		     \encr\vbthree\vbfive\cdot\encr\vbthree\vbfour \\
		&= 2\cdot\left( \encr\vbthree\vbfour\cdot\encr\vbthree\vbfive + \encr\vbthree\vbsix\cdot\encr\vbthree\vbseven \right) \\
		&= \vbeight
	\end{align}
\end{solution}
