% This is an empty shell file placed for you by the 'examiner' script.
% You can now fill in the TeX for your question here.

% Now, down to brasstacks. ** Writing good solutions is an Art **. 
% Eventually, you will find your own style. But here are some thoughts 
% to get you started: 
%
%   1. Write to be understood - but be crisp. Your own solution should not take 
%      more space than you will give to the student. Hence, if you take more than 
%      a half-page to write a solution, then give the student a full-page and so on...
%
%   2. Use margin-notes to "talk" to students about the critical insights
%      in the question. The tone can be - in fact, should be - informal
%
%   3. Don't shy away from creating margin-figures you think will help
%      students understand. Yes, it is a little more work per question. 
%      But the question & solution will be written only once. Make that
%      attempt at writing a solution count.
%      
%      3b. Use bc_to_fig.tex. Its an easier way to generate plots & graphs 
% 
%   4. Ensure that there are *no spelling mistakes anywhere*. We are an 
%      education company. Bad spellings suggest that we ourselves 
%      don't have any education. Also, use American spellings by default
% 
%   5. If a question has multiple parts, then first delete lines 40-41
%   6. If a question does not have parts, then first delete lines 43-69
%   
%   7. Create versions of the question when possible. Use commands defined in 
%      tufte-tweaks.sty to do so. Its easier than you think

%\noprintanswers
%\setcounter{rolldice}{3}
%\printrubric

\ifnumequal{\value{rolldice}}{0}{
  % variables 
  \renewcommand{\vbone}{1}
  \renewcommand{\vbtwo}{2}
  \renewcommand{\vbthree}{4}
  \renewcommand{\vbfour}{6}
}{
  \ifnumequal{\value{rolldice}}{1}{
    % variables 
    \renewcommand{\vbone}{3}
    \renewcommand{\vbtwo}{7}
    \renewcommand{\vbthree}{2}
    \renewcommand{\vbfour}{4}
  }{
    \ifnumequal{\value{rolldice}}{2}{
      % variables 
      \renewcommand{\vbone}{2}
      \renewcommand{\vbtwo}{6}
      \renewcommand{\vbthree}{5}
      \renewcommand{\vbfour}{9}
    }{
      % variables 
      \renewcommand{\vbone}{5}
      \renewcommand{\vbtwo}{4}
      \renewcommand{\vbthree}{3}
      \renewcommand{\vbfour}{7}
    }
  }
}

\SQUARE\vbtwo\tp
\gcalcexpr[0]\tq{\vbfour * \tp}
\ADD\tq\vbthree\tr
\MULTIPLY\vbtwo{2}\ts

\question[3] Given two functions - $f(x)$ and $g(x)$ - find $f'(x)$ given that 
$g(\vbone) = \vbtwo$, $\left.\dfrac{d}{dx} f(x)g(x)\right\vert_{x=\vbone} = \vbthree$ and 
$\dfrac{d}{dx}\left.\dfrac{f(x)}{g(x)}\right\vert_{x=\vbone} = \vbfour$

\insertQR[-10pt]{QRC}

\watchout

\ifprintanswers
\fi 

\begin{solution}[\halfpage]
	We know that, 
	\begin{align}
		\text{(Product Rule) }\dfrac{d}{dx} f\cdot g &= f'g + g'f \\
		\text{(Quotient Rule) }\dfrac{d}{dx}\dfrac{f}{g} &= \dfrac{gf' - g'f}{g^2}
	\end{align}
	
	And therefore, when $x=\vbone$, the following would hold 
	\begin{align}
		\vbtwo\cdot f' + g'f &= \vbthree \Rightarrow g'f = \vbthree - \vbtwo\cdot f' \\
		\dfrac{\vbtwo\cdot f' - g'f }{\vbtwo^2} &= \vbfour \Rightarrow g'f = \vbtwo\cdot f' - \tq \\
		\Rightarrow \vbthree - \vbtwo\cdot f' &= \vbtwo\cdot f' - \tq \\
		\text{ or } f'(x) &= \WRITEFRAC\tr\ts
	\end{align}
\end{solution}

\ifprintrubric
  \begin{table}
  	\begin{tabular}{ p{5cm}p{5cm} }
  		\toprule % in brief (4-6 words), what should a grader be looking for for insights & formulations
  		  \sc{\textcolor{blue}{Insight}} & \sc{\textcolor{blue}{Formulation}} \\ 
  		\midrule % ***** Insights & formulations ******
  			& Identified and applied product \& quotient rules \\
  		\toprule % final numerical answers for the various versions
        \sc{\textcolor{blue}{If question has $\ldots$}} & \sc{\textcolor{blue}{Final answer}} \\
  		\midrule % ***** Numerical answers (below) **********
  			$g(1) = 2$ & $f'(x) = 7$ \\ 
  			$g(3) = 7$ & $f'(x) = \frac{99}{7}$ \\ 
  			$g(2) = 6$ & $f'(x) = \frac{329}{12}$ \\ 
  			$g(5) = 4$ & $f'(x) = \frac{115}{8}$ \\ 
  		\bottomrule
  	\end{tabular}
  \end{table}
\fi
