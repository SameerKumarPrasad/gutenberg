


\ifnumequal{\value{rolldice}}{0}{
  % variables 
  \renewcommand{\vbone}{\frac{1}{\sqrt{3}}} % tan \vbfour 
  \renewcommand{\vbtwo}{\sqrt{3}} % 1/ \vbone 
  \renewcommand{\vbthree}{8\sqrt{3}}
  \renewcommand{\vbfour}{30} % angle of \perp with x-axis
  \renewcommand{\vbfive}{4} %m 
  \renewcommand{\vbsix}{4\sqrt{3}} %n 
}{
  \ifnumequal{\value{rolldice}}{1}{
    % variables 
    \renewcommand{\vbone}{\sqrt{3}}
    \renewcommand{\vbtwo}{\dfrac{1}{\sqrt{3}}}
    \renewcommand{\vbthree}{\dfrac{8}{\sqrt{3}}}
    \renewcommand{\vbfour}{60}
    \renewcommand{\vbfive}{4}
    \renewcommand{\vbsix}{\frac{4}{\sqrt{3}}}
  }{
    \ifnumequal{\value{rolldice}}{2}{
      % variables 
      \renewcommand{\vbone}{\frac{1}{\sqrt{3}}}
      \renewcommand{\vbtwo}{\sqrt{3}}
      \renewcommand{\vbthree}{32\sqrt{3}}
      \renewcommand{\vbfour}{30} 
      \renewcommand{\vbfive}{8}
      \renewcommand{\vbsix}{8\sqrt{3}}
    }{
      % variables 
      \renewcommand{\vbone}{\sqrt{3}}
      \renewcommand{\vbtwo}{\dfrac{1}{\sqrt{3}}}
      \renewcommand{\vbthree}{\dfrac{32}{\sqrt{3}}}
      \renewcommand{\vbfour}{60}
      \renewcommand{\vbfive}{8}
      \renewcommand{\vbsix}{\frac{8}{\sqrt{3}}}
    }
  }
}

\question[4] If the perpendicular from the origin to a line $L_1$ makes an angle of $\ang{\vbfour}$ 
with the $x$-axis and the triangle formed by $L_1$ and the axes has an area of $\vbthree$ units,
then find the equation of $L_1$


\watchout

\ifprintanswers
	\begin{marginfigure}
		% 1. Definition of characteristic points
\figinit{pt}
\def\Xmin{-13.33333}
\def\Ymin{-13.33333}
\def\Xmax{66.66666}
\def\Ymax{66.66666}
\def\Xori{13.33333}
\def\Yori{13.33333}
\figpt0:(\Xori,\Yori)
\figpt 100: $M$(66.2,13.8)
\figpt 101: $N$(13.5,66)
\figpt 102: $L_1$(80,0)
\figptorthoprojline 200 :$K$= 0 /100, 101/
\figdrawbegin{}
\def\Xmaxx{\Xmax} % To customize the position
\def\Ymaxx{\Ymax} % of the arrow-heads of the axes.
\figdrawline [0,200]
\figset arrowhead(length=4, fillmode=yes) % styling the arrowheads
\figdrawaxes 0(\Xmin, \Xmaxx, \Ymin, \Ymaxx)
\figdrawlineC(
0 79.99999, % y = 3.75
2.75862 77.24137, % y = 3.59
5.51724 74.48275, % y = 3.43
8.27586 71.72413, % y = 3.28
11.03448 68.96551, % y = 3.12
13.79310 66.20689, % y = 2.97
16.55172 63.44827, % y = 2.81
19.31034 60.68965, % y = 2.66
22.06896 57.93103, % y = 2.50
24.82758 55.17241, % y = 2.35
27.58620 52.41379, % y = 2.19
30.34482 49.65517, % y = 2.04
33.10344 46.89655, % y = 1.88
35.86206 44.13793, % y = 1.73
38.62068 41.37931, % y = 1.57
41.37931 38.62068, % y = 1.42
44.13793 35.86206, % y = 1.26
46.89655 33.10344, % y = 1.11
49.65517 30.34482, % y = .95
52.41379 27.58620, % y = .80
55.17241 24.82758, % y = .64
57.93103 22.06896, % y = .49
60.68965 19.31034, % y = .33
63.44827 16.55172, % y = .18
66.20689 13.79310, % y = .02
68.96551 11.03448, % y = -.12
71.72413 8.27586, % y = -.28
74.48275 5.51724, % y = -.43
77.24137 2.75862, % y = -.59
79.99999 0
)
\figdrawend
\figvisu{\figBoxA}{}{%
\figptsaxes 1:0(\Xmin, \Xmaxx, \Ymin, \Ymaxx)
\figwritee 1:(5pt)     \figwriten 2:(5pt)
\figptsaxes 1:0(\Xmin, \Xmax, \Ymin, \Ymax)
\figwrites 102:(1)
\figset write(mark=$\bullet$)
\figwritesw 100:(2)
\figwritene 101:(2)
\figwritesw 0:$O$(2)
\figwritene 200:(2)
}
\centerline{\box\figBoxA}

	\end{marginfigure}
\fi 

\begin{solution}[\halfpage]
	The situation is \asif. $OK \perp L_1 \Rightarrow$ slope of $L_1 = -\dfrac{1}{\tan\ang{\vbfour}} = -\vbtwo$
	
	Now, if $M = (m,0)$ and $N = (0,n)$, then looking at $L_1$
	\begin{align}
		\dfrac{n-0}{0-m} &= -\vbtwo \Rightarrow n = \vbtwo m
	\end{align}
	Moreover, the area of $\triangle OMN$ is 
	\begin{align}
		\dfrac{1}{2}m\cdot n &= \dfrac{1}{2}\vbtwo m\cdot m = \vbthree \\
		\Rightarrow m &= \pm\vbfive \text{ and } \therefore n = \pm\vbsix
	\end{align}
	There would be \textit{two} lines that meet our criterion. And their equations would be 
	\begin{align}
		\dfrac{x}{\vbfive} + \dfrac{y}{\vbsix} &= 1 \\
		\text{ and } -\dfrac{x}{\vbfive} - \dfrac{y}{\vbsix} &= 1
    \end{align}
\end{solution}

