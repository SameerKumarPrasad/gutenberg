


\ifnumequal{\value{rolldice}}{0}{
  % variables 
  \renewcommand{\va}{1}
  \renewcommand{\vb}{1}
}{
  \ifnumequal{\value{rolldice}}{1}{
    % variables 
    \renewcommand{\va}{4}
    \renewcommand{\vb}{3}
  }{
    \ifnumequal{\value{rolldice}}{2}{
      % variables 
      \renewcommand{\va}{9}
      \renewcommand{\vb}{5}
    }{
      % variables 
      \renewcommand{\va}{1}
      \renewcommand{\vb}{9}
    }
  }
}

\MULTIPLY{2}{\vb}{\m}
\SQUARE{\vb}{\n}
\EXPR[0]\p{-(\va - \n)}

\question[4] Suppose $f(x) = \sqrt{x^2 - \va} = (g \circ h)(x)$. If $h(x) = (x + \vb)$, 
then find $g(x)$. The notation $(g\circ h)(x)$ means $g(h(x))$


\watchout

\ifprintanswers
\fi 

\begin{solution}[\halfpage]
	The question boils down to finding $g(x)$ so that $g(x+\vb) = \sqrt{x^2-\va}$. And the key 
	is to understand that $g(x)$ has to bring to the table whatever $h(x)$ does \textit{not} - like 
	a square root and $x^2$ term 
	
	Given that $x^2$ is the highest power term of $x$ in $f(x)$, we can safely guess that 
	$g(x)$ is of the form 
	
	\begin{align}
		g(x + \vb) &= \sqrt{A\cdot\underbrace{(x + \vb)^2}_{\texttt{will give } x^2} 
		+ B\cdot(x + \vb) + C}
	\end{align}			
	
	
	And so,
	\begin{align}
		\sqrt{x^2 - \va} &= \sqrt{A\cdot(x + \vb)^2 + B\cdot(x + \vb) + C} \\ 
		 &= \sqrt{A\cdot(x^2 + \m\cdot x + \n) + B\cdot(x + \vb) + C} \\
		 &= \sqrt{A\cdot x^2 + (\m A + B)\cdot x + (\n\cdot A + \vb B + C)}
	\end{align}
	
	Now its a simple question of comparing \textit{coefficients of x} on both sides. 
	Which means that if,
  \[ \underbrace{\sqrt{x^2 - \va}}_{f(x)} = 
		\underbrace{\sqrt{A\cdot x^2 + (\m A + B)\cdot x + (\n\cdot A + \vb B + C)}}_{(g\circ h)(x)} \]
  then, 
	\begin{align}
    A &= 1 \\
		\m A + B &= 0 \implies B = -\m \\
		\text{and } \n\cdot A + \vb B + C &= -\va \implies C = \p
	\end{align}
	
	Hence, $g(x + \vb)$ = $\sqrt{(x+\vb)^2 -\m\cdot(x + \vb) + \p}$. Or, generally speaking, 
	\[g(x) = \sqrt{x^2 - \m x + \p} \]
	
\end{solution}

\ifprintanswers\begin{codex}
	$g(x) = \sqrt{x^2 - \m x + \p}$
\end{codex}\fi
