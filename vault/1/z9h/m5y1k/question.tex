% This is an empty shell file placed for you by the 'examiner' script.
% You can now fill in the TeX for your question here.

% Now, down to brasstacks. ** Writing good solutions is an Art **. 
% Eventually, you will find your own style. But here are some thoughts 
% to get you started: 
%
%   1. Write to be understood - but be crisp. Your own solution should not take 
%      more space than you will give to the student. Hence, if you take more than 
%      a half-page to write a solution, then give the student a full-page and so on...
%
%   2. Use margin-notes to "talk" to students about the critical insights
%      in the question. The tone can be - in fact, should be - informal
%
%   3. Don't shy away from creating margin-figures you think will help
%      students understand. Yes, it is a little more work per question. 
%      But the question & solution will be written only once. Make that
%      attempt at writing a solution count.
%      
%      3b. Use bc_to_fig.tex. Its an easier way to generate plots & graphs 
% 
%   4. Ensure that there are *no spelling mistakes anywhere*. We are an 
%      education company. Bad spellings suggest that we ourselves 
%      don't have any education. Also, use American spellings by default
% 
%   5. If a question has multiple parts, then first delete lines 40-41
%   6. If a question does not have parts, then first delete lines 43-69
%   
%   7. Create versions of the question when possible. Use commands defined in 
%      tufte-tweaks.sty to do so. Its easier than you think

%\noprintanswers
%\setcounter{rolldice}{3}
%\printrubric

\ifnumequal{\value{rolldice}}{0}{
  % variables 
  \renewcommand{\vbone}{1}
  \renewcommand{\vbtwo}{1}
}{
  \ifnumequal{\value{rolldice}}{1}{
    % variables 
    \renewcommand{\vbone}{4}
    \renewcommand{\vbtwo}{3}
  }{
    \ifnumequal{\value{rolldice}}{2}{
      % variables 
      \renewcommand{\vbone}{9}
      \renewcommand{\vbtwo}{5}
    }{
      % variables 
      \renewcommand{\vbone}{1}
      \renewcommand{\vbtwo}{9}
    }
  }
}

\MULTIPLY{2}{\vbtwo}{\m}
\SQUARE{\vbtwo}{\n}
\gcalcexpr[0]\p{-(\vbone - \n)}

\question[4] Suppose $f(x) = \sqrt{x^2 - \vbone} = (g \circ h)(x)$. If $h(x) = (x + \vbtwo)$, 
then find $g(x)$. The notation $(g\circ h)(x)$ means $g(h(x))$

\insertQR[-15pt]{QRC}

\watchout

\ifprintanswers
\fi 

\begin{solution}[\halfpage]
	The question boils down to finding $g(x)$ so that $g(x+\vbtwo) = \sqrt{x^2-\vbone}$. And the key 
	is to understand that $g(x)$ has to bring to the table whatever $h(x)$ does \textit{not} - like 
	a square root and $x^2$ term 
	
	Given that $x^2$ is the highest power term of $x$ in $f(x)$, we can safely guess that 
	$g(x)$ is of the form 
	
	\begin{align}
		g(x + \vbtwo) &= \sqrt{A\cdot\underbrace{(x + \vbtwo)^2}_{\texttt{will give } x^2} 
		+ B\cdot(x + \vbtwo) + C}
	\end{align}			
	
	
	And so,
	\begin{align}
		\sqrt{x^2 - \vbone} &= \sqrt{A\cdot(x + \vbtwo)^2 + B\cdot(x + \vbtwo) + C} \\ 
		 &= \sqrt{A\cdot(x^2 + \m\cdot x + \n) + B\cdot(x + \vbtwo) + C} \\
		 &= \sqrt{A\cdot x^2 + (\m A + B)\cdot x + (\n\cdot A + \vbtwo B + C)}
	\end{align}
	
	Now its a simple question of comparing \textit{coefficients of x} on both sides. 
	Which means, if 
	\begin{align}
		\underbrace{\sqrt{x^2 - \vbone}}_{f(x)} &= 
		\underbrace{\sqrt{A\cdot x^2 + (\m A + B)\cdot x + (\n\cdot A + \vbtwo B + C)}}_{(g\circ h)(x)} \\
		\text{then } A &= 1 \\
		\m A + B &= 0 \Rightarrow B = -\m \\
		\text{and } \n\cdot A + \vbtwo B + C &= -\vbone \Rightarrow C = \p
	\end{align}
	
	Hence, $g(x + \vbtwo)$ = $\sqrt{(x+\vbtwo)^2 -\m\cdot(x + \vbtwo) + \p}$. Or, generally speaking, 
	$g(x) = \sqrt{x^2 - \m x + \p}$
	
\end{solution}


\ifprintrubric
  \begin{table}
  	\begin{tabular}{ p{5cm}p{5cm} }
  		\toprule % in brief (4-6 words), what should a grader be looking for for insights & formulations
  		  \sc{\textcolor{blue}{Insight}} & \sc{\textcolor{blue}{Formulation}} \\ 
  		\midrule % ***** Insights & formulations ******
  			$g(x + m) = \sqrt{A\cdot(x + m)^2 + B\cdot(x + m) + C}$ & \\
  		\toprule % final numerical answers for the various versions
        \sc{\textcolor{blue}{If question has $\ldots$}} & \sc{\textcolor{blue}{Final answer}} \\
  		\midrule % ***** Numerical answers (below) **********
  			$h(x) = x + 1$ & $g(x) = \sqrt{x^2-2x} $\\
  			$h(x) = x + 3$ & $g(x) = \sqrt{x^2-6x+5}$\\
  			$h(x) = x + 5$ & $g(x) = \sqrt{x^2-10x+16}$\\
  			$h(x) = x + 9$ & $g(x) = \sqrt{x^2-18x+80}$\\
  		\bottomrule
  	\end{tabular}
  \end{table}
\fi
