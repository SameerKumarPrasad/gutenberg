

\ifnumequal{\value{rolldice}}{0}{
  % variables 
  \renewcommand{\va}{99}
  \renewcommand{\vb}{9}
  \renewcommand{\vc}{1}
  \renewcommand{\vd}{600}
}{
  \ifnumequal{\value{rolldice}}{1}{
    % variables 
    \renewcommand{\va}{95}
    \renewcommand{\vb}{4}
    \renewcommand{\vc}{3}
    \renewcommand{\vd}{250}
  }{
    \ifnumequal{\value{rolldice}}{2}{
      % variables 
      \renewcommand{\va}{97}
      \renewcommand{\vb}{3}
      \renewcommand{\vc}{1}
      \renewcommand{\vd}{750}
    }{
      % variables 
      \renewcommand{\va}{98}
      \renewcommand{\vb}{7}
      \renewcommand{\vc}{3}
      \renewcommand{\vd}{400}
    }
  }
}

\EXPR[0]\p{(10 * \va * \vc)}
\EXPR[0]\q{(\vb * (\vd - \vc))}
\EXPR[0]\r{\p + \q}

\question[5] It is standard practice to take a chest X-ray of a person 
suspected of having tuberculosis (TB). The probability of detecting TB in 
a person actually suffering from it is $\frac{\va}{100}$. However, the 
probability of a healthy person being diagnosed with TB is $\frac{\vb}{1000}$. 
In a certain city, $\vc$ in $\vd$ people suffers from TB. A person is 
selected at random and is diagnosed with the disease. What is the probability
that the person actually has TB?  

\watchout

\begin{solution}[\fullpage]
	Let event $X =$ X-ray says person has TB and event $T=$ person actually has TB. Given these 
	definitions, here is what we know
	\begin{align}
		P(X\vert T) &= \dfrac{\va}{100} \\
		P(T) &= \dfrac{\vc}{\vd} \\
		P(X\vert{T'}) &= \dfrac{\vb}{1000}
	\end{align}
	What we need to find is $P(T\vert X)$. Now, from Baye's theorem, we have 
	\begin{align}
		P(T\vert X) &= \dfrac{P(X\vert T)\cdot P(T)}{P(X\vert\,T')\cdot P(T') + P(X\vert T)\cdot P(T)} \\
		&= \dfrac{\frac{\va}{100}\cdot\frac{\vc}{\vd}}
		   {\frac{\vb}{1000}\cdot\left(1-\frac{\vc}{\vd}\right) + \frac{\va}{100}\cdot\frac{\vc}{\vd}} \\
		&= \WRITEFRAC\p\r
	\end{align}

\end{solution}

\ifprintanswers\begin{codex}$\WRITEFRAC\p\r$\end{codex}\fi
