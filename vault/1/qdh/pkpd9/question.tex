
%\noprintanswers
%\setcounter{rolldice}{2}
%\printrubric

\ifnumequal{\value{rolldice}}{0}{
  % variables 
  \renewcommand{\vbone}{4}
  \renewcommand{\vbtwo}{7}
  \renewcommand{\vbthree}{10}
}{
  \ifnumequal{\value{rolldice}}{1}{
    % variables 
    \renewcommand{\vbone}{6}
    \renewcommand{\vbtwo}{10}
    \renewcommand{\vbthree}{30}
  }{
    \ifnumequal{\value{rolldice}}{2}{
      % variables 
      \renewcommand{\vbone}{6}
      \renewcommand{\vbtwo}{8}
      \renewcommand\vbthree{20}
    }{
      % variables 
      \renewcommand{\vbone}{8}
      \renewcommand{\vbtwo}{12}
      \renewcommand{\vbthree}{25}
    }
  }
}

\FRACTIONSIMPLIFY\vbone{100}\p\q
\FRACTIONSIMPLIFY\vbtwo{100}\r\s
\FRACTIONSIMPLIFY\vbthree{100}\m\n
\FRACMINUS{1}{1}\m\n\a\b

\FRACMULT\p\q\m\n\k\j
\FRACMULT\r\s\a\b\y\z
\FRACDIV\y\z\k\j\tp\tq
\ADD\tp\tq\tr

\question[3] If $\vbone$\% of the people with blood group $O$ are left-handed and $\vbtwo$\% 
of those with other blood groups are left-handed, then what is the probability that a left-handed
person selected at random has blood group $O$ given that overall $\vbthree$\% of people have 
blood group $O$?


\insertQR[5pt]{QRC}

\watchout[-40pt]

\ifprintanswers
\fi 

\begin{solution}[\halfpage]
  If $L =$ event that a person is left-handed and $O=$ event that a person has blood group $O$,
  then we have been told the following
  \begin{align}
    P(L\vert\,O) &= \WRITEFRAC\p\q \\
    P(L\vert\,O') &= \WRITEFRAC\r\s \\
    P(O) &= \WRITEFRAC\m\n \Rightarrow P(O') = \WRITEFRAC\a\b
  \end{align}
  What we need is $P(O\vert\,L)$ - which can be found using Baye's theorem
  \begin{align}
    P(O\vert\,L) &= \dfrac{P(L\vert\,O)\cdot P(O)}{P(L\vert\,O)\cdot P(O) + P(L\vert\,O')\cdot P(O')} \\
       &= \dfrac{\frac\p\q\cdot\frac\m\n}{\frac\p\q\cdot\frac\m\n + \frac\r\s\cdot\frac\a\b} \\
       &= \WRITEFRAC\tq\tr 
  \end{align}
\end{solution}


\ifprintrubric
  \begin{table}
  	\begin{tabular}{ p{5cm}p{5cm} }
  		\toprule % in brief (4-6 words), what should a grader be looking for for insights & formulations
  		  \sc{\textcolor{blue}{Insight}} & \sc{\textcolor{blue}{Formulation}} \\ 
  		\midrule % ***** Insights & formulations ******
        Correctly identified $P(L\vert\,O)$, $P(L\vert\, O')$ and $P(O)$ & \\
  		\toprule % final numerical answers for the various versions
        \sc{\textcolor{blue}{If question has $\ldots$}} & \sc{\textcolor{blue}{Final answer}} \\
  		\midrule % ***** Numerical answers (below) **********
        $P(O) = 10\%$ & $\dfrac{4}{67}$ \\
        $P(O) = 30\%$ & $\qquad\dfrac{9}{44}$ \\
        $P(O) = 20\%$ & $\dfrac{3}{19}$ \\
        $P(O) = 25\%$ & $\qquad\dfrac{2}{11}$ \\
  		\bottomrule
  	\end{tabular}
  \end{table}
\fi
