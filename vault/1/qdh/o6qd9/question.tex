
\ifnumequal{\value{rolldice}}{0}{
  % variables 
  \renewcommand{\va}{2}
  \renewcommand{\vb}{4}
  \renewcommand{\vc}{5}
}{
  \ifnumequal{\value{rolldice}}{1}{
    % variables 
    \renewcommand{\va}{7}
    \renewcommand{\vb}{3}
    \renewcommand{\vc}{8}
  }{
    \ifnumequal{\value{rolldice}}{2}{
      % variables 
      \renewcommand{\va}{3}
      \renewcommand{\vb}{9}
      \renewcommand{\vc}{11}
    }{
      % variables 
      \renewcommand{\va}{4}
      \renewcommand{\vb}{13}
      \renewcommand{\vc}{5}
    }
  }
}

\question[3] If the slope of tangents to a curve $C$ at any \textbf{non-origin} point is given by 
\[\va y + \dfrac{\vb y}{\vc x} \]then find the equation of $C$

\watchout

\begin{solution}[\halfpage]
  In effect, we have been told that for a curve $C: y = f(x)$ 
  \begin{align}
    \dfrac{dy}{dx} &= \va y + \dfrac{\vb y}{\vc x} 
    = y\cdot\left(\va + \dfrac{\vb}{\vc x} \right), (x,y)\neq(0,0) \\
    \implies \dfrac{dy}{y} &= \left(\va + \dfrac{\vb}{\vc x}\right)\cdot dx \\
    \implies\int\dfrac{dy}{y} = \ln y &= \int \left(\va + \dfrac{\vb}{\vc x}\right)\cdot dx =  \va x + \WRITEFRAC\vb\vc\ln x + D \\
    \therefore y &= e^{\va x + \frac\vb\vc\ln x + D} = e^{\va x}\cdot e^{\frac\vb\vc\ln x}\cdot e^{D} \\
    \text{or, } y &= k\cdot e^{\va x}\cdot e^{\frac\vb\vc\ln x}, \text{ where } k = e^D
  \end{align}
  
  Now, $e^{\ln x} = x \implies (e^{\ln x})^{\frac\vb\vc} = x^{\frac\vb\vc}$
  
  And therefore, our final answer is simply
  \begin{align}
    y &= k\cdot e^{\va x}\cdot x^{\frac\vb\vc}
  \end{align}
\end{solution}

\ifprintanswers\begin{codex}
	$y = k\cdot e^{\va x}\cdot x^{\frac\vb\vc}$
\end{codex}\fi
