
\ifnumequal{\value{rolldice}}{0}{
  % variables 
  \renewcommand{\vbone}{2}
  \renewcommand{\vbtwo}{4}
  \renewcommand{\vbthree}{5}
}{
  \ifnumequal{\value{rolldice}}{1}{
    % variables 
    \renewcommand{\vbone}{7}
    \renewcommand{\vbtwo}{3}
    \renewcommand{\vbthree}{8}
  }{
    \ifnumequal{\value{rolldice}}{2}{
      % variables 
      \renewcommand{\vbone}{3}
      \renewcommand{\vbtwo}{9}
      \renewcommand{\vbthree}{11}
    }{
      % variables 
      \renewcommand{\vbone}{4}
      \renewcommand{\vbtwo}{13}
      \renewcommand{\vbthree}{5}
    }
  }
}

\question[3] If the slope of tangents to a curve $C$ at any \textbf{non-origin} point is given by 
$\vbone y + \dfrac{\vbtwo y}{\vbthree x}$ then find the equation of $C$


\watchout

\ifprintanswers
\fi 

\begin{solution}[\halfpage]
  In effect, we have been told that for a curve $C: y = f(x)$ 
  \begin{align}
    \dfrac{dy}{dx} &= \vbone y + \dfrac{\vbtwo y}{\vbthree x} 
    = y\cdot\left(\vbone + \dfrac{\vbtwo}{\vbthree x} \right), (x,y)\neq(0,0) \\
    \Rightarrow \dfrac{dy}{y} &= \left(\vbone + \dfrac{\vbtwo}{\vbthree x}\right)\cdot dx \\
    \Rightarrow\int\dfrac{dy}{y} = \ln y &= \int \left(\vbone + \dfrac{\vbtwo}{\vbthree x}\right)\cdot dx =  \vbone x + \WRITEFRAC\vbtwo\vbthree\ln x + D \\
    \therefore y &= e^{\vbone x + \WRITEFRAC\vbtwo\vbthree\ln x + D} \\
    \text{or, } y &= k\cdot e^{\vbone x}\cdot e^{\WRITEFRAC\vbtwo\vbthree\ln x}, \text{ where } k = e^D
  \end{align}
  
  Now, $e^{\ln x} = x \implies (e^{\ln x})^{\WRITEFRAC\vbtwo\vbthree} = x^{\WRITEFRAC\vbtwo\vbthree}$
  
  And therefore, our final answer is simply
  \begin{align}
    y &= k\cdot e^{\vbone x}\cdot x^{\WRITEFRAC\vbtwo\vbthree}
  \end{align}
\end{solution}

