

\ifnumequal{\value{rolldice}}{0}{
  % variables 
  \renewcommand{\va}{30}
  \renewcommand{\vb}{40}
  \renewcommand{\vc}{3}
  \renewcommand{\vd}{50}
}{
  \ifnumequal{\value{rolldice}}{1}{
    % variables 
    \renewcommand{\va}{60}
    \renewcommand{\vb}{80}
    \renewcommand{\vc}{5}
    \renewcommand{\vd}{100}
  }{
    \ifnumequal{\value{rolldice}}{2}{
      % variables 
      \renewcommand{\va}{80}
      \renewcommand{\vb}{60}
      \renewcommand{\vc}{4}
      \renewcommand{\vd}{100}
    }{
      % variables 
      \renewcommand{\va}{40}
      \renewcommand{\vb}{30}
      \renewcommand{\vc}{10}
      \renewcommand{\vd}{50}
    }
  }
}

\question[4] $A$, $B$ and $C$ are three trains on three separate tracks $T_1, T_2$ and $T_3$ (see figure). 
$T_2\parallel T_3$ and $T_1\perp$ to both $T_2$ and $T_3$. Train $A$ has developed engine 
trouble and is therefore being towed towards $M$ by train $B$ using an iron chain welded to the 
the two trains. But when $B$ is $\va$ meters from $N$ (and $\vb$ meters from $M$), it too
develops engine problems. So now, train $B$ is attached to train $C$ and is pulled towards $N$.
If $C$ is moving at a speed of $\vc$ meters per second and the two chains ($AB$ and $BC$)
are $\vd$ meters long, then how fast is $A$ moving towards $M$?

\watchout[-5cm]

\vspace{1cm}
\figinit{pt}
  \figpt 100:(0,80) % line 1
  \figpt 101:$A$(30,80)
  \figpt 102:$M$(70,80)
  \figpt 103:$T_2$(85,80)
  \figpt 104:$T_1$(70,95) % line 2
  \figpt 105:$B$(70,40)
  \figpt 106:$N$(70,10)
  \figpt 107:(70,0)
  \figpt 108:$T_3$(85,10) % line 3
  \figpt 109:$C$(30,10)
  \figpt 110:(0,10)
\figdrawbegin{}
  \figdrawline [100,101,102,103]
  \figdrawline [104,105,106,107]
  \figdrawline [108,109,110]
  \figdrawline [101,105,109]
\figdrawend
\figvisu{\figBoxA}{}{%
  \figwriten 104:(3)
  \figwritee 103:(3)
  \figwritee 108:(3)
  \figset write(mark=$\bullet$)
  \figwrites 101:(3)
  \figwritese 102:(3)
  \figwritee 105:(3)
  \figwritene 106:(3)
  \figwrites 109:(3)
}
\centerline{\box\figBoxA}

\ADD\va\vb\p

\POWER\vd{2}\l % L^2
\POWER\va{2}\bns
\POWER\vb{2}\bms

\SUBTRACT\l\bns\cn
\SUBTRACT\l\bms\am
\SQRT\cn\q
\SQRT\am\r

\FRACMULT\va\q\r\vb\a\b
\FRACMULT\b\a\vc{1}\e\f

\begin{solution}[\fullpage]
  The facts of the case are as follows
  \begin{align}
    BM +BN = \p&\implies \dfrac{\ud}{\ud t}BM + \dfrac{\ud}{\ud t}BN = 0 \\
      \implies \dfrac{\ud}{\ud t}BM &= -\dfrac{\ud}{\ud t}BN \\
    AM^2 + BM^2 = \vd^2 \implies 2AM\dfrac{\ud}{\ud t}AM &+ 2BM\dfrac{\ud}{\ud t}BM = 0 \\
    \implies \dfrac{\ud}{\ud t}AM &= -\WRITEFRAC[false]\vb\r\times\dfrac{\ud}{\ud t}BM \\
    CN^2 + BN^2 &= \vd^2 \\
    \implies 2CN\dfrac{\ud}{\ud t}CN + 2BN\dfrac{\ud}{\ud t}BN &= 0 \nonumber\\
    \implies \dfrac{\ud}{\ud t}CN &= -\WRITEFRAC\va\q\times\dfrac{\ud}{\ud t}BN \\
      &= \WRITEFRAC\va\q\times\dfrac{\ud}{\ud t}BM \to (2) \nonumber\\
      &= \WRITEFRAC\va\q\times\left(-\WRITEFRAC\r\vb\dfrac{\ud}{\ud t}AM\right)\to (4) \nonumber 
  \end{align}
  Now, 
  \[ \text{A's speed}=\dfrac{\ud}{\ud t} AM\text{ and }\dfrac{\ud}{\ud t} CN = \text{C's speed} = \vc\text{ m/s} \] 

  \begin{align}
    \dfrac{\ud}{\ud t}AM &= -\WRITEFRAC[false]\b\a\dfrac{\ud}{\ud t}CN = -\WRITEFRAC[false]\e\f
  \end{align}
  Hence, $A$ is being pulled at a speed of $\WRITEFRAC[false]\e\f$ meters per second. The negative 
  sign is because $A$ moves in opposite direction to $C$.
\end{solution}

\ifprintanswers\begin{codex}$\WRITEFRAC[false]\e\f\text{ m/s}$\end{codex}\fi
