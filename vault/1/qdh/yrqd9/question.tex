

\ifnumequal{\value{rolldice}}{0}{
  % variables 
  \renewcommand{\va}{2}
  \renewcommand{\vb}{3}
  \renewcommand{\vc}{7}
  \renewcommand{\vd}{3}
}{
  \ifnumequal{\value{rolldice}}{1}{
    % variables 
    \renewcommand{\va}{3}
    \renewcommand{\vb}{4}
    \renewcommand{\vc}{2}
    \renewcommand{\vd}{7}
  }{
    \ifnumequal{\value{rolldice}}{2}{
      % variables 
      \renewcommand{\va}{2}
      \renewcommand{\vb}{5}
      \renewcommand{\vc}{4}
      \renewcommand{\vd}{3}
    }{
      % variables 
      \renewcommand{\va}{3}
      \renewcommand{\vb}{7}
      \renewcommand{\vc}{2}
      \renewcommand{\vd}{9}
    }
  }
}

\EXPR[0]{\p}{(\vb * \vc) + (\vd * \va)}
\MULTIPLY\va\p\q
\MULTIPLY\vc\p\r

\question[3] Evaluate the following \[ \int\dfrac{x\ud x}{(\va x - \vb)\cdot (\vc x + \vd)} \]

\watchout

\begin{solution}[\halfpage]
  \begin{align}
    \dfrac{x}{(\va x - \vb)\cdot (\vc x + \vd)} &= 
    \dfrac{A}{\va x - \vb} + \dfrac{B}{\vc x + \vd} \\
    \implies x &= A\cdot (\vc x + \vd) + B\cdot (\va x - \vb) \\
                &= (\vc A + \va B)x + (\vd A - \vb B)
  \end{align}
  Comparing coefficients on both sides, we get 
  \begin{align}
    \vd A - \vb B &= 0 \implies B = \WRITEFRAC\vd\vb A \\
    \vc A + \WRITEFRAC\vd\vb A\cdot\va &= 1 
    \implies A = \WRITEFRAC\vb\p \text{ and } B = \WRITEFRAC\vd\p
  \end{align}
  And therefore
  \begin{align}
    \int\dfrac{x\ud x}{(\va x - \vb)\cdot (\vc x + \vd)} &= 
    \WRITEFRAC\vb\p\int\dfrac{\ud x}{\va x - \vb} + \WRITEFRAC\vd\p\int\dfrac{\ud x}{\vc x + \vd} \\
    &= \WRITEFRAC\vb\q\ln\vert\va x - \vb\vert + \WRITEFRAC\vd\r\ln\vert\vc x + \vd\vert + C
  \end{align}
\end{solution}

\ifprintanswers\begin{codex}
	$\WRITEFRAC\vb\q\ln\vert\va x - \vb\vert + \WRITEFRAC\vd\r\ln\vert\vc x + \vd\vert + C$
\end{codex}\fi
