
%\noprintanswers
%\setcounter{rolldice}{3}
%\printrubric

\ifnumequal{\value{rolldice}}{0}{
  % variables 
  \renewcommand{\vbone}{23}
  \renewcommand{\vbtwo}{2}
  \renewcommand{\vbthree}{10}
}{
  \ifnumequal{\value{rolldice}}{1}{
    % variables 
    \renewcommand{\vbone}{25}
    \renewcommand{\vbtwo}{1}
    \renewcommand{\vbthree}{25}
  }{
    \ifnumequal{\value{rolldice}}{2}{
      % variables 
      \renewcommand{\vbone}{30}
      \renewcommand{\vbtwo}{2}
      \renewcommand{\vbthree}{12}
    }{
      % variables 
      \renewcommand{\vbone}{27}
      \renewcommand{\vbtwo}{3}
      \renewcommand{\vbthree}{10}
    }
  }
}

\POWER\vbone{2}\p
\POWER\vbtwo{2}\q
\POWER\vbthree{2}\r
\MULTIPLY\vbone\vbtwo\m
\MULTIPLY\q\r\n
\ADD\p\n\j

\question[3] A balloon is released at eye-level $\vbone$ meters before you. You watch it rise, 
tracking it with your eyes. The balloon rises vertically at a rate of $\vbtwo\frac{m}{s}$.
After $\vbthree$ seconds, how fast is your neck tilting back (in radians per second)? Draw
an explanatory diagram for full credit

\insertQR[10pt]{QRC}

\watchout[-40pt]

\ifprintanswers
  % stuff to be shown only in the answer key - like explanatory margin figures
  \begin{marginfigure}
    \figinit{pt}
      \figpt 100:$A$(0,0)
      \figpt 101:$B$(70,0)
      \figpt 102:$C$(70,40)
      \figpt 103:$\theta$(15,5)
    \figdrawbegin{}
      \figdrawline [100,101, 102, 100]
    \figdrawend
    \figvisu{\figBoxA}{}{%
      \figwritesw 100:(2)
      \figwritese 101:(2)
      \figwritee 102:(2)
      \figwritee 103:(2)
    }
    \centerline{\box\figBoxA}
  \end{marginfigure}
\fi 

\begin{solution}[\halfpage]
  The situation is \asif. At any point in time, your head is tilted up an angle $\angle CAB = \theta$
  - if $A$ is the eye-level

  \begin{align}
    \tan\theta(T) &= \dfrac{BC}{AB} = \dfrac{\vbtwo\cdot T}{\vbone} \text{ where 'T' is time } \\
    \Rightarrow \sec^2\theta\cdot\dfrac{d\theta}{dt} &= \dfrac{\vbtwo}{\vbone} \\
    \Rightarrow \dfrac{d\theta}{dt} &= \dfrac{\vbtwo}{\vbone\cdot (1+\tan^2\theta)} = \dfrac\m{\p + \q T^2}
  \end{align}
  And so, when $T=\vbthree$ seconds, then 
  \begin{align}
    \dfrac{d\theta}{dt} &= \dfrac\m{\p+\q\cdot\vbthree^2} = \WRITEFRAC\m\j \dfrac{\text{radians}}{\text{sec}}
  \end{align}
\end{solution}

\ifprintrubric
  \begin{table}
  	\begin{tabular}{ p{5cm}p{5cm} }
  		\toprule % in brief (4-6 words), what should a grader be looking for for insights & formulations
  		  \sc{\textcolor{blue}{Insight}} & \sc{\textcolor{blue}{Formulation}} \\ 
  		\midrule % ***** Insights & formulations ******
        $\tan\theta = \dfrac{BC (t)}{AB}$ & $\sec^2\theta = 1 + \tan^2\theta$ \\
  		\toprule % final numerical answers for the various versions
        \sc{\textcolor{blue}{If question has $\ldots$}} & \sc{\textcolor{blue}{Final answer}} \\
  		\midrule % ***** Numerical answers (below) **********
        $AB=23$ & $\dfrac{46}{929}$ \\
        $AB=25$ & $\qquad\dfrac{1}{50}$ \\
        $AB=27$ & $\dfrac{9}{181}$ \\
        $AB=30$ & $\qquad\dfrac{5}{123}$ \\
  		\bottomrule
  	\end{tabular}
  \end{table}
\fi
