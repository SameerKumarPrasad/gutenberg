% This is an empty shell file placed for you by the 'examiner' script.
% You can now fill in the TeX for your question here.

% Now, down to brasstacks. ** Writing good solutions is an Art **. 
% Eventually, you will find your own style. But here are some thoughts 
% to get you started: 
%
%   1. Write the solution as if you are writing it for your favorite
%      14-17 year old to help him/her understand. Could be your nephew, 
%      your niece, a cousin perhaps or probably even you when you 
%      were that age. Just write for them.
%
%   2. Use margin-notes to "talk" to students about the critical insights
%      in the question. The tone can be - in fact, should be - informal
%
%   3. Don't shy away from creating margin-figures you think will help
%      students understand. Yes, it is a little more work per question. 
%      But the question & solution will be written only once. Make that
%      attempt at writing a solution count.
%
%   4. At the same time, do not be too verbose. A long solution can
%      - at first sight - make the student think, "God, that is a lot to know".
%      Our aim is not to scare students. Rather, our aim should be to 
%      create many "Aha!" moments everyday in classrooms around the world
% 
%   5. Ensure that there are *no spelling mistakes anywhere*. We are an 
%      education company. Bad spellings suggest that we ourselves 
%      don't have any education. Also, use American spellings by default
% 
%   6. If a question has multiple parts, then first delete lines 40-41
%   7. If a question does not have parts, then first delete lines 43-69

\question[2] An aircraft at an altitude of $200$m above a river observes that the angle 
of depression of two points on opposite banks are $\ang{45}$ and $\ang{60}$ respectively.
Find, in meters, the width of the river

\insertQR{QRC}

\ifprintanswers
  % stuff to be shown only in the answer key - like explanatory margin figures
  \begin{marginfigure}
    \figinit{pt}
      \figpt 100: $B$(0,0)
      \figpt 101: $C$(30,0)
      \figpt 102: $D$(80,0)
      \figpt 103: $A$(30,60)
      \figpt 104: $M$(0,60)
      \figpt 105: $N$(80,60)
      \figpt 106: $\ang{45}$(39,50) % angle labels
      \figpt 107: $\ang{60}$(20,48)
    \figdrawbegin{}
      \figdrawline [100,103,102,101,100]
      \figdrawline [103,101]
      \figdrawline [104,105]
      \figdrawarccircP 103 ; 10 [102,105] 
      \figdrawarccircP 103 ; 10 [104,100] 
      \figdrawarccircP 103 ; 12 [104,100] 
    \figdrawend
    \figvisu{\figBoxA}{}{%
      \figwrites 100:(2)
      \figwrites 101:(2)
      \figwrites 102:(2)
      \figwriten 103:(2)
      \figwritee 106:(2)
      \figwritew 107:(2)
    }
    \centerline{\box\figBoxA}
  \end{marginfigure}
\fi 

\begin{solution}[\halfpage]
  If the angles of depression are \asif, then $\angle{BAC} = \ang{30}$ and $\angle{DAC} = \ang{45}$.
  And therefore, the width of the river is, 
  
  \begin{align}
  	BD = BC + CD &= AC\cdot\tan\angle{BAC} + AC\cdot\tan\angle{DAC} \\
  	  &= \text{200m}\cdot\left( \tan\ang{30} + \tan\ang{45} \right) \\
  	  &= \text{200m}\cdot\left( \dfrac{1}{\sqrt{3}} + 1\right) \\
  	  &= 315.47\text{m}
  \end{align}
\end{solution}
