
\ifnumequal{\value{rolldice}}{0}{
  % variables 
  \renewcommand{\va}{25} % initial angle 
  \renewcommand{\vb}{39} % final angle  
  \renewcommand{\vh}{200} % Height 
}{
  \ifnumequal{\value{rolldice}}{1}{
    % variables 
    \renewcommand{\va}{27}
    \renewcommand{\vb}{43}
    \renewcommand{\vh}{250} % Height 
  }{
    \ifnumequal{\value{rolldice}}{2}{
      % variables 
      \renewcommand{\va}{42}
      \renewcommand{\vb}{35}
      \renewcommand{\vh}{300} % Height 
    }{
      % variables 
      \renewcommand{\va}{22}
      \renewcommand{\vb}{51}
      \renewcommand{\vh}{350} % Height 
    }
  }
}

\SUBTRACT{90}\va\vc
\SUBTRACT{90}\vb\vd
\DEGREESTAN\vc\vm
\DEGREESTAN\vd\vn
\ROUND[2]\vm\vx
\ROUND[2]\vn\vy
\ADD\vx\vy\vz
\MULTIPLY\vh\vz\vv
\ROUND[2]\vv\vw

\question[2] An aircraft at an altitude of $\vh$m above a river observes that the angle 
of depression of two points on opposite banks are $\ang\va$ and $\ang\vb$ respectively.
Find the width of the river (in meters).

\watchout
\marginnote[-70pt]{
  \begin{tabular}{c c}
    $\tan\ang\vc=\vx$ & $\tan\ang\vd=\vy$ 
  \end{tabular}
}

\ifprintanswers
  % stuff to be shown only in the answer key - like explanatory margin figures
  \begin{marginfigure}
    \figinit{pt}
      \figpt 100: $B$(0,0)
      \figpt 101: $C$(30,0)
      \figpt 102: $D$(80,0)
      \figpt 103: $A$(30,60)
      \figpt 104: $M$(0,60)
      \figpt 105: $N$(80,60)
      \figpt 106: $\ang\va$(39,50) % angle labels
      \figpt 107: $\ang\vb$(20,48)
    \figdrawbegin{}
      \figdrawline [100,103,102,101,100]
      \figdrawline [103,101]
      \figdrawline [104,105]
      \figdrawarccircP 103 ; 10 [102,105] 
      \figdrawarccircP 103 ; 10 [104,100] 
      \figdrawarccircP 103 ; 12 [104,100] 
    \figdrawend
    \figvisu{\figBoxA}{}{%
      \figwrites 100:(2)
      \figwrites 101:(2)
      \figwrites 102:(2)
      \figwriten 103:(2)
      \figwritee 106:(2)
      \figwritew 107:(2)
    }
    \centerline{\box\figBoxA}
  \end{marginfigure}
\fi 

\begin{solution}[\halfpage]
  If the angles of depression are \asif, then 
    \begin{align}
      \angle{BAC} &= \ang{90} - \ang\vb=\ang\vd \\
      \angle{DAC} &= \ang{90} - \ang\va = \ang\vc
    \end{align}

  The width of the river is would therefore be
  \begin{align}
  	BD = BC + CD &= AC\cdot\tan\angle{BAC} + AC\cdot\tan\angle{DAC} \\
  	  &= \vh\text{ meters}\cdot\left( \tan\ang\vc + \tan\ang\vd \right) \\
  	  &= \vh\text{ meters}\cdot(\vy + \vx) = \vw
  \end{align}
\end{solution}

\ifprintanswers\begin{codex}$\vw$ meters\end{codex}\fi
