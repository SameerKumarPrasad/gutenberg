
\ifnumequal{\value{rolldice}}{0}{
  % variables 
  \renewcommand{\va}{10} % assumed speed 
  \renewcommand{\vb}{12} % actual speed
  \renewcommand\vz{-\frac{60}{11}} % negative spped - to be ignored
}{
  \ifnumequal{\value{rolldice}}{1}{
    % variables 
    \renewcommand{\va}{12}
    \renewcommand{\vb}{16}
    \renewcommand\vz{-\frac{16}{5}} % negative spped - to be ignored
  }{
    \ifnumequal{\value{rolldice}}{2}{
      % variables 
      \renewcommand{\va}{10}
      \renewcommand{\vb}{15}
      \renewcommand\vz{-\frac{15}{7}} % negative spped - to be ignored
    }{
      % variables 
      \renewcommand{\va}{12}
      \renewcommand{\vb}{14}
      \renewcommand\vz{-\frac{84}{13}} % negative spped - to be ignored
    }
  }
}

\LCM\va\vb\vc
\MULTIPLY\vc{2}\vd % distance 
\ADD\vd{2}\ve 
\SUBTRACT\vb{1}\a
\FRACTIONSIMPLIFY\a\va\vp\vq
\FRACMULT\vp\vq{2}{1}\vr\vs
\FRACMULT\vp\vq\vd{1}\k\m % \m = 1, \k is what we need 
\SUBTRACT\k\ve\c 
\FRACMULT\a\va{60}{1}\vt\b % \b = 1 guaranteed

\MULTIPLY\vd\vs\d % constant in quadratic 
\MULTIPLY\c\vs\b % x-term coefficient in quadratic

\question[5] A cyclist covered a distance of $\vd$ km two hours faster than he assumed. Every hour
he travelled 1 km more than he intended to in $\vt$ minutes. What was his speed?

\watchout

\begin{calcaid}
  \begin{tabular}{c c c c} 
    $\sqrt{12996}=114$ & $\sqrt{28900}=170$ & 
    $\sqrt{5776}=76$ & $\sqrt{57600}=240$
  \end{tabular}
\end{calcaid}

\ifprintanswers
  % stuff to be shown only in the answer key - like explanatory margin figures
  	\begin{tabular}{cccc}
  	    \toprule
  		& Speed (km/h) & Number of 1-hour periods & Distance covered \\
  		\midrule
  		Assumed & $s$ & $N$ & $\vd$ \\
  		Actual & $\frac\vp\vq s + 1$ & $N-2$ & $\vd$ \\
  		\bottomrule
  	\end{tabular}
\fi 

\begin{solution}[\fullpage]
	If $s$ be the cyclist's speed on any other day, then this time around he was 
	travelling at 
  \[ \left(\dfrac{\vt\text{ min}}{60\text{ min}}s + 1\right) \text{km/hour} = \left(\dfrac\vp\vq s + 1\right) \]
	
	The second insight is that 
		\[ s\cdot N = \left( \dfrac\vp\vq s + 1 \right)\cdot (N-2) = \vd\text{ km} \]
	
  Which means,
	\begin{align}
		\dfrac\vp\vq sN - \dfrac\vr\vs s + (N - 2) &= \vd \\
		\underbrace{\dfrac\vp\vq\cdot \vd}_{s\cdot N=\vd} - \dfrac\vr\vs s + \underbrace{\dfrac{\vd}{s}}_{s\cdot N=\vd} - 2 &= \vd \\
    \dfrac\vr\vs s - \dfrac\vd{s} = \c\implies \vr s^2 - \b s -\d &= 0  \\
    \text{or } s &= \vz,\va
	\end{align}

  As speed can only be positive, the cyclist's \textbf{assumed} speed is $s = \va$ km per hour.
  \textbf{But his actual speed} was $\frac\vp\vq\cdot\va + 1 = \vb$ km per hour.
\end{solution}

\ifprintanswers\begin{codex}$\vb$ km/h\end{codex}\fi
