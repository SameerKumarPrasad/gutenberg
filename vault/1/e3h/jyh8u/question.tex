% This is an empty shell file placed for you by the 'examiner' script.
% You can now fill in the TeX for your question here.

% Now, down to brasstacks. ** Writing good solutions is an Art **. 
% Eventually, you will find your own style. But here are some thoughts 
% to get you started: 
%
%   1. Write the solution as if you are writing it for your favorite
%      14-17 year old to help him/her understand. Could be your nephew, 
%      your niece, a cousin perhaps or probably even you when you 
%      were that age. Just write for them.
%
%   2. Use margin-notes to "talk" to students about the critical insights
%      in the question. The tone can be - in fact, should be - informal
%
%   3. Don't shy away from creating margin-figures you think will help
%      students understand. Yes, it is a little more work per question. 
%      But the question & solution will be written only once. Make that
%      attempt at writing a solution count.
%
%   4. At the same time, do not be too verbose. A long solution can
%      - at first sight - make the student think, "God, that is a lot to know".
%      Our aim is not to scare students. Rather, our aim should be to 
%      create many "Aha!" moments everyday in classrooms around the world
% 
%   5. Ensure that there are *no spelling mistakes anywhere*. We are an 
%      education company. Bad spellings suggest that we ourselves 
%      don't have any education. Also, use American spellings by default
% 
%   6. If a question has multiple parts, then first delete lines 40-41
%   7. If a question does not have parts, then first delete lines 43-69

\question[3] To a person on the ground, the angle of elevation of an aircraft flying at an altitude of 1km 
is $\ang{60}$ initially and $\ang{30}$ 10 seconds later. What is the speed of the
aircraft in kilometers per hour? The aircraft maintains its altitude for the 10 seconds

\insertQR{QRC}

\ifprintanswers
  % stuff to be shown only in the answer key - like explanatory margin figures
  \begin{marginfigure}
    \figinit{pt}
      \figpt 100: $A$(0,0)
      \figpt 101: $B$(25,0)
      \figpt 102: $C$(65,0)
      \figpt 103: $E$(25,45)
      \figpt 104: $F$(65,45)
    \figdrawbegin{}
      \figdrawline [100,101,102,104,103,100]
      \figdrawline [100,103]
      \figdrawline [100,104]
      \figdrawline [103,101]
      \figdrawarccircP 100 ; 10 [101,103] 
      \figdrawarccircP 100 ; 12 [101,104] 
    \figdrawend
    \figvisu{\figBoxA}{}{%
      \figwrites 100:(2)
      \figwrites 101:(2)
      \figwrites 102:(2)
      \figwriten 103:(2)
      \figwriten 104:(2)
    }
    \centerline{\box\figBoxA}
  \end{marginfigure}
\fi 

\begin{solution}[\halfpage]
	The situation is as \asif. And in the 10 seconds, the aircraft covers 
	a distance given by, 
	
	\begin{align}
		BC = AC - AB &= FC\cdot\dfrac{1}{\tan\angle{FAC}} - EB\cdot\dfrac{1}{\tan\angle{EAB}}
	\end{align}
	
	As the aircraft maintains its altitude, $EB = FC = 1km$. And therefore, 
	\begin{align}
		BC &= \text{1 km}\cdot\left(\cot\angle{FAC} - \cot\angle{EAB}\right) \\
		   &= \text{1 km}\cdot\left(\cot\ang{30} - \cot\ang{60}\right) \\
		   &= \dfrac{2}{\sqrt{3}}\text{km} \\
		\Rightarrow \text{Speed} &= \dfrac{\frac{2}{\sqrt{3}}\text{km}}{10\text{s}}\cdot 3600\dfrac{\text{s}}{\text{hour}} \\
		   &= 415.7\text{ km per hour}
	\end{align}
\end{solution}
