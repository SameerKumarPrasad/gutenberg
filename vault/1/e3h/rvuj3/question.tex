
\question A family has two children. We know that the older child is 7-years old while 
the younger one is 4-years old. But we don't know whether they are both girls or both boys 
or a boy and a girl. Given this, what is the probability of $\ldots$

\ifprintanswers
  % stuff to be shown only in the answer key - like explanatory margin figures
  	\begin{tabular}{ccc}
  	   \toprule
  	   Combination & First child & Second child \\
  	   \midrule
  	   1 & Girl & Girl \\
  	   2 & Girl & Boy \\
  	   3 & Boy & Girl \\
  	   4 & Boy & Boy \\
  	   \bottomrule
  	\end{tabular}
\fi 

\begin{parts}
  \part[1] $\ldots$ the older child being a girl?

  \begin{solution}[\mcq]
     If the first child was a girl, then our set of possible outcomes is down to just
     the first two. And therefore, the required probability is $\dfrac{1}{2}$
  \end{solution}

  \part[1] $\ldots$ both the children being girls if we come to know that the family has atleast one girl child?

  \begin{solution}[\mcq]
    The above table lists all possible ways of having two children. 

    Of these, combinations \#1, \#2 and \#3 represent the case where the family has atleast 
    one girl child. 

    And of these three, only combination \#1 represents the case where the family has two girls. 
    Hence, the required probability is simply $\dfrac{1}{3}$
  \end{solution}

\end{parts}
