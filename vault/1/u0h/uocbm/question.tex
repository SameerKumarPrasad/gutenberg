% This is an empty shell file placed for you by the 'examiner' script.
% You can now fill in the TeX for your question here.

% Now, down to brasstacks. ** Writing good solutions is an Art **. 
% Eventually, you will find your own style. But here are some thoughts 
% to get you started: 
%
%   1. Write the solution as if you are writing it for your favorite
%      14-17 year old to help him/her understand. Could be your nephew, 
%      your niece, a cousin perhaps or probably even you when you 
%      were that age. Just write for them.
%
%   2. Use margin-notes to "talk" to students about the critical insights
%      in the question. The tone can be - in fact, should be - informal
%
%   3. Don't shy away from creating margin-figures you think will help
%      students understand. Yes, it is a little more work per question. 
%      But the question & solution will be written only once. Make that
%      attempt at writing a solution count.
%
%   4. At the same time, do not be too verbose. A long solution can
%      - at first sight - make the student think, "God, that is a lot to know".
%      Our aim is not to scare students. Rather, our aim should be to 
%      create many "Aha!" moments everyday in classrooms around the world
% 
%   5. Ensure that there are *no spelling mistakes anywhere*. We are an 
%      education company. Bad spellings suggest that we ourselves 
%      don't have any education. Also, use American spellings by default
% 
%   6. If a question has multiple parts, then first delete lines 40-41
%   7. If a question does not have parts, then first delete lines 43-69

\question[4] Compute the area contained between the parabola $y=x^2-2x+2$, the
tangent to it at point $(3,5)$ and the $x$ and $y$ axes

\insertQR{QRC}

\ifprintanswers
  \begin{marginfigure}
% 1. Definition of characteristic points
\figinit{pt}
\def\Xmin{-10.00000}
\def\Ymin{-5.58139}
\def\Xmax{80.00000}
\def\Ymax{74.41860}
\def\Xori{0}
\def\Yori{5.58139}
\figpt0:(\Xori,\Yori)
\figpt 100: (10,10)
\figpt 101: (20,20)
% 2. Creation of the graphical file
\figdrawbegin{}
\def\Xmaxx{\Xmax} % To customize the position
\def\Ymaxx{\Ymax} % of the arrow-heads of the axes.
\figset arrowhead(length=4, fillmode=yes) % styling the arrowheads
\figdrawaxes 0(\Xmin, \Xmaxx, \Ymin, \Ymaxx)
\figdrawline[100,101]
\figdrawarrow[100,101]
\figdrawlineC(
0 -46.51162,
2.75862 -42.40577,
5.51724 -38.29991,
8.27586 -34.19406,
11.03448 -30.08821,
13.79310 -25.98235,
16.55172 -21.87650,
19.31034 -17.77064,
22.06896 -13.66479,
24.82758 -9.55894,
27.58620 -5.45308,
30.34482 -1.34723,
33.10344 2.75862,
35.86206 6.86447,
38.62068 10.97032,
41.37931 15.07618,
44.13793 19.18203,
46.89655 23.28789,
49.65517 27.39374,
52.41379 31.49959,
55.17241 35.60545,
57.93103 39.71130,
60.68965 43.81716,
63.44827 47.92301,
66.20689 52.02886,
68.96551 56.13472,
71.72413 60.24057,
74.48275 64.34643,
77.24137 68.45228,
79.99999 72.55813
)
\figdrawlineC(
0 20.46511,
2.75862 18.55377,
5.51724 16.92558,
8.27586 15.58056,
11.03448 14.51870,
13.79310 13.74001,
16.55172 13.24447,
19.31034 13.03210,
22.06896 13.10289,
24.82758 13.45684,
27.58620 14.09396,
30.34482 15.01424,
33.10344 16.21768,
35.86206 17.70428,
38.62068 19.47404,
41.37931 21.52697,
44.13793 23.86306,
46.89655 26.48231,
49.65517 29.38473,
52.41379 32.57030,
55.17241 36.03904,
57.93103 39.79094,
60.68965 43.82601,
63.44827 48.14423,
66.20689 52.74562,
68.96551 57.63017,
71.72413 62.79788,
74.48275 68.24876,
77.24137 73.98280,
79.99999 79.99999
)
\figdrawend
% 3. Writing text on the figure
\figvisu{\figBoxA}{}{%
\figptsaxes 1:0(\Xmin, \Xmaxx, \Ymin, \Ymaxx)
% Points 1 and 2 are the end points of the arrows
\figwritee 1:(5pt)     \figwriten 2:(5pt)
\figptsaxes 1:0(\Xmin, \Xmax, \Ymin, \Ymax)
\figwriten 101: $R$(2)
}
\centerline{\box\figBoxA}

  \end{marginfigure}
\fi 

\begin{solution}[\fullpage]
   First, lets find the equation for the tangent to the parabola at $(3,5)$
   \begin{align}
      \text{Slope} &= \dfrac{\ud}{\ud x}(x^2-2x+2) \\
                   &= (2x-2) \\
                   &= 4 \text{ when } x = 3
   \end{align}
   The equation of the tangent, therefore, would be 
   \begin{align}
       \dfrac{y-5}{x-3} &= 4 \\
       \Rightarrow y &= 4x - 7
   \end{align}
   And note that the tangent cuts the $x-axis$ when $y = 4x - 7 = 0 \Rightarrow x = \frac{7}{4}$
   
   Therefore, the required area $A$ of region $R$ is given by
   \begin{align}
      A &= \int_0^3 (x^2-2x+2)\ud x - \int_{\frac{7}{4}}^3 (4x - 7) \ud x \\
        &= \left[ \dfrac{x^3}{3}-x^2+2x\right]_0^3 - \left[ 2x^2-7x \right]_{\frac{7}{4}}^3 \\
        &= \left[(9-9+6)-0 \right] - \left[(18-21)-(2\cdot\frac{49}{16}-\frac{49}{4}) \right] \\
        &= \frac{23}{8} = 2\frac{7}{8}
   \end{align}
\end{solution}
