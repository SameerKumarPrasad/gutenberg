% This is an empty shell file placed for you by the 'examiner' script.
% You can now fill in the TeX for your question here.

% Now, down to brasstacks. ** Writing good solutions is an Art **. 
% Eventually, you will find your own style. But here are some thoughts 
% to get you started: 
%
%   1. Write the solution as if you are writing it for your favorite
%      14-17 year old to help him/her understand. Could be your nephew, 
%      your niece, a cousin perhaps or probably even you when you 
%      were that age. Just write for them.
%
%   2. Use margin-notes to "talk" to students about the critical insights
%      in the question. The tone can be - in fact, should be - informal
%
%   3. Don't shy away from creating margin-figures you think will help
%      students understand. Yes, it is a little more work per question. 
%      But the question & solution will be written only once. Make that
%      attempt at writing a solution count.
%
%   4. At the same time, do not be too verbose. A long solution can
%      - at first sight - make the student think, "God, that is a lot to know".
%      Our aim is not to scare students. Rather, our aim should be to 
%      create many "Aha!" moments everyday in classrooms around the world
% 
%   5. Ensure that there are *no spelling mistakes anywhere*. We are an 
%      education company. Bad spellings suggest that we ourselves 
%      don't have any education. Also, use American spellings by default
% 
%   6. If a question has multiple parts, then first delete lines 40-41
%   7. If a question does not have parts, then first delete lines 43-69

\question[4] The circle $x^2+y^2=8$ is divided into two parts by the parabola 
$y=\frac{x^2}{2}$ - as shown in the figure alongside. Find the areas of both parts - $A$ and $B$

\insertQR{QRC}

\ifprintanswers
  % stuff to be shown only in the answer key - like explanatory margin figures
\fi 
\begin{marginfigure}
	% 1. Definition of characteristic points
\figinit{pt}
\def\Xmin{-39.99999}
\def\Ymin{-31.37254}
\def\Xmax{39.99999}
\def\Ymax{48.62745}
\def\Xori{39.99999}
\def\Yori{31.37254}
\figpt0:(\Xori,\Yori)
\figpt 100: (40,33)
\figpt 101: (40,45)
\figpt 102: (40,20)
% 2. Creation of the graphical file
\figdrawbegin{}
\def\Xmaxx{\Xmax} % To customize the position
\def\Ymaxx{\Ymax} % of the arrow-heads of the axes.
\figset arrowhead(length=4, fillmode=yes) % styling the arrowheads
\figdrawaxes 0(\Xmin, \Xmaxx, \Ymin, \Ymaxx)
\figdrawcirc 100 (25)
\figdrawlineC(
0 79.99999,
2.75862 73.52404,
5.51724 67.51066,
8.27586 61.95985,
11.03448 56.87160,
13.79310 52.24592,
16.55172 48.08281,
19.31034 44.38227,
22.06896 41.14429,
24.82758 38.36888,
27.58620 36.05604,
30.34482 34.20577,
33.10344 32.81807,
35.86206 31.89293,
38.62068 31.43037,
41.37931 31.43037,
44.13793 31.89293,
46.89655 32.81807,
49.65517 34.20577,
52.41379 36.05604,
55.17241 38.36888,
57.93103 41.14429,
60.68965 44.38227,
63.44827 48.08281,
66.20689 52.24592,
68.96551 56.87160,
71.72413 61.95985,
74.48275 67.51066,
77.24137 73.52404,
79.99999 79.99999
)
\figdrawend
% 3. Writing text on the figure
\figvisu{\figBoxA}{}{%
\figptsaxes 1:0(\Xmin, \Xmaxx, \Ymin, \Ymaxx)
% Points 1 and 2 are the end points of the arrows
\figwritee 1:(5pt)     \figwriten 2:(5pt)
\figptsaxes 1:0(\Xmin, \Xmax, \Ymin, \Ymax)
\figwritee 101: $A$(2)
\figwritee 102: $B$(2)
}
\centerline{\box\figBoxA}

\end{marginfigure}

\begin{solution}[\fullpage]
  First, the parabola and the circle will intersect at points where
  \begin{align}
     x^2 = 2y &= 8-y^2 \\
     \Rightarrow y^2+2y-8 &= 0 \\
     \Rightarrow y &= 2, -4
  \end{align}
  As is apparent from the figure, only a $y > 0$ makes sense and therefore
  we will go with $y = 2$
  
  For this $y$, $x = \sqrt{8-2^2} = \pm 2$. The area of $A$ can now be calculated
  \begin{align}
     A &= 2\cdot\left[ \int_0^2 \sqrt{8-x^2}\ud x - \int_0^2\dfrac{x^2}{2}\ud x\right] \\
       &= 2\cdot\left[ \left(\dfrac{x}{2}\sqrt{8-x^2}+\dfrac{8}{2}\sin^{-1}\dfrac{x}{\sqrt{8}}\right)_0^2 
       - \left( \dfrac{x^3}{6}\right)_0^2\right] \\
       &= 2\cdot\left[ \dfrac{2}{2}\sqrt{4}+4\sin^{-1}\dfrac{2}{2\sqrt{2}} - \dfrac{8}{6}\right] \\
       &= 2\pi + \frac{4}{3} \\
    \Rightarrow B &= \pi\cdot R^2 - A \\ 
                  &= 8\pi - \left( 2\pi + \frac{4}{3}\right) \\
                  &= 6\pi - \frac{4}{3}
  \end{align}
  
\end{solution}
