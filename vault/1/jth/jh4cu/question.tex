\question The following table presents output from a regression. Three models  were used to evaluate 
loan approvals as a function of several explanatory variables. The scale factors for logit and 
probit are 0.0883 and 0.166 respectively 

\begin{parts}
  \part[2] How much more likely is a married white male to be approved for a loan compared 
  to an unmarried-nonwhite male? Compare the values from all three models

\begin{solution}[\halfpage]
    The coefficient on \texttt{male} does not matter (why?).  For the \textbf{linear model}, being white 
    is associated with a 12.9\% advantage over non-whites, and being married is associated with a 
    4.6\% increase in approval rate. Thus the net impact is an approximately $\mathbf{17.5\%} (=12.9\% + 4.6\%)$ 
    higher likelihood of being approved
    
    Using the scale factors, the \textbf{Probit model} gives us a higher likelihood of 
    $0.166\times(0.520+0.266) = .1304 \text{, that is } \mathbf{13.04\%}$ 
    
    Similarly for \textbf{Logit} we get $0.0883\times(0.938+0.503) = .1272 \text{, that is, } \textbf{12.72\%}$
  \end{solution}

  \part[2] What do you think is the rationale behind using \textit{unemployment rate in applicant's industry}
  rather than employment status? Is it statistically significant?

\begin{solution}[\halfpage]
    In most cases, unemployed people will not be seeking a home loan. So employment status is not that useful. 
    But the nature of ones job is a relevant factor. One measure of job stability is the unemployment rate in that 
    field.  Banks may be wary of extending loans to people in industries with high unemployment rates
  \end{solution}

  \part[2] Do any of the coefficients have unexpected signs? Are they statistically significant? 

\begin{solution}[\mcq]
    I would have imagined some type of discrimination against female borrowers. 
    So the negative sign on \emph{male} is unexpected. However, the effect is not significant at all.
  \end{solution}

  \part[2] How many variables have \textit{not} been reported in the model? 

\begin{solution}[\mcq]
    Use the degrees of freedom, note that the constant term has also not been mentioned
    \[df = n-k-1 \implies k = 1971-1955 -1 = 15\]
    
    So I have not reported $6 (= 15 - 9)$ variables
  \end{solution}

\end{parts}

\begin{attachment}
  \begin{tabular}{lccc}
  \toprule
    & \multicolumn{3}{c}{Dependent variable: loan approval decision} \\
    \cline{2-4} \\
    \\[-3ex]& \textit{Linear} & \textit{Probit} & \textit{Logit} \\
  \midrule
   \texttt{Is white/Caucasian} & 0.129& 0.520 & 0.938 \\ 
    & (0.020) & (0.097) & (0.173) \\ 
    & & & \\ 
   \texttt{Current housing expenditure (\% income)} & 0.002 & 0.008 & 0.013 \\ 
    & (0.001) & (0.007) & (0.013) \\ 
    & & & \\ 
   \texttt{Other obligations (\% income)} & -0.005 & -0.028 & -0.053\\ 
    & (0.001) & (0.006) & (0.011) \\ 
    & & & \\ 
   \texttt{Loan requested (\% of price)} & -0.147 & -1.012 & -1.905 \\ 
    & (0.038) & (0.240) & (0.460) \\ 
    & & & \\ 
   \texttt{Unemployment rate in industry} & -0.007 & -0.037 & -0.067 \\ 
    & (0.003) & (0.018) & (0.033) \\ 
    & & & \\ 
   \texttt{Male} & -0.004 & -0.037 & -0.066 \\ 
    & (0.019) & (0.110) & (0.206) \\ 
    & & & \\ 
   \texttt{Married} & 0.046 & 0.266 & 0.503 \\ 
    & (0.016) & (0.095) & (0.178) \\ 
    & & & \\ 
   \texttt{Good credit history} & 0.133 & 0.585 & 1.067 \\ 
    & (0.019) & (0.096) & (0.171) \\ 
    & & & \\ 
   \texttt{Prior bankruptcy} & -0.242 & -0.779 & -1.341 \\ 
    & (0.028) & (0.127) & (0.217) \\ 
    & & & \\ 
  \midrule
  Observations & \multicolumn{1}{c}{1,971} & \multicolumn{1}{c}{1,971} & \multicolumn{1}{c}{1,971} \\ 
  R$^{2}$ & \multicolumn{1}{c}{0.166} &  &  \\ 
  Adjusted R$^{2}$ & \multicolumn{1}{c}{0.159} &  &  \\ 
  Log likelihood &  & \multicolumn{1}{c}{-600.271} & \multicolumn{1}{c}{-600.496} \\ 
  Residual Std. Error & \multicolumn{1}{c}{0.302 (df = 1955)} &  &  \\ 
  F statistic & \multicolumn{1}{c}{25.863$^{***}$ (df = 15; 1955)} &  &  \\ 
  \bottomrule 
  \end{tabular} 
\end{attachment}
