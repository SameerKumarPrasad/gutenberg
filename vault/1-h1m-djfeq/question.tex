% This is an empty shell file placed for you by the 'examiner' script.
% You can now fill in the TeX for your question here.

% Now, down to brasstacks. ** Writing good solutions is an Art **. 
% Eventually, you will find your own style. But here are some thoughts 
% to get you started: 
%
%   1. Write the solution as if you are writing it for your favorite
%      14-17 year old to help him/her understand. Could be your nephew, 
%      your niece, a cousin perhaps or probably even you when you 
%      were that age. Just write for them.
%
%   2. Use margin-notes to "talk" to students about the critical insights
%      in the question. The tone can be - in fact, should be - informal
%
%   3. Don't shy away from creating margin-figures you think will help
%      students understand. Yes, it is a little more work per question. 
%      But the question & solution will be written only once. Make that
%      attempt at writing a solution count.
%
%   4. At the same time, do not be too verbose. A long solution can
%      - at first sight - make the student think, "God, that is a lot to know".
%      Our aim is not to scare students. Rather, our aim should be to 
%      create many "Aha!" moments everyday in classrooms around the world
% 
%   5. Ensure that there are *no spelling mistakes anywhere*. We are an 
%      education company. Bad spellings suggest that we ourselves 
%      don't have any education. Also, use American spellings by default
% 
%   6. If a question has multiple parts, then first delete lines 40-41
%   7. If a question does not have parts, then first delete lines 43-69

\question Fuel expenditures for a steamship are propotional to the cube of it speed.
It is known that at a speed of $10$ km/hour, fuel costs are \texteuro 30/hour. Other 
expenses (independent of speed) amount to \texteuro 480 per hour 

\insertQR{}

\ifprintanswers
  % stuff to be shown only in the answer key - like explanatory margin figures
  \marginnote[1cm]{Keeping track of the units in which values are expressed is always 
  a good policy. Sometimes, you can crack the logic by just looking at the units} 
\fi 

\begin{parts}
  \part At what speed of the ship will the sum of expenses \textit{per kilometer} of
  travel be the lowest? 

  \insertQR{}
  \begin{solution}
    Let $F(s)$ and $E(s)$ be the fuel expenditure per hour and total expenditure per hour 
    - expressed as a function of speed $s$
    \begin{align}
        F(s) \propto s^3 \Rightarrow F(s) &= ks^3 \\ 
        \text{Given that } F(10) = k\cdot 10^3 &= \text{\texteuro 30 per hour} \\
        \text{ we get } k &= 0.03 \\ 
        \text{And therefore, } E(s) &= (0.03s^3 + 480)\text{ \texteuro per hour}
    \end{align}
    
    However, we need to express the expenditure in \texteuro per kilometer
    \begin{align}
       \dfrac{\text{\texteuro}}{\text{km}} &= \dfrac{\frac{\text{\texteuro}}{\text{hour}}}
                                                    {\frac{\text{km}}{\text{hour}}} \\
       \text{which suggests } G(s) &= \dfrac{E(s)}{s} \dfrac{\text{\texteuro}}{\text{km}} \\
                                   &= 0.03s^2 + \dfrac{480}{s} \\
       \text{Now, } \dfrac{\ud G(s)}{\ud s} &= 0.06s - \dfrac{480}{s^2} \\
       \text{And, }\dfrac{\ud^2 G(s)}{\ud s^2} &= 0.06 + \dfrac{960}{s^3} > 0 \Rightarrow \text{ minima }
    \end{align}
    
    And hence, the speed at which total expenditure \textit{ per km } is minimized is the one at which
    \begin{align}
       \dfrac{\ud G(s)}{\ud s} = 0.06s - \dfrac{480}{s^2} &= 0 \\
       \Rightarrow s &= 20 \text{ km per hour }
    \end{align}
  \end{solution}

  \part What will the total expenditure per hour at that speed be? 

  \insertQR{}
  \begin{solution}
    The total expenditure per hour at this speed would be
    \begin{align}
       E(20) &= 0.03\times 20^3 + 480 \\
             &= \text{\texteuro 720 per hour}
    \end{align}
  \end{solution}

\end{parts}

