% This is an empty shell file placed for you by the 'examiner' script.
% You can now fill in the TeX for your question here.

% Now, down to brasstacks. ** Writing good solutions is an Art **. 
% Eventually, you will find your own style. But here are some thoughts 
% to get you started: 
%
%   1. Write the solution as if you are writing it for your favorite
%      14-17 year old to help him/her understand. Could be your nephew, 
%      your niece, a cousin perhaps or probably even you when you 
%      were that age. Just write for them.
%
%   2. Use margin-notes to "talk" to students about the critical insights
%      in the question. The tone can be - in fact, should be - informal
%
%   3. Don't shy away from creating margin-figures you think will help
%      students understand. Yes, it is a little more work per question. 
%      But the question & solution will be written only once. Make that
%      attempt at writing a solution count.
%
%   4. At the same time, do not be too verbose. A long solution can
%      - at first sight - make the student think, "God, that is a lot to know".
%      Our aim is not to scare students. Rather, our aim should be to 
%      create many "Aha!" moments everyday in classrooms around the world
% 
%   5. Ensure that there are *no spelling mistakes anywhere*. We are an 
%      education company. Bad spellings suggest that we ourselves 
%      don't have any education. Also, use American spellings by default
% 
%   6. If a question has multiple parts, then first delete lines 40-41
%   7. If a question does not have parts, then first delete lines 43-69

\question[3] The sum of the first three terms of an \text{increasing} geometric progression 
is 13 and their product is 27. Find the sum of the first five terms of the progression

\insertQR{QRC}

\ifprintanswers
\fi 

\begin{solution}[\halfpage]
	The first three terms would be $\dfrac{a_2}{r}$, $a_2$ and $a_2r$, where $a_2$ is the second term
	\begin{align}
		\dfrac{a_2}{r}\cdot a_2\cdot a_2r &= 27 \Rightarrow a_2^3 = 27 \Rightarrow a_2 = 3 \\
		\text{Also }, \dfrac{a_2}{r} + a_2 + a_2r &= 13 \Rightarrow r + \dfrac{1}{r} = \dfrac{13}{3} - 1 \\
		\Rightarrow 3r^2 - 10r + 3 &= 0 \text{ or } r = \frac{1}{3}, 3
	\end{align}
	
	With $r=\frac{1}{3}$, the sequence cannot be increasing - as is required. Hence, $r = 3$ is the only
	acceptable solution
	
	The sequence, therefore, is $\lbrace 1,3,9,27,81\ldots\rbrace$. And the sum of the first 
	5 terms is
	\begin{align}
		S_5 &= \dfrac{a_1\cdot(r^5 - 1)}{r-1} = \dfrac{1\cdot(3^5 - 1)}{2} = 121
	\end{align}
	
\end{solution}
