% This is an empty shell file placed for you by the 'examiner' script.
% You can now fill in the TeX for your question here.

% Now, down to brasstacks. ** Writing good solutions is an Art **. 
% Eventually, you will find your own style. But here are some thoughts 
% to get you started: 
%
%   1. Write the solution as if you are writing it for your favorite
%      14-17 year old to help him/her understand. Could be your nephew, 
%      your niece, a cousin perhaps or probably even you when you 
%      were that age. Just write for them.
%
%   2. Use margin-notes to "talk" to students about the critical insights
%      in the question. The tone can be - in fact, should be - informal
%
%   3. Don't shy away from creating margin-figures you think will help
%      students understand. Yes, it is a little more work per question. 
%      But the question & solution will be written only once. Make that
%      attempt at writing a solution count.
%
%   4. At the same time, do not be too verbose. A long solution can
%      - at first sight - make the student think, "God, that is a lot to know".
%      Our aim is not to scare students. Rather, our aim should be to 
%      create many "Aha!" moments everyday in classrooms around the world
% 
%   5. Ensure that there are *no spelling mistakes anywhere*. We are an 
%      education company. Bad spellings suggest that we ourselves 
%      don't have any education. Also, use American spellings by default
% 
%   6. If a question has multiple parts, then first delete lines 40-41
%   7. If a question does not have parts, then first delete lines 43-69

\question A cyclist covered a distance of 96 km two hours faster than he assumed. Every hour
he travelled 1 km more than he intended to in 1h 15min. What was his speed?

\insertQR{}

\ifprintanswers
  % stuff to be shown only in the answer key - like explanatory margin figures
  \begin{table}
  	\begin{tabular}{cccc}
  	    \toprule
  		& Speed (km/h) & Number of 1-hour periods & Distance covered \\
  		\midrule
  		Normally & $s$ & $N$ & 96 \\
  		Now & $\frac{5}{4}s + 1$ & $N-2$ & 96 \\
  		\bottomrule
  	
  	\end{tabular}
  \end{table}
\fi 

\begin{solution}
	If $s$ be the cyclist's speed on any other day, then this time around he was 
	travelling at $\left(\dfrac{75\text{ min}}{60\text{ min}}s + 1\right) \text{km/hour} = \left(\dfrac{5}{4}s + 1\right)$.
	
	The second insight is that 
	
	\begin{align}
		s\cdot N &= \left( \dfrac{5}{4}s + 1 \right)\cdot (N-2) = 96\text{ km} \\
		\Rightarrow \dfrac{5}{4}sN - \dfrac{5}{2}s + N - 2 &= 96 \\
		\dfrac{5}{4}\cdot 96 - \dfrac{5}{2}s + \dfrac{96}{s} - 2 &= 96 \\
		\Rightarrow 5s^2 - 44s - 192 &= 0 \\
		\Rightarrow s &= \dfrac{44\pm\sqrt{44^2-4\cdot 5 \cdot (-192)}}{2\cdot 5} \\
		              &= 12,\, -3.2
	\end{align}
	Speed can only be positive and hence the cyclist's normal speed is $s = 12$ km per hour.
	And this time around, it was $\frac{5}{4}\cdot 12 + 1 = 16$ km per hour
\end{solution}
