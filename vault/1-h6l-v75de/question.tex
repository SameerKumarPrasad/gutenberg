% This is an empty shell file placed for you by the 'examiner' script.
% You can now fill in the TeX for your question here.

% Now, down to brasstacks. ** Writing good solutions is an Art **. 
% Eventually, you will find your own style. But here are some thoughts 
% to get you started: 
%
%   1. Write the solution as if you are writing it for your favorite
%      14-17 year old to help him/her understand. Could be your nephew, 
%      your niece, a cousin perhaps or probably even you when you 
%      were that age. Just write for them.
%
%   2. Use margin-notes to "talk" to students about the critical insights
%      in the question. The tone can be - in fact, should be - informal
%
%   3. Don't shy away from creating margin-figures you think will help
%      students understand. Yes, it is a little more work per question. 
%      But the question & solution will be written only once. Make that
%      attempt at writing a solution count.
%
%   4. At the same time, do not be too verbose. A long solution can
%      - at first sight - make the student think, "God, that is a lot to know".
%      Our aim is not to scare students. Rather, our aim should be to 
%      create many "Aha!" moments everyday in classrooms around the world
% 
%   5. Ensure that there are *no spelling mistakes anywhere*. We are an 
%      education company. Bad spellings suggest that we ourselves 
%      don't have any education. Also, use American spellings by default
% 
%   6. If a question has multiple parts, then first delete lines 40-41
%   7. If a question does not have parts, then first delete lines 43-69

\question The ratio of the sum of the cubes of the terms of an infinitely decreasing 
geometric progression to the sum of the squares of its terms is 12:13. The sum of the 
first two terms of the progression is equal to $\frac{4}{3}$. Find the progression

\insertQR{}

\ifprintanswers
\fi 

\begin{solution}
   If $s = \lbrace a, ar, ar^2 \ldots\rbrace$ be the original sequence, then the sequence
   resulting from taking the cube and squares of the terms are
   
   $s_2 = \lbrace a^2, a^2r^2,a^2r^4 \ldots \rbrace$ and $ s_3 = \lbrace a^3, a^3r^3, a^3r^6 \ldots \rbrace$.
   These are themselves geometric progressions, \textit{but} with different first terms and common ratios
   
   Now, if $S_3$ be the sum of $s_3$ and $S_2$ the sum of $s_2$, then
   \begin{align}
   		\dfrac{S_3}{S_2} &= \dfrac{\frac{a^3}{1-r^3}}{\frac{a^2}{1-r^2}} = \dfrac{12}{13} \\
   		\Rightarrow \dfrac{a\cdot(1-r)\cdot(1+r)}{a\cdot(1-r)\cdot(1+r+r^2)} &= \dfrac{12}{13} \\
   		\text{ or, } \dfrac{a\cdot(1+r)}{1+r+r^2} &= \dfrac{12}{13} \\
   		\text{Also, } \underbrace{a\cdot(1+r)}_{\text{sum of first two terms}} &= \dfrac{4}{3} \\
   		\text{ And so, } \dfrac{\frac{4}{3}}{1+r+r^2} &= \dfrac{12}{13} \\
   		\Rightarrow 9r^2 + 9r - 4 &= 0 \Rightarrow r = \frac{1}{3}, -3
   \end{align}
   The sequence can be infinitely decreasing if $r = \frac{1}{3}$
   
   And therefore 
   \begin{align}
   	a\cdot(1+r) &= \dfrac{4}{3} \Rightarrow a = 1
   \end{align}
   Hence, the sequence is $\lbrace 1, \frac{1}{3}, \frac{1}{9}, \frac{1}{27} \ldots\rbrace$
\end{solution}
