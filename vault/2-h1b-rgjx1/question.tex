% This is an empty shell file placed for you by the 'examiner' script.
% You can now fill in the TeX for your question here.

% Now, down to brasstacks. ** Writing good solutions is an Art **. 
% Eventually, you will find your own style. But here are some thoughts 
% to get you started: 
%
%   1. Write the solution as if you are writing it for your favorite
%      14-17 year old to help him/her understand. Could be your nephew, 
%      your niece, a cousin perhaps or probably even you when you 
%      were that age. Just write for them.
%
%   2. Use margin-notes to "talk" to students about the critical insights
%      in the question. The tone can be - in fact, should be - informal
%
%   3. Don't shy away from creating margin-figures you think will help
%      students understand. Yes, it is a little more work per question. 
%      But the question & solution will be written only once. Make that
%      attempt at writing a solution count.
%
%   4. At the same time, do not be too verbose. A long solution can
%      - at first sight - make the student think, "God, that is a lot to know".
%      Our aim is not to scare students. Rather, our aim should be to 
%      create many "Aha!" moments everyday in classrooms around the world
% 
%   5. Ensure that there are *no spelling mistakes anywhere*. We are an 
%      education company. Bad spellings suggest that we ourselves 
%      don't have any education. Also, use American spellings by default
% 
%   6. If a question has multiple parts, then first delete lines 40-41
%   7. If a question does not have parts, then first delete lines 43-69

\question In the adjoining figure, $R$ is the region in the first quadrant bounded by the graph of  
$y =4\ln (3-x)$, the horizontal line $y=6$, and the vertical line $x=2$. Answer the following questions 
based on this information 

\calculator{\ln 3 = 1.0986}

\begin{marginfigure}
% 1. Definition of characteristic points
\figinit{pt}
\def\Xmin{0}
\def\Ymin{-11.29448}
\def\Xmax{79.99999}
\def\Ymax{68.70551}
\def\Xori{0}
\def\Yori{11.29448}
\figpt0:(\Xori,\Yori)
\figpt 100:$y=6$(80,71)
\figpt 101:$y=4\cdot\ln[3-x]$(80,0)
\figpt 102:(70,14)
\figpt 103:$x=2$(70,81)
\figpt 104:$R$(38,50)
% 2. Creation of the graphical file
\figdrawbegin{}
\def\Xmaxx{\Xmax} % To customize the position
\def\Ymaxx{\Ymax} % of the arrow-heads of the axes.
\figset arrowhead(length=4, fillmode=yes) % styling the arrowheads
\figdrawaxes 0(\Xmin, \Xmaxx, \Ymin, \Ymaxx)
\figdrawlineC(
0 70.18492,
2.75862 70.18492,
5.51724 70.18492,
8.27586 70.18492,
11.03448 70.18492,
13.79310 70.18492,
16.55172 70.18492,
19.31034 70.18492,
22.06896 70.18492,
24.82758 70.18492,
27.58620 70.18492,
30.34482 70.18492,
33.10344 70.18492,
35.86206 70.18492,
38.62068 70.18492,
41.37931 70.18492,
44.13793 70.18492,
46.89655 70.18492,
49.65517 70.18492,
52.41379 70.18492,
55.17241 70.18492,
57.93103 70.18492,
60.68965 70.18492,
63.44827 70.18492,
66.20689 70.18492,
68.96551 70.18492,
71.72413 70.18492,
74.48275 70.18492,
77.24137 70.18492,
79.99999 70.18492
)
\figdrawlineC(
0 54.42632,
2.75862 53.39761,
5.51724 52.34121,
8.27586 51.25561,
11.03448 50.13912,
13.79310 48.98996,
16.55172 47.80614,
19.31034 46.58552,
22.06896 45.32572,
24.82758 44.02416,
27.58620 42.67796,
30.34482 41.28395,
33.10344 39.83863,
35.86206 38.33805,
38.62068 36.77783,
41.37931 35.15304,
44.13793 33.45808,
46.89655 31.68664,
49.65517 29.83148,
52.41379 27.88429,
55.17241 25.83546,
57.93103 23.67380,
60.68965 21.38614,
63.44827 18.95690,
66.20689 16.36737,
68.96551 13.59490,
71.72413 10.61167,
74.48275 7.38298,
77.24137 3.86478,
79.99999 0
)
\figdrawline [102,103]
\figdrawend
% 3. Writing text on the figure
\figvisu{\figBoxA}{}{%
\figptsaxes 1:0(\Xmin, \Xmaxx, \Ymin, \Ymaxx)
% Points 1 and 2 are the end points of the arrows
\figwritee 1:(5pt)     \figwriten 2:(5pt)
\figptsaxes 1:0(\Xmin, \Xmax, \Ymin, \Ymax)
\figwritee 100:(2)
\figwrites 101:(3)
\figwriten 103:(2)
\figwritee 104:(2)
}
\centerline{\box\figBoxA}


\end{marginfigure}

\begin{parts}
  \part[3] Find the area of $R$
  \insertQR[-50pt]{QRC}
\begin{solution}[\halfpage]
	\begin{align}
	    R &= \int_0^2 6-4\cdot\ln(3-x)\ud x \\
	      &= 6\left[ x\right]_0^2 - 
	      4\cdot\underbrace{\left[ x\cdot\ln(3-x) - \int\dfrac{-x}{3-x}\ud x\right]_0^2}_{\textit{integration by parts}} \\
	     &= 6\left[ x\right]_0^2 - 4\cdot\left[ x\cdot\ln(3-x) - (x + 3\ln(3-x))\right]_0^2 \\
	     &= 6\left[ x\right]_0^2 - 4\cdot\left[(x-3)\cdot\ln (3-x) - x \right]_0^2 \\
	     &= 12 - 4\cdot(3\ln 3 - 2) = 6.8167
	\end{align}

\end{solution}

  \part[3] Find the volume of the solid generated when $R$ is revolved about the horizontal line $y=8$.
  \insertQR{QRC}
\begin{solution}[\halfpage]
    The required volume is the sum of volumes of many \textit{infinitesimally} thin rings, where
    the volume of an individual ring is given by
    \begin{align}
      \ud V(x) &= \pi\cdot\left\lbrace (8 - 4\ln (3-x))^2 - (8-6)^2 \right\rbrace  \\
              &= \pi\cdot\left\lbrace (8-4\ln (3-x))^2 - 4\right\rbrace
    \end{align}
    And so, 
    \begin{align}
      V &= \pi\cdot\int_0^2\left\lbrace (8-4\ln (3-x))^2 - 4\right\rbrace \ud x \\
        &= 168.179
    \end{align}
  \end{solution}

\newpage
  \part[3] The region $R$ is the base of a solid. For this solid, each cross section perpendicular 
  to the $x-axis$ is a square. Find the volume of the solid.
  \insertQR{QRC}
\begin{solution}[\halfpage]
    The said solid is made by lining up squares with sides $= 6-4\cdot\ln(3-x)$ from left-to-right.
    And therefore, the required volume - $V$ - equal to
    \begin{align}
      V &= \int_0^2\underbrace{(6 - 4\cdot\ln(3-x))^2}_{\texttt{cross-section}}\underbrace{\ud x}_{\texttt{width}} \\
       &= 26.266
    \end{align}    
  \end{solution}
\end{parts}
