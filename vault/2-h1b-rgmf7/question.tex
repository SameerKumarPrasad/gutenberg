% This is an empty shell file placed for you by the 'examiner' script.
% You can now fill in the TeX for your question here.

% Now, down to brasstacks. ** Writing good solutions is an Art **. 
% Eventually, you will find your own style. But here are some thoughts 
% to get you started: 
%
%   1. Write the solution as if you are writing it for your favorite
%      14-17 year old to help him/her understand. Could be your nephew, 
%      your niece, a cousin perhaps or probably even you when you 
%      were that age. Just write for them.
%
%   2. Use margin-notes to "talk" to students about the critical insights
%      in the question. The tone can be - in fact, should be - informal
%
%   3. Don't shy away from creating margin-figures you think will help
%      students understand. Yes, it is a little more work per question. 
%      But the question & solution will be written only once. Make that
%      attempt at writing a solution count.
%
%   4. At the same time, do not be too verbose. A long solution can
%      - at first sight - make the student think, "God, that is a lot to know".
%      Our aim is not to scare students. Rather, our aim should be to 
%      create many "Aha!" moments everyday in classrooms around the world
% 
%   5. Ensure that there are *no spelling mistakes anywhere*. We are an 
%      education company. Bad spellings suggest that we ourselves 
%      don't have any education. Also, use American spellings by default
% 
%   6. If a question has multiple parts, then first delete lines 40-41
%   7. If a question does not have parts, then first delete lines 43-69
\question  Let $R$ be the region in the first quadrant enclosed by the graphs of $y=2x$ and $y=x^2$, as shown in the figure.

\begin{marginfigure}
\figinit{mm}
%Origin
\figpt 1:$O$ (0,0)
%Straight line
\figpt 2: (20,40)
\figpt 3: (22,44)
%axes
\figpt 4: (10,0)
\figpt 5: (20,0)
\figpt 6: (0,10)
\figpt 7: (0,20)
\figpt 8: (0,30)
\figpt 9: (0,40)
%parabola
\figpt 10: (10,10)
\figpt 11: (20.2,44)
\figpt 12: (-2,0.2)
\figpt 13: (0.2,0.2)

%draw
\figdrawbegin{}
%\figset curve(0.15)
\figdrawcurve [12,13,10,2,3,11]
\figdrawline [1,3]
\figdrawaxes 1(-2,24, -5,45)
\figdrawend
%write
\figvisu{\figBoxA}{Figure}{
\figwritesw 1:(3)
\figsetmark{$\bullet$}
\figwritew 2:$[2, 4]$ (3)
\figsetmark{$|$}
\figwrites 4:$1$(3)
\figwrites 5:$2$(3)
\figsetmark{$-$}
\figwritew 6:$1$(3)
\figwritew 7:$2$(3)
\figwritew 8:$3$(3)
\figwritew 9:$4$(3)
\figsetmark{}
\figwritee 5:$x$(5)
\figwriten 9:$y$(6)
\figwritee 6:$R$(6)
}
\centerline{\box\figBoxA}
\end{marginfigure}

\ifprintanswers
  % stuff to be shown only in the answer key - like explanatory margin figures
\fi 

\begin{parts}
	\part[3] Find the area of $R$.
\insertQR{QRC}
\begin{solution}[\halfpage]
		R = Area under line ($A_1$) - Area under parabola ($A_2$)
		\begin{align}
				\text{A}_1 &= \int_0^2 2x\ud x 					\\
						   &= 4									
		\end{align}
		\begin{align}			
				\text{A}_2 &= \int_0^2 x^2\ud x 				\\
						   &= \left[\dfrac{x^3}{3}\right]_0^2	\\
						   &= \dfrac{8}{3}
		\end{align}
		\begin{align}
				\text{R} &= \text{A}_1 - \text{A}_2				\\
						 &= \dfrac{4}{3}
		\end{align}
	\end{solution}

	\part[3] The region $R$ is the base of a solid. For this solid, at each $x$ the cross section perpendicular to the x-axis has an area $A(x)=\sin{\dfrac{\pi}{2}x}$. Find the volume of the solid.
\insertQR{QRC}
\begin{solution}[\halfpage]
	    Let Volume of the solid be $V$.
	    \begin{align}
	    		\text{V} &= \int \sin \left(\dfrac{\pi}{2}x\right)\ud x							\\
	    				 &= \left[\dfrac{- \cos \left(\frac{\pi}{2}x\right)}
	    				 	{\frac{\pi}{2}}\right]_0^2	\\
	    				 &= \left(-\dfrac{2}{\pi}\cos \pi + \dfrac{2}{\pi}\cos 0\right)			\\
	    				 &= \dfrac{4}{\pi}
	    \end{align}
	\end{solution}
	
\newpage
	\part[2] Another solid has the same base $R$. For this solid, the cross sections perpendicular to the y-axis are squares. Write. but do not evaluate, an integral expression for the volume of the solid.
\insertQR{QRC}
\begin{solution}[\halfpage]
	Cross-sections are squares means height of the solid is equal to the base for any given cross-section. Further, since the areas are perpendicular to the $y$-axis, the integral for Volume $V$ would be with respect to an infinitesimal width of $\ud y$
		\begin{align}
				\text{V} = \int_0^2 \left(\frac{y}{2} - \sqrt{y}\right)^2\ud y
		\end{align}
	\end{solution}
\end{parts}

