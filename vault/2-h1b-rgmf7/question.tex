% This is an empty shell file placed for you by the 'examiner' script.
% You can now fill in the TeX for your question here.

% Now, down to brasstacks. ** Writing good solutions is an Art **. 
% Eventually, you will find your own style. But here are some thoughts 
% to get you started: 
%
%   1. Write the solution as if you are writing it for your favorite
%      14-17 year old to help him/her understand. Could be your nephew, 
%      your niece, a cousin perhaps or probably even you when you 
%      were that age. Just write for them.
%
%   2. Use margin-notes to "talk" to students about the critical insights
%      in the question. The tone can be - in fact, should be - informal
%
%   3. Don't shy away from creating margin-figures you think will help
%      students understand. Yes, it is a little more work per question. 
%      But the question & solution will be written only once. Make that
%      attempt at writing a solution count.
%
%   4. At the same time, do not be too verbose. A long solution can
%      - at first sight - make the student think, "God, that is a lot to know".
%      Our aim is not to scare students. Rather, our aim should be to 
%      create many "Aha!" moments everyday in classrooms around the world
% 
%   5. Ensure that there are *no spelling mistakes anywhere*. We are an 
%      education company. Bad spellings suggest that we ourselves 
%      don't have any education. Also, use American spellings by default
% 
%   6. If a question has multiple parts, then first delete lines 40-41
%   7. If a question does not have parts, then first delete lines 43-69
\question  Let $R$ be the region in the first quadrant enclosed by the graphs of $y=2x$ and $y=x^2$, \asif

\begin{marginfigure}
% 1. Definition of characteristic points
\figinit{pt}
\def\Xmin{-13.33333}
\def\Ymin{.00380}
\def\Xmax{66.66666}
\def\Ymax{80.00380}
\def\Xori{13.33333}
\def\Yori{-.00380}
\figpt0:(\Xori,\Yori)
\figpt 100:$y=2x$(80,64)
\figpt 101:$y=x^2$(80,80)
\figpt 102:$R$(33,9)
% 2. Creation of the graphical file
\figdrawbegin{}
\def\Xmaxx{\Xmax} % To customize the position
\def\Ymaxx{\Ymax} % of the arrow-heads of the axes.
\figset arrowhead(length=4, fillmode=yes) % styling the arrowheads
\figdrawaxes 0(\Xmin, \Xmaxx, \Ymin, \Ymaxx)
\figdrawlineC(
0 -12.80441,
2.75862 -10.15601,
5.51724 -7.50761,
8.27586 -4.85920,
11.03448 -2.21080,
13.79310 .43759,
16.55172 3.08599,
19.31034 5.73439,
22.06896 8.38280,
24.82758 11.03120,
27.58620 13.67960,
30.34482 16.32800,
33.10344 18.97640,
35.86206 21.62480,
38.62068 24.27321,
41.37931 26.92161,
44.13793 29.57001,
46.89655 32.21841,
49.65517 34.86681,
52.41379 37.51522,
55.17241 40.16362,
57.93103 42.81202,
60.68965 45.46042,
63.44827 48.10882,
66.20689 50.75722,
68.96551 53.40563,
71.72413 56.05403,
74.48275 58.70243,
77.24137 61.35083,
79.99999 63.99923
)
\figdrawlineC(
0 3.19634,
2.75862 2.00913,
5.51724 1.09589,
8.27586 .45662,
11.03448 .09132,
13.79310 0,
16.55172 .18264,
19.31034 .63926,
22.06896 1.36986,
24.82758 2.37442,
27.58620 3.65296,
30.34482 5.20547,
33.10344 7.03196,
35.86206 9.13242,
38.62068 11.50684,
41.37931 14.15525,
44.13793 17.07762,
46.89655 20.27397,
49.65517 23.74429,
52.41379 27.48858,
55.17241 31.50684,
57.93103 35.79908,
60.68965 40.36529,
63.44827 45.20547,
66.20689 50.31963,
68.96551 55.70776,
71.72413 61.36986,
74.48275 67.30593,
77.24137 73.51598,
79.99999 79.99999
)
\figdrawend
% 3. Writing text on the figure
\figvisu{\figBoxA}{}{%
\figptsaxes 1:0(\Xmin, \Xmaxx, \Ymin, \Ymaxx)
% Points 1 and 2 are the end points of the arrows
\figwritee 1:(5pt)     \figwriten 2:(5pt)
\figptsaxes 1:0(\Xmin, \Xmax, \Ymin, \Ymax)
\figwritee 100:(2)
\figwriten 101:(2)
\figwriten 102:(1)
}
\centerline{\box\figBoxA}

\end{marginfigure}

\ifprintanswers
  % stuff to be shown only in the answer key - like explanatory margin figures
\fi 

\begin{parts}
	\part[3] Find the area of $R$.
\insertQR{QRC}
\begin{solution}[\halfpage]
   The two curves intersect at points where,
   \begin{align}
       y = 2x &= x^2 \\
       \Rightarrow x\cdot(x-2) &= 0 \text{ or } x = 0,2
   \end{align}
   The required area - $R$ - is therefore, 
   \begin{align}
      R &= \int_0^2 2x\ud x - \int_0^2 x^2\ud x \\
        &= \left[ 2\cdot\dfrac{x^2}{2}\right]_0^2 - \left[ \dfrac{x^3}{3}\right]_0^2 \\
        &= 4 - \dfrac{8}{3} = \dfrac{4}{3}
   \end{align}
		
\end{solution}

	\part[3] The region $R$ is the base of a solid. For this solid, at each $x$ the cross section perpendicular to the x-axis has an area $A(x)=\sin{\dfrac{\pi}{2}x}$. Find the volume of the solid.
\insertQR{QRC}
\begin{solution}[\halfpage]
	    Let Volume of the solid be $V$.
	    \begin{align}
	    		\text{V} &= \int \sin \left(\dfrac{\pi}{2}x\right)\ud x							\\
	    				 &= \left[\dfrac{- \cos \left(\frac{\pi}{2}x\right)}
	    				 	{\frac{\pi}{2}}\right]_0^2	\\
	    				 &= \left(-\dfrac{2}{\pi}\cos \pi + \dfrac{2}{\pi}\cos 0\right)			\\
	    				 &= \dfrac{4}{\pi}
	    \end{align}
	\end{solution}
	
\newpage
	\part[2] Another solid has the same base $R$. For this solid, the cross sections perpendicular to the y-axis are squares. Write. but do not evaluate, an integral expression for the volume of the solid.
\insertQR{QRC}
\begin{solution}[\halfpage]
	Cross-sections are squares means height of the solid is equal to the base for any given cross-section. Further, since the areas are perpendicular to the $y$-axis, the integral for Volume $V$ would be with respect to an infinitesimal width of $\ud y$
		\begin{align}
				\text{V} = \int_0^2 \left(\frac{y}{2} - \sqrt{y}\right)^2\ud y
		\end{align}
	\end{solution}
\end{parts}

