% This is an empty shell file placed for you by the 'examiner' script.
% You can now fill in the TeX for your question here.

% Now, down to brasstacks. ** Writing good solutions is an Art **. 
% Eventually, you will find your own style. But here are some thoughts 
% to get you started: 
%
%   1. Write to be understood - but be crisp. Your own solution should not take 
%      more space than you will give to the student. Hence, if you take more than 
%      a half-page to write a solution, then give the student a full-page and so on...
%
%   2. Use margin-notes to "talk" to students about the critical insights
%      in the question. The tone can be - in fact, should be - informal
%
%   3. Don't shy away from creating margin-figures you think will help
%      students understand. Yes, it is a little more work per question. 
%      But the question & solution will be written only once. Make that
%      attempt at writing a solution count.
%      
%      3b. Use bc_to_fig.tex. Its an easier way to generate plots & graphs 
% 
%   4. Ensure that there are *no spelling mistakes anywhere*. We are an 
%      education company. Bad spellings suggest that we ourselves 
%      don't have any education. Also, use American spellings by default
% 
%   5. If a question has multiple parts, then first delete lines 40-41
%   6. If a question does not have parts, then first delete lines 43-69
%   
%   7. Create versions of the question when possible. Use commands defined in 
%      tufte-tweaks.sty to do so. Its easier than you think

%\noprintanswers
%\setcounter{rolldice}{3}

\ifnumequal{\value{rolldice}}{0}{
  % variables 
  \renewcommand{\vbone}{6}
  \renewcommand{\vbtwo}{30,240}
}{
  \ifnumequal{\value{rolldice}}{1}{
    % variables 
    \renewcommand{\vbone}{7}
    \renewcommand{\vbtwo}{2,540,160}
  }{
    \ifnumequal{\value{rolldice}}{2}{
      % variables 
      \renewcommand{\vbone}{5}
      \renewcommand{\vbtwo}{25,200}
    }{
      % variables 
      \renewcommand{\vbone}{8}
      \renewcommand{\vbtwo}{29,030,400}
    }
  }
}

\gcalcexpr[0]\tp{\vbone + 2}
\gcalcexpr[0]\tq{\tp - 1}
\gcalcexpr[0]\tr{\tp - 2}

\question[3] At a buffet, guests get one type of soup and one type of dessert (or sweet-dish) alongwith $\vbone$ other 
main-course dishes. In how many ways can the buffet table be laid if the soup and the dessert should \textit{never} be
placed together? 

\insertQR{QRC}

\watchout

\ifprintanswers
\fi 

\begin{solution}[\mcq]
  Including the soup and the dessert, there are a total of $\tp = (\vbone + 2)$ dishes. The number of ways, therefore, 
  of laying the table in which the dessert and the soup are not placed together is 
  \begin{align}
  	N &= N_{\texttt{all possible ways}} - N_{\texttt{soup \& dessert together}} \\
  	  &= \tp ! - 2!\cdot\left( \tp - 2 + 1\right) ! = \tp ! - 2!\cdot\tq ! = \tq !\cdot(\tp - 2) \\
  	  &= \tp!\cdot\tr = \vbtwo
  \end{align}
\end{solution}
