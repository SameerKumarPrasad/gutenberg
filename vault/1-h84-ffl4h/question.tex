% This is an empty shell file placed for you by the 'examiner' script.
% You can now fill in the TeX for your question here.

% Now, down to brasstacks. ** Writing good solutions is an Art **. 
% Eventually, you will find your own style. But here are some thoughts 
% to get you started: 
%
%   1. Write to be understood - but be crisp. Your own solution should not take 
%      more space than you will give to the student. Hence, if you take more than 
%      a half-page to write a solution, then give the student a full-page and so on...
%
%   2. Use margin-notes to "talk" to students about the critical insights
%      in the question. The tone can be - in fact, should be - informal
%
%   3. Don't shy away from creating margin-figures you think will help
%      students understand. Yes, it is a little more work per question. 
%      But the question & solution will be written only once. Make that
%      attempt at writing a solution count.
%      
%      3b. Use bc_to_fig.tex. Its an easier way to generate plots & graphs 
% 
%   4. Ensure that there are *no spelling mistakes anywhere*. We are an 
%      education company. Bad spellings suggest that we ourselves 
%      don't have any education. Also, use American spellings by default
% 
%   5. If a question has multiple parts, then first delete lines 40-41
%   6. If a question does not have parts, then first delete lines 43-69
%   
%   7. Create versions of the question when possible. Use commands defined in 
%      tufte-tweaks.sty to do so. Its easier than you think

%\printrubric
%\noprintanswers
%\setcounter{rolldice}{3}

\ifnumequal{\value{rolldice}}{0}{
  % variables 
  \renewcommand{\vbone}{5}
  \renewcommand{\vbtwo}{12}
  \renewcommand{\vbthree}{26}
  \renewcommand{\vbfour}{4}
}{
  \ifnumequal{\value{rolldice}}{1}{
    % variables 
    \renewcommand{\vbone}{6}
    \renewcommand{\vbtwo}{8}
    \renewcommand{\vbthree}{11}
    \renewcommand{\vbfour}{6}
  }{
    \ifnumequal{\value{rolldice}}{2}{
      % variables 
      \renewcommand{\vbone}{3}
      \renewcommand{\vbtwo}{4}
      \renewcommand{\vbthree}{12}
      \renewcommand{\vbfour}{6}
    }{
      % variables 
      \renewcommand{\vbone}{12}
      \renewcommand{\vbtwo}{5}
      \renewcommand{\vbthree}{14}
      \renewcommand{\vbfour}{7}
     }
  }
}

\gcalcHypotenuse[0]{\vbfive}{\vbone}{\vbtwo}
\gcalcexpr[0]{\vbsix}{\vbfour * \vbfive }
\gcalcexpr[0]{\vbseven}{\vbthree - \vbsix}
\gcalcexpr[0]{\vbeight}{\vbsix + \vbthree}

\question[3] Find the equation of the straight lines parallel to $\vbone x - \vbtwo y + \vbthree = 0$ 
and at a distance of $\vbfour$ units from it

\insertQR[-15pt]{QRC}

\watchout

\ifprintanswers
\fi 

\begin{solution}[\halfpage]
	The distance between two parallel lines - $Ax + By + C_1 = 0$ and $Ax+By+C_2 = 0$ is given by 
	$d = \dfrac{\vert C_1 - C_2 \vert}{\sqrt{A^2+ B^2}}$. Note that there would be \textit{two} such lines, 
	one on either side of the original line. So,
	\begin{align}
		\vbfour &= \dfrac{\vert \vbthree - C_2 \vert}{\sqrt{\vbone^2 + \vbtwo^2}} \\
		\Rightarrow \vbthree - C_2 &= \vbsix, \text{ for when } C_2 < \vbthree \\
		\text{ and } C_2 - \vbthree &= \vbsix, \text{ for when } C_2 > \vbthree
	\end{align}
	Solving (2) and (3) above, we get 
	\begin{align}
		C_2 &= \vbseven \text{ and } C_2 = \vbeight
	\end{align}
	
	Hence, the two lines are $\vbone x - \vbtwo y + \vbseven = 0$ and $\vbone x - \vbtwo y + \vbeight = 0$
\end{solution}

\ifprintrubric
  \begin{table}
  	\begin{tabular}{ p{5cm}p{5cm} }
  		\toprule % in brief (4-6 words), what should a grader be looking for for insights & formulations
  		  \sc{\textcolor{blue}{Insight}} & \sc{\textcolor{blue}{Formulation}} \\ 
  		\midrule % ***** Insights & formulations ******
        Two lines satisfy the condition & Used formula for distance b/w parallel lines \\
  		\toprule % final numerical answers for the various versions
        \sc{\textcolor{blue}{If question has $\ldots$}} & \sc{\textcolor{blue}{Final answer}} \\
  		\midrule % ***** Numerical answers (below) **********
        $5x - 12y + 26 = 0$ & $5x - 12y - 26 = 0$ and $5x - 12y + 78 = 0$ \\
        $6x - 8y + 11 = 0$ & $6x - 8y - 49 = 0$ and $6x - 8y + 71 = 0$ \\
        $3x - 4y + 12 = 0$ & $3x - 4y-18 =0$ and $3x - 4y+42 = 0$ \\
        $12x - 5y + 14 = 0$ & $12x - 5y - 77 = 0$ and $12x - 5y + 105 = 0$ \\
  		\bottomrule
  	\end{tabular}
  \end{table}
\fi
