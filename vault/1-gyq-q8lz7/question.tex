% This is an empty shell file placed for you by the 'examiner' script.
% You can now fill in the TeX for your question here.

% Now, down to brasstacks. ** Writing good solutions is an Art **. 
% Eventually, you will find your own style. But here are some thoughts 
% to get you started: 
%
%   1. Write the solution as if you are writing it for your favorite
%      14-17 year old to help him/her understand. Could be your nephew, 
%      your niece, a cousin perhaps or probably even you when you 
%      were that age. Just write for them.
%
%   2. Use margin-notes to "talk" to students about the critical insights
%      in the question. The tone can be - in fact, should be - informal
%
%   3. Don't shy away from creating margin-figures you think will help
%      students understand. Yes, it is a little more work per question. 
%      But the question & solution will be written only once. Make that
%      attempt at writing a solution count.
%
%   4. At the same time, do not be too verbose. A long solution can
%      - at first sight - make the student think, "God, that is a lot to know".
%      Our aim is not to scare students. Rather, our aim should be to 
%      create many "Aha!" moments everyday in classrooms around the world
% 
%   5. Ensure that there are *no spelling mistakes anywhere*. We are an 
%      education company. Bad spellings suggest that we ourselves 
%      don't have any education. And, use American spellings

\question[3] A small aircraft encounters bad weather a third of the way
into a \SI{600}{\kilo\meter} flight. As a result, it has to slow down by
\SI{50}{\kilo\meter\per\hour} and it therefore lands 40 minutes later than usual. How
long does the aircraft take to fly the same route on a normal day?

\texttt{Hint: $1225 = 35^2$}

\ifprintanswers
  % stuff to be shown only in the answer key - like explanatory margin figures
  \marginnote[1cm]{If you were mindful of the units, then you would have realized
  that 40 minutes = $\frac{2}{3}$ hour}
  \marginnote[0.3cm]{ Lesson: Units can change the whole calculation. So, always write them}
\fi 

\begin{solution}
	Let $S$ be the aircraft's speed in \si{km\per hour} and $T$ the time - in hours -
	the plane ordinarily takes for the 600 km trip
	
	Which means
	\begin{align}
		T &= \dfrac{\SI{600}{\kilo\meter}}{S}
	\end{align}
	
	On this day, however,
	\begin{align}
		T + \dfrac{40 \si{\minute}}{\SI{60}{\minute\per\hour}} &= \dfrac{200}{S} + \dfrac{400}{S-50}
	\end{align}
	
	Using (1) and (2), we get 
	\begin{align}
		\dfrac{600}{S} + \dfrac{2}{3} &= \dfrac{200}{S} + \dfrac{400}{S-50} \\
		\Rightarrow \dfrac{1800+2S}{3S} &= \dfrac{200S-10,000+400S}{S\cdot(S-50)} \\
		\Rightarrow (1800+2S)\cdot(S-50) &= (1800S - 30,000) \\
		\Rightarrow 2S^2-100S-60,000 &= 0 \\
		\Rightarrow S^2-50S-30,000 &= 0 \\
		\Rightarrow S &= \dfrac{50\pm\sqrt{50^2-4\times-30,000}}{2} \\
		              &= \dfrac{50\pm\sqrt{122500}}{2}
	\end{align}
	Speed can only be positive. And therefore
	\begin{align}
		S &= \dfrac{50+350}{2} = \SI{200}{\kilo\meter\per\hour}
	\end{align}
	Which means, that on a normal day, the aircraft will take 
	$\dfrac{\SI{600}{\kilo\meter}}{\SI{200}{\kilo\meter\per\hour}}$ = 3 hours
	
\end{solution}
