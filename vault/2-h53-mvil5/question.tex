% This is an empty shell file placed for you by the 'examiner' script.
% You can now fill in the TeX for your question here.

% Now, down to brasstacks. ?? Writing good solutions is an Art ??. 
% Eventually, you will find your own style. But here are some thoughts 
% to get you started: 
%
%   1. Write to be understood - but be crisp. Your own solution should not take 
%      more space than you will give to the student. Hence, if you take more than 
%      a half-page to write a solution, then give the student a full-page and so on...
%
%   2. Use margin-notes to "talk" to students about the critical insights
%      in the question. The tone can be - in fact, should be - informal
%
%   3. Don't shy away from creating margin-figures you think will help
%      students understand. Yes, it is a little more work per question. 
%      But the question & solution will be written only once. Make that
%      attempt at writing a solution count.
%      
%      3b. Use bc_to_fig.tex. Its an easier way to generate plots & graphs 
% 
%   4. Ensure that there are ?no spelling mistakes anywhere?. We are an 
%      education company. Bad spellings suggest that we ourselves 
%      don't have any education. Also, use American spellings by default
% 
%   5. If a question has multiple parts, then first delete lines 40-41
%   6. If a question does not have parts, then first delete lines 43-69
%   
%   7. Create versions of the question when possible. Use commands defined in 
%      tufte-tweaks.sty to do so. Its easier than you think

% \noprintanswers
% \setcounter{rolldice}{0}

\ifnumequal{\value{rolldice}}{0}{
  % variables 
  \renewcommand{\vbone}{WORLD}
  \renewcommand{\vbtwo}{top}
  \renewcommand{\vbthree}{bottom}
  \renewcommand{\vbfour}{5!}
  \renewcommand{\vbfive}{7}
  \renewcommand{\vbsix}{113}
  \renewcommand{\vbseven}{}
  \renewcommand{\vbeight}{}
  \renewcommand{\vbnine}{}
  \renewcommand{\vbten}{}
}{
  \ifnumequal{\value{rolldice}}{1}{
    % variables 
    \renewcommand{\vbone}{GLOBE}
    \renewcommand{\vbtwo}{top}
    \renewcommand{\vbthree}{bottom}
    \renewcommand{\vbfour}{5!}
    \renewcommand{\vbfive}{49}
    \renewcommand{\vbsix}{71}
    \renewcommand{\vbseven}{}
    \renewcommand{\vbeight}{}
    \renewcommand{\vbnine}{}
    \renewcommand{\vbten}{}
  }{
    \ifnumequal{\value{rolldice}}{2}{
      % variables 
      \renewcommand{\vbone}{EARTH}
      \renewcommand{\vbtwo}{bottom}
      \renewcommand{\vbthree}{top}
      \renewcommand{\vbfour}{5!}
      \renewcommand{\vbfive}{27}
      \renewcommand{\vbsix}{93}
      \renewcommand{\vbseven}{}
      \renewcommand{\vbeight}{}
      \renewcommand{\vbnine}{}
      \renewcommand{\vbten}{}
    }{
      % variables 
      \renewcommand{\vbone}{PLANET}
      \renewcommand{\vbtwo}{top}
      \renewcommand{\vbthree}{bottom}
      \renewcommand{\vbfour}{6!}
      \renewcommand{\vbfive}{189}
      \renewcommand{\vbsix}{531}
      \renewcommand{\vbseven}{}
      \renewcommand{\vbeight}{}
      \renewcommand{\vbnine}{}
      \renewcommand{\vbten}{}
    }
  }
}

\question[3] The letters in the word $\vbone$ are written in all possible 
arrangements. If all these arrangements were ranked alphabetically, where
would $\vbone$ be ranked starting from the $\vbtwo$.

\insertQR{QRC}

\watchout

\ifprintanswers
  % stuff to be shown only in the answer key - like explanatory margin figures
  \begin{marginfigure}
    \figinit{pt}
      \figpt 100:(0,0)
      \figpt 101:(0,0)
    \figdrawbegin{}
      \figdrawline [100,101]
    \figdrawend
    \figvisu{\figBoxA}{}{%
    }
    \centerline{\box\figBoxA}
  \end{marginfigure}
\fi 

\begin{solution}[\halfpage]
  Since $\vbone$ would be closer to the $\vbthree$ than the $\vbtwo$, 
  we can attempt to compute its rank from the $\vbthree$ and subtract 
  from the total number of permutations of $\vbone$. \\
  Pattern of words that would appear before $\vbone$ starting from the
  $\vbthree$
  would be:
  
  \begin{align}
    &\textit{Pattern} &\textit{Combinations} \\
    \ifnumequal{\value{rolldice}}{0}{
      &\text{W R ? ? ?} &\Rightarrow 3! \\
      &\text{W O R D L} &\Rightarrow 1 \\
      &\text{W O R L D} 
    }{
      \ifnumequal{\value{rolldice}}{1}{
        &\text{L ? ? ? ?} &\Rightarrow 4! \\
        &\text{O ? ? ? ?} &\Rightarrow 4! \\
        &\text{G L O E B} &\Rightarrow 1 \\
        &\text{G L O B E}
      }{
        \ifnumequal{\value{rolldice}}{2}{
          &\text{A ? ? ? ?} &\Rightarrow 4! \\
          &\text{E A H ? ?} &\Rightarrow 2! \\
          &\text{E A R H T} &\Rightarrow 1 \\
          &\text{E A R T H}
        }{
          &\text{T ? ? ? ? ?} &\Rightarrow 5! \\
          &\text{P T ? ? ? ?} &\Rightarrow 4! \\
          &\text{P N ? ? ? ?} &\Rightarrow 4! \\
          &\text{P L T ? ? ?} &\Rightarrow 3! \\
          &\text{P L N ? ? ?} &\Rightarrow 3! \\
          &\text{P L E ? ? ?} &\Rightarrow 3! \\
          &\text{P L A T ? ?} &\Rightarrow 2! \\
          &\text{P L A N T E} &\Rightarrow 1 \\
          &\text{P L A N E T}
        }
      }        
    }  
  \end{align}
  Subtracting from all possible combinations we get a rank of
  \begin{align}
	\text{Rank} &= \vbfour - \vbfive \\
		        &= \vbsix
  \end{align}

\end{solution}
