% This is an empty shell file placed for you by the 'examiner' script.
% You can now fill in the TeX for your question here.

% Now, down to brasstacks. ** Writing good solutions is an Art **. 
% Eventually, you will find your own style. But here are some thoughts 
% to get you started: 
%
%   1. Write to be understood - but be crisp. Your own solution should not take 
%      more space than you will give to the student. Hence, if you take more than 
%      a half-page to write a solution, then give the student a full-page and so on...
%
%   2. Use margin-notes to "talk" to students about the critical insights
%      in the question. The tone can be - in fact, should be - informal
%
%   3. Don't shy away from creating margin-figures you think will help
%      students understand. Yes, it is a little more work per question. 
%      But the question & solution will be written only once. Make that
%      attempt at writing a solution count.
%      
%      3b. Use bc_to_fig.tex. Its an easier way to generate plots & graphs 
% 
%   4. Ensure that there are *no spelling mistakes anywhere*. We are an 
%      education company. Bad spellings suggest that we ourselves 
%      don't have any education. Also, use American spellings by default
% 
%   5. If a question has multiple parts, then first delete lines 40-41
%   6. If a question does not have parts, then first delete lines 43-69
%   
%   7. Create versions of the question when possible. Use commands defined in 
%      tufte-tweaks.sty to do so. Its easier than you think

%\noprintanswers
%\setcounter{rolldice}{1}

\ifnumequal{\value{rolldice}}{0}{
  % variables 
  \renewcommand{\vbone}{-1}
  \renewcommand{\vbtwo}{0}
  \renewcommand{\vbthree}{7}
  \renewcommand{\vbfour}{2}
  \renewcommand{\vbfive}{5}
  \renewcommand{\vbsix}{-2}
  \renewcommand{\vbseven}{\frac{20}{7}}
  \renewcommand{\vbeight}{\frac{11}{7}}
}{
  \ifnumequal{\value{rolldice}}{1}{
    % variables 
    \renewcommand{\vbone}{1}
    \renewcommand{\vbtwo}{0}
    \renewcommand{\vbthree}{-7}
    \renewcommand{\vbfour}{2}
    \renewcommand{\vbfive}{-5}
    \renewcommand{\vbsix}{-2}
    \renewcommand{\vbseven}{-\frac{20}{7}}
    \renewcommand{\vbeight}{\frac{11}{7}}
  }{
    \ifnumequal{\value{rolldice}}{2}{
      % variables 
      \renewcommand{\vbone}{-1}
      \renewcommand{\vbtwo}{0}
      \renewcommand{\vbthree}{7}
      \renewcommand{\vbfour}{-2}
      \renewcommand{\vbfive}{5}
      \renewcommand{\vbsix}{2}
      \renewcommand{\vbseven}{\frac{20}{7}}
      \renewcommand{\vbeight}{-\frac{11}{7}}
    }{
      % variables 
      \renewcommand{\vbone}{1}
      \renewcommand{\vbtwo}{0}
      \renewcommand{\vbthree}{-7}
      \renewcommand{\vbfour}{-2}
      \renewcommand{\vbfive}{-5}
      \renewcommand{\vbsix}{2}
      \renewcommand{\vbseven}{-\frac{20}{7}}
      \renewcommand{\vbeight}{-\frac{11}{7}}
    }
  }
}

\gcalcexpr[0]{\mxone}{(\vbone + \vbthree) / 2}
\gcalcexpr[0]{\myone}{(\vbtwo + \vbfour) / 2}
\gcalcexpr[0]{\mxtwo}{(\vbthree + \vbfive) / 2}
\gcalcexpr[0]{\mytwo}{(\vbfour + \vbsix) / 2}
\gcalcexpr[0]{\mxthree}{(\vbone + \vbfive) / 2}
\gcalcexpr[0]{\mythree}{(\vbtwo + \vbsix) / 2}

\gcalcexpr[0]{\mab}{-(\vbthree - \vbone)/(\vbfour - \vbtwo)}
\gcalcexpr[2]{\mbc}{-(\vbfive - \vbthree)/(\vbsix - \vbfour) }
\gcalcexpr[0]{\mca}{-(\vbone - \vbfive)/(\vbtwo - \vbsix)} 

\gcalcexpr[0]{\cone}{\myone - (\mab * \mxone)}
\gcalcexpr[0]{\ctwo}{\mytwo - (\mbc * \mxtwo)}
\gcalcexpr[0]{\cthree}{\mythree - (\mca * \mxthree)}

\question A triangle has vertices $A = (\vbone, \vbtwo)$, $B = (\vbthree, \vbfour)$ and 
$C = (\vbfive, \vbsix)$ - \asif. $X$, $Y$ and $Z$ are the mid-points of the three sides 


  \begin{marginfigure}[-47pt]
    \figinit{pt}
      \figpt 10:$A$(-10, 0)
      \figpt 20:$B$(70, 20)
      \figpt 30:$C$(50, -20)
      \figpt 11: $X$(30,10)
      \figpt 21: $Y$(60,0)
      \figpt 31: $Z$(20,-10)
    \figdrawbegin{}
      \figdrawline [10, 20]
      \figdrawline [20, 30]
      \figdrawline [30, 10]
    \figdrawend
    \figvisu{\figBoxA}{}{%
      \figwritesw 10:(5pt)
      \figwritene 20:(5pt)
      \figwritese 30:(5pt)
      \figset write(mark=$\bullet$)
      \figwriten 11:(4)
      \figwritee 21:(4)
      \figwrites 31:(4)
    }
    \centerline{\box\figBoxA}
  \end{marginfigure}

\watchout
\begin{parts}
	\part[1] Find the coordinates of the mid-points $X$, $Y$ and $Z$
	
	\insertQR[-30pt]{QRC}
\begin{solution}[\mcq]
		The mid-point of a line-segment joining two points $(x_1, y_1)$ and $(x_2, y_2)$ is given 
		by $\left( \frac{x_1 + x_2}{2}, \frac{y_1 + y_2}{2}\right)$
		
		And so, the coordinates of $X$, $Y$ and $Z$ would be 
		\begin{align}
			X &= \left( \frac{\vbone + \vbthree}{2}, \frac{\vbtwo + \vbfour}{2} \right) = (\mxone, \myone) \\
			Y &= \left( \frac{\vbthree + \vbfive}{2}, \frac{\vbfour + \vbsix}{2} \right) = (\mxtwo, \mytwo) \\
			Z &= \left( \frac{\vbone + \vbfive}{2}, \frac{\vbtwo + \vbsix}{2} \right) = (\mxthree, \mythree)
		\end{align}
	\end{solution}
	
	\part[3] Find the equations of the \textit{perpendicular} bisectors of the three sides of the triangle
	
	\insertQR[-15pt]{QRC}
\begin{solution}[\halfpage]
		If $AB_\perp$, $BC_\perp$ and $CA_\perp$ be the perpendicular bisectors of $AB$, $BC$ and $CA$ respectively, then 
		their slopes are given by
		\begin{align}
			m_{AB,\perp} &= -\frac{1}{m_{AB}} = -\dfrac{\vbthree - \vbone}{\vbfour - \vbtwo} = \mab \\
			m_{BC,\perp} &= -\frac{1}{m_{BC}} = -\dfrac{\vbfive - \vbthree}{\vbsix - \vbfour} = \mbc \\
			m_{CA,\perp} &= -\frac{1}{m_{CA}} = -\dfrac{\vbone - \vbfive}{\vbtwo - \vbsix} = \mca
		\end{align}
		And their equations, therefore, would be 
		\begin{align}
			AB_\perp &= \dfrac{y-\myone}{x-\mxone} = \mab \Rightarrow y = \mab x + \cone \\
			BC_\perp &= \dfrac{y-\mytwo}{x-\mxtwo} = \mbc \Rightarrow y = \mbc x + \ctwo \\
			CA_\perp &= \dfrac{y-\mythree}{x-\mxthree} = \mca \Rightarrow y = \mca x + \cthree					
		\end{align}
	\end{solution}
	
	\part[2] Are the perpendicular bisectors concurrent? If yes, then find their point of intersection
	
	\insertQR[-15pt]{QRC}
\begin{solution}[\mcq]
		$AB_\perp$ and $BC_\perp$ intersect when $ y = \mab x + \cone = \mbc x + \ctwo \Rightarrow x = \vbseven$ 
		and $y = \vbeight$
		
		$(\vbseven, \vbeight)$ also satisfies the equation of $CA_\perp$. You can check this by plugging in 
		$(x,y) = (\vbseven, \vbeight)$ into the equation for $CA_\perp$. Which means that \textit{all three} 
		perpendicular bisectors intersect at one point \textit{and} that point is $(\vbseven, \vbeight)$
	\end{solution}
\end{parts}

