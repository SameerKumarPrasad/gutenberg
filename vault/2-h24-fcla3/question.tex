% This is an empty shell file placed for you by the 'examiner' script.
% You can now fill in the TeX for your question here.

% Now, down to brasstacks. ** Writing good solutions is an Art **. 
% Eventually, you will find your own style. But here are some thoughts 
% to get you started: 
%
%   1. Write the solution as if you are writing it for your favorite
%      14-17 year old to help him/her understand. Could be your nephew, 
%      your niece, a cousin perhaps or probably even you when you 
%      were that age. Just write for them.
%
%   2. Use margin-notes to "talk" to students about the critical insights
%      in the question. The tone can be - in fact, should be - informal
%
%   3. Don't shy away from creating margin-figures you think will help
%      students understand. Yes, it is a little more work per question. 
%      But the question & solution will be written only once. Make that
%      attempt at writing a solution count.
%
%   4. At the same time, do not be too verbose. A long solution can
%      - at first sight - make the student think, "God, that is a lot to know".
%      Our aim is not to scare students. Rather, our aim should be to 
%      create many "Aha!" moments everyday in classrooms around the world
% 
%   5. Ensure that there are *no spelling mistakes anywhere*. We are an 
%      education company. Bad spellings suggest that we ourselves 
%      don't have any education. Also, use American spellings by default
% 
%   6. If a question has multiple parts, then first delete lines 40-41
%   7. If a question does not have parts, then first delete lines 43-69

\question[4] The sum of three numbers which form a geometric progression is 13 and the sum of their squares is 91. Find the numbers.

\insertQR{QRC}

\ifprintanswers
  % stuff to be shown only in the answer key - like explanatory margin figures
\fi 

\begin{solution}[\fullpage]
  Let the three numbers be $\dfrac{a}{r}$, $a$ and $ar$. The conditions we have been given are:
  \begin{align}
    \dfrac{a}{r} + a + ar                       &= 13 \\
    \dfrac{a^2}{r^2} + a^2 + {ar}^2 &= 91
  \end{align}

  Consider equation (1) and take square on both sides,
  \begin{align}
    a^2(\dfrac{1}{r} + 1 + r)^2 &= 169
  \end{align}
  
  Subtract equation (2) from equation (3),
  \begin{align}
    a^2(\dfrac{1}{r} + 1 + r)^2 - a^2(\dfrac{1}{r^2} + 1 + r^2) &= 169 - 91 \\
                                                                 a(\dfrac{a}{r} + a + ar) &= 39
  \end{align}
  
  Substitute equation (1) into equation (5),
  \begin{align}
    a = 3
  \end{align}  
  
  Substitute the value of $a$ into equation (1),
  \begin{align}
    \dfrac{3}{r} + 3 + 3r &= 13 \\
    (3r - 1)(r - 3)                &= 0
  \end{align}  
  
  $r$ may be equal to $\dfrac{1}{3}$ or $3$. Either way the numbers would be $1$, $3$ and $9$.
  
\end{solution}

