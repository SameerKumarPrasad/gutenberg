% This is an empty shell file placed for you by the 'examiner' script.
% You can now fill in the TeX for your question here.

% Now, down to brasstacks. ** Writing good solutions is an Art **. 
% Eventually, you will find your own style. But here are some thoughts 
% to get you started: 
%
%   1. Write the solution as if you are writing it for your favorite
%      14-17 year old to help him/her understand. Could be your nephew, 
%      your niece, a cousin perhaps or probably even you when you 
%      were that age. Just write for them.
%
%   2. Use margin-notes to "talk" to students about the critical insights
%      in the question. The tone can be - in fact, should be - informal
%
%   3. Don't shy away from creating margin-figures you think will help
%      students understand. Yes, it is a little more work per question. 
%      But the question & solution will be written only once. Make that
%      attempt at writing a solution count.
%
%   4. At the same time, do not be too verbose. A long solution can
%      - at first sight - make the student think, "God, that is a lot to know".
%      Our aim is not to scare students. Rather, our aim should be to 
%      create many "Aha!" moments everyday in classrooms around the world
% 
%   5. Ensure that there are *no spelling mistakes anywhere*. We are an 
%      education company. Bad spellings suggest that we ourselves 
%      don't have any education. Also, use American spellings by default
% 
%   6. If a question has multiple parts, then first delete lines 40-41
%   7. If a question does not have parts, then first delete lines 43-69

\question Chords $AB$ and $CD$ are of lengths 5cm and 11cm respectively. They are also 
parallel to each other. If the distance between them (the chords) is 3cm, then what is 
the radius of the circle? 

\insertQR{}

  % stuff to be shown only in the answer key - like explanatory margin figures
  \begin{marginfigure}
    \figinit{pt}
      \figpt 100: $O$(50,50)
      \figpt 200: (95,50) % ref. pt 
      \figptrot 101: $D$= 200 /100,20/
      \figptrot 102: $B$= 200 /100,50/
      \figptrot 103: $C$= 200 /100,160/
      \figptrot 104: $A$= 200 /100,130/
      \figptorthoprojline 300:$M$= 100 /102,104/
      \figptorthoprojline 301:$N$= 100 /101,103/
    \figdrawbegin{}
      \figdrawcirc 100(45)
      \figdrawline [101,103]
      \figdrawline [102,104]
      \figdrawaltitude 5 [100,102,104]
      \figdrawaltitude 5 [100,101,103]
      \ifprintanswers
        \figset (dash=5)
        \figdrawline [100,101]
        \figdrawline [100,102]
      \fi
    \figdrawend
    \figvisu{\figBoxA}{}{%
      \figset write(mark=$\bullet$)
      \figwrites 100:(2)
      \figwritee 101:(3)
      \figwritee 102:(3)
      \figwritew 103:(3)
      \figwritew 104:(3)
      \figwritene 300:(3)
      \figwritene 301:(3)
    }
    \centerline{\box\figBoxA}
  \end{marginfigure}

\begin{solution}
	Here is what we know,
	\begin{align}
		MN &= 3cm \\
		OB^2 &= OM^2 + MB^2 \\
		OD^2 &= ON^2 + ND^2 \\
		OB = OD &= \text{ Radius } \\
		MB &= \frac{1}{2}AB = 2.5cm \\
		ND &= \frac{1}{2}CD = 5.5cm
	\end{align}
	And therefore, 
	\begin{align}
		OM^2 + MB^2 = (ON + 3)^2 + 2.5^2 &= ON^2 + 5.5^2 \\
		\Rightarrow (ON+3)^2 - ON^2 &= 30.25 - 6.25 = 24 \\
		\Rightarrow \underbrace{(2\cdot ON + 3)\cdot 3}_{a^2-b^2 = (a-b)(a+b)} &= 24 \Rightarrow ON = 2.5cm \\
		\therefore OD = \text{ Radius } &= \sqrt{ON^2 + ND^2} \\ 
		&= \sqrt{2.5^2 + 5.5^2} = 6.04cm
	\end{align}
\end{solution}

