% This is an empty shell file placed for you by the 'examiner' script.
% You can now fill in the TeX for your question here.

% Now, down to brasstacks. ** Writing good solutions is an Art **. 
% Eventually, you will find your own style. But here are some thoughts 
% to get you started: 
%
%   1. Write the solution as if you are writing it for your favorite
%      14-17 year old to help him/her understand. Could be your nephew, 
%      your niece, a cousin perhaps or probably even you when you 
%      were that age. Just write for them.
%
%   2. Use margin-notes to "talk" to students about the critical insights
%      in the question. The tone can be - in fact, should be - informal
%
%   3. Don't shy away from creating margin-figures you think will help
%      students understand. Yes, it is a little more work per question. 
%      But the question & solution will be written only once. Make that
%      attempt at writing a solution count.
%
%   4. At the same time, do not be too verbose. A long solution can
%      - at first sight - make the student think, "God, that is a lot to know".
%      Our aim is not to scare students. Rather, our aim should be to 
%      create many "Aha!" moments everyday in classrooms around the world
% 
%   5. Ensure that there are *no spelling mistakes anywhere*. We are an 
%      education company. Bad spellings suggest that we ourselves 
%      don't have any education. And, use American spellings

\question[1] A gentleman buys Bank's cash certificates every year and each year their value exceeds the previous year's purchase value by Rs.300. After 20 years, he finds that the total value of the certificates purchased by him is Rs.83,000. Find the value of the certificates purchased by him in the 13$^{th}$ year.

\ifprintanswers
  % stuff to be shown only in the answer key - like explanatory margin figures
  \marginnote[2cm] {$\text{S}_{n} = \dfrac{n}{2}
  \left(2\text{a}+\left(\text{n}-1\right)\text{d}\right)$}
  \marginnote[3cm] {$\text{a}_{n} = \text{a} + \left(\text{n}-1\right)\text{d}$} 
\fi 

\begin{solution}
	Let the value of the certificate in the first year of purchase be \textit{a}. Applying the formula for the sum of terms of an Arthmetic Progression we get, 
	\begin{align}
			   83000 &= \dfrac{20}{2}\left(2\text{a}+19\times 300\right)\\
			   83000 &= 10\left(2\text{a} + 5700\right) \\
			\text{a} &= 1300   
	\end{align}
	Value of certificate purchased in the n$^{th}$ year,
	\begin{align}
		\text{a}_{13}&= 1300 + \left(12\times 300\right) \\
		\text{a}_{13}&= 4900
	\end{align}
	The certificate purchased in the 13$^{th}$ year was worth Rs.4,900.
\end{solution}
