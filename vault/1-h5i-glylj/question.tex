% This is an empty shell file placed for you by the 'examiner' script.
% You can now fill in the TeX for your question here.

% Now, down to brasstacks. ** Writing good solutions is an Art **. 
% Eventually, you will find your own style. But here are some thoughts 
% to get you started: 
%
%   1. Write the solution as if you are writing it for your favorite
%      14-17 year old to help him/her understand. Could be your nephew, 
%      your niece, a cousin perhaps or probably even you when you 
%      were that age. Just write for them.
%
%   2. Use margin-notes to "talk" to students about the critical insights
%      in the question. The tone can be - in fact, should be - informal
%
%   3. Don't shy away from creating margin-figures you think will help
%      students understand. Yes, it is a little more work per question. 
%      But the question & solution will be written only once. Make that
%      attempt at writing a solution count.
%
%   4. At the same time, do not be too verbose. A long solution can
%      - at first sight - make the student think, "God, that is a lot to know".
%      Our aim is not to scare students. Rather, our aim should be to 
%      create many "Aha!" moments everyday in classrooms around the world
% 
%   5. Ensure that there are *no spelling mistakes anywhere*. We are an 
%      education company. Bad spellings suggest that we ourselves 
%      don't have any education. Also, use American spellings by default
% 
%   6. If a question has multiple parts, then first delete lines 40-41
%   7. If a question does not have parts, then first delete lines 43-69

\question[4] Prove that $\dfrac{\cos{8A}\cos{5A}-\cos{12A}\cos{9A}}{\sin{8A}\cos{5A}+\cos{12A}\sin{9A}} = \tan{4A}$

\insertQR{QRC}

\ifprintanswers
\fi 

\begin{solution}[\halfpage]
	Given that,
	\begin{fullwidth}
	\begin{align}
		\fProdOfCos{x}{y} \\
		\fProdSinCos{x}{y}
     \end{align}
     , one can \textit{re-write} the above expression as 
     \begin{align}
     & \dfrac{\frac{1}{2}\cdot\left[\eProdOfCos{8A}{5A}\right]-\frac{1}{2}\cdot\left[\eProdOfCos{12A}{9A}\right]}
     	{\frac{1}{2}\cdot\left[\eProdSinCos{8A}{5A}\right] + \frac{1}{2}\cdot\left[\eProdSinCos{9A}{12A}\right]} \\
     	&= \dfrac{\cos 13A - \cos 21A}{\sin 13A + \sin 21A} \\
     	&= \dfrac{\eDiffOfCos{13A}{21A}}{\eSumOfSin{13A}{21A}} \\
     	&= \dfrac{\sin 4A}{\cos 4A} = \tan 4A
     \end{align}
     \end{fullwidth}
\end{solution}
