% This is an empty shell file placed for you by the 'examiner' script.
% You can now fill in the TeX for your question here.

% Now, down to brasstacks. ** Writing good solutions is an Art **. 
% Eventually, you will find your own style. But here are some thoughts 
% to get you started: 
%
%   1. Write the solution as if you are writing it for your favorite
%      14-17 year old to help him/her understand. Could be your nephew, 
%      your niece, a cousin perhaps or probably even you when you 
%      were that age. Just write for them.
%
%   2. Use margin-notes to "talk" to students about the critical insights
%      in the question. The tone can be - in fact, should be - informal
%
%   3. Don't shy away from creating margin-figures you think will help
%      students understand. Yes, it is a little more work per question. 
%      But the question & solution will be written only once. Make that
%      attempt at writing a solution count.
%
%   4. At the same time, do not be too verbose. A long solution can
%      - at first sight - make the student think, "God, that is a lot to know".
%      Our aim is not to scare students. Rather, our aim should be to 
%      create many "Aha!" moments everyday in classrooms around the world
% 
%   5. Ensure that there are *no spelling mistakes anywhere*. We are an 
%      education company. Bad spellings suggest that we ourselves 
%      don't have any education. Also, use American spellings by default
% 
%   6. If a question has multiple parts, then first delete lines 40-41
%   7. If a question does not have parts, then first delete lines 43-69

\question[4] Find the sum of all \text{three digit} natural numbers which leave a 
remainder 2 when divided by 3

\insertQR{QRC}

\ifprintanswers
\fi 

\begin{solution}[\halfpage]
	The first such number is $101 = 3\times 33 + 2$. The last such number is $998 = 3\times 332 + 2$
	What does this mean? It means that to find the required sum, we should ignore the first $32$ such numbers
	and any number after the $332^{nd}$ one
	
	\begin{align}
		S &= S_{332} - S_{32} \\
		&= \sum_{k=32}^{332}(3n + 2) = \left[ 3\eSumOfN{n} + 2n \right]_{32}^{332} \\
		&= \left[ \dfrac{n}{2}\cdot(3n+7)\right]_{32}^{332} \\
		&= \dfrac{332}{2}\times 1003 - \dfrac{32}{2}\times 103 = 164,850
	\end{align}
\end{solution}
