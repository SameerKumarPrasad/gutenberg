% This is an empty shell file placed for you by the 'examiner' script.
% You can now fill in the TeX for your question here.

% Now, down to brasstacks. ** Writing good solutions is an Art **. 
% Eventually, you will find your own style. But here are some thoughts 
% to get you started: 
%
%   1. Write to be understood - but be crisp. Your own solution should not take 
%      more space than you will give to the student. Hence, if you take more than 
%      a half-page to write a solution, then give the student a full-page and so on...
%
%   2. Use margin-notes to "talk" to students about the critical insights
%      in the question. The tone can be - in fact, should be - informal
%
%   3. Don't shy away from creating margin-figures you think will help
%      students understand. Yes, it is a little more work per question. 
%      But the question & solution will be written only once. Make that
%      attempt at writing a solution count.
%      
%      3b. Use bc_to_fig.tex. Its an easier way to generate plots & graphs 
% 
%   4. Ensure that there are *no spelling mistakes anywhere*. We are an 
%      education company. Bad spellings suggest that we ourselves 
%      don't have any education. Also, use American spellings by default
% 
%   5. If a question has multiple parts, then first delete lines 40-41
%   6. If a question does not have parts, then first delete lines 43-69
%   
%   7. Create versions of the question when possible. Use commands defined in 
%      tufte-tweaks.sty to do so. Its easier than you think

% \noprintanswers
% \setcounter{rolldice}{3}

\ifnumequal{\value{rolldice}}{0}{
  % variables 
  \renewcommand{\vbone}{even}
  \renewcommand{\vbtwo}{1,7,6,8,4,3}
  \renewcommand{\vbthree}{3} % # odd 
  \renewcommand{\vbfour}{3} % # even
  \renewcommand{\vbsix}{360}
}{
  \ifnumequal{\value{rolldice}}{1}{
    % variables 
    \renewcommand{\vbone}{odd}
    \renewcommand{\vbtwo}{5,8,3,2,4,1,9}
    \renewcommand{\vbthree}{4}
    \renewcommand{\vbfour}{3}
    \renewcommand{\vbsix}{2880}
  }{
    \ifnumequal{\value{rolldice}}{2}{
      % variables 
      \renewcommand{\vbone}{even}
      \renewcommand{\vbtwo}{7,8,3,4,1,6}
      \renewcommand{\vbthree}{3}
      \renewcommand{\vbfour}{3}
      \renewcommand{\vbsix}{360}
    }{
      % variables 
      \renewcommand{\vbone}{odd}
      \renewcommand{\vbtwo}{2,5,7,9,3,4}
      \renewcommand{\vbthree}{4}
      \renewcommand{\vbfour}{2}
      \renewcommand{\vbsix}{480}
    }
  }
}

\gcalcexpr[0]\tp{\vbthree + \vbfour}
\ifthenelse{\equal\vbone{odd}}{\renewcommand{\vbfive}{\vbthree}}{\renewcommand{\vbfive}{\vbfour}}

\question How many \textit{\vbone} numbers can be formed using $\lbrace \vbtwo \rbrace$. All digits 
must be used and no digit repeats

\insertQR{}

\watchout

\ifprintanswers
\fi 

\begin{solution}
	There are a total of $\tp$ digits available - of which $\vbthree$ are odd and $\vbfour$ are even. 
	To form an \textit{\vbone} number, it must end with one of the $\vbfive$ \vbone digits. The other 
	digits can be in any order
	
	And therefore
	\begin{align}
		N &= (\tp - 1)!\times\vbfive = \vbsix
	\end{align}
\end{solution}
